% Options for packages loaded elsewhere
\PassOptionsToPackage{unicode}{hyperref}
\PassOptionsToPackage{hyphens}{url}
%
\documentclass[
  10pt,
  letterpaper,
  twoside]{scrbook}

\usepackage{amsmath,amssymb}
\usepackage{setspace}
\usepackage{iftex}
\ifPDFTeX
  \usepackage[T1]{fontenc}
  \usepackage[utf8]{inputenc}
  \usepackage{textcomp} % provide euro and other symbols
\else % if luatex or xetex
  \usepackage{unicode-math}
  \defaultfontfeatures{Scale=MatchLowercase}
  \defaultfontfeatures[\rmfamily]{Ligatures=TeX,Scale=1}
\fi
\usepackage{lmodern}
\ifPDFTeX\else  
    % xetex/luatex font selection
  \setmainfont[ItalicFont=EB Garamond Italic,BoldFont=EB Garamond
Bold]{EB Garamond Math}
  \setsansfont[]{Europa-Bold}
  \setmathfont[]{Garamond-Math}
\fi
% Use upquote if available, for straight quotes in verbatim environments
\IfFileExists{upquote.sty}{\usepackage{upquote}}{}
\IfFileExists{microtype.sty}{% use microtype if available
  \usepackage[]{microtype}
  \UseMicrotypeSet[protrusion]{basicmath} % disable protrusion for tt fonts
}{}
\usepackage{xcolor}
\usepackage[left=1in, right=1in, top=0.8in, bottom=0.8in,
paperheight=9.5in, paperwidth=6.5in, includemp=TRUE, marginparwidth=0in,
marginparsep=0in]{geometry}
\setlength{\emergencystretch}{3em} % prevent overfull lines
\setcounter{secnumdepth}{5}
% Make \paragraph and \subparagraph free-standing
\ifx\paragraph\undefined\else
  \let\oldparagraph\paragraph
  \renewcommand{\paragraph}[1]{\oldparagraph{#1}\mbox{}}
\fi
\ifx\subparagraph\undefined\else
  \let\oldsubparagraph\subparagraph
  \renewcommand{\subparagraph}[1]{\oldsubparagraph{#1}\mbox{}}
\fi


\providecommand{\tightlist}{%
  \setlength{\itemsep}{0pt}\setlength{\parskip}{0pt}}\usepackage{longtable,booktabs,array}
\usepackage{calc} % for calculating minipage widths
% Correct order of tables after \paragraph or \subparagraph
\usepackage{etoolbox}
\makeatletter
\patchcmd\longtable{\par}{\if@noskipsec\mbox{}\fi\par}{}{}
\makeatother
% Allow footnotes in longtable head/foot
\IfFileExists{footnotehyper.sty}{\usepackage{footnotehyper}}{\usepackage{footnote}}
\makesavenoteenv{longtable}
\usepackage{graphicx}
\makeatletter
\def\maxwidth{\ifdim\Gin@nat@width>\linewidth\linewidth\else\Gin@nat@width\fi}
\def\maxheight{\ifdim\Gin@nat@height>\textheight\textheight\else\Gin@nat@height\fi}
\makeatother
% Scale images if necessary, so that they will not overflow the page
% margins by default, and it is still possible to overwrite the defaults
% using explicit options in \includegraphics[width, height, ...]{}
\setkeys{Gin}{width=\maxwidth,height=\maxheight,keepaspectratio}
% Set default figure placement to htbp
\makeatletter
\def\fps@figure{htbp}
\makeatother
% definitions for citeproc citations
\NewDocumentCommand\citeproctext{}{}
\NewDocumentCommand\citeproc{mm}{%
  \begingroup\def\citeproctext{#2}\cite{#1}\endgroup}
\makeatletter
 % allow citations to break across lines
 \let\@cite@ofmt\@firstofone
 % avoid brackets around text for \cite:
 \def\@biblabel#1{}
 \def\@cite#1#2{{#1\if@tempswa , #2\fi}}
\makeatother
\newlength{\cslhangindent}
\setlength{\cslhangindent}{1.5em}
\newlength{\csllabelwidth}
\setlength{\csllabelwidth}{3em}
\newenvironment{CSLReferences}[2] % #1 hanging-indent, #2 entry-spacing
 {\begin{list}{}{%
  \setlength{\itemindent}{0pt}
  \setlength{\leftmargin}{0pt}
  \setlength{\parsep}{0pt}
  % turn on hanging indent if param 1 is 1
  \ifodd #1
   \setlength{\leftmargin}{\cslhangindent}
   \setlength{\itemindent}{-1\cslhangindent}
  \fi
  % set entry spacing
  \setlength{\itemsep}{#2\baselineskip}}}
 {\end{list}}
\usepackage{calc}
\newcommand{\CSLBlock}[1]{\hfill\break\parbox[t]{\linewidth}{\strut\ignorespaces#1\strut}}
\newcommand{\CSLLeftMargin}[1]{\parbox[t]{\csllabelwidth}{\strut#1\strut}}
\newcommand{\CSLRightInline}[1]{\parbox[t]{\linewidth - \csllabelwidth}{\strut#1\strut}}
\newcommand{\CSLIndent}[1]{\hspace{\cslhangindent}#1}

\setlength\heavyrulewidth{0ex}
\setlength\lightrulewidth{0ex}
\usepackage[automark]{scrlayer-scrpage}
\clearpairofpagestyles
\cehead{
  \leftmark
  }
\cohead{
  \rightmark
}
\ohead{\bfseries \pagemark}
\cfoot{}
\makeatletter
\newcommand*\NoIndentAfterEnv[1]{%
  \AfterEndEnvironment{#1}{\par\@afterindentfalse\@afterheading}}
\makeatother
\NoIndentAfterEnv{itemize}
\NoIndentAfterEnv{enumerate}
\NoIndentAfterEnv{description}
\NoIndentAfterEnv{quote}
\NoIndentAfterEnv{equation}
\NoIndentAfterEnv{longtable}
\renewcommand{\chaptermark}[1]{\markboth{#1}{}}
\renewcommand{\sectionmark}[1]{\markright{#1}}
\makeatletter
\@ifpackageloaded{bookmark}{}{\usepackage{bookmark}}
\makeatother
\makeatletter
\@ifpackageloaded{caption}{}{\usepackage{caption}}
\AtBeginDocument{%
\ifdefined\contentsname
  \renewcommand*\contentsname{Table of contents}
\else
  \newcommand\contentsname{Table of contents}
\fi
\ifdefined\listfigurename
  \renewcommand*\listfigurename{List of Figures}
\else
  \newcommand\listfigurename{List of Figures}
\fi
\ifdefined\listtablename
  \renewcommand*\listtablename{List of Tables}
\else
  \newcommand\listtablename{List of Tables}
\fi
\ifdefined\figurename
  \renewcommand*\figurename{Figure}
\else
  \newcommand\figurename{Figure}
\fi
\ifdefined\tablename
  \renewcommand*\tablename{Table}
\else
  \newcommand\tablename{Table}
\fi
}
\@ifpackageloaded{float}{}{\usepackage{float}}
\floatstyle{ruled}
\@ifundefined{c@chapter}{\newfloat{codelisting}{h}{lop}}{\newfloat{codelisting}{h}{lop}[chapter]}
\floatname{codelisting}{Listing}
\newcommand*\listoflistings{\listof{codelisting}{List of Listings}}
\makeatother
\makeatletter
\makeatother
\makeatletter
\@ifpackageloaded{caption}{}{\usepackage{caption}}
\@ifpackageloaded{subcaption}{}{\usepackage{subcaption}}
\makeatother
\ifLuaTeX
  \usepackage{selnolig}  % disable illegal ligatures
\fi
\IfFileExists{bookmark.sty}{\usepackage{bookmark}}{\usepackage{hyperref}}
\IfFileExists{xurl.sty}{\usepackage{xurl}}{} % add URL line breaks if available
\urlstyle{same} % disable monospaced font for URLs
\hypersetup{
  pdftitle={Normative Externalism},
  pdfauthor={Brian Weatherson},
  hidelinks,
  pdfcreator={LaTeX via pandoc}}

\title{Normative Externalism}
\author{Brian Weatherson}
\date{2019}

\begin{document}
\frontmatter
\maketitle

\renewcommand*\contentsname{Table of contents}
{
\setcounter{tocdepth}{2}
\tableofcontents
}
\setstretch{1.1}
\mainmatter
\bookmarksetup{startatroot}

\chapter*{Preface}\label{preface}
\addcontentsline{toc}{chapter}{Preface}

\markboth{Preface}{Preface}

Philosophy is hard. Ethics is hard; epistemology is hard; decision
theory is hard; logic is hard. All the parts of philosophy are hard, but
those four are going to be particularly relevant to the story I'm
telling here. They matter because they are all evaluative. Someone who
violates ethical principles is immoral; someone who violates
epistemological principles is irrational; someone who violates the
principles of decision theory is imprudent; someone who violates logical
principles is illogical. And to say that someone is immoral, irrational,
imprudent or illogical is to negatively evaluate them.

But it is easy to feel uneasy with this set of facts. If it is so hard
to figure out the truth in these fields, why should we negatively
evaluate someone for failing to conform to these hard to find standards?
Doesn't fairness require that we only judge people by standards they can
know about? I'm going to argue this is not right - that to evaluate
someone is necessarily to impose a standard on them, and they may not
even know what the standard is. Indeed, they may not have any reason to
believe the truth about what the standard is, and in extreme cases may
have good reason to endorse a false standard.

This position is uncomfortable, since it is easy to feel the unfairness
of holding someone to a standard that they do not accept, and could not
reasonably accept. Many philosophers think that we should either
supplement or replace these external standards with internal standards.
An `internal standard' here is one that the person being evaluated
either accepts, or has good reason to accept. To supplement the external
standards is to say that there are two ways to evaluate people. It is
good to live up to the correct standards in ethics, epistemology and
decision theory, and bad to violate them. But it is also, say the
supplementers, good to live up to one's own standards, and bad to
violate them. The replacers say that conformity to one's own standards
is more important than conformity to external standards; in some deep
sense (at least some of) the heroes of ethics, epistemology and decision
theory are people who abide by their own standards.

I am going to press two problems against this kind of view. The problems
are most pressing for the replacers, but they undermine the position of
the supplementers too.

The first problem is that this kind of view has problems with fanatics
and ideologues. Every ideologue who thought that they had figured out
the one true way things must be done and reacted violently against those
who didn't agree were doing well by their own lights. It's not good, in
any way, to be that kind of ideologue. We shouldn't look back at the
Reign of Terror and say, ``Well, at least {Robespierre} and {Saint-Just}
were living in accordance with their own values.'' Aiming to fit the
world to one's own values is a dangerous game; it's only worth playing
if you've got the values right. When we focus our attention on
ideologues who have gone off the rails, the idea that it is unfair to
hold people to a standard they can't see feels like something that's
problem in theory but not in practice.

The second problem with the the internal view is that it leads to a
nasty regress. It is, to be sure, hard to tell what the true values are.
But choosing some values does not end our problems. Morality is hard
even once you've settled on a moral theory. This is a point familiar
from, for example, Sartre's discussion of the young man torn between
duty to his mother and his country.

\begin{quote}
What could help him make that choice? The Christian doctrine? No.~The
Christian doctrine tells us we must be charitable, love our neighbour,
sacrifice ourselves for others, choose the ``narrow way,'' et cetera.
But what is the narrow way? Whom should we love like a brother--the
solider or the mother? \(\dots\) Who can decide that \emph{a priori}? No
one. No code of ethics on record answers that question.
~(\citeproc{ref-Sartre1946}{Sartre 1946/2007, 31})
\end{quote}

We can evaluate the young man by his own lights and still be in a way
unfair to him. Perhaps it turns out that the truly Christian thing to do
is to fight Nazis, but the young man concludes (reasonably but falsely)
that it is to help his mother. And he does that. If we are moved by the
unfairness of holding him to a standard he does not endorse, we should
also find it unfair to hold him to a consequence of his own standard
that he doesn't recognise. But now what is left of the internal
standard? It must be that it is good to do not what is best by one's own
lights, but what one thinks is best by one's own lights. But perhaps one
could even be wrong about \emph{that}. (I'll discuss an example of this
in chapter 1.) And the internal view collapses into the view that we
should evaluate people by what they think they think they think
\(\dots\) their own views support.

This is all absurd, and it makes the problem with fanatics and
ideologues even worse. Perhaps we could argue that some ideologues take
actions that are incompatible with what they say their values are. But
they do not act against what they think their own values require.

Perhaps we can motivate the importance of the internal point of view not
by thinking about fairness, but by focussing on an analogy with reckless
agents. If I fire a cannon down Fifth Avenue at peak hour, I do
something morally horrible even if miraculously I don't hit anyone. My
action is wrong because it is reckless. Perhaps if I do something that
is probably morally wrong, I am morally reckless in just the same way.
And that's true even if my action turns out not to be wrong. So what
matters is not just what is right and wrong, but probabilities of
rightness and wrongness. I think this kind of reasoning fails too, and
there are important asymmetries between physical risk (as is involved in
firing cannons down busy streets) and moral risk. I'll spend chapters
three and four outlining these asymmetries, and why they tell against
the idea that there is a distinctive wrong of moral recklessness.

The first half of the book discusses the significance of the internal
point of view in ethics. As I've indicated, I don't think it is
particularly important, though we'll spend a bit of time towards the end
of part one looking at some more limited, and hence more plausible,
claims for its usefulness. The second part of the book turns to
epistemology, and to the idea that one cannot reasonably have beliefs
that one believes (or should believe) to be unreasonable.

Again, the issue turns on how important is conformity to one's own
standards. The most common philosophical view around here is a kind of
supplementing view, not a replacing view. It is important, say several
philosophers, to have beliefs that are both actually reasonable and also
reasonable by one's own lights. And I'm going to push back against that.
One reason comes from work by Timothy Williamson. What's reasonable to
believe turns on empirical facts about one's situation. Since we don't
have God-like perfect access to our own empirical situation, we might
not realise what is reasonable to do in our own situation just because
we don't know precisely what situation we are in. In such cases, it
seems we should react to the situation we are actually in, not to our
best guess about what situation that is.

There will be two primary themes of part two of the book. One echoes the
first part of the book. Sometimes we cannot know what it would be to be
reasonable by our own lights. So adding a requirement that reasonable
people are doing well by their own lights threatens to trigger a vicious
regress. I'm going to argue that this threat is realised. The other
theme is that the phenomena that philosophers have thought could only be
explained by adding an internal constraint onto belief can be adequately
explained by a more careful attention to the nature of evidence, and
what it takes for one to have evidence and for that evidence to support
a belief. I'll argue that such explanations are preferable to
explanations in terms of internal constraints (such as only believe what
you believe is reasonable to believe). This is in part because they
avoid regress and implausible knowledge about one's own situation; in
part because they only commit us to things we are independently
committed to; and in part because they explain a much broader range of
cases than are explained by the alleged internal constraints.

I have more people to thank for help with this book than I could
possibly list here. I'm not even sure at which point of time I should
start the thanks. Twenty-odd years ago as a graduate student at Monash I
wasn't working on \emph{this} project. But the picture that pervades
this book, that in philosophy everything is contestable and there are no
safe stopping points, owes a lot to the amount of time I spent as a
graduate student thinking about, and being taught about, heterodox
approaches to logic and to decision theory.

Most of the best feedback I've received on the various parts of the book
has come from graduate students. Some of the second part of the book is
based on an epistemology seminar I taught at Rutgers. I taught a
graduate seminar at Michigan off an early draft of the book manuscript.
And I've taught several mini-courses at St Andrews, and presented at
even more workshops and symposia there, off parts of the book. In every
case the feedback I received from colleagues and, even more frequently,
graduate students, changed the book for the better.

Parts of the book are based on presentations at or organised by the
University of Aberdeen, University of Oxford, University of Vienna,
University of Konstanz, University of Zurich, University of Graz,
Massachusetts Institute of Technology, Princeton University, Ohio State
University, University of Sydney, Australian National University and
University of Melbourne. I've presented parts of it at the Bellingham
Summer Philosophy Conference, the Night of Philosophy in New York City
and the Australasian Association of Philosophy annual conference. And
I've discussed it with the Corridor Reading Group in New York, and the
Ethics Lunch group in Ann Arbor. I'm very grateful for all the feedback
I got at those presentations.

As well as all those audiences, I'd like to particularly thank Derek
Ball, Jessica Brown, Sarah Buss, Herman Cappelen, Ruth Chang, Stewart
Cohen, Josh Dever, Tom Donaldson, Andy Egan, Claire Field, Katherine
Hawley, Scott Hershowitz, Torfinn Huvenes, Jonathan Jenkins Ichikawa,
Jim Joyce, Zoe Johnson King, Maria Lasonen-Aarnio, Ben Levinstein, Julia
Markovits, Matthew McGrath, Sarah Moss, Jill North, Caroline Perry,
Quentin Pharr, Lewis Ross, Andrew Sepielli, Joe Shin, Holly Smith,
Martin Smith and Elia Zardini for particularly valuable feedback. (And
I'm already dreading finding out who I should have included on this list
but didn't.) Ralph Wedgwood read the whole manuscript and provided
comments that improved it in innumerable ways. Thanks to him, and to
Peter Momtchiloff for making such an astute choice of reader for the
manuscript.

The idiosyncratic workflow I used for writing this would have been
impossible without Fletcher Penney's Multimarkdown (both the language
and the Composer software) and John MacFarlane's Pandoc, and I'm very
grateful to both of them for building such valuable tools. Much of the
book was drafted under the dome in the La Trobe Reading Room at the
State Library of Victoria, and I'm so grateful that Victoria has
maintained that space, and that building.

Early in the development of this book project, I was honoured to become
the first Marshall M. Weinberg Professor of Philosophy at the University
of Michigan, Ann Arbor. Without the support Marshall has provided to my
research, and to the research project of the University of Michigan more
broadly, this project would have been unimaginable. My inaugural lecture
was ``Running Risks Morally'', most of which appears in one way or
another in part one of the book. The first draft of the book was written
while on a sabbatical funded through the Weinberg Professorship. But
beyond that, the vibrant intellectual community here at Michigan relies
in ever so many ways on Marshall's support. I couldn't tell you how much
this book relies on feedback from graduate students who have received
Weinberg fellowships, or who came to Michigan in part because of the
Weinberg Center for Cognitive Science. While this is by no means a work
of cognitive science, it is influenced in many ways by what I've learned
from cognitive scientists talking at the Weinberg Center. And I really
cannot thank Marshall enough for his support for Michigan, and for its
research.

Finally, I'd like to thank Ishani Maitra and Nyaya Maitra Weatherson
for, well, everything. Ishani didn't just talk through all the things in
this book with me, and improved it in so many ways, but she also talked
through all the things I cut from the book. And she improved those
portions too.

\bookmarksetup{startatroot}

\chapter{Introduction}\label{introduction}

\section{To Thine Own Self Be True}\label{tothineownselfbetrue}

Early in \emph{Hamlet}, Laertes departs Elsinore for Paris. As he
prepares to go his father, Lord Polonius, offers him some paternal
advice. He tells him to talk less and smile more. He tells him to spend
all his money on clothes, since that's how they roll in Paris. He tells
him to neither a borrower nor a lender be, though the latter is
presumably redundant if he's taken the advice to date. And he concludes
with this advice, destined to adorn high school yearbooks for centuries
to come.

\begin{quote}
This above all: to thine own self be true,\\
And it must follow, as the night the day,\\
Thou canst not then be false to any man.
\end{quote}

It isn't completely clear what Polonius means when he advises Laertes to
be true to himself, but it is plausible that he means something like
this:

\begin{quote}
Follow your own principles!
\end{quote}

Or perhaps something like this:

\begin{quote}
Do what you think is right!
\end{quote}

And unlike the rest of the advice Polonius gives, many philosophers have
followed him in thinking this is a very good idea.

The primary aim of this book is argue against this idea. Following one's
own principles, or doing what one thinks is right, are not in general
very good ideas at all. I will call \textbf{normative internalism} the
view that we should be guided by norms that are internal to our own
minds, in the sense that our beliefs, and our (normative evidence) is
internal to our minds. And I will oppose that view, arguing for
\textbf{normative externalism}.

Normative externalism is the view that the most important standards for
evaluating actions, mental states and agents are typically external to
the actor, believer or agent being evaluated. It can be appropriate to
hold someone to a moral, or an epistemic, standard that they do not
endorse, or even that they could not be reasonably expected to endorse.
If one has bad standards, there need be nothing wrong in violating them,
and there is nothing good about upholding them.

That last paragraph made a lot of distinct claims, and it is worth
spending some time teasing them apart. But before we get too deep in the
weeds, I want to have on the table the guiding principle of the book.
Being true to yourself, in the sense of conforming to the principles one
has, or even to the principles one has reason to have, is just not that
important. What is important is doing the right thing, being a good
person, and having rational beliefs. If one has misguided views about
the right, the good, and the rational, then there is nothing good about
conforming to those misguided views. And this matters, because many
people have views about the right, the good, and the rational, that are
very misguided indeed.

\section{Four Questions}\label{fourquestions}

\subsection{Actions, Agents or Advice}\label{actionsagentsoradvice}

If one says, with Polonius, that it is good to conform to one's own
principles, there are a number of distinct things one could be meaning.

One could be making a claim about particular \emph{actions}. (Or about
particular beliefs, but we'll focus on actions for the next few
paragraphs.) So one could be saying that actions that conform to the
actor's principles are thereby in some sense right or good, and those
that violate the actor's principles are in some sense wrong or bad.

Alternatively, one could be making a claim about \emph{agents}. So one
could be saying that people who (typically) conform their actions to
their principles are in some sense good (or less bad) people, and those
who violate their own principles are in some sense bad.

Or alternatively again, one could be making a claim about \emph{advice}.
One could be saying that whether or not the claims in the previous two
paragraphs are strictly correct, it is excellent to advise people to act
according to their principles. There are plenty of cases where advising
people to do the optimal thing is bad, especially if aiming for the
optimal result is likely to lead to catastrophe. So this view about
advise is in principle distinct from the views about actions and agents.

The form of externalism I will defend is opposed to the views in all of
the last three paragraphs. But it is most strongly opposed to the view
about actions, and least strongly opposed to the view about advice.
Indeed, I won't have a lot to say about advice throughout the book;
except to note occasionally when intuitions about advice seem to be
getting used illegitimately to justify conclusions about actions. But I
don't mean to imply that the views have to stand or fall together. A
view that is externalist about actions - it thinks it doesn't make any
difference to the correct evaluation of an action whether the actor
endorsed it or not - but internalist about agents - it thinks there is
something good about people who stick to their principles and bad about
those who do not - is certainly worth considering. But it isn't my view;
I mean to oppose all three precisifications of what Polonius says.

\subsection{Above All?}\label{aboveall}

Polonius does not just say Laertes should be true to himself. He says
this is something `above all'. This suggests that he is elevating
\emph{Do what you think is right} to a central place, making it more
important than principles like \emph{Respect other people}, or
\emph{Make the world better}, or even \emph{Do the right thing}.

The externalist view I support takes completely the opposite tack. The
principle \emph{Do what you think is right} is of no importance at all.

But there is a large middle ground position. This is easiest to see if
we assume the debate is about agents, not actions or advice, so I'll
present it for agents. But it shouldn't be too hard to see how to
generalise the idea.

We could hold that doing what one thinks is right is one of the virtues,
something that contributes to a person being a good person. Or we might
think that failing to do what one thinks is right is a vice, something
that contributes to a person being a bad person. And we might think one
or other (or both) of those things without thinking them particularly
important virtues or vices. One could coherently hold that there is a
virtue in holding to one's principles, even if one thinks that other
virtues to do with honesty, courage, respect and the like are more
important. And one could coherently hold that doing what one thinks is
wrong is a vice, even in the case where one has false enough views about
first-order moral questions that doing what one thinks it right would
manifest even more serious vices.

Indeed, one might think that ordinary English goes along with this. We
do talk somewhat admiringly about people who are principled or resolute,
and somewhat disdainfully about people who are hypocritical.\footnote{Though
  to be clear, I don't think the English words `principled' and
  `resolute' actually pick out the so-called virtue of upholding one's
  own principles. Following Richard Holton
  (\citeproc{ref-Holton1999}{1999}), I think those words pick out
  diachronic properties of a person. They apply to a person in part due
  to that person's constancy over time in some respect. Following one's
  principles isn't like this; it is a purely synchronic affair.}

I'm going to classify this kind of view, the one that says that doing
what one thinks is right is important to character, but not of maximal
importance, as a moderate internalist view. And my externalism will be
opposed to it, like it is opposed to the view that being principled, and
avoiding hypocrisy, are the most important virtues.

The possibility of such a moderate internalist view is important,
because otherwise we might think the argument against internalism would
be too easy. History is full of fanatics who convinced themselves that
they were doing the right thing while causing immense harm. It is hard
to believe that the one principle they did conform to, \emph{Follow your
own principles}, is the most important principle of all. But perhaps,
just perhaps, their resoluteness is in a small way a virtue. At least, a
philosophical view that says that it is a virtue, albeit one offset by
mountains of vice, is not absurd.

\subsection{Ethics, Epistemology and
More}\label{ethicsepistemologyandmore}

I've been interpreting Polonius's dictum as being primarily about ethics
so far. But views like his are available in many other areas of
philosophy. I'll mention three more here, the first of which will be a
major focus of this book.

Belief is subject to evaluation on a number of fronts. Beliefs are true
or false, but that hardly exhausts their virtues or vices. Some true
beliefs are bad in virtue of being lucky guesses, or leaps to
unwarranted conclusions. Some false beliefs are the result of sensibly
following the evidence where it leads, and just being unluckily misled
into error. So as well as evaluating a belief for truth, we can evaluate
it for responsiveness to the evidence. I'm going to argue, somewhat
indirectly, that a belief is rational just in case it is responsive to
the evidence in this way.\footnote{Though getting clear on just what
  this last sentence commits me to will require saying more about what
  evidence is. For now, it won't do much harm to equate evidence with
  basic knowledge. A proposition \emph{p} is part of the subject's
  evidence if the subject knows \emph{p}, and doesn't know \emph{p}
  because she inferred it from something else.}

But if that's what rationality is, then subjects can also have beliefs
about the rationality of their own beliefs. And we can ask whether
subjects are doing well at believing by their own lights. To believe
something just is to believe it is true, so if our only standard for
belief is truth, then everyone will believe well by their own lights.
But it is possible to believe something, and even rationally believe it,
while believing that that very belief is irrational. Or, at least, so
I'll argue.

Is this a bad thing? Should we mark someone down for believing in a way
that they take to be irrational? I'm going to argue that we should not.
It's good to believe truths. It's good to believe in accord with one's
evidence. And that's as far as we should go. It's not good to believe in
accord with what one believes the evidence supports, unless one thereby
ends up with a belief that is good for some other reason. And it's not
bad to believe something that one believes is not supported by one's
evidence, unless one ends up with a belief that is bad for some other
reason.

Just as in the ethics case, we can separate out a number of distinct
questions here. Assume you think there is something philosophically
important about beliefs that are irrational by the lights of the
believer themselves. You could say that this is a bad-making feature of
the belief itself, or a bad-making feature of the believer, or, perhaps
that it is bad to advise people to have beliefs that are irrational by
their own lights. That is, we can replicate the act, agent or advice
distinction inside epistemology, though the `acts' are really the states
of holding particular beliefs. And if you do think these beliefs, or
believers, are bad in some way, there is a further question about how
much badness is involved. Is believing in a way that one thinks is
irrational as bad as not following the (first-order) evidence, or more
bad, or less bad. (Or is badness the wrong concept to be using here?)

We will see different philosophical views that take different stands on
these questions throughout part II of the book. I'm going to defend a
fairly simple, and fairly extreme, position. It isn't a bad making
feature, in any way, of a belief that the believer thinks it is
irrational, nor is it a bad making feature of believers that they have
beliefs they think are irrational. It isn't even a bad habit to
routinely have beliefs that one thinks are irrational; though I'm going
to be a little more tentative in defending that last conclusion. The
general principle throughout is to motivate and defend a picture where
what matters is conformity to the actual rules - be they rules of action
or rules of belief - rather than conformity to what one takes (or even
rationally takes) the rules to be.

The disputes of the last few paragraphs have all been over epistemology,
fairly narrowly construed. But there are some other disputes that we
might have to, where the difference between conformity to external rules
and conformity to one's version of the rules matters. I'm not going to
say much about the next two disputes, but they are helpful to have on
the table.

Some lives go better than others. When we act for the sake of others,
when we act benevolently, we aim to improve the lives of others. Call
someone's \emph{welfare} that quantity we improve when we act
benevolently.\footnote{There are a lot of different things that people
  call welfare in the philosophical literature. I'm taking the idea of
  tying it definitionally to benevolent action from Simon Keller
  (\citeproc{ref-Keller2009}{2009}).} Philosophers disagree a lot about
what welfare is, so some of them are wrong. And though I'm not going to
argue for this, it seems to me that the disagreeing parties each have
such good arguments that at least some of the philosophers who are wrong
are nevertheless rational in holding the position they do. So that
implies that a rational person could have a choice between two actions,
one of which actually produces more welfare, and the other of which
produces more welfare according to the theory of welfare they
(rationally) hold. Assuming the person wants to act benevolently, or, if
the act is directed to their own good, they want to act prudentially, is
there anything good about doing the thing that produces more welfare
according to the theory of welfare they hold? My position, though I'm
not going to argue for this in this book, is that there is not. What
matters for benevolent or prudential action is how well one's act does
according to the correct theory of welfare. It doesn't make an action
benevolent, or prudent, if the action is good according to a mistaken
theory of welfare. That's true even if the theory of welfare is one's
own, or even if it is the one that is rational for one to hold. If one's
theory of welfare is a purely hedonistic experiential theory of welfare,
then you might think you are improving the welfare of others by
force-feeding them happy pills. But if that theory of welfare is false,
and welfare involves preference satisfaction, or autonomy, then such an
action will not be benevolent, nor will it be rational to perform on
benevolent grounds.

We can make the same kind of distinction within decision theory. Let's
assume for now that a person has rational beliefs, and when they lack
belief they assign a rational probability to each uncertain outcome, and
they value the right things. There is still a question about how they
should act in the face of uncertainty. Unlike the questions about
ethics, epistemology, or welfare, there is an orthodox answer here. They
should maximise expected utility. That is, for each act, they should
multiply the probability of each outcome given that act, by the
(presumably numerical) value of that outcome-act pair, and add up the
resulting products to get an expected value of the act. Then they should
choose the act with the highest expected value. But while this is the
orthodox view of decision theory, there are dissenters from
it\footnote{I'm suppressing disputes within orthodoxy about how just to
  formulate the view, though those disputes would also suffice to get
  the kind of example I want going.}. The best recent statement of
dissent is in a book-length treatment by Lara Buchak
(\citeproc{ref-BuchakRisk}{2013}). And someone who has read Buchak's
book can think that her view is true, or, perhaps, think that there is
some probability that it is true and some probability that the orthodoxy
is true.

So now we can ask the same kind of question about conformity to the
correct rules versus conformity to the rules one thinks are
correct.\footnote{If the moral theories one gives credence to reject
  expected value maximisation, then there will be even more
  complications at the intersection of ethics and decision theory. Ittay
  Nissan-Rozen (\citeproc{ref-NissanRozen2015}{2015}) has a really nice
  case showing the complications that arise for the internalist when
  moral theories do not assume orthodox decision theory.} Assume that
someone does not have the correct beliefs about how to rationally make
decisions. And assume that they perform an act which is not rational,
according to the true decision theory, but is rational according to the
decision theory they accept. Is there something good about that
decision, and would there have been something bad about them doing the
thing that correct theory recommended? My position is that there is not.
The rational decisions are the ones recommended by correct decision
theory. There is nothing to be said for conforming to one's own
preferred decision theory, if that theory is false.

\subsection{Actual or Rational}\label{actualorrational}

So far I've focussed on the distinction between principles that are
external to the agent, and principles that are internal to the agent in
the sense of being believed by the agent, or being the agent's own
principles. When I call my view externalist, it is to indicate that I
think it is the external principles that matter. But there is another
category of principles that I haven't focussed on, and which are in some
sense internal. These are the principles that the agent should,
rationally, accept.

Now if we say that the agent should rationally accept all and only the
true principles, then there won't be a distinction between \emph{Follow
the true principles} and \emph{Follow the principles it is rational to
accept}. But let's work for now with the assumption that there is a
difference here; that just like with anything else, agents can be
rationally misled about the nature of ethics, epistemology, welfare, and
decision theory.\footnote{Julia Markovits
  (\citeproc{ref-Markovits2014}{2014}) argues that agents have rational
  reason to accept the fundamental moral truths. Michael Titelbaum
  (\citeproc{ref-Titelbaum2015}{2015}) argues that agents have rational
  reason to accept the fundamental epistemological truths. I'm assuming
  for now that both of these positions are false, because it gives my
  opponents more room to move if they are false. Claire Field
  (\citeproc{ref-Field2017}{forthcoming}) responds to Titelbaum's
  arguments. Note here that when I say that an agent can be rationally
  misled about morality and epistemology, I am not claiming that they
  can rationally have outright false beliefs about morality and
  epistemology. I just mean that rationality is consistent with having
  something other than complete certainty in the claims that are
  actually true.} Then there is another possibility; that agents should
follow the principles that they have most reason to believe are true.

This gives another way for the internalist to respond to the problem of
historical monsters. Let's think about one particular case, one that
I'll return to occasionally in the book: Maximilien
{Robespierre}\footnote{There are more historical sources on
  {Robespierre} than would be remotely possible to list. The things I
  say here are largely drawn from recent work by Peter McPhee
  (\citeproc{ref-McPhee2012}{2012}), Ruth Scurr
  (\citeproc{ref-Scurr2006}{2006}) and especially Marisa Linton
  (\citeproc{ref-Linton2013}{2013}). The study of the Committee of
  Public Safety by R. R. Palmer (\citeproc{ref-Palmer1941}{1941}) is
  helpful for seeing {Robespierre} in context, and especially seeing him
  alongside men with even more extreme characteristics than his.}.
Whatever else one can say about him, no one seriously doubts that
{Robespierre} always did what he thought was right.\footnote{Most
  revolutionary leaders are either power-hungry or bloodthirsty. But
  {Robespierre} genuinely seems to have been neither of those, except
  perhaps at the very very end. Linton (\citeproc{ref-Linton2013}{2013,
  97--99}) is particularly clear on this point.} But doing what he
thought was right involved setting off the Reign of Terror, and
executing ever so many people on incredibly flimsy pretexts. We can't
really say that the principle he did well by, \emph{Do what you think is
right}, is one that should be valued above all. We mentioned above that
we could reasonably say it is a good-making feature of {Robespierre}
that he was principled, even if it is outweighed by how abhorrent his
set of principles turned out to be. But the interest here is in whether
we can find some internalist principle that can be said to be true
`above all' in his case.\footnote{One thing that won't rescue intuitions
  about the case is to say that \emph{Do what you think is right} is
  important only if the agent is `procedurally rational'. {Robespierre}
  used the right methods to form moral beliefs: he read widely, talked
  to lots of people, and reflected on what he heard and saw. He just got
  things catastrophically wrong. Gideon Rosen
  (\citeproc{ref-Rosen2003}{2003}, \citeproc{ref-Rosen2004}{2004})
  places a lot of emphasis on procedural rationality in defending a form
  of internalism, though his aim is very much not to track intuitions
  about particular cases.}

{Robespierre} had ample reason to believe that he had ended up on the
wrong track. He wasn't brainwashed into believing that the Terror was
morally justifiable; the reasons for it were clearly present to him. The
results of the Terror weren't playing out in some distant land, or in
the hold of a slave ship, they were right in front of him. And he knew a
lot of moral and political theory. He was well educated in the classics.
He read Montesquieu. He read, and adored, Rousseau. He sat through hours
upon hours of debate every day about the efficacy and morality of
government actions, both before and during his reign. Even if one
thinks, as I do, that sometimes the reasons for the immorality of an
action are hidden from the actor, that can hardly be said to be true in
{Robespierre}'s case.

So I think we can reasonably say in {Robespierre}'s case that he
violated the rule \emph{Follow the principles it is rational to accept}.
And that rule is an internal rule, in some sense. If we take it to be
the primary rule, then we won't judge people by standards that are
hidden from them. We may judge them by standards they don't accept, but
only when they have reason to accept the standards. So I'll treat it as
another internalist approach, though very different from the approach
that says it is most important for people to follow their own
principles.

So we have two very different kinds of internalist approaches to ethics,
epistemology, welfare and decision theory. One says that it is (most)
important that people follow their own principles. The other says that
it is (most) important that people follow the principles they have
rational reason to accept. The first, in its strongest form, says absurd
things about the case of fanatics. As I'll argue at length in what
follows, it also leads to nasty regresses. The second does not have
these problems. But it is very hard to motivate. We will spend some time
on the reasons philosophers have had for wanting views like Polonius's.
All of these, I'll argue, push towards the idea that the most important
thing is that people follow the principles they actually accept. None of
them, when considered carefully, give us a reason to prefer principles
the actor or believer has reason to accept to the principles that are
actually true. Retreating from \emph{Follow your own principles} to
\emph{Follow the principles it is rational to accept} lets the
internalist avoid harsh cases like {Robespierre}, but at the cost of
abandoning the interesting reasons they have for their view.

\subsection{Some Caveats}\label{somecaveats}

I've spoken freely in this section about the true moral principles. That
way of speaking presupposes that there are moral truths. I mean to be
using the phrase `moral truths' in as non-committing as sense as is
possible. I don't mean to say that the moral truths are
mind-independent. If it is true that murder is wrong in virtue of our
disapproval of murder, it is still true that murder is wrong, and that's
enough for current purposes. Nor do I mean to insist that the moral
truths are invariant across space and time. There are hard questions
about how we should evaluate actors from different times and places if a
form of moral relativism is true. But those questions are largely
orthogonal to the one's I'm interested in.

I am in effect assuming away a very strong form of moral relativism, one
that makes moral truth relative to the moral principles of the actor
being evaluated. But that's not a plausible form of moral relativism. If
moral relativism is true, then what morality is relative to is much more
inclusive than a single person; it is something like a culture, or a
practice. And that is enough for there to be a difference between what a
person accepts, and what is true in their culture or practice.

As briefly noted above, I'm also assuming that there is a difference
between what is true and what it is rational to accept. All I really
need here is that it can be rational to be less than fully certain in
some moral and epistemic truths. I'm not going to assume, for example,
that one can rationally believe moral or epistemic falsehoods. I've
spoken above as if that is possible, but that was a convenient
simplification. What's going to really matter is just the existence of a
gap between what's true and what's reasonable to believe, and that gap
can arise even if all the things that are reasonable to believe are
true.

Finally, you may have noticed that we ended up a long way from anything
that could be plausibly attributed to Lord Polonius. When he tells
Laertes to be true to himself, I'm pretty sure he's not saying anything
about whether Laertes should have beliefs that are rational by the
standards that Laertes should rationally accept. Yet whether Laertes (or
anyone else) should have such beliefs is one of the questions we ended
up being interested in. The good Lord's role in this play was just to
introduce the distinction between following one's own principles and
following the true principles. With that distinction on stage, we can
let Polonius exit the scene.

\section{Normative Externalism
Defined}\label{normativeexternalismdefined}

Normative externalism is the view that the most important evaluations of
actions and actors, and of beliefs and believers, are independent both
of the actor or believer's belief about the value of their action or
belief, and of the evidence the actor or believer has about the value of
their action or belief. The aim of this book is to defend normative
externalism, and explore why it is philosophically important.

It is tempting to strengthen this kind of normative externalism further,
and say that what one should do and believe is completely independent of
what one believes one should do and believe. But this strong
independence claim can't be right. (I'm grateful here to Derek Ball.) If
one thinks that one should murder one's neighbours, then one ought to
get professional help. Sometimes normative beliefs change the normative
significance of other actions. So the externalist claim I'm defending is
a little weaker than this general independence claim. It allows that a
normative belief \emph{B} may change the normative status of actions and
beliefs that are not part of the content of \emph{B}. But the
externalism I'm defending is still going to be strong enough to allow a
lot of critics.

The strongest kind of normative internalism says that the value of
actions and beliefs is tightly tied to the beliefs that actors and
believers have about their own actions and beliefs. It says that the
most important moral precept is to do what you think is right, and the
most important epistemological precept is to believe what you think the
evidence supports. The strong version of internalism is not a popular
position. But it has an important role to play in the narrative here.
That's because there are many interesting, and popular, moderate
versions of internalism. Yet once we look at the motivations for those
moderate versions, we'll see that they really are arguments for the
strongest, and least plausible, version.

We can generate those moderate forms of normative internalism by looking
at the four questions from the previous section. Some internalists say
that internalism is true just for actors (or believers), not for actions
(or beliefs). Some say that internalist principles are part of the moral
(or epistemological) truth, not principles to put above all. Some say
that internalism principles apply to just one of ethics or epistemology,
not both. And some say that what matters is not conformity to the
principles one actually holds, but conformity to the principles one has
evidence for. And answers to these questions can be mixed and matched
indefinitely to produce varieties of internalist theses. Here, for
example, are three principles that are both widely believed, and which
you can get by mixing and matching answers to the four questions.

\begin{itemize}
\tightlist
\item
  It is a vice to frequently do things one believes are wrong, even if
  those actions are actually right.
\item
  Wrong actions are blameless, and hence do not reflect badly on the
  actor who performs them, if that actor believes the action is right,
  and has good reason for that belief.
\item
  A belief is irrational if the believer has good evidence that the
  belief is not supported by their evidence, even if that `higher-order'
  evidence is misleading.
\end{itemize}

And I'm going to argue that the best arguments for those positions
overgeneralise; they are equally good as arguments for the implausible
strong version of internalism. So they are no good.

Part of the argument here will be piecemeal: showing for a particular
internalist thesis that there are no good arguments for it but for the
arguments that lead all the way to the strongest form of internalism.
And I can't hope to do that for all the possible theses you could get by
mixing and matching answers to the four questions. But I can hope to
make the strong form of externalism more plausible, both by showing how
it handles some difficult cases, and by showing that the most general
arguments against it do not work.

\section{Guidance}\label{guidance}

To illustrate the kind of storyline I sketched in the previous section,
let's consider one popular argument against externalism. The externalist
says that people should do the right thing, whatever that is, whether or
not they know that the right thing is in fact right. It is often
objected that this is not particularly helpful guidance, and morality
should be more guiding than this. We see versions of this objection made
by Ted Lockhart (\citeproc{ref-Lockhart2000}{2000, 8--9}), Michael M.
Smith (\citeproc{ref-Smith2006}{2006, 143}), Andrew Sepielli
(\citeproc{ref-Sepielli2009}{2009, 8}), William MacAskill
(\citeproc{ref-MacAskillThesis}{2014, 7}) and by Hillary Greaves and
Toby Ord (\citeproc{ref-GreavesOrd2017}{2017}). These authors differ
between themselves about both why norms that are not guiding are bad,
some saying they are unfair, others that they are unhelpful, and about
what conclusion we should draw from this fact. But they agree there is
something bad about \emph{Do the right thing} in virtue of it not being
guiding, and think we need something more internalist.

But if you think \emph{Do the right thing} is not guiding, and we need
norms that are guiding in just that sense, some very strong conclusions
follow. After all, if non-guiding rules are bad, then they shouldn't be
any part of our moral theory. So it isn't just that we should take
hypocrisy to be one vice alongside cowardice, dishonesty, and so on, but
to be the only vice. After all, if there are other vices at all, then
morality as a whole may not be guiding. Now who is \emph{Do the right
thing} not guiding to? Presumably to people who lack full moral
knowledge. But some of these people won't have full epistemological
knowledge either. So by the standard that \emph{Do the right thing} is
not guiding, principles like \emph{Do whatever the evidence best
suggests is right}, or \emph{Do whatever maximises expected rightness}
won't be guiding either. If we can't expect people to know what's right,
we can't really expect them to know what's probably right either.

So taking guidance to be a constraint in this way pushes us to a version
of internalism that relies on actual beliefs about rightness, not
beliefs the evidence supports, and relies on a version that takes
conformity to one's own values to be `above all'. But if we do that, we
can't say either of the plausible things I suggested various moderate
internalists could say about {Robespierre}. The two suggestions were to
say that conformity to one's own value is merely one virtue among many,
and that good people should conform not to their actual principles, but
to the principles their evidence supports. If we take guidance to be a
constraint, then both ways out are blocked. {Robespierre} failed by some
very important standards, but he couldn't be guided (in whatever sense
the internalist means) by those standards.

We'll see this storyline a few times in what follows. The externalist
view seems to have some unattractive features. But when we spell out
just what the features are, we'll see they are shared by all but some
very implausible theories. This won't just hold in ethics. The
epistemological picture I'm going to draw allows for kinds of reasoning
that appear on their face to be unacceptably circular. But when we try
to say just what this kind of circularity comes to, we'll see that
blocking it would provide enough resources to ground an argument for
Pyrrhonian scepticism.

\section{Symmetry}\label{symmetry}

In general, one's evidence is relevant to what one should do. The
normative externalist denies a natural generalisation of this little
platitude. Although evidence about matters of fact is relevant to what
one should do, evidence about the normative, about the nature of
morality and rational, is not. Evidence about whether to turn left or
right is relevant to rational decision making, evidence about what is
wrong or right is irrelevant. Or so says the externalist.

This looks like an argument against externalism: it denies a very
plausible symmetry principle. The principle says that we should treat
all kinds of uncertainty, and all kinds of evidence, the same. I'm going
to spend much of the first half of this book arguing against the
symmetry principle, but for now let's quickly set up why we might think
there is a puzzle here.

We'll start by thinking through an example of where evidence is relevant
to mundane action. A person, we'll call him {Baba}, is looking for his
car keys. He can remember leaving them in the drawer this morning, and
has no reason to think they will have moved. So the natural thing to do
is to look in the drawer. If he does this, however, he will be sadly
disappointed, for his two year old daughter has moved the car keys into
the cookie jar.

Things would go best for {Baba} if he looked in the cookie jar; that way
he would find his car keys. But that would be a very odd thing for him
to do. It would be irrational to look there. It wouldn't make any sense.
If he walked down the steps, walked straight to the cookie jar, and
looked in it for his car keys, it would shock any onlookers because it
would make no sense. It used to be thought that it would not shock his
two year old daughter, since children that young had no sense that
different people have different views on the world. But this isn't true;
well before age two children know that evidence predicts action, and are
surprised by actions that don't make sense given a person's evidence
~(\citeproc{ref-HeBolzBaillargeon2011}{He, Bolz, and Baillargeon 2011}).
This is because from a very young age, humans expect other humans to act
rationally ~(\citeproc{ref-ScottBaillargeon2013}{Scott and Baillargeon
2013}).

In this example, {Baba} has a well-founded but false belief about a
matter of fact: where the car keys are. Let's compare this to a case
where the false beliefs concern normative matters. The example is going
to be more than a little violent, though after this the examples will
usually be more mundane. And the example will, in my opinion, involve
three different normative mistakes.

\begin{quote}
{Gwenneg} is at a conference, and is introduced to a new person. ``Hi,''
he says, ``I'm {Gwenneg},'' and extends his hand to shake the stranger's
hand. The stranger replies, ``Nice to meet you, but you shouldn't shake
my hand. I have disease D, and you can't be too careful about
infections.'' At this point {Gwenneg} pulls out his gun and shoots the
stranger dead.
\end{quote}

Now let's stipulate that {Gwenneg} has the following beliefs, the first
of which is about a matter of fact, and the next three are about
normative matters.

First, {Gwenneg} knows that disease D is so contagious, and so bad for
humans both in terms of what it does to its victims' quality and
quantity of life, that the sudden death of a person with the disease
will, on average, increase the number of quality-adjusted-life-years
(QALYs) of the community.\footnote{QALYs are described in McKie et al.
  (\citeproc{ref-McKie1998}{1998}), who go on to defend some
  philosophical theses concerning them that I'm about to assign to
  {Gwenneg}.} That is, although the sudden death of the person with the
disease obviously decreases their QALYs remaining, to zero in fact, the
death reduces everyone else's risk of catching the disease so much that
it increases the remaining QALYs in the community by a more than
offsetting amount.

Second, {Gwenneg} believes in a strong version of the `straight rule'.
The straight rule says that given the knowledge that \emph{x}\% of the
Fs are Gs, other things equal it is reasonable to have credence that
this particular F is a G. Just about everyone believes in some version
of the straight rule, and just about everyone thinks that it needs to be
qualified in certain circumstances. When I say that {Gwenneg} believes
in a strong version of it, I mean he thinks the circumstances that
trigger qualifications to the rule rarely obtain. So he thinks that it
takes quite a bit of additional information to block the the transition
from believing \emph{x}\% of the Fs are Gs to having credence that this
particular F is a G.\footnote{Nick Bostrom
  (\citeproc{ref-Bostrom2003}{2003}) endorses, and uses to interesting
  effect, what I'm calling a strong version of the straight rule. In my
  reply to his paper I argue that only a weak version is plausible,
  since other things are rarely equal
  ~(\citeproc{ref-Weatherson2003-sim}{Weatherson 2003}). {Gwenneg}
  thinks that Bostrom has the better of that debate.}

Third, {Gwenneg} thinks that QALYs are a good measure of welfare. So the
most beneficent action, the one that is best for well-being, is the one
that maximises QALYs. This is hardly an uncontroversial view, but it
does have some prominent defenders ~(\citeproc{ref-McKie1998}{McKie et
al. 1998}).

And fourth, {Gwenneg} endorses a welfarist version of {Frank} Jackson's
decision-theoretic consequentialism ~(\citeproc{ref-Jackson1991}{Jackson
1991}). That is, {Gwenneg} thinks the right thing to do is the thing
that maximises expected welfare.

Putting these four beliefs together, we can see why {Gwenneg} shot the
stranger. He believed that, on average, the sudden death of someone
suffering from disease D increases the QALYs remaining in the community.
By the straight rule, he inferred that each particular death of someone
suffering from disease D increases the expected QALYs remaining in the
community. By the equation of QALYs with welfare he inferred that each
particular death of someone suffering from disease D increases the
expected welfare of the community. And by his welfarist
consequentialism, he inferred that bringing about such a death is a good
thing to do. So not only do these beliefs make his action make sense,
they make it the case that doing anything else would be a moral failing.

Now I think the second, third and fourth beliefs I've attributed to
{Gwenneg} are false. The first is a stipulated fact about the world of
{Gwenneg}'s story. It is a fairly extreme claim, but far from fantastic.
There are probably diseases in reality that are like disease D in this
respect\footnote{At least, there probably were such diseases at some
  time. I suspect cholera had this feature during some epidemics.}. So
we'll assume he hasn't made a mistake there, but from then on every
single step is wrong. But none of these steps are utterly absurd. It is
not too hard to find both ordinary reasonable folk who endorse each
individual step, and careful argumentation in professional journals in
support of those steps. Indeed, I have cited just such argumentation.
Let's assume that {Gwenneg} is familiar with those arguments, so he has
reason to hold each of his beliefs. In fact, and here you might worry
that the story I'm telling loses some coherence, let's assume that
{Gwenneg}'s exposure to philosophical evidence has been so tilted that
he has only seen the arguments for the views he holds, and not any good
arguments against them. So not only does he have these views, but in
each case he is holding the view that is best supported by the
(philosophical) evidence available.

Now I don't mean to use {Gwenneg}'s case to argue against internalism.
It wouldn't be much use in such an argument for two reasons. First,
there are plenty of ways for internalists to push back against my
description of the case. For example, perhaps it is plausible for
{Gwenneg} to have any one of the the normative beliefs I've attributed
to him, but not to have all of them at once. Second, not all of the
internalist views I described so far would even endorse his actions
given that my description of the case is right.

But the case does illustrate three points that will be important going
forward. One is that it isn't obvious that the symmetry claim above,
that all uncertainty should be treated alike, is true. Maybe that claim
is true, but it needs to be argued for. Second, the symmetry claim has
very sweeping implications, once we realise that people can be uncertain
about so many philosophical matters. Third, externalist views look more
plausible the more vivid the case becomes. It is one thing to say
abstractly that {Gwenneg} should treat his uncertainty about morality
and epistemology the same way he treats his uncertainty about how many
people the stranger will infect. At that level of abstraction, that
sounds plausible. It is another to say that the killing was a good
thing. And we'll see this pattern a lot as we go forward; the more vivid
cases are, the more plausible the externalist position looks. But from
now on I'll keep the cases vivid enough without being this
violent.\footnote{One exception: {Robespierre} will return from time to
  time, along with other Terrorists.}

\section{Regress}\label{regress}

In this book I'm going to focus largely on ethics and epistemology.
{Gwenneg}'s case illustrates a third possible front in the battle
between normative internalists and externalists: welfare theory. There
is a fourth front that also won't get much discussion, but is I think
fairly interesting: decision theory. I'm going to spend a bit of time on
it right now, as a way of introducing regress arguments for externalism.
And regress arguments are going to be very important indeed in the rest
of the book.

Imagine that {Llinos} is making trying to decide how much to value a bet
with the following payoffs: it returns £10 with probability 0.6, £13
with probability 0.3, and £15 with probability 0.1. Assume that for the
sums involved, each pound is worth as much to {Llinos} as the
next.\footnote{Technically, what I'm saying here is that the marginal
  utility of money to {Llinos} is constant. There is a usual way of
  cashing out what it is for the marginal utility of money to be
  constant in terms of betting behaviour. It is that the marginal
  utility of money is constant iff the agent is indifferent between a
  bet that returns 2\emph{x} with probability 0.5, and getting \emph{x}
  for sure. But we can't adopt that definition here, because it takes
  for granted a particular method of valuing bets. And whether that
  method is correct is about to come into question.} Now the normal way
to think about how much this bet is worth to {Llinos} is to multiply
each of the possible outcomes by the probability of that outcome, and
sum the results. So this bet is worth 10 × 0.6 + 13 × 0.3 + 15 × 0.1 = 6
+ 3.9 + 1.5 = 11.4. This is what is called the \emph{expected} return of
the bet, and the usual theory is that the expected return of the bet is
its value. (It's not the most helpful name, since the expected return is
not in any usual sense the return we expect to get. But it is the common
name throughout philosophy, economics and statistics, and it is the name
I'll use here.)

There's another way to get to calculate expected values. Order each of
the possible outcomes from worst to best, and at each step, multiply the
probability of getting at least that much by the difference between that
amount and the previous step. (At the first step, the `previous' value
is 0.) So {Llinos} gets £10 with probability 1, has an 0.4 chance of
getting another £3, and has an 0.1 chance of getting another £2.
Applying the above rule, we work out her expected return is 10 + 0.4 × 3
+ 0.1 × 2 = 10 + 1.2 + 0.2 = 11.4. It isn't coincidence that we got the
same result each way; these are just two ways of working out the same
sum. But the latter approach makes it easier to understand an
alternative approach to decision theory, one recently defended by Lara
Buchak (\citeproc{ref-BuchakRisk}{2013}).

She thinks that the standard approach, the one I've based around
expected values, is appropriate only for agents who are neutral with
respect to risk. Agents who are risk seeking, or risk averse, should use
slightly different methods.\footnote{The orthodox view is that the
  agent's attitude to risk should be incorporated into their utility
  function. That's what I think is correct, but Buchak does an excellent
  job of showing why there are serious reasons to question the
  orthodoxy.} In particular, when we multiplied each possible gain by
the probability of getting that gain, Buchak thinks we should instead
multiply by some function \emph{f} of the probability. If the agent is
risk averse, then \emph{f}(\emph{x})~\textless~\emph{x}. To use one of
Buchak's standard examples, a seriously risk averse agent might set
\emph{f}(\emph{x}) = \emph{x}\textsuperscript{2}. (Remember that
\emph{x}~∈~{[}0, 1{]}, so \emph{x}\textsuperscript{2}~\textless~\emph{x}
everywhere except the extremes.) If we assume that this is {Llinos}'s
risk function, the bet I described above will have value 10 +
0.4\textsuperscript{2} × 3 + 0.1\textsuperscript{2} × 2 = 10 + 0.48 +
0.02 = 10.5.

Now imagine a case that is simpler in one respect, and more complicated
in another. {Iolana} has to choose between getting £1 for sure, and
getting £3 iff a known to be fair coin lands heads. (The marginal
utility of money to Iolana is also constant over the range in question.)
And she doesn't know whether she should use standard decision theory, or
a version of Buchak's decision theory, with the risk function set at
\emph{f}(\emph{x}) = \emph{x}\textsuperscript{2}. Either way, the £1 is
worth 1. (I'm assuming that £1 is worth 1 util, expressing values of
choices in utils, and not using any abbreviation for these utils.) On
standard theory, the bet is worth 0.5~×~3~=~1.5. On Buchak's theory, it
is worth 0.5\textsuperscript{2}~×~3~= 0.75. So until she knows which
decision theory to use, she won't know which option is best to take.
That's not merely to say that she won't know which option will return
the most. She can't know which option has the best returns until the
coin is flipped. It's to say also that she won't know which bet is
rational to take, given her knowledge about the setup, until knows which
is the right theory of rational decision making.

In the spirit of normative internalism, we might imagine we could solve
this problem for {Iolana} without resolving the dispute between Buchak
and her orthodox rivals. Assume that {Iolana} has, quite rationally,
credence 0.5 that Buchak's theory is correct, and credence 0.5 that
orthodox theory is correct. (I'm assuming here that a rational agent
could have positive credence in Buchak's views. But that's clearly true,
since Buchak herself is rational.) Then the bet on the coin has, in some
sense, 0.5 chance of being worth 1.5, and 0.5 chance of being worth
0.75. Now we could ask ourselves, is it better to take the £1 for sure,
or to take the bet that has, in some sense, 0.5 chance of being worth
1.5, and 0.5 chance of being worth 0.75?

The problem is that we need a theory of decision to answer that very
question. If {Iolana} takes the bet, she is guaranteed to get a bet
worth at least 0.75, and she has, by her lights, an 0.5 chance of
getting a bet worth another 0.75. (That 0.75 is the difference between
the 1.5 the bet is worth if orthodox theory is true, and the 0.75 it is
worth if Buchak's theory is true.) And, by orthodox lights, that is
worth 0.75~+~0.5~×~0.75~=~1.125. But by Buchak's lights, that is worth
0.75~+~0.5\textsuperscript{2}~×~0.75~=~0.9375. We still don't know
whether the bet is worth more or less than the sure £1.

Over the course of this book, we'll see a lot of theorists who argue
that in one way or other, we can resolve practical normative questions
like the one {Iolana} faces without actually resolving the hard
theoretical issues that make the practical questions difficult. And one
common way to think this can be done traces back to an intriguing
suggestion by Robert Nozick (\citeproc{ref-Nozick1994}{1994}). Nozick
suggested we could use something like the procedure I described in the
previous paragraph. Treat making a choice under normative uncertainty as
taking a kind of bet, where the odds are the probabilities of each of
the relevant normative theories, and the payoffs are the values of the
choice given the normative theory.\footnote{Nozick's own application of
  this was to the Newcomb problem ~(\citeproc{ref-Nozick1969}{Nozick
  1969}). (Going into the details of what the Newcomb problem is would
  take us too far afield; Paul Weirich
  (\citeproc{ref-Weyrich2012}{2016}) has a nice survey of it if you want
  more details.) He noted that if causal decision theory is correct,
  then two--boxing is fractionally better than one--boxing, while if
  evidential decision theory is correct, then one--boxing is
  considerably better than two--boxing. If we think the probability that
  evidential decision theory is correct is positive, and we use this
  approach, we will end up choosing one box. And that will be true even
  if the probability we assign to evidential decision theory is very
  very small.} And the point to note so far is that this won't actually
be a technique for resolving practical problems without a theory of
decision making. At some level, we simply need a theory of decision.

The fully internalist `theory' turns out to not have anything to say
about cases like {Iolana}'s. If it had a theory of second order
decision, of how to make decisions when you don't know how to make
decisions, it could adjudicate between the cases. But there can't be a
theory of how to make decisions when you don't know how to make
decisions. Or, more precisely, any such theory will be externalist.

Let's note one striking variant on the case. {Wikolia} is like {Iolana}
is almost every respect. She gives equal credence to orthodox decision
theory and Buchak's alternative, and no credence to any other
alternative, and she is facing a choice between £1 for sure, and £3 iff
a fair coin lands heads. But she has a third choice: 55 pence for sure,
plus another £1.60 iff the coin lands heads. It might be easiest to
label her options A, B and C, with A being the sure pound, B being the
bet {Iolana} is considering, and C the new choice. Then her payoffs,
given each choice and the outcome of the coin toss, are as follows.

\begin{longtable}[]{@{}lcc@{}}
\toprule\noalign{}
& Heads & Tails \\
\midrule\noalign{}
\endhead
\bottomrule\noalign{}
\endlastfoot
Option A & £1 & £1 \\
Option B & £3 & £0 \\
Option C & £2.15 & £0.55 \\
\end{longtable}

The expected value of Option C is 0.55~+~0.5~×~1.6~=~1.35. (I'm still
assuming that £1 is worth 1 util, and expressing values of choices in
utils.) It's value on Buchak's theory is
0.55~+~0.5\textsuperscript{2}~×~1.6~=~0.95. Let's add those facts to the
table, using \textbf{EV} for expected value, and \textbf{BV} for value
according to Buchak's theory.

\begin{longtable}[]{@{}lcccc@{}}
\toprule\noalign{}
& Heads & Tails & EV & BV \\
\midrule\noalign{}
\endhead
\bottomrule\noalign{}
\endlastfoot
Option A & £1 & £1 & 1 & 1 \\
Option B & £3 & £0 & 1.50 & 0.75 \\
Option C & £2.15 & £0.55 & 1.35 & 0.95 \\
\end{longtable}

Now rememeber that {Wikolia} is unsure which of these decision theories
to use, and gives each of them equal credence. And, as above, whether we
use orthodox theory or Buchak's alternative at this second level affects
how we might incorporate this fact into an evaluation of the options. So
let \textbf{EV2} be the expected value of each option if it is construed
as a bet with an 0.5 chance of returning its expected value, and an 0.5
chance of returning its value on Buchak's theory, and \textbf{BV2} the
value of that same bet on Buchak's theory.

\begin{longtable}[]{@{}lcccccc@{}}
\toprule\noalign{}
& Heads & Tails & EV & BV & EV2 & BV2 \\
\midrule\noalign{}
\endhead
\bottomrule\noalign{}
\endlastfoot
Option A & £1 & £1 & 1 & 1 & 1 & 1 \\
Option B & £3 & £0 & 1.50 & 0.75 & 1.125 & 0.9375 \\
Option C & £2.15 & £0.55 & 1.35 & 0.95 & 1.15 & 1.05 \\
\end{longtable}

And now something interesting happens. In each of the last two columns,
Option C ranks highest. So arguably\footnote{Ironically, it isn't at all
  obvious in this context that this is acceptable reasoning on Wikolia's
  part. The argument by cases she goes on to give is not strictly
  speaking valid on Buchak's theory, so it isn't clear that Wikolia can
  treat it as valid here, given that she isn't sure which decision
  theory to use. This goes to the difficulty of saying anything about
  what should be done without making substantive normative assumptions,
  a difficulty that will recur frequently in this book.}, {Wikolia} can
reason as follows: \emph{Whichever theory I use at the second order,
option C is best. So I should take option C}. On the other hand,
{Wikolia} can also reason as follows. If expected value theory is
correct, then I should take option B, and not take option C. And if
Buchak's theory is correct, then I should take option A, and not take
option C. So either way, I should not take option C. {Wikolia} both
should and should not take option C.

That doesn't look good, but again I don't want to overstate the
difficulty for the internalist. The puzzle isn't that internalism leads
to a contradiction, as it might seem here. After all, the term `should'
is so slippery that we might suspect there is some kind of fallacy of
equivocation going on. And so our conclusion is not really a
contradiction. It really means that {Wikolia} should-in-some-sense take
option C, and should-not-in-some-other-sense take option C. And that's
not a contradiction. But it does require some finesse for the
internalist to say just what these senses are. This kind of challenge
for the internalist, the puzzle of ending up with more senses of should
than one would have hoped for, and needing to explain each of them, will
recur a few times in the book.

\section{Two Recent Debates}\label{tworecentdebates}

I think the question of whether \emph{Do the right thing} or
\emph{Follow your principles} is more fundamental is itself an
interesting question. But it has become relevant to two other debates
that have become prominent in recent philosophy as well. These are
debates about moral uncertainty, and about higher-order evidence.

Many of the philosophers who have worried that \emph{Do the right thing}
is insufficiently guiding have looked to have a theory that makes moral
uncertainty more like factual uncertainty. And since it is commonly
agreed that an agent facing factual uncertainty, and only concerned with
outcomes, should maximise factual uncertainty, a common conclusion has
been that a morally uncertain agent should also maximise some kind of
expected value. In particular, they should aim to maximise the expected
moral value of their action, where probabilities about moral theories
can affect the expected moral value.

In the recent literature, we see the view that people should be
sensitive to the probabilities of moral theories sometimes described as
`moral hedging'. This terminology is used by Christian Tarsney
(\citeproc{ref-Tarsney2017}{2017}), who is fairly supportive of the
idea, and Ittay Nissan-Rozen (\citeproc{ref-NissanRozen2015}{2015}), who
is not. It's not, I think, the happiest term. After all, {Robespierre}
maximised expected moral value, at least relative to the credences that
he had. And it would be very odd to describe the Reign of Terror as a
kind of moral hedging.

The disputes about moral uncertainty have largely focussed on cases
where a person is torn between two (plausible) moral theories, and has
to choose between a pair of actions. In one important kind of case, the
first is probably marginally better, but it might be much much worse. In
that case, maximising moral value may well involve taking the second
option. And that's the kind of case where it seems right to describe the
view as a kind of moral hedging.

But the general principle that one should maximise expected moral value
applies in many more cases than that. It applies, for example, to people
who are completely convinced that some fairly extreme moral theory is
correct. And in those cases, maximising expected moral value, rather
than actual moral value, could well be disastrous.

When it is proposed that probabilities matter to a certain kind of
decision, it is a useful methodology to ask what the proposal says in
the cases where the probabilities are all 1 or 0. That's what I'm doing
here. If probabilities of moral theories matter, they should still
matter when the probability (in the relevant sense) of some horrid
theory is 1. So my investigation of Polonius's principle will have
relevance for the debate over moral uncertainty, since it will have
consequences for what theories of moral uncertainty can plausibly say in
extreme cases.

There is one dispute about moral uncertainty that crucially involves
intermediate probabilities. Maximising expected moral value requires
putting different theories' moral evaluations of actions on a common
scale. There is no particularly good way to do this, and it has been
argued that there is no possible way to do this. This is sometimes held
to be a reason to reject `moral hedging'
~(\citeproc{ref-Hedden2015}{Hedden 2016a}). I'll return to this question
in chapter 6, offering a tentative defence of the `hedger'. The question
of how to find this common scale is hard, but there are reasons to think
it is not impossible. And what matters for the current debate is whether
it is in fact impossible.

The other recent dispute that normative externalism bears on concerns
peer disagreement. Imagine that two friends, {Ankita} and {Bojan}, both
regard themselves and each other as fairly knowledgable about a certain
subject matter. And let \emph{p} be a proposition in that subject, that
they know they have equally good evidence about, and that they are
antecedentl equally likely to form true beliefs about. Then it turns out
that {Ankita} believes \emph{p}, while {Bojan} believes ¬\emph{p}. What
should they do in response to this news?

One response goes via beliefs about their own rationality. Each of them
should think it is equally likely that believing \emph{p} and believing
¬\emph{p} is rational given their common evidence. They should think
this because they have two examples of rational people, and they ended
up with these two conclusions. So they should think that holding on to
their current belief is at most half-likely to be rational. And it is
irrational, say some theorists, to believe things that you only think
are half-likely to be rational. So both of them should become completely
uncertain about whether \emph{p} is true.

I'm going to argue that there are several mistakes in this reasoning.
They shouldn't always think that holding on to their current belief is
half-likely to be rational. Whether they should or not depends, among
other things, on why they have their current belief. But even if they
should change their belief about how likely it is that their belief is
rational, nothing follows about what they should do to their first-order
beliefs. In some strange situations, the thing to do is to hold on to a
belief, while being sceptical that it is the right belief to have. This
is the key externalist insight, and it helps us resolve several puzzles
about disagreement.

\section{Elizabeth and Descartes}\label{elizabethanddescartes}

Although the name `normative externalism' is new, the view is not. It
will be obvious in what follows how much the arguments I have to offer
are indebted to earlier work by, among others, Nomy Arpaly
(\citeproc{ref-Arpaly2003}{2003}), Timothy Schroeder
~(\citeproc{ref-ArpalySchroeder2014}{Arpaly and Schroeder 2014}), Maria
Lasonen-Aarnio (\citeproc{ref-Lasonen-Aarnio2010}{2010a},
\citeproc{ref-Lasonen-Aarnio2014}{2014a}), Miriam Schoenfield
(\citeproc{ref-Schoenfield2014}{2015}) and Elizabeth Harman
(\citeproc{ref-Harman2011a}{2011}, \citeproc{ref-Harman2014}{2015}). It
might not be as obvious, because they aren't directly cited as much, but
much of the book is influenced by the pictures of normativity developed
by Thomas Kelly (\citeproc{ref-Kelly2005}{2005}) and by Amia Srinivasan
(\citeproc{ref-Srinivasan2015b}{2015b}).

Many of the works just cited address just one of the two families of
debates this book joins: i.e., debates about ethics and debates about
epistemology. One of the nice features about taking on both of these
debates at once is that it is possible to blend insights from the
externalist side of each of those debates. So chapter 4, which is the
main argument against normative internalism in ethics, is modelled on an
argument Miriam Schoenfield (\citeproc{ref-Schoenfield2014}{2015})
develops to make a point in epistemology. And much of what I say
epistemic akrasia in chapter 10 is modelled on what Nomy Arpaly
(\citeproc{ref-Arpaly2003}{2003}) says about practical akrasia.

There are also some interesting historical references to normative
externalism. I'm just going to talk about the one that is most
interesting to me. In the correspondence between Descartes and
Elizabeth, we see Descartes taking a surprisingly internalist view in
ethics, and Elizabeth the correct externalist view.\footnote{All
  translations are from the recent edition of the correspondence by Lisa
  Shapiro ~(\citeproc{ref-Shapiro2007}{Elizabeth and Descartes 2007}).}

On 15 September, 1645, Descartes wrote:

\begin{quote}
For it is irresolution alone that causes regret and repentance.
\end{quote}

This had been a theme of the view he had been putting forward. The good
person, according to the view Descartes put forward in the
correspondence, is one who makes a good faith effort to do the best they
can. Someone who does this, and who is not irresolute, has no cause to
regret their actions. He makes this clear in an earlier letter, on 4
August 1645, where he is also more explicit that it is only careful and
resolute actors who are immune to regret.

\begin{quote}
But if we always do all that our reason tells us, we will never have any
grounds to repent, even though events afterward make us see that we were
mistaken. For our being mistaken is not our fault at all.
\end{quote}

Elizabeth disagrees with Descartes both about regret, and with what it
shows us about the nature of virtue. She writes, on 16 August 1645,

\begin{quote}
On these occasions regret seems to me inevitable, and the knowledge that
to err is as natural to man as it is to be sick cannot protect us. For
we also are not unaware that we were able to exempt ourselves of each
particular fault.
\end{quote}

Over the course of the correspondence, Elizabeth seems to be promoting a
view of virtue on which being sensible in forming intentions, and
resolute in carrying them out, does not suffice for being good. One must
also form the right intentions. If that is really her view, then she is
a very important figure in the history of normative externalism. Indeed,
if that is her view, perhaps I should be calling this book a defence of
Elizabethan philosophy.

But it would be a major diversion from the themes of this book to
investigate exactly how much credit Elizabeth is due. And in any case, I
don't want to suggest that I'm defending exactly the view Elizabeth is.
The point about possibility she makes in the above quote is very
important. It's possible that we ought to be good, and we can't know
just what is good, but this isn't a violation of \emph{Ought implies
can}, because for any particular good thing we ought do, we can with
effort come to know that that thing is good. That's a nice problem to
raise for particular internalists, but it's not my motivation for being
externalist. I don't think it matters at all whether we know what is
good, so the picture of virtue I'm working with is very different to the
Stoic picture that Elizabeth has. (It's much more like the picture that
Nomy Arpaly (\citeproc{ref-Arpaly2003}{2003}) has developed.)

So it would be misleading to simply say this book is a work in
Elizabethan philosophy. But Elizabeth is at the very least an important
figure in the history of the views I'm defending, and she is to me the
most fascinating of my historical predecessors.

\section{Why Call This Externalism?}\label{whycallthisexternalism}

There are so many views already called externalist in the literature
that I feel I should offer a few words of defence of my labelling my
view externalist. In the existing literature I'm not sure there is any
term, let alone an agreed upon term, for the view that higher-order
considerations are irrelevant to both ethical and epistemological
evaluation. So we needed some nice term for my view. And using
`externalist' suggested a useful term for the opposing view. And there
is something evocative about the idea that what's distinctive of my view
is that it says that agents are answerable to standards that are
genuinely \emph{external} to them. More than that, it will turn out that
there are similarities between the debates we'll see here and familiar
debates between internalists and externalists about both content and
about the nature of epistemic norms.

In debates about content, we should not construe the
internalism/externalism debate as a debate about which of two kinds of
content are, as a matter of fact, associated with our thought and talk.
To set up the debate that way is to concede something that is at issue
in the debate. That is, it assumes from the start that there is an
internalist friendly notion of content, and it really is a kind of
content. But this is part of what's at issue. The same point is true
here. I very much do not think the debate looks like this: The
externalist identifies some norms, and the internalist identifies some
others, and then we debate which of those norms are really our norms. At
least against some internalist opponents, I deny that they have so much
as identified a kind of norm that we can debate whether it is our norm.

In debates in epistemology, there is a running concern that internalist
norms are really not normative. If we identify justified belief in a way
that makes it as independent of truth as the internalist wants
justification to be, there is a danger that that we should not care
about justification. Internalists have had interesting things to say
about this danger ~(\citeproc{ref-Conee1992}{Conee 1992}), and I don't
want to say that that it is a compelling objection to (first-order
epistemological) internalism. But it is a danger. And I will argue that
it's a danger that the normative internalist can't avoid.

Let's say we can make sense of a notion that tracks what the internalist
thinks is important. In section 6.1 I'll argue that \emph{not being a
hypocrite} is such a notion; the internalist cares a lot about it, and
it is a coherent notion. There is a further question of whether this
should be relevant to our belief, our action or our evaluation of
others. If someone is a knave, need we care further about whether they
are a sincere or hypocritical knave? I'll argue that at the end of the
day we should not care; it isn't worse to be hypocritical.\footnote{My
  instinct is that there is something preferable about the hypocrite
  compared to the person who does wrong while thinking they are doing
  the right thing. After all, the hypocrite has figured out a moral
  truth, and figuring out moral truths typically reflects well on a
  person. But I'm not going to try to turn this instinct into an
  argument in this book.}

The debates I'm joining here have something else in common with familiar
internalist/externalist debates. Many philosophers will be tempted to
react to them by saying the parties are talking at cross-purposes. In
fact, there are two ways that it might be thought the parties are at
cross-purposes.

First, it might be thought the parties are both right, but they are
talking about different things. The normative internalist is talking
about subjective normativity, and saying true things about it, while the
normative externalist is talking about objective normativity, and saying
true things about it. One of the running themes of this book will be
that this isn't a way of dissolving the debate, it is a way of taking
the internalist's side. Just like in debates about content, and in
debates about epistemic justification, the externalist denies that there
is any notion that plays the role the internalist wants their notion to
play. To say the notion exists, but isn't quite as important as the
internalist says it is, is to concede the vast majority of what the
externalist wants to contest.

The second way to say that the theorists are talking at cross-purposes
is to say that their differences merely turn on first-order questions
about ethics and epistemology. What the internalist calls misleading
evidence about morality, the externalist calls first-order reasons to
act a different way. And what the internalist calls higher-order
evidence, the externalist calls just more first-order evidence. This is,
I'm going to argue, an externalist position, and not one that the
internalist should happily sign on to. It is, very roughly, the view I
want to defend in epistemology. What has been called higher-order
evidence in epistemology is, when it is anything significant at all,
just more first-order evidence. It is also a possible externalist view
in ethics, though not one I want to defend. In particular, it is the
view that misleading evidence about morality changes the objective
circumstances in a way that changes what is good to do. I don't think
that's typically true, but it is a possible externalist view.

All that said, there are two ways in which what I'm saying differs from
familiar internalist/externalist debates. One is that what I'm saying
cross-cuts the existing debates within ethics and epistemology that
often employ those terms. Normative externalism is compatible with an
internalist theory of epistemic justification. It is consistent to hold
the following two views:

\begin{itemize}
\tightlist
\item
  Whether S is justified in believing \emph{p} depends solely on S's
  internal states.
\item
  There is a function from states of an agent to permissible beliefs,
  and whether an agent's beliefs are justified depends solely on the
  nature of that function, and the agent could in principle be mistaken,
  and even rationally mistaken, about the nature of the function.
\end{itemize}

The first bullet point defines a kind of internalism in epistemology.
The second bullet point defines a kind of externalism about epistemic
norms. But the two bullet points are compatible, as long as the function
in question does not vary between agents with the same internal states.
The two bullet points may appear to be in some tension, but their
conjunction is more plausible than many theses that have wide
philosophical acceptance. Ralph Wedgwood
(\citeproc{ref-Wedgwood2012}{2012}), for example, defends the
conjunction, and spends some time arguing against the idea that the
conjuncts are in tension.

And normative externalism is compatible in principle with the view in
ethics that there is an internal connection between judging that
something is right, and being motivated to do it. This view is sometimes
called motivational internalism ~(\citeproc{ref-Rosati2014}{Rosati
2016}). But again, there is a tension, in this case so great that it is
hard to see why one would be a normative externalist and a motivational
internalist. The tension is that to hold on to both normative
externalism and motivational internalism simultaneously, one has to
think that `rational' is not an evaluative term, in the sense relevant
for the definition of normative externalism. That is, one has to hold on
to the following views.

\begin{itemize}
\tightlist
\item
  It is irrational to believe that one is required to φ, and not be
  motivated to φ; that's what motivational internalism says.
\item
  An epistemically good agent will follow their evidence, so if they
  have misleading moral evidence, they will believe that φ is required,
  even when it is not. The possibility of misleading moral evidence is a
  background assumption of the debate between normative internalists and
  normative externalists. And the normative externalist says that the
  right response to misleading evidence is to be misled.
\item
  An agent should be evaluated by whether they do, and are motivated to
  do, what is required of them, not whether they do, or are motivated to
  do, what they believe is required of them. Again, this is just what
  normative externalism says.
\end{itemize}

Those three points are consistent, but they entail that judging someone
to be irrational is not, in the relevant sense, to evaluate them. Now
that's not a literally incoherent view. It is a souped-up version of
what Niko Kolodny (\citeproc{ref-Kolodny2005}{2005}) argues for. (It
isn't Kolodny's own view; he thinks standards of rationality are
evaluative but not normative. I'm discussing the view that they are
neither evaluative nor normative.) But it is a little hard to see the
attraction of the view. So normative externalism goes more happily with
motivational externalism.

And that's the common pattern. Normative externalism is a genuinely
novel kind of externalism, in that it is neither entailed by, nor
entails, other forms of externalism. But some of the considerations for
and against it parallel considerations for and against other forms of
externalism. And it sits most comfortably with other forms of
externalism. So the name is a happy one.

\section{Plan of Book}\label{planofbook}

This book is in two parts: one about ethics, the other about
epistemology.

The ethics part starts with a discussion of the motivations of
internalism in ethics. It then spends two chapters arguing against
strong forms of internalism. By strong forms, I mean views where some
key moral concept is identified with acting in accord with one's own
moral beliefs. So this internalist-friendly condition (I'm doing what I
think I should do) is both necessary and sufficient for some moral
concept to apply. After this, I spend two chapters on weak forms. In
chapter 5, I discuss a view where blameworthiness requires that one not
believe one was doing the wrong thing. In chapter 6, I discuss a view
where doing what one thinks is wrong manifests a vice, even if the
action is right. These don't cover the field of possible views, but they
are important versions of views that hold that internalist-friendly
conditions have a one-way connection to key moral concepts. The
internalist-friendly conditions in these cases provide either a
necessary or a sufficient condition for the application of a key moral
concept, but not both.

I then turn to epistemology. The organising principle that I'll be
defending is something I'll call Change Evidentialism: only new evidence
that bears on \emph{p} can compel a rational agent to change their
credences in \emph{p}. The forms of internalism that I'll be opposing
all turn out to reject that. And the reason they reject it is that they
think a rational person can be compelled to change their credences for
much more indirect reasons. In particular, the rational person could get
misleading evidence that the rational attitude to take towards \emph{p}
is different to the attitude they currently take, and that could compel
them to change their attitude towards \emph{p}. I'm going to argue that
this is systematically mistaken. And this has consequences for how to
think about circular reasoning (it's not as bad as you think!),
epistemic akrasia (it's not as bad as you think!), and standing one's
ground in the face of peer disagreement (it's really not as bad as you
think!).

\part{Ethics}

\chapter{All About Internalism}\label{allaboutinternalism}

This chapter has two related aims. The first is to clarify, and
classify, the range of internalist positions that are available. The
second is to set out more carefully the reasons for adopting one or
other of these positions. We'll end by putting the two parts together,
seeing which motivations push towards what kinds of internalism. These
themes were all introduced briefly in the introduction, but they need
fuller treatment before we proceed.

It is always good practice to state as carefully and as persuasively as
possible the view one means to oppose. But there is a particular reason
to adopt that general practice here. Some of the appeal of internalism
comes from sliding between different versions of the view. Once we get
some key distinctions on the table, we get a better look at which
versions are defensible.

The conclusion of the chapter will be that the best arguments for
normative internalism in ethics make heavy use of the idea that moral
uncertainty and factual uncertainty should be treated symmetrically. So
to get started, we'll look at how factual uncertainty matters morally.

\section{Some Distinctions}\label{somedistinctions}

It helps to have some mildly technical language on the table to begin
with. The terminology I'll use here is standard enough. But the terms
are somewhat ambiguous, and theoretically loaded. I want to stipulate
away some possible ambiguities, and simultaneously avoid at least some
theoretical disputes. So take the following elucidations of the
distinctions to be definitional of the bolded terms as they'll be used
here.

\begin{itemize}
\tightlist
\item
  \textbf{Useful vs Harmful} \emph{Outcomes}. Some outcomes involve more
  welfare, others involve less. I'll say an action is more useful to the
  extent that it involves more welfare, and harmful to the extent it
  involves less.\footnote{I'm going to stay neutral about just what
    outcomes are. I prefer to think of them as possible worlds, but
    there are many other choices that would do just as well for current
    purposes.}
\item
  \textbf{Good vs Bad} \emph{Outcomes}. Some outcomes are better, all
  things considered, than others. I'll use good and bad as predicates of
  outcomes, ones that track whether the outcome is better or worse. It
  is common enough to talk about good and bad actions, and good and bad
  agents, but I'll treat those usages as derivative. What's primary is
  whether outcomes are good or bad. I will \emph{not} assume that the
  goodness of an outcome is agent-independent. Perhaps an outcome where
  a person lies to prevent a great harm is bad relative to that person,
  since they have violated a categorical moral imperative. That is
  consistent with saying the lie was very useful, and even that it was
  good relative to other people.
\item
  \textbf{Right vs Wrong} \emph{Actions}. Unlike \emph{good} and
  \emph{bad}, I'll use \emph{right} and \emph{wrong} exclusively as
  predicates of actions.
\item
  \textbf{Rational vs Irrational} \emph{Actions} and \emph{States}. This
  is a bit of a stretch of ordinary usage, but I'll talk both about
  mental states (beliefs, intentions, etc.) being rational or
  irrational, and the actions that issue from these states being
  rational or irrational. So it is both irrational to believe that the
  moon is made of green cheese, and to bet that it is.
\item
  \textbf{Praiseworthy vs Blameworthy} \emph{Agents}. Again, there is an
  ordinary usage where actions are praiseworthy or blameworthy. But I'll
  treat that as derivative. What's primary is that an agent is
  praiseworthy or blameworthy, perhaps in virtue of having performed a
  particular action.
\end{itemize}

In conditions of full knowledge, it is very plausible that there are
close connections between these five distinctions. There is a natural
form of consequentialism that says the five are co-extensive under
conditions of full knowledge. A good outcome just is a useful one; a
right action is one that promotes the good; it is rational to promote
the good, and blameworthy to do not so. Those who are not sympathetic to
classical consequentialism will not be happy with this equation between
the good and the useful, but they might support many of the other
equations. Michael Smith (\citeproc{ref-Smith2006}{2006},
\citeproc{ref-Smith2009}{2009}) for example, has argued that if we allow
goodness to be agent-relative, then even non-consequentialists can allow
that, under conditions of full knowledge, right actions are those that
maximise the good. Smith's argument is not uncontroversial. Campbell
Brown (\citeproc{ref-Brown2011-BROCT}{2011}) notes there will be
problems with this attempt to `consequentialize' a theory that allows
for moral dilemmas. But I'm going to set that issue aside.

Under conditions of uncertainty, the connections between the
distinctions becomes much more murky, even for a consequentialist. There
are cases where the useful comes apart from the right, the rational, and
the praiseworthy. Here are two such cases.

\begin{quote}
{Cressida} is going to visit her grandmother, who is unwell, and who
would like a visit from her granddaughter. She knows the more time she
spends with her grandmother, the better things will be. So she drives as
fast as she can to get there, not worrying about traffic lights or any
other kind of traffic regulation. Normally this kind of driving would
lead to several serious injuries, and possibly to fatalities, but by
sheer good fortune, no one is harmed by {Cressida}'s driving. And her
grandmother does get some enjoyment from spending a few more minutes
with her granddaughter.
\end{quote}

\begin{quote}
{Botum} is the chief executive of a good, well-run, charity. She has
just been given a £10,000 donation, in cash. She is walking home her
normal way, through the casino. As she is walking past the roulette
table, it occurs to her that if she put the £10,000 on the right number,
she could turn it into £360,000, which would do much more good for the
charity. She has 38 choices: Do nothing, bet on 0, bet on 1, \ldots, bet
on 36. Of these, she knows the one with the most useful outcome will be
one of the last 37. But she keeps the money in her pocket, and deposits
it with the charity's bank account the next morning.
\end{quote}

{Cressida} acts wrongly, and is seriously blameworthy for her driving.
That's even though the outcome is the best possible outcome. So there's
no simple connection, given uncertainty, between usefulness and
rightness.

But in some ways the case of {Cressida} is simple. After all, it is very
improbable that driving this way will be useful. We might think that
there is still a duty to maximise the probability of being maximally
useful. The case of {Botum} shows this isn't true. She does the one
thing she knows cannot be maximally useful. But that one thing is the
one and only right thing for her to do. All the other alternatives are
both wrong and blameworthy, and that includes the one very useful one.

This way of talking about right and wrong is not universally adopted. In
part this is an unimportant matter of terminological regimentation, but
I suspect in part it reflects a deeper disagreement. Here's the kind of
case that motivates the way of talking I'm not going to use.

\begin{quote}
{Adelajda} is a doctor, and {Francesc} her patient. {Francesc} is in a
lot of pain, so {Adelajda} provides pain medication to {Francesc}.
Unfortunately, someone wants to kill {Francesc}, so the pain medication
has been adulterated. In fact, when {Adelajda} gives {Francesc} this
medicine, she kills him.
\end{quote}

A common verdict on this kind of case is that {Adelajda} acts wrongly,
since she kills someone, but blamelessly, since she was ignorant of what
she was injecting {Francesc} with ~(\citeproc{ref-Rosen2008}{Rosen
2008}; \citeproc{ref-Graham2012}{Graham 2014};
\citeproc{ref-Harman2014}{E. Harman 2015}). The picture seems to be that
an action is wrong if it brings about a bad outcome, and considerations
of what was known are irrelevant to the wrongnes of the act. So
{Adelajda}'s act is wrong because it is a killing, independent of her
knowledge.

I think this is at best an unhelpful way to think about {Adelajda}. In
any case, I'm not going to use `right' and `wrong' in that way. On my
preferred picture, {Adelajda}'s ignorance doesn't provide her an excuse,
because she didn't do anything wrong. (I follow orthodoxy in thinking
that excuses are what make wrong actions less blameworthy.) I think the
picture where {Adelajda} doesn't do anything wrong makes best sense of
cases like {Botum}'s. I'm here following {Frank} Jackson
(\citeproc{ref-Jackson1991}{1991}), who supports this conclusion with a
case like this one.

\begin{quote}
{Billie} is a doctor, and {Jack} her patient. {Jack} has a very serious
disease. He is suffering severe stomach pains, and the disease will soon
kill him if untreated. There are three drugs that would cure the
disease, A, B and C. One of A and B would stop {Jack}'s pain
immediately, and cure the disease with no side effects. The other would
have side effects so severe they would kill {Jack}. {Billie} has no idea
which is which, and it would take two days of tests to figure out which
to use, during which time {Jack} would suffer greatly. Drug C would cure
the disease, but cause {Jack} to have one day of severe headaches, which
would be just as painful as the stomach pains he now has.
\end{quote}

The thing for {Billie} to do is to give {Jack} drug C. (I'm saying
`thing to do' rather than using a term like `good or 'right' because
what's at issue is figuring out what's good and right.) Giving {Jack}
drug A or B would be a horribly reckless act. Waiting to find out which
of them would have no side effect would needlessly prolong {Jack}'s
suffering. So the thing to do is give him drug C.

But now consider things from the perspective of someone with full
knowledge. (Maybe we could call that the objective perspective, but I
suspect the terminology of `objective' and `subjective' obscures more
than it reveals here.) {Billie} directly causes {Jack} to have severe
headaches for a day. This was avoidable; there was a drug that would
have cured the disease with no side effects at all. Given full
knowledge, we can see that {Billie} caused someone in her care severe
pain, when this wasn't needed to bring about the desired result. This
seems very bad.

And things get worse. We can imagine {Billie} knows everything I've said
so far about A, B and C. So she knows, or at least could easily figure
out, that providing drug C would be the wrong thing to do if she had
full knowledge. So unlike {Adelajda}, we can't use her ignorance as an
excuse. She is ignorant of something all right, namely whether A or B is
the right drug to use. But she isn't ignorant of the fact that providing
C is wrong given full information. Now assume that we should say what
{Adelajda} does is wrong (since harmful), but excusable (because she
does not and could not know it is wrong). It follows that what {Billie}
does is also wrong (since harmful) but not excused (since she does know
it is wrong).

This all feels like a reductio of that picture of wrongness and excuse.
The full knowledge perspective, independent of all considerations about
individual ignorance, is not constitutive of right or wrong. Something
can be the right thing to do even if one knows it will produce a
sub-optimal outcome. So it can't be ignorance of the effects of one
action provides an excuse which makes a wrong action blameless. {Billie}
needs no excuse, even though she needlessly causes {Jack} pain. That's
because {Billie} does nothing wrong in providing drug C. Similarly,
{Adelajda} does nothing wrong in providing the pain medication. In both
cases the outcome is unfortunate, extremely unfortunate in {Adelajda}'s
case. But this doesn't show that their actions need excusing, and
doesn't show that what they are doing is wrong.

The natural solution here is to say that what is right for {Botum} or
{Billie} to do is not to maximise the probability of a useful outcome,
but to maximise something like expected utility. It won't matter for
current purposes whether we think {Botum} should maximise expected
utility itself, or some other risk-adjusted value, along the lines
suggested by John Quiggin (\citeproc{ref-Quiggin1982}{1982}) or Lara
Buchak (\citeproc{ref-BuchakRisk}{2013}). The point is, we can come up
with a `subjective' version of usefulness, and this should not be
identified with the probability of being useful. We'll call cases like
{Botum} and {Billie}'s, where what's right comes apart from even the
probability of being best, Jackson cases, and return to them frequently
in what follows.\footnote{Similar cases were discussed by Donald Regan
  (\citeproc{ref-Regan1980}{1980}) and Derek Parfit
  (\citeproc{ref-Parfit1984}{1984}). But I'm using the terminology
  `Jackson case' since my use of the cases most closely resembles
  Jackson's, and because the term `Jackson case' is already in the
  literature.}

Expected values are only defined relative to a probability function. So
when we ask which action maximises expected value, the question only has
a clear answer if we make clear which probability functions we are
talking about. Two probability functions in particular will be relevant
going forward. One is the `subjective' probability defined by the
agent's credences. The other is the `evidential' probability that tracks
how strongly the agent's evidence supports one proposition or another.
These will generate subjective expected values, and evidential expected
values, for each possible action. And both values will have a role to
play in later discussion.

\section{Two Ways of Maximising Expected
Goodness}\label{twowaysofmaximisingexpectedgoodness}

So far we have only looked at agents who are uncertain about a factual
question. {Cressida} does not know who she will harm by driving as she
does, {Botum} does not know which number will come up on the roulette
wheel, and {Adelajda} and {Billie} are ignorant of the effects of some
medication. But we could also imagine that agents are uncertain about
normative questions.

{Deorsa} is deciding whether to have steak or tofu for dinner. He is a
remarkably well informed eater, and so he knows a lot about the process
that goes into producing a steak. But try as he might, he can't form an
opinion on the moral appropriateness of eating meat. He thinks meat
eating results in outcomes that are probably not bad, but like many
carnivores, he has his doubts.

To simplify the story, I'm going to make three assumptions. The first
assumption is that Deòrsa is actually in a world where meat eating is
not bad. The second assumption is that Deòrsa is perfectly reasonable in
having a high, but not maximal, credence in meat eating not being bad.
You may think that this requires Deòrsa to live in a world very unlike
this one, or even an impossible world. But that's OK for the story I'm
telling; I just need Deòrsa's situation to be conceivable. (We will
spend a lot of time thinking about impossible worlds as this book goes
on, so it's useful to warm up with one that might be impossible now.)
And the third assumption is that there is a large asymmetry between
Deòrsa's choices. If meat eating is not bad, it would be ever so
slightly better for Deòrsa to have the steak, since he would get some
enjoyment from it, and it wouldn't be bad in any other respect. But if
meat eating is bad, then having the steak would be a much much worse
outcome, since it would involve Deòrsa in an unjustified killing.

Which action, having the steak or having the tofu maximises expected
goodness? That question is ambiguous. In one sense the answer is tofu.
After all, there is a non-trivial probability that having the steak
leads to a disasterous outcome. In another sense, the answer is steak.
After all, there is a thing goodness, and Deòrsa knows enough to know of
it that it is maximised by steak eating. Since Deòrsa is to some extent
morally ignorant, he doesn't know what goodness is, so he thinks
goodness might be something else, something that is not maximised by
steak eating. But given his (perfectly reasonable, rational) credences,
the thing that is goodness has its expected (and actual) value maximised
by steak eating.

We might put the distinction in the previous paragraph by saying that
the action that maximises the expected value of goodness de re, that is,
of the thing that is goodness, is different from the action that
maximises the expected value of goodness de dicto, that is, of whatever
it is that goodness turns out to be. And using the de dicto/de re
terminology, we can see that this distinction applies across a lot of
realms. Here are two more examples where we can use it.

\begin{quote}
{Monserrat} is playing the board game Settlers of Catan. She has to
decide between two moves. She is uncertain how the moves will affect the
later game play. This is reasonable, since the game play includes dice
rolls that she couldn't possibly predict. But she's also forgotten what
the victory condition is. She can't remember if it is first to 10 points
wins, or first to 12 points. The standard is 10, but some games are
played under special house rules that change this. In {Monserrat}'s
game, there aren't any special house rules, so it is actually 10 points
that wins. Call the moves that she is choosing between A and B. If she
plays A, she has a 30\% chance of being first to 10 points, and a 50\%
chance of being first to 12 points. If she plays B, she has a 40\%
chance of being first to 10 points, but only a 10\% chance of being
first to 12 points. She thinks it is 60\% likely that the winner is the
first to 10, and 40\% likely that the winner is the first to 12. So
playing B maximises the probability of winning de re. That is, it
maximises the probability of doing the thing that is actually winning,
i.e., being first to 10. But playing A maximises the probability of
winning de dicto. Given {Monserrat}'s uncertainty about the victory
conditions, she thinks her probability of winning is 38\% if she plays
A, and only 34\% if she plays B.
\end{quote}

\begin{quote}
A professor is deciding which music to put on. She would prefer lowbrow,
trashy music. But, suffering from a common enough kind of false
consciousness, she thinks she would prefer highbrow, classy music. So
playing the lowbrow music would maximise expected preference
satisfaction de re. That is, it would maximise the expected value of the
satisfaction level of the preferences she actually has. But playing the
classy music would maximise expected preference satisfaction de dicto.
That is, given her beliefs about her preferences, it seems that the
classy music would do a better job at satisfying her preferences.
\end{quote}

The key internalist idea is that in situations that call for maximising
expected goodness (or utility, or anything else), it is the de dicto
version, not the de re version, that matters. The key externalist idea
is that it is the de re version that matters. For the rest of this
chapter, while I'm setting up and motivating internalism, I'll leave it
tacit that we are talking about expected values de dicto.

\section{Varieties of Internalism}\label{varietiesofinternalism}

The chapter started with a five-way distinction between the useful, the
good, the right, the rational and the praiseworthy. And we noted that
for each of those, there were three separate questions we can ask in any
practical situation. First, we can ask what action would be most
useful/good/right/rational/praiseworthy. Second, we can ask what action
has the highest expected
usefulness/goodness/rightness/rationality/praiseworthiness given the
credences of the agent. Third, we can ask that same question, but
relativise the answer to the agent's evidence, not the agent's
credences. Multiplying the five way distinction by the three types of
question gives us fifteen questions. And each of those fifteen questions
picks out a kind of standard. It is an interesting feature of a possible
choice that it actually is the rational one, or that it maximises credal
expected praiseworthiness, or evidential expected usefulness. For now,
call the questions about what actually is most useful etc objective
questions, and the standard that an action or choice meets in virtue of
being the answer to such a question an objective standard. (This is just
to distinguish the first class of questions from the credal and
evidential questions.)

Having these fifteen standards in mind, the five objective standards,
the five credal standards and the five evidential standards, we have the
resources to formulate a number of interesting internalist theses. The
theses I have in mind are of the form:

\begin{itemize}
\tightlist
\item
  X objectively meets normative standard N1 when she meets
  credal/evidential standard N2.
\end{itemize}

Philosophers who endorse these theses usually take it that the
explanatory direction here goes from right-to-left. It is because the
agent meets credal/evidential standard N2 that she objectively meets
standard N1. But my primary focus will be on the truth of these claims,
and not yet the claims about explanatory priority.

Michael Zimmerman (\citeproc{ref-Zimmerman2008}{2008}) endorses the
following two theses, which exemplify this schema.

\begin{itemize}
\tightlist
\item
  An action is right when it maximises evidentially expected goodness,
  and it is wrong when it does not.
\item
  A person is praiseworthy for maximising credally expected goodness,
  and blameworthy for not doing so.
\end{itemize}

Michael Smith (\citeproc{ref-Smith2006}{2006},
\citeproc{ref-Smith2009}{2009}) argues (against the arguments from
Jackson I gave above) that right action is just action that maximises
the good. But what an agent is responsible for is whether they maximise
evidential expected goodness de dicto. Indeed, what they should do, in
`the sense most relevant for action', is maximise evidential expected
goodness de dicto ~(\citeproc{ref-Smith2006}{M. Smith 2006, 144}).
Moreover, this is what rationality requires
~(\citeproc{ref-Smith2009}{M. Smith 2009}).

There are obviously a lot of other possibilities for N1 and N2 that we
could use, and that gives us a lot of internalist theses. Before we go
on, three clarifications on what I am, and what I am not, counting as an
internalist thesis.

First, I've put the statements above in ways that are naturally
interpreted as universal quantifications. That makes them very strong,
perhaps implausibly strong. A view that said that theses like the above
held ceteris paribus, or held subject to side constraints, or held in a
well defined range of cases, would still be internalist in the sense I'm
interested in.

Second, the theses listed above are biconditionals. We could weaken them
to one-way conditionals, and still get something recognisably
internalist, as long as we think that the conditional is still somewhat
explanatory. For instance, a view that said an agent is blameless for
what they do if they maximise evidential expected goodness would be
internalist, even if it didn't give necessary and sufficient conditions
for blamelessness. Such a view might also add some externalist
conditions to blamelessness; perhaps it would go on to say that someone
is blameless as long as they actually don't make things worse, or
actually do anything wrong. It's a matter of terminological preference
whether we count these hybrid views as internalist or externalist, but
since I plan to argue against them, I'm counting them as internalist.
(Chapters 5 and 6 will be dedicated to a discussion of some such views.)

Third, I'm not counting a view as internalist unless both N1 and N2 are
\textbf{person-evaluative}. What I mean by saying a term is
person-evaluative is that it is a term we use for evaluations that
essentially apply to persons, or actions or states of persons. So truth
is not person-evaluative, since we can ask whether the output of a
measuring device is true, and harmfulness is not person-evaluative,
since earthquakes and volcanoes are harmful. But rationality,
praiseworthiness, moral goodness, and moral rightness are
person-evaluative (at least if they are evaluative).

So the view Jackson (\citeproc{ref-Jackson1991}{1991}) defends, where
rightness is a matter of maximising expected benefits, is not
internalist in my sense, because being a benefit is not a
person-evaluative notion. Put another way, we don't positively evaluate
{Cressida} the reckless driver, even if we note that her actions
actually had a small benefit to the world.

A harder case to judge is whether this should count as an internalist
thesis.

\begin{itemize}
\tightlist
\item
  It is a requirement of rationality that one does the thing that
  maximises expected goodness (de dicto).
\end{itemize}

Is that an internalist thesis, or not? It depends on what one thinks
about rationality. Is rationality person-evaluative. Well, it
essentially applies to people. (If we judge a machine is thinking
rationally, and not just accurately, we are treating it as a person.)
But is it evaluative? It's easy to think this question is easy. Ideal
agents are rational, and it is good to be like ideal agents, so of
course it is good to be rational. But that's too quick. An ideal taker
of a logic quiz would make an even number of errors, since they would
make 0 errors, and 0 is even. But that doesn't mean the property of
making an even number of errors is an evaluative notion in any sense. We
shouldn't say, ``Good for you, you made an even number of errors.''
Making an even number of errors seems completely epiphenomenal from an
evaluative standpoint. And it would be an absurd thing to aim at, as
such. It's surprising how common it is that properties of the ideal are
actually bad to aim at, since they often make things worse in the
absence of other features of the ideal
~(\citeproc{ref-LipseyLancaster}{Lipsey and Lancaster 1956-1957}). If
one thinks being rational is like possessing the property \emph{makes an
even number of mistakes}, then one could agree that rationality involves
maximising expected goodness, without thereby disagreeing with
externalism.

Now as a matter of fact, I personally think rationality is evaluative,
and is not a matter of maximising expected goodness. So I think the
thesis is internalist, and is false. But the classificatory question is
still important. After all, this thesis is certainly true:

\begin{itemize}
\tightlist
\item
  An action maximises expected goodness iff it maximises expected
  goodness.
\end{itemize}

This looks like it has the structure of my canonical internalist theses,
with N1 being \emph{maximises expected goodness} and N2 being
\emph{goodness}. So doesn't this show that some internalist theses are
true? No, I say. This isn't internalist because maximising expected
goodness, where this is understood de dicto and not de re, is not a
positive feature of a person. It is a feature that ideal agents have,
but it is also a feature that political fanatics like {Robespierre}
have. And it isn't a good-making feature in either of them. Rather, it
is like making an even number of errors; something that can be
instantiated in very good ways, or very bad ways.

\section{An Initial Constraint}\label{aninitialconstraint}

The internalist schema above has some interesting instances when N1 =
N2. For instance, we could consider the following theories, where we use
the same kind of evaluation on both sides of the biconditional.

\begin{itemize}
\tightlist
\item
  It is right to maximise the expected rightness of one's actions, and
  wrong to do otherwise.
\item
  It is blameworthy to do what is most probably blameworthy.
\end{itemize}

But there is a quick argument that all such principles are mistaken. The
brief version of the argument is that no such principle is compatible
with the conjunction of knowledge of one's own mental states, plus
uncertainty about what I'll call morally asymmetric choices. But there
is nothing wrong with knowing one's own mental states when faced with a
morally asymmetric choice, so the principles must be wrong.

A morally asymmetric choice is where we know that one side of the choice
is not in any way morally problematic. A simple case, for most people,
is the choice between meat eating and vegetarianism. Very few people
would think that it is immoral, bad, wrong, or blameworthy to be
vegetarian on ethical grounds. On the other hand, it is easy to feel
some qualms about eating meat. So it looks like this is a choice where
all the moral risk falls on one side.

(I'm more interested in the general principle than the particular case,
but let me note two quick complications before moving on. It's
imaginable that there is a person who puts either their own health or,
if they are pregnant or nursing, their child's health, at risk by not
eating any meat. In the situations most readers of this book find
themselves, those situations will be vanishingly rare since there are so
many meat alternatives available. But it's at least conceivable. In the
cases I'm discussing I want it to be explicitly part of the case that
the person making the choice faces no health complications from being
vegetarian. Second, I'm ignoring the possibility that denying oneself
pleasures for spurious reasons is immoral. It would merely complicate,
but not overturn, the argument to allow for that possibility.)

Now let's think about the first bulleted principle above, which I'll
call \textbf{ProbWrong}. And consider an agent who is deciding between
steak and tofu for dinner. Imagine that she has the following mental
states:

\begin{enumerate}
\def\labelenumi{\arabic{enumi}.}
\tightlist
\item
  She is sure that \textbf{ProbWrong} is true.
\item
  She is almost, but not completely, sure that eating meat is not wrong
  in her exact circumstances.
\item
  She is sure that eating vegetables is not wrong in her exact
  circumstances.
\item
  She is sure that she has states 1--3.
\end{enumerate}

A little reflection shows that this is an incoherent set of states.
Given \textbf{ProbWrong}, it is simply wrong for someone with states 2
and 3 to eat meat. And the agent knows that she has states 2 and 3. So
she can deduce from her other commitments and mental states that eating
meat is, right now, wrong. So she shouldn't be almost sure that eating
meat is not wrong; she should be sure that it is wrong.

This argument generalises. If 1, 3 and 4 are true of any agent, the only
ways to maintain coherence are to be completely certain that meat eating
is not wrong, or completely certain that it is wrong. But that is
absurd; these are hard questions, and it is perfectly reasonable to be
uncertain about them. At least, there is nothing incoherent about being
uncertain about them. But \textbf{ProbWrong} implies that this kind of
uncertainty is incoherent, at least for believers in the truth of
\textbf{ProbWrong} itself. Indeed, it implies that in any asymmetric
moral risk case, an agent who knows the truth of \textbf{ProbWrong} and
is aware of her own mental states cannot have any attitude between
certainty that both options are not wrong, and certainty that the risky
action is not, in her exact circumstances, wrong. That is absurd.

I conclude that any version of the normative internalist thesis where N1
= N2 is also absurd. Happily, that view seems to be shared by existing
defenders of internalism, who usually defend versions where N1 \(\neq\)
N2. So I'll set the N1 = N2 versions of internalism aside and focus just
on the versions where they come apart.

\section{Motivation One: Guidance}\label{motivationone:guidance}

The externalist offers a fairly simple piece of advice to people facing
a moral challenge: Do the right thing. But as a general piece of advice,
\emph{Do the right thing} might sound not much more helpful than
\emph{Buy low, sell high}. We need, it might be thought, more helpful
advice.

That kind of consideration plays a big role in our thinking about
factual uncertainty. Think again about {Botum} the charity director. The
best outcome for her, and for the cause she is working for, would be for
her to bet the £10,000 on the number that will actually win. But we
don't think she's under an obligation to do that. Indeed, we think she
is under an obligation to not even try to do that. One reason for that,
arguably, is that the strategy \emph{Bet on the winning number} is not
one she is in a position to carry out.

Now the externalist does think that agents should carry out the strategy
\emph{Do the right thing}. But in cases where the moral evidence is
murky, arguably this is no more a reasonable demand than the demand that
{Botum} bet on the winning number. Here is how Michael Smith puts the
point. He has just rehearsed {Frank} Jackson's argument, involving cases
like {Billie}, for the conclusion that right action does not involve
maximising the probability of the best outcome, but maximising expected
value.

\begin{quote}
Indeed, anyone impressed by Jackson's argument on the non-evaluative
facts side of things should surely suppose that an equally impressive
argument could be made for the conclusion that right action consists not
in the maximization of expected \emph{value}, but rather in the the
maximization of expected \emph{value-as-the-agent-sees-things}. For no
mere exercise of such capacities as an agent has looks like it will
ensure that what is really valuable will manifest itself to her either.
There are, after all, cultural circumstances in which it would be wildly
optimistic to suppose that agents could, merely through the exercise of
their own rational capacities, come to judge to be valuable what's
really valuable \ldots{} If this is right, however, then it seems that
the most that we could ever expect of a normal agent \ldots{} is that
they form their evaluative commitments in a way that is sensitive to
such evidence as is available to them and that they form their desires
in a way that is sensitive to their evaluative commitments.
~(\citeproc{ref-Smith2006}{M. Smith 2006, 143})
\end{quote}

Andrew Sepielli expresses a similar sentiment.

\begin{quote}
The problem is that we cannot base our actions on the correct normative
standards; our relationship to such standards is limited to mere
conformity to them. This follows from a quite general point---that we
cannot guide ourselves by the way the world is, but only by our
representations of the world. ~(\citeproc{ref-Sepielli2009}{Sepielli
2009, 8})
\end{quote}

And we saw in the previous chapter that similar sentiments are expressed
by Ted Lockhart (\citeproc{ref-Lockhart2000}{2000, 8--9}), William
MacAskill (\citeproc{ref-MacAskillThesis}{2014, 7}) and by Hillary
Greaves and Toby Ord (\citeproc{ref-GreavesOrd2017}{2017}). We might try
to turn this idea into an argument for internalism as follows.

\begin{enumerate}
\def\labelenumi{\arabic{enumi}.}
\tightlist
\item
  Our most important norms should be sources of usable advice.
\item
  If normative externalism is true, our norms are not sources of usable
  advice.
\item
  If normative internalism is true, our norms are sources of usable
  advice.
\item
  So normative externalism is false, and we have a reason to believe
  normative internalism is true.
\end{enumerate}

Note that I'm not here assuming that normative externalism and normative
internalism are contradictories; there are positions that might best be
classified as falling into neither camp. If they were contradictories,
the second conjunction of the conclusion would be highly redundant.

One problem for this argument is that it relies on a slippery notion of
usability. If we have rather generous standards for what counts as a
usable norm, then premise 2 of the argument is false. After all, we can
often tell what is the right thing to do. If we have rather strict
standards, then premise 1 is false, since it amounts to the claim that
the application conditions for the most important norms must be
luminous. (A norm is luminous if whenever it applies, it is possible to
know that it applies.) But Timothy Williamson
(\citeproc{ref-Williamson2000}{2000}) has shown that nothing interesting
is luminous, and our most important norms are interesting. I suspect
that there is no reading of `usable' that makes both premises 1 and 2
true.

The slipperiness also extends to premise 3. The internalist needs
standards that are usable, in their preferred sense, and which
{Robespierre} violates. (Unless they are happy saying that Robespierre
did well, in the sense that's most important to them.) But they need
that sense of usability to be one in which \emph{Do the right thing} is
not usable. And it is hard to see what that sense could be.

The regress arguments that will recur throughout this book are designed,
in part, to back up this conclusion. (See particularly the discussion of
inter-theoretic value comparisons in section 6.2.) I'm going to be
arguing that everyone except the most radical subjectivist will be have
to acknowledge standards for evaluating agents that those very agents
are not in a position to accept. The only options, I'll argue, are
radical subjectivism, and norms that are not guaranteed to be able to
usable in the internalist's preferred sense. That is, the norms will
only be usable in the sense that \emph{Do the right thing} is usable.
Since this radical subjectivism is false, some monsters really do well
by their own lights, the connection between evaluation and guidance must
be more tenuous than the internalist assumes.

\section{Motivation Two: Recklessness}\label{motivationtwo:recklessness}

A different argument against externalism is that it licences a form of
moral recklessness. And this kind of moral recklessness should not be
licenced, says the objector, it should be condemned.

To see the problem, start with the example of Deòrsa, the uncertain
carnivore. (This case is discussed by Guerrero
(\citeproc{ref-Guerrero2007}{2007}), who uses it in mounting an attack
on moral recklessness.) And let's assume that Deòrsa does end up
deciding that he will eat meat. Deòrsa knows that the moral risks are
largely, if not universally, on one side. He knows that eating meat
provides him with just a small benefit, but puts him at risk of being a
moral monster. And yet he does it.

Now by hypothesis, a fully informed agent in Deòrsa's position would do
the same thing. And yet it is easy to feel some unease with the
externalist verdict that Deòrsa's actions are right, rational and
blameless. There is a whiff of recklessness about Deòrsa's actions, and
this kind of recklessness may seem to be a moral vice.

We can make this whiff stronger by tightening the analogy with reckless
action. For example, imagine that Deòrsa isn't mostly certain that meat
eating is acceptable. In fact, in the revised case he is very confident
that meat eating is wrong. And yet, he eats meat anyway. The analogies
between Deòrsa and {Cressida}, the reckless driver, start to feel
compelling at this point. And yet the externalist says that Deòrsa is
not doing anything wrong, or irrational, or blameworthy. (This variant,
and its importance, was suggested by Andy Egan.)

Or perhaps we can build the analogy directly into Deòrsa's case. Imagine
that as well as choosing what to eat, Deòrsa is choosing how to cook it.
Deòrsa is considering trying out a new technique from a modernist
cookbook. He knows that a side effect of this technique is that a
distinctive kind of chemical is released into his building's
ventilation. This chemical will build up in large quantities in his
apartment and the apartment next door. The chemical is odorless, and
harmless to everyone who doesn't have a particular allergy. But the
quantities Deòrsa would release would be fatal to anyone with the
allergy. And Deòrsa knows the boy in the next apartment has some kind of
rare allergy, though he can never remember which one it is. He thinks it
is probably some other allergy the boy has, and in fact he is right. So
he cooks the meat using the modernist technique.

To make the analogy explicit, assume that Deòrsa has equal credence in
these two propositions.

\begin{enumerate}
\def\labelenumi{\arabic{enumi}.}
\tightlist
\item
  Meat eating is morally acceptable.
\item
  The boy in the next apartment will not have a fatal reaction to the
  chemical that will be released by the modernist cooking technique.
\end{enumerate}

In each case, this credence is high, but far from maximal. Unless Deòrsa
knows that 2 is true, what he does is horribly reckless. It's not worth
risking killing one of your neighbours to get the benefits of a new
method of meat preparation. Similarly, says the internalist, the
gustatory benefits of meat aren't worth the risk that goes along with
joining the meat-eating team.

D. Moller (\citeproc{ref-Moller2011}{2011}) similarly argues that
internalism is motivated by considerations about recklessness. I'll
respond to Moller's own example at more length below, so let me start
with my own variant of the kind of case that motivates his position. Two
CEOs are trying to choose between more aggressive and more conservative
business strategies. Each commissions internal inquiries to determine
some properties of the aggressive strategy. (They know how conservative
strategies work, since those strategies are familiar.) One of the CEOs
doesn't know exactly what the practical consequences of the aggressive
strategy will be, so she commisions an inquiry into those practical
consequences. And the other CEO doesn't know what the right moral
evaluation of the aggressive strategy is, so he commissions an inquiry
into its moral evaluation. Both inquiries come back with a 3--2 split.
In the first case, all five agree the aggressive strategy will slightly
raise profits relative to the conservative strategy. But two members
think that a side effect will be that ten people in nearby communities
fall sick and die as a consequence of the company's operations. In the
second case, all five think the conservative strategy is morally
acceptable. But three think the aggressive strategy is good enough,
while the other two think it is as bad as being responsible for ten
avoidable deaths. In each case, it turns out, the majority members of
the committee are right, though the CEO has no extra evidence for that.
The intuition these cases seem to support is that neither CEO should
carry out the aggressive strategy. Indeed one might hold (though Moller,
interestingly, does not) that we should think of the CEO's who carry out
these strategies as being equally culpable for their recklessness.

\section{Motivation Three: Symmetry}\label{motivationthree:symmetry}

Both the guidance considerations and the recklessness considerations
push one towards thinking that factual uncertainty and moral uncertainty
should be treated symmetrically, or at least as symmetrically as
possible. I briefly mentioned that Moller expressly rejects the symmetry
claim, and the failure of N1=N2 versions of internalism make it hard, at
least for non-consequentialists, to endorse perfect symmetry. But there
is something to the idea that moral uncertainty and factual uncertainty
should get very similar theoretical treatments, and the externalist
offers very different theoretical treatment of them.

We could get to this idea in a few ways. We could try to argue that it
follows from considerations about guidance or recklessness. We could try
to argue that it best explains intuitions about guidance or
recklessness. Or we could just argue for it directly, either my appeal
to the intuitive plausibility of the symmetry claim, or the intuitive
plausibility of what it says about a number of cases. For instance, we
could just argue that it is plausible that whatever negative attitude we
have towards {Cressida}'s actions, and to {Cressida}, we should have
towards Deòrsa's actions, and to Deòrsa. And we could argue that
whatever positive attitude we have towards {Billie}'s actions, and to
{Billie}, we should have to the person who successfully manages to
maximise evidential expected goodness. In short, we should have
symmetric attitudes about the philosophical significance of normative
uncertainty and factual uncertainty.

Ths idea that symmetry (or near-symmetry) should be built into our
theories will not, I suspect, strike most people as absurd. Indeed, I
suspect it strikes many people as so plausible it barely needs defence.
It certainly does a lot of work, without much argument, in works by
Jacob Ross (\citeproc{ref-Ross2006}{2006}) and Michael Zimmerman
(\citeproc{ref-Zimmerman2008}{2008}). If the symmetry thesis is both
intuitive and true, there's nothing wrong with this approach. And I
concede it is, at least prima facie, highly intuitive. But I don't think
it is true. Indeed, I don't think it is even particularly intuitive,
once we reflect on it in more detail.

But it is intuitive enough to use as the foundation for discussions of
internalism. And while I'll cycle back around to other motivations for
internalism, I'll use symmetry-based considerations as the main focus of
discussion. That's because the symmetry-based considerations do such a
good job of both being independently intuitive, and capturing what is
best worth capturing in the other arguments.

So in the next chapter I'll push back against the intuitiveness of this
symmetry claim, arguing that the closer we look at it, the less similar
factual and moral uncertainty seem. And in the chapter after that, I'll
argue that even if symmetry is plausible it should be rejected, for it
leads to unacceptable regresses.

\chapter{Against Symmetry}\label{againstsymmetry}

In the previous chapter, I suggested that one of the key motivations for
normative internalism that it allows for a symmetry between the way we
treat factual uncertainty and ignorance, and the way we might think
about treating normative uncertainty and ignorance. Some writers have
found it so obvious that these cases should be treated symmetrically
that they have simply incorporated this symmetric treatment into their
theory without arguing for it. Those who have argued for it have usually
found the symmetry very intuitive.

In this chapter, I'll try to undermine that intuitive symmetry. The
first three sections will introduce three considerations that undermine
the idea that the factual and normative uncertainty should be treated
symmetrically, and the last three sections deal with some complications
that the first three sections introduce. In the next chapter, I'll argue
that even if we found the symmetry intuitive, we should ultimately
reject it, because there is no way to incorporate it into a theory that
is even remotely plausible. That is, I'll argue that any internalist
theory that can handle even very simple cases has to reject the symmetry
thesis, and so cannot be motivated by symmetry considerations.

\section{Guilt and Shame}\label{guiltandshame}

If normative and factual uncertainty have the same normative
implications, then we should feel similarly about our own past actions
that were done due to factual ignorance, and those that were done due to
moral ignorance. But this doesn't seem to be how we do, or should, feel.
We can see this by comparing a pair of cases. The second of the cases is
a minor modification of a case Elizabeth E. Harman
(\citeproc{ref-Harman2014}{2015}) uses in making a similar argument to
the one I'm presenting in this section.

{Prasad} is a father of two children, an older daughter and a younger
son. In the division of parental labour in his house, teaching the
children to read is primarily his responsibility. He takes this very
seriously, and reads the latest studies on which techniques are most
effective at teaching reading. He doesn't have a strong enough
background in statistics to be able to evaluate many of the papers he
reads, but he can tell what techniques are being approved by the leading
figures in the field, and those are the techniques he uses in teaching
his children to read.

Unfortunately, the relevant science around here moves slowly and
fitfully. The technique that {Prasad} followed when his daughter was
learning to read was soon shown to be mostly ineffective. It was better
than not spending time on reading, but wasn't any better than
unstructured reading time. By the time his son was learning to read,
educational science had advanced substantially, and {Prasad} was able to
use a technique that led to his son learning to read relatively quickly.
This gave his son an advantage that persisted throughout his schooling,
and led to him being admitted to an exclusive college, and subsequently
earning much more than he would have without the benefit of early
reading. {Prasad}'s daughter did well at school, as you'd expect with
this level of parental attention, but would have been even better off
had been trained to read the way her brother was trained.

{Archie} is a 1950s father who, like many other 1950s fathers, thinks it
is more important to look after his son's interests than his daughter's.
So while he puts aside a substantial college fund for his son, he puts
aside less for his daughter. As a consequence, his daughter cannot
afford to go to as good a college as his son goes to, and subsequently
is materially less well off throughout her life than {Archie}'s son.

{Prasad} was mistaken about a matter of fact; about which techniques are
most effective at teaching a child to read. {Archie} was mistaken about
a moral matter; whether one should treat one's sons and daughters
equally. Now consider what happens when both see the error of their
ways. {Prasad} may feel bad for his son, but there is no need for any
kind of self-reproach. It's hard to imagine he would feel ashamed for
what he did. And there's no obligation for him to feel guilty, though
it's easier to imagine him feeling guilty than feeling ashamed.
{Archie}, on the other hand, should feel both ashamed and guilty. And
it's natural that a father who realised too late that he had been guilty
of this kind of sexism would in fact feel the shame and guilt he should
feel. The fact that his earlier sexist attitudes were widely shared, and
firmly and sincerely held, simply seems irrelevant here.

If the symmetry thesis were correct, there should not be any difference
in {Prasad} and {Archie}'s attitudes. Both of them behaved in just the
way we should expect, given their factual and normative beliefs. And
both of them had beliefs that were sincere, and widely shared in their
community. But there is still a difference between the two of them, as
revealed by the emotional reactions they both do and should have.

\section{Jackson Cases}\label{jacksoncases}

As Zimmerman (\citeproc{ref-Zimmerman2008}{2008}) argues, the kinds of
cases discussed by Jackson (\citeproc{ref-Jackson1991}{1991}) are
important for seeing how factual uncertainty is normatively significant.
It isn't just that when an agent doesn't know what is true, and so
doesn't know which action produces the best outcome, she thereby doesn't
know what is right to do. In some cases of decision making under
uncertainty, the thing that is clearly right to do is the one thing she
knows will not produce the best outcome. Gambling the charitable
donation on the roulette wheel is wrong, although the best outcome would
be to gamble on the number that will actually come up. In the previous
chapter we dubbed cases like this, where the right thing to do is
something one knows will not produce the best outcome, Jackson cases.
Jackson cases are ubiquitous when making decisions under factual
uncertainty.

If we should treat factual uncertainty and moral uncertainty
symmetrically, then Jackson cases for moral uncertainty would be easy to
find. But it is far from clear that there are any such cases. That is,
it is far from clear that there are cases where we want to say anything
positive about an agent who hedges their moral bets.

A simple way to generate Jackson cases is to set up a decision problem
with the following features:

\begin{itemize}
\tightlist
\item
  There are three option: A, B and C;
\item
  There are two epistemic possibilities, \emph{w}\textsubscript{1} and
  \emph{w}\textsubscript{2}, the agent knows that precisely one of them
  is realised, and the she reasonably thinks each is fairly likely.
\item
  In \emph{w}\textsubscript{1}, A is optimal, C is a little worse, and B
  is a catastrophe.
\item
  In \emph{w}\textsubscript{2}, B is optimal, C is a little worse, and A
  is a catastrophe.
\end{itemize}

If the agent's uncertainty about \emph{w}\textsubscript{1} or
\emph{w}\textsubscript{2} is grounded in a straightforwardly factual
uncertainty, it seems the agent should do C. Just what that `should'
amounts to is up for debate, but there is something awful about doing A
or B - even if it produces the optimal outcome.

What happens, though, if \emph{w}\textsubscript{1} and
\emph{w}\textsubscript{2} are factually alike, but differ in the correct
moral theory? (As has come up a few times, it is unlikely that both
\emph{w}\textsubscript{1} and \emph{w}\textsubscript{2} will be
\emph{possible} worlds in this case, but I don't think this matters for
current purposes.) Well, let's look at some cases and see.

\subsection{Case One - Abortion}\label{caseone-abortion}

{Marilou} is 12 weeks pregnant, and lives in a state where abortion is
criminalised and, on occasion, heavily punished. {Marilou} deeply
desires to have an abortion. {Marilou} is reasonably well off, and as is
the norm in states that criminalise abortion, reasonably well off people
are able to obtain abortions with a little assistance. {Marilou} asks
her friend {Shila} for such assistance. {Shila} now has to make a
choice. {Shila} is torn between two moral views about abortions 12 weeks
into pregnancy. According to one, the potential that the fetus has to
develop into a fully functioning human being means that aborting it is
the moral equivalent of murder. According to another, the fetus has
little or no moral standing on its own, the importance of {Marilou}'s
autonomy means that {Marilou} should be able to get an abortion, and her
friends should assist her in avoiding the oppressive laws against
abortion. {Shila} now has three choices.

\begin{enumerate}
\def\labelenumi{\arabic{enumi}.}
\tightlist
\item
  Assist {Marilou} in getting the abortion, which is either a way of
  respecting {Marilou}'s autonomy and honouring their friendship, or is
  a way of being an accomplice to murder.
\item
  Report {Marilou}'s plans to the authorities, which is either horribly
  disrespectful to {Marilou} and a gross violation of their friendship,
  or bravely preventing a murder. (Assume that Shila knows that although
  the authorities aren't maximally vigilant about preventing abortions,
  they are obliged to act on incriminating information, so this tip-off
  will lead to Marilou's imprisonment.)
\item
  Do nothing, suspecting that without her help, {Marilou} will carry the
  child to term and quietly adopt it out.
\end{enumerate}

In either \emph{w}\textsubscript{1}, the world where abortion is
permissible, or \emph{w}\textsubscript{2}, the world where it is not, C
is bad. In \emph{w}\textsubscript{1}, {Shila} is a bad friend, and is
tacitly collaborating in state oppression. In \emph{w}\textsubscript{2},
{Shila} is not taking simple steps that would remove the mortal danger
facing an innocent human. But option C isn't catastrophic in either
world. In \emph{w}\textsubscript{1}, {Shila} is not personally stopping
{Marilou} get an abortion, she just isn't helping {Marilou} break the
law. (You can be a good enough friend and still draw the line between
helping one move houses and helping one move bodies.) And in
\emph{w}\textsubscript{2}, she's not killing anyone, or even letting
someone be killed, just not being maximally vigilent in preventing a
killing. So the case has the structure of a Jackson case.

And yet there is little to be said for C. The situation calls for moral
bravery, one way or the other. (I think in the direction of A, but it
doesn't matter for these purposes whether you agree with that.) And C is
moral cowardice. Unlike in the cases involving factual uncertainty, it
doesn't seem at all like the safe, prudent, commendable option.

\subsection{Case Two - Theft}\label{casetwo-theft}

{Eurydice} and {Pandora} are acquaintances, and they are planning to go
to a party. {Eurydice} is worried because {Pandora} plans to wear some
very expensive jewellery, and the party features a number of thieves,
several of whom are {Eurydice}'s friends. {Eurydice} tells {Pandora}
this, but {Pandora} is unmoved, and insists she won't be deterred from
living her life the way she wants by the existence of petty criminals.
{Eurydice} is much more observant than {Pandora}, and knows that if
someone tries to steal the jewellery, she'll be able to prevent them,
but only by using a non-trivial amount of physical force. For example,
she could punch the would-be thief hard in the jaw while he was making
his escape, revealing his thievery. (Realistically, she can't know
exactly how she would prevent a theft, but assume that's the level of
force that would be needed.)

{Eurydice} is torn between two moral theories. One of them is a fairly
mainstream view on which a moderate amount of physical force is
warranted if it is the only way to prevent the theft of expensive goods.
On the other moral theory, the demands of friendship and bodily autonomy
completely outweigh considerations arising from property, so punching a
friendly thief to prevent a theft would be a completely unjustified
assault. Given all this, {Eurydice} has three options.

\begin{enumerate}
\def\labelenumi{\arabic{enumi}.}
\tightlist
\item
  Go to the party and plan to prevent (using violence if necessary) any
  theft of {Pandora}'s jewellery.
\item
  Go to the party and plan to refrain from any violence, even if this
  means standing by while a theft occurs.
\item
  Prevent {Pandora} going to the party. The most morally acceptable way
  to do that, {Eurydice} thinks, would be to tell {Pandora} a small lie
  that leads to {Pandora} going on a wild goose chase for half the
  night, leaving it impossible to go to the party.
\end{enumerate}

Again, this feels like a Jackson case. C is a moral misdemeanour - you
shouldn't lie to people for the purpose of distracting them away from a
party they have every right to be at. But it's worse to stand by and
watch a theft take place that you could easily (and properly) prevent,
or to unjustifiedly punch a friend in the jaw.

Yet again it seems like C would be a terrible option to take. Either the
amount of violence needed to apprehend the thief would be justified or
it wouldn't be. In neither case does it seem like sending {Pandora} on a
wild goose chase to prevent the theft would be a good way to prevent the
problem arising. This seems true even though it would guarantee that
things don't go badly morally wrong, while either alternative runs a
substantial moral risk.

\subsection{An Asymmetry}\label{anasymmetry}

When welfare is on the line, it is not just acceptable, but laudable, to
sacrifice the chance of the best outcome for a certainty of a very good
outcome. But it isn't at all clear that this is true when virtue is on
the line. Committing a moral misdemeanour because you don't know which
of the other options is a moral felony and which is the right thing to
do is, still, committing a moral misdemeanour.

\section{Motivation}\label{motivation}

Moral uncertainty, at least of the kind I'm focussing on, is a kind of
constitutive uncertainty. An agent who is morally uncertain is uncertain
about what kind of things constitute goodness, rightness,
praiseworthiness, and so on. It's very plausible that these are indeed
constituted by something else. It's hard to imagine that rightness is a
free-floating feature of reality.

Cases of constitutive uncertainty are useful test cases for thinking
about what's really valuable. If we know that A constitutes B, and hence
have equally strong desires for A and for B, it isn't always easy to
tell which of these desires is more fundamental, and which is derived.
Of course, neither of the desires will be an \emph{instrumental} desire,
since getting A isn't a means to getting B. But one of them could be
derivative on the other.

And the simplest way to tell which is which, is to look to people who do
not know that A constitutes B, and see what makes sense from their
perspective. Think again about {Monserrat}, who has forgotten the
victory conditions for her game. We know that being first to 10 points
constitutes winning. But she doesn't. What action makes sense for her to
do? I think it is doing the thing that maximises her probability of
winning, given her credal distribution. It turns out that isn't the
thing that maximises her probability of being first to 10 points, which
is what actually amounts to winning. But she has no motivation to be
first to 10 points, unless that amounts to winning. Or, at least, she
has no such motivation on the most natural telling of the story. Perhaps
she has an odd psychological tick that means she always values being
first to \emph{n} figures in points in any game she plays. But the more
natural story is that she wants to win, and she should do the thing that
maximises the probability of winning.

Things are rather different when it comes to moral uncertainty. There it
seems that agents should be moved to produce the outcome that actually
constitutes goodness or rightness, not the thing that maximises expected
goodness or rightness. This is a point well made by Michael Smith. He
compared the person who desires to do what is actually right, as he put
it, desires the right de re, with the person who desires to do what is
right whatever that turns out to be, as he put it, desires the right de
dicto.

\begin{quote}
Good people care non-derivatively about honesty, the weal and woe of
their children and friends, the well-being of their fellows, people
getting what they deserve, justice, equality, and the like, not just one
thing: doing what they believe to be right, where this is read \emph{de
dicto} and not \emph{de re}. Indeed, commonsense tells us that being so
motivated is a fetish or moral vice, not the one and only moral virtue.
~(\citeproc{ref-Smith1994}{M. Smith 1994, 75})
\end{quote}

I think that's all true. A good person will dive into a river to rescue
a drowning child. (Assuming that is that it is safe enough to do so;
it's wrong to create more rescue work for onlookers.) And she won't do
so because it's the right thing to do. She'll do it because there's a
child who needs to be rescued, and that child is valuable.

Not everyone agrees with Smith that commonsense has this verdict about
moral motivation. It helps to see the point made less abstractly, about
a particular case. Here is the initial description of {Saint-Just} from
Palmer's classic study of the Committee of Public Safety, \emph{Twelve
Who Ruled}.

\begin{quote}
Saint-Just was an idea energised by a passion. All that was abstract,
absolute and ideological in the Revolution was embodied in his slender
figure and written upon his youthful face, and was made terrible by the
unceasing drive of his almost demonic energy. He was a Rousseauist, but
what he shared with Rousseau was the Spartan rigor of the \emph{Social
Contract}, not the soft day-dreaming of the \emph{Nouvelle Héloïse},
still less the self-pity of the \emph{Confessions}. He was no lover of
blood, as Collot d'Herbois seems to have become. \emph{Blood to him
simply did not matter}. The individual was irrelevant to his picture of
the world. The hot temperament that had disturbed his adolescence now
blazed beneath the calm exterior of the political fanatic.
~(\citeproc{ref-Palmer1941}{Palmer 1941, 74}, emphasis added)
\end{quote}

That's what someone who is only motivated by the good, as such, looks
like. And it's terrifying. Commonsense morality prefers a view where
blood matters, and the individual is relevant, and where all of
Rousseau's works have something to teach us about how to live.
\footnote{I've mentioned {Robespierre} a few times in this context, so
  it's interesting to note that Palmer thinks {Robespierre} is not as
  extreme as {Saint-Just}. He compares the two in the paragraph
  preceding this one, mostly saying that {Saint-Just} is a more extreme
  version of {Robespierre}. {Saint-Just} is similar to his hero, but
  ``without the saving elements of kindness and sincerity''. I think
  `saving' is a little strong, but otherwise that judgment seems right.
  Collot positively desired actually bad things, {Robespierre} cared
  insufficiently about actually good things, and {Saint-Just} simply did
  not care about anything beyond ideology.}

We need to distinguish here two theses one might have about moral
motivation. One is that the good, as such, should not be one's only
motivation. That's what Smith says commonsense says, and it's what the
example of {Saint-Just} supports. Another is that the good, as such,
should not be among one's motivations. I think this latter claim is
mostly true as well. But I'll come back to that; for now I want to spell
out the consequences of the weaker claim, that the good should not be
one's only motivation.

This claim already makes trouble for normative internalists, including
Smith himself. It makes trouble because it offers us a nice explanation
of why there should be the kind of asymmetry between factual and
normative uncertainty that we see in cases like {Shila}'s. Think again
about the situation she is facing. She has to choose between respecting
{Marilou}'s autonomy, and respecting the foetus's life. And she doesn't
know what to do, in no small part because she doesn't know which form of
respect constitutes moral rightness. But one thing she does know is that
the moderate option maximises expected goodness. If we thought that this
was an important motivation, we presumably should think it could be
decisive in some cases, and {Shila} might take that moderate option. But
intuitively she should never do that, and not have any motivation to do
that. A pro-choice theorist may think {Shila} should believe that
respecting autonomy is the right thing to do, and so {Shila} should be
motivated to do what's right because she's motivated to respect
autonomy. A pro-life theorist may think {Shila} should believe that
respecting life is the right thing to do, and so {Shila} should be
motivated to do what's right because she's motivated to respect life.
But neither will hold that {Shila} should have a motivation to do what's
right that floats free of her motivation to respect autonomy, and to
respect life.

This way of thinking about {Shila}`s case suggests a prediction, one
that is borne out by the cases. It isn't always the case that moral
'hedging', of the kind I've been criticising since the start of section
3.2, is bad. Imagine an agent faces a choice between competing values,
both of which are values that she holds dear. For instance, consider an
administrator who faces a student in a somewhat unusual situation. (The
point of it being unusual is to ensure there is no clear precedent for
what to do in such cases.) The administrator has to choose between being
compassionate to the person in front of her, and doing the thing she
thinks would best treat the case in front of her like previous cases.
She may well care both about compassion and equality, and in such a
case, it would make sense to look for a way to minimise the distance
between how she treats this case and how she has treated past cases,
while also being highly compassionate to the person in front of her. And
that is true even if the outcome she comes up with is neither the most
compassionate thing she can do, nor the most respecting of her desire to
treat like cases alike. The reason this makes sense is that the
administrator doesn't think rightness is either exclusively constituted
by compassion, or by treating like cases as alike as possible. Rather,
she has plural values, like most of us do. And plural values, as opposed
to uncertainty about what is the one true value, can produce moral
Jackson cases.

What is the difference between {Monserrat}'s case and {Shila}'s? Why
should {Monserrat} aim for what maximises the constituted quantity,
while {Shila} aims for what maximises (or perhaps best respects) the
constituting quantity? The answer comes from what it means for something
to be right. It just is for it to be valuable. One of the striking
things about games is that they turn something otherwise pointless, like
being first to 10 points, into something that rational people can value.
But morality isn't like that. It can't make value out of something that
wasn't valuable, because if it wasn't valuable, it wouldn't be fit to
constitute rightness. So whatever rightness is, be it respecting
autonomy or maximising welfare or whatever, must be something already
valuable. And it is hard to see how having the property of being most
valuable can be more valuable than the valuable thing itself.

So we get an explanation of Smith's observation. (And here I'm not
saying anything that hasn't been said before, by for example Nomy Arpaly
(\citeproc{ref-Arpaly2003}{2003}) and Julia Markovits
(\citeproc{ref-Markovits2010}{2010}).) It is good to aim at what is
actually right and good, not at rightness and goodness themselves,
because the constitutors are where the value lies. But that means moral
uncertainty should not affect our motivations. And that's a striking
asymmetry with factual uncertainty, which quite clearly should affect
our motivations.

\section{Welfare and Motivation}\label{welfareandmotivation}

Smith's insight, that there is something wrong about being motivated to
do what's good as such, generalises. There are plenty of other things
where we do and should care about their constituents, but we should not
(and typically do not) care about them as such. Welfare, for instance,
is like this.

It's plausible that deliberately undermining your own welfare, for no
gain of any kind to anyone, is irrational. Indeed, it may be the
paradigmatic form of irrationality. There is a radically Humean view
that says that welfare just consists of preference satisfaction, and
rationality is just a matter of means-end reasoning. If that's right
then what I just said is plausible is not only true, but almost
definitional of rationality. You don't have to be that radical a Humean,
or really any kind of Humean at all, to think there is a connection
between welfare and rationality. But if rationality is connected to
welfare, it is because it is connected to the constituents of welfare,
not to welfare as such. To see this, consider two examples, {Bruce} and
{Oberon}.

{Bruce} has thought a bit about philosophical views on welfare. In
particular, he has spent a lot of time arguing with a colleague who has
the G. E. Moore-inspired view that all that matters to welfare is the
appreciation of beauty, and personal love.\footnote{It would be a bit of
  a stretch to say this is Moore's own view, but you can see how a
  philosopher might get from Moore (\citeproc{ref-Moore1903}{1903}) to
  here. Appreciation of beauty is one of the constituents of welfare in
  the objective list theory of welfare put forward by John Finnis
  (\citeproc{ref-Finnis2011}{2011, 87--88}).} {Bruce} is pretty sure
this isn't right, but he isn't certain, since he has a lot of respect
for both his colleague and for Moore.

{Bruce} also doesn't care much for visual arts. He thinks that art is
something he should learn something about, both because of the value
other people get from art, and because of what you can learn about the
human condition from it. And while he's grateful for what he learned
while trying to inculcate an appreciation of art, and he has become a
much more reliable judge of what's beautiful and what isn't, the art
itself just leaves him cold. I suspect most of us are like {Bruce} about
some fields of art; there are genres that we feel have at best a kind of
sterile beauty. That's how {Bruce} feels about visual art in general.
This is unfortunate; we should feel sorry for {Bruce} that he doesn't
get as much pleasure from great art as we do. But it doesn't make
{Bruce} irrational, just unlucky.

Finally, we will suppose, {Bruce} is right to reject his colleague's
Moorean view on welfare. Appreciation of beauty isn't a constituent of
welfare. We'll for the sake of the example that welfare is a matter of
health, happiness and friendship. That is, a fairly restricted version
of an objective list theory of welfare is correct in {Bruce}'s world.
And for people who like art, appreciating art can produce a lot of
goods. Some of these are direct - art can make you happy. And some are
indirect - art can teach you things and that learning can contribute to
your welfare down the line. But if the art doesn't make you happy, as it
doesn't make {Bruce} happy, and one has learned all one can from a
genre, as {Bruce} has, there is no welfare gain from going to see art.
It doesn't in itself make you better off, in the way that {Bruce}'s
Moorean colleague thinks it does.

Now {Bruce} has to decide whether to spend some time at an art gallery
on his way home. He knows the art there will be beautiful, and he knows
it will leave him cold. There isn't any cost to going, but there isn't
anything else he'll gain by going either. Still, {Bruce} decides it
isn't worth the trouble, and stays out. He doesn't have anything else to
do, so he simply takes a slightly more direct walk home, which (as he
knows) makes at best a trifling gain to his welfare.

{Bruce} is perfectly rational to do this. He doesn't stand to gain
anything at all from going to the gallery. In fact, it would be a little
perverse, in a sense we'll return to, if he did go.

{Oberon} is also almost, but not completely certain, that health,
happiness and friendship are the sole constituents of
welfare.\footnote{Thanks to Julia Markovits for suggesting the central
  idea behind the Oberon example, and to Jill North for some comments
  that showed the need for it.} But he worries that this is undervaluing
art. He isn't so worried by the Moorean considerations of {Bruce}'s
colleagues. But he fears there is something to the Millian distinction
between higher and lower pleasures, and thinks that perhaps higher
pleasures contribute more to welfare than lower pleasures. Now most of
{Oberon}'s credence goes to alternative views. He is mostly confident
that people think higher pleasures are more valuable than lower
pleasures because they are confusing causation and constitution. It's
true that experiencing higher pleasures will, typically, be part of
experiences with more downstream benefits than experiences of lower
pleasures. But that's the only difference between the two that's
prudentially relevant. (Oberon also suspects the Millian view goes along
with a pernicious conservatism that values the pop culture of the past
over the pop culture of the present solely because it is past. But
that's not central to his theory of welfare.) And like {Bruce}, we'll
assume {Oberon} is right about the theory of welfare in the world of the
example.

Now {Oberon} can also go to the art gallery. And, unlike {Bruce}, he
will like doing so. But going to it will mean he has to miss a night
playing video games that he often goes to. {Oberon} knows he will enjoy
the video games more. And since playing video games with friends helps
strengthen friendships, he has a further reason to skip the gallery and
play games. Like {Bruce}, {Oberon} knows that there can be very good
consequences of seeing great art. But also like {Bruce}, {Oberon} knows
that none of that relevant here. Given {Oberon}'s background knowledge,
he will have fun at the exhibition, but won't learn anything
significant.

Still, {Oberon} worries that he should take a slightly smaller amount of
higher pleasure rather than a slightly larger amount of lower pleasure.
And he's worried about this even though he doesn't give a lot of
credence to the whole theory of higher and lower pleasures. But he
doesn't go to the gallery. He simply decides to act on the basis of his
preferred theory of welfare, and since that theory of welfare is
correct, he maximises his welfare by doing this.

Now distinguish the following two claims about welfare and rationality.
The first of these claims is plausibly true; the second is false.

\begin{itemize}
\tightlist
\item
  A person's welfare is such that it is irrational for them to do
  something that might undermine it for no compensating gain.
\item
  It is irrational for a person to do something that might undermine
  their welfare, whatever that turns out to be, for no compensating
  gain.
\end{itemize}

If welfare turns out to be health, happiness and learning, then the
first claim says that it is irrational to risk undermining one's health,
happiness and learning for no compensating gain. And that is correct.
But the second claim says that for any thing, if that thing might be
welfare, and an action might undermine it, it is irrational to perform
the action without a compensating gain. That's a much stronger, and a
much less plausible, claim. The examples of {Bruce} and of {Oberon} show
that it is false; they act rationally even though they do things that
might undermine what welfare turns out to be.

One caveat to all this. On some theories of welfare, it will not be
obvious that even the first claim is right. Consider a view (standard
among economists) that welfare is preference satisfaction. Now you might
think that even the first claim is ambiguous, between a claim that one's
preferences are such that it is irrational to undermine them (plausibly
true), and a claim that it is irrational to undermine one's preference
satisfaction. The latter claim is not true. If someone offers a person a
pill that will make her have preferences for things that are sure to
come out true (she wants the USA to stay being more populous than
Monaco, she wants to have fewer than ten limbs; etc.), it is rational to
refuse it. And that's true even though taking the pill will ensure that
she has a lot of satisfied preferences. What matters is that taking the
pill does not satisfy her actual preferences. If she prefers X to Y, she
should aim to bring about X. But she shouldn't aim to bring about a
state of having satisfied preferences; that could lead to rather
perverse behaviour, like taking this pill.

\section{Motivation, Virtues and Vices}\label{motivationvirtuesandvices}

So far in this chapter I have relied heavily on Michael Smith's
principle that a certain kind of motivation would be unreasonably
fetishistic. In this section I'm going to defend Smith's principle in
more detail. Since Smith's principle has been extensively discussed, I'm
going to spend some time on the existing literature. But one key point
of this section will be that I need a much weaker principle for my
broader conclusion than Smith needs for his. So even if the existing
objections to Smith are correct, and I will concede at least one has
some force against the strong principle Smith defends, they may not
affect my argument for externalism.

That Smith and I need different versions of the principle should not be
too surprising. As we saw in chapter 2, Smith defends some of the
internalist principles I'm arguing against. Since we have different
conclusions, one might hope we had different premises. The passage from
Smith I quoted about moral fetishism is in defence of his motivational
internalism. As I noted in chapter 1, the different theses called
internalism are dissociable, but they do have some affinities.
Motivational internalism is consistent with normative externalism, but
is in some tension with it. So again, it isn't surprising that I'll be
using Smith's idea in a slightly different way.

Let's start by setting out three theses that one might try to draw from
considerations starting from Smith's reflections.

\begin{description}
\tightlist
\item[Weak Motivation Principle (WMP)]
In equilibrium, it is permissible to not be intrinsically motivated by
maximally thin moral properties de dicto.
\item[Strong Motivation Principle (SMP)]
In most circumstances, it is impermissible to be at all intrinsically
motivated by moderately thin (or thinner) moral properties de dicto.
\item[Ideal Motivation Principle (IMP)]
In all circumstances, it is impermissible to be at all intrinsically
motivated by maximally thin moral properties de dicto.
\end{description}

The SMP and IMP are both stronger than the WMP, though neither is
stronger than the other. As I read him, Smith needs the IMP to get his
argument for motivational internalism to work. Since I'm not interested
in that, I'll set it aside from now on.

In the next section I'll discuss the WMP, with a focus on clarifying the
term `equilibrium'. The aim is to argue that it is is true, and that if
it is true, there is an asymmetry between factual and moral uncertainty.

After that, I'll discuss the SMP. I also think the SMP is true, and if
it is true, then there is a huge asymmetry between factual and moral
uncertainty. But I need to stress at this point that defending the SMP
isn't strictly necessary for the major argument of the chapter; the WMP
is enough to raise problems.

After that, I'll discuss a few examples that help clarify the boundaries
of the two principles, and which I think provide some argument for the
principles. But I'm discussing them at the end, because I don't really
want the case for or against the principles to rest on intuitions about
disputed examples like the ones I'll bring up.

The principles appeal to the notion of `intrinsic motivation', and it's
worth spending a few words on that. Just about everything I say here is
drawn from Arpaly and Schroeder
(\citeproc{ref-ArpalySchroeder2014}{2014, 6--14}), and they go into more
detail than I do about some of the important distinctions.

There is a distinction in everyday English between ends and means. And
to a first approximation, to desire something as an end is to desire it
intrinsically, and to desire it as a means it to desire it
instrumentally. But here we need to make a slightly finer distinction
than that.

Parents typically desire that their children be well educated. For some
people this will be an instrumental desire; they want their children to
be, say, very rich, and think that education is a means to wealth. But
for others it will be intrinsic; a good education is part of what is
good for their children.

Now consider the desire (again widely held among parents) that one's
children be well educated in arithmetic. How does this relate to the
general desire that they be well educated? It isn't exactly a means to
that end. It is part of what it is to be well educated. To desire that a
child be well educated, and to know what it is to be well educated, just
means that you desire that the child be well educated in arithmetic.
Call desires like this, ones which have a constitutive rather than
causal connection to intrinsic desires, \emph{realizer} desires.

The most obvious cases of realizer desires are when the intrinsic desire
is more general, and the realizer desire is more specific. But we can go
the other way around too. Consider again the perfectly normal parent who
wants their child to be well educated, to be healthy, to be happy, to
have lots of friendships, and generally wants all the things that make
up a good life for their child. That parent will want their child to
have a good life. This might be an intrinsic desire; maybe all those
other desires are realizers of it. It might even be an instrumental
desire, though this would be a little perverse. Or it might be a
realizer desire, and I think this is the most natural case. If one wants
the child to be happy, healthy, befriended, educated, etc, and one has a
sensible balance between those desires, then in virtue of all that, one
has the desire that the child have a good life. To desire all these
things just is to desire the child have a good life. It's a very
different way of desiring that the child have a good life than having
that desire instrumentally, as one might if one wanted the child to have
a good life solely so one would be rewarded in the afterlife. And it is
a somewhat different way of desiring that the child have a good life
than having that desire intrinsically. The difference shows up in two
ways. One concerns the order of explanation: does one want the child to
have a good life in virtue of wanting the child to be happy, healthy
etc, or is it the other way around? The other concerns how one's desires
for the child change when one's conception of the good life changes.

So the SMP and WMP concern themselves neither with instrumental desires
nor with realizer desires. A good person will typically desire that they
do the right thing, but they will desire that because the things they
desire are actually the right thing to do, and they will (typically)
know this. The principles say that the desires to do things that are
actually right could be, or in the case of the latter two principles
should be, explanatorily prior to the desire to do the right thing as
such.

\section{The Weak Motivation Principle
(WMP)}\label{theweakmotivationprinciplewmp}

\subsection{Equilibrium}\label{equilibrium}

The WMP is restricted to equilibrium states. This restriction is there
to deal with an important class of cases that Sigrún Svavarsdóttir
(\citeproc{ref-Svavarsdottir1999}{1999}) discusses.

\begin{quote}
{[}Smith argues that{]} the externalist account ``re-describe{[}s{]}
familiar psychological processes in ways that depart radically from the
descriptions that we would ordinarily give of them''
~(\citeproc{ref-Smith1996}{M. Smith 1996, 180}) \ldots{} Smith tells a
story of a friend (let's call him{Mike}) who has radically changed his
moral view over the years from act-utilitarianism to a view that
sanctions, in some instances, favoring family and friends, even when
this cannot be given utilitarian justification. Since {Mike} is a
moralist, his motivational dispositions have changed correspondingly
\ldots{} I would like to offer an illustration of what sort of
description externalists might give of {Mike}'s mental states before,
during, and after his two moral conversions. I venture the following
speculation: {Mike} has always had some inclination to favor family and
friends, but at one point he developed strong inhibitions against acting
on these inclinations. These inhibitions were largely the result of
being convinced that act-utilitarianism specifies the correct criterion
for moral rightness. Having a strong desire to do the right thing and a
rigid temperament, {Mike} quickly developed an avid interest in
maximizing total happiness in the world, taking the interest of each
person equally into account. In due time, his desire to maximize
happiness actually started to dominate all other desires to the point
that his friends thought of him as a utilitarian monster. But slowly
doubts started to emerge as a result of exposure to arguments against
utilitarianism. By and by {Mike}'s conviction eroded and in the end he
accepted a moral view according to which it is often right to be partial
to family and friends, even when doing so cannot be given a utilitarian
justification. At the same time, he came to see himself as a utilitarian
monster, ever ready to sacrifice the interests of friends and family for
the utilitarian project. Motivational dispositions he formerly took
pride in having developed now became distasteful to him. However, since
his desire to do the right thing has continued to be operative in his
psyche, these dispositions are slowly eroding and the inhibitions on his
inclinations to favor family and friends are undergoing radical change.
They are gradually falling in line with his view of when it is right to
give extra benefits to family and friends.
~(\citeproc{ref-Svavarsdottir1999}{Svavarsdóttir 1999, 208--10})
\end{quote}

Smith had argued that it is always a bad thing to be moved by the desire
to do the right thing, as such. Svavarsdóttir's reply here is that this
isn't bad at the very moment of major change in one's moral outlook.
(Since this was the very example that Smith used against the
motivational externalist, such examples were rather relevant to her
debate with Smith.) Adopting a moral theory wholeheartedly requires
adjusting one's motivations to align with it. But this need not be an
instantaneous process; it can take time and effort. And the motivation
to engage in this process of adjustment may come from a desire to do the
right thing.

The defender of the WMP can concede all this. What the defender says is
that {Mike}, in Svavarsdóttir's example, is not in equilibrium. What do
we mean here by being in equilibrium?

For current purposes, it means having fairly settled moral views, and
having had enough time and space since one's views became settled to
make suitable adjustments in the rest of one's mind. Equilibrium
requires the absence of felt pressure to change one's desires in light
of changes to one's moral outlook.

Here are two cases that I take to not be in equilibrium, in the sense
relevant to the WMP.

\begin{itemize}
\tightlist
\item
  Our hero faces a choice between competing values, and is torn about
  how to resolve them. She does not know which value is stronger, and
  she either lacks a clear disposition to resolve the tension in one
  particular way, or has such a disposition but does not trust it.
\item
  Our hero systematically does not do what they believe to be best, and
  is trying to change their attitudes and behaviour to conform to their
  beliefs about the good.
\end{itemize}

On the other hand, the following two cases are cases of equilibrium in
the relevant sense, albeit highly imperfect equilibrium.

\begin{itemize}
\tightlist
\item
  Our hero does not do what they believe to be best, but they have
  learned to live with this, perhaps feeling guilty about the gap
  between their thoughts and their deeds.
\item
  Our hero is disposed to act one way, but would change their
  disposition if the reasons for acting a different way, reasons they
  already possess, were made salient to them.
\end{itemize}

In all four cases, the person already possesses something like reasons
to change. But what makes for being in disequilibrium is the feeling
that things must and will change.

Our ultimate interest here is in cases where moral beliefs do or don't
line up with action, but we can come up with mundane, non-moral,
illustrations of each of them. Here's a (schematic) illustration of the
fourth kind of case.

I have a particular route I usually use going from B to C. I have a
different route I use going from A to C. That route goes via B, but it
does not take the usual route I use from B to C. This can't be optimal;
if there is a best way to get from B to C, I should use it in parts of
journeys as well as wholes. I could, nevertheless, be in equilibrium,
even if a small suggestion (hey, why don't you do something different
for the second part of the A-C route?) would push me to change my
behaviour. The point is that equilibrium in the relevant sense just
requires that the agent isn't trying to change, and isn't feeling
pressure to change, even if they possess perfectly good reasons to
change, and could easily be changed.

But in Svavarsdóttir's example, we do not have someone in equilibrium
even in this weak sense. {Mike} wants to change his dispositions to line
up with his moral theory, and he is making progress at this, but he
still isn't there. The WMP does not deny that in cases like this, it is
permissible to have goodness itself as a motivation.

\subsection{Why Engage in Moral
Reflection?}\label{whyengageinmoralreflection}

The following kind of consideration is sometimes advanced as a reason to
be motivated by goodness as such. Sometimes people engage in practically
directed moral reflection. That is, they think hard about what is the
right thing to do, and the intended result of that thinking is that they
do the thing they think is right. The most obvious analysis of what's
going on in these cases is that the people involved want to do the right
thing, and the point of engaging in reflection and acting on it is to
bring it about that they do the right thing. And at least in cases where
this leads to the thinker acting well, it seems this kind of moral
reflection is a very good thing to engage in.

In the next section I'm going to say a lot more about this kind of case,
because the SMP has to give a very different analysis of what is going
on in moral reflection. But the defender of the WMP does not need to say
much about these cases because they can simply endorse the `obvious
analysis'. The defender of the WMP can say that it is good, even
optimal, to engage in moral reflection, motivated by the desire to do
the right thing, when not in equilibrium.

The WMP is only making the following claim. When the storm is over and
the seas are flat, a good person may be motivated by the things that
make their actions right, not by the rightness itself. People who don't
know what to do, and are torn between competing values, could not be a
counterexample to such a principle.

\subsection{The WMP and Two Kinds of Motivation
Gaps}\label{thewmpandtwokindsofmotivationgaps}

But why should we believe the WMP? I think the best reason is the simple
intuition that Smith put forward: good people are motivated by things
around them in the world, not by abstract notions of virtue and
rightness. Another reason comes from reflection on fanatics like
{Robespierre} and {Saint-Just}. But not everyone accepts those reasons.
So let's look at a pair of cases that need explaining, and which the WMP
can explain.

The first case is a petty crook who won't cross certain lines. In
particular, while he'll steal anything from anyone, he won't engage in
violence. This isn't just because he is scared of getting punished for
violent acts. He has a kind of moral objection to violence. Perhaps
speaking loosely, let's say that he has no respect for property rights,
but a fitting and proper respect for rights involving bodily autonomy.

The thief's colleagues are planning a violent robbery. Feeling
uncomfortable with this turn of events, the thief informs the police,
who prevent the violence. This was a right and praiseworthy action by
the thief. But what could make it right and praiseworthy? Not that he
was trying to do the right thing - he's a thief who would have happily
gone along with a non-violent plan to steal the goods. What makes his
actions right and praiseworthy is that his motivation, prevention of
violence against (relative) innocents, was good. There is nothing
mysterious, and nothing wrong, with having this motivation without
having a general motivation to be moral.

The second case is a person who has a desire to do what's right, but no
underlying motivations. There are a couple of interesting variants of
this case. Nomy Arpaly (\citeproc{ref-Arpaly2003}{2003}) spends some
time on examples of `misguided conscience'; people who want to do the
right thing and are wrong about what it is. But we can also imagine
someone who does want to do the right thing, and is broadly correct
about what is right, but lacks any direct desire to do the thing that's
actually right. Let's think about such a case for a bit.

Our protagonist, call him {Rowly}, was brought up well enough that he
knows it is wrong to use violence to get things you want. And a desire
to avoid wrongdoing was inculcated at a young age. So when {Rowly} wants
a beer, but could only get one by punching someone, he declines to take
the opportunity. But he is upset by this; he has no desire to avoid
violence, or to avoid causing suffering, and wishes it was not wrong to
punch someone to get a beer.

There is something deeply wrong with {Rowly}. We can see this by
thinking about our interpretative practices. When someone says they did
something because ``it was the right thing to do'', we do not normally
interpret them as having no other-directed desires other than the desire
to avoid wrong-doing. We do not normally think of such a person as being
like {Rowly}. Someone who has to be taught what's right and wrong, and
who has this belief as the only barrier stopping serious wrongdoing, is
a deeply flawed human being. Even when people are too inarticulate to
say what desires they have beyond a desire to do the right thing, we
normally interpret this as inarticulateness, not a lack of respect for
others, nor a lack of desire that others not suffer. This
inarticulateness is not surprising; it's really hard to describe what
makes actions right or wrong. But not wishing well for others is
surprising; it's a serious character flaw.

So a desire to do the right thing is, in equilibrium, either unnecessary
or insufficient. If one wants to prevent suffering to others, and acts
on this, that's great, and it makes the desire to do the right thing
unnecessary. If one lacks a desire to prevent (causing) suffering, then
it is perhaps fortunate to have a desire to do the right thing, but that
is insufficient for virtue.

Since a desire to do the right thing seems so useless, at least in
equilibrium and in the presence of other good desires, it seems
permissible to not have such a desire. And that's all WMP says.

\subsection{Against Symmetry}\label{against-symmetry}

I've argued so far that the WMP is true. I'm now going to argue that,
assuming the WMP is true, there is an asymmetry between factual and
moral uncertainty. The role the WMP plays is to block one of three
possible routes out of a problem facing the defender of symmetry.

We know that having the probability of some factual proposition move
from 0\% to 5\% can (rationally) change behaviour. If I think the
probability of rain is 0\%, I don't have to check whether there is an
umbrella in the car. If I think it is 5\%, I will check the trunk to see
the umbrella is still there before heading out. If symmetry holds, then
changing the probability of a moral proposition from 0\% to 5\% should
also change behaviour. And it is hard to see how that could happen.

I'm going to mostly assume here a broadly Humean picture of motivation:
people do things that promote their desires assuming their beliefs are
true. The relevant contrast here is with the view that beliefs, or at
least belief-like states, can promote action without an underlying
desire. So the Humean thinks I pack the umbrella because I believe it
prevents me getting wet, and I have a desire to avoid getting wet, while
the anti-Human thinks I pack it because I believe it prevents me getting
wet, and I believe that it is good to avoid getting wet (or something
similar).

I'm assuming the Humean view partially because it is implicit in our
best formal models, partially because it seems intuitive, and partially
because there are technical problems with the anti-Human view. David
(\citeproc{ref-Lewis1988b}{Lewis 1988},
\citeproc{ref-Lewis1996a}{1996a}) showed that the view that beliefs
about the good played the role of values in expected value theory led to
problems with updating mental states. Recently Jeffrey Sanford Russell
and John Hawthorne (\citeproc{ref-RussellHawthorne2016}{2016}) have
shown that these results rely on much weaker premises, and apply much
more broadly, than a casual reading of Lewis's papers would suggest.
Anyone who thinks that belief-like states alone can drive action has to
adopt a rather implausible seeming picture of how beliefs are updated.

So I think rejecting belief-desire psychology is a high price to pay.
But let's note it is one way out of the argument I'm about to give. I'll
call it Option One for the symmetry defender.

If we don't take option one, then the symmetry defender must say which
desires interact with a change in credence to produce a change in
action. An obvious choice is to say that it is a desire to do the right
thing. But that's blocked by the WMP. If symmetry is true, then there
are times when a change in credence from 0\% to 5\% makes it compulsory
to change actions. And it is not compulsory to have a desire to do the
right thing. So that won't work. For the record, Option Two for the
symmetry defender is to reject the WMP, but that's also a bad move.

What the symmetry defender needs is to identify desires, other than
desires to do the right thing, that can generate the action. These will
be tricky to find. If someone thinks that it is 0\% likely that doing X
is wrong, then presumably it is completely rational to have no desire to
avoid X, or avoid what X involves. So it looks like this route won't
work either.

But that's too quick. All the symmetry defender needs is that after the
change in credence, there is a desire that drives the change in action.
Perhaps a change in credence could be correlated with a change in
desires that produced, via orthodox belief-desire reasoning, the outcome
the internalist wants.

But thinking there will always be such a change in desires is too much
to hope for. Indeed, in some cases having such a change would be bad, as
we can see using an example from Lara Buchak
(\citeproc{ref-Buchak2013}{2014}).

{Malai} has a good friend, who she has known since childhood, and she
values the friendship highly\footnote{I'm assuming throughout this
  paragraph that to value the friendship is a matter of having the right
  desires concerning the friend and the friendship, not having beliefs
  about the value of the friend or friendship.}. Then {Malai} learns
that someone committed a horrible crime, and there is some very weak
evidence that it was her friend. It's reasonable for {Malai} to have a
slightly greater than zero credence that it was her friend who committed
the crime, while not changing at all how much she values the friendship.
Indeed, if the evidence is strong enough to move her credence, but not
much more, it would be bad to have any other attitude. It's wrong to
devalue friendships because you get some almost certainly misleading
evidence about your friend. It's true the expected value of the
friendship goes down when the evidence comes in, and if the friendship
had only instrumental value, then that's a reason to devalue it. If
{Malai}'s only interest was in, say, getting to heaven, and she only
valued the friendship insofar as she thought it likely it was a
friendship with a good person, and that's the kind of thing that helps
get you to heaven, then she should reduce how much she values the
friendship. But most of us do not have quite that transactional an
attitudes towards our friends or our friendships. {Malai} should have
just as strong a desire to respect her friend and promote her friend's
interests, and to respect and promote the friendship, as she had before
getting the evidence. The evidence should not make her value the
friendship less, and that's because friendships are intrinsically
valuable, and how much something is intrinsically valued is not
proportionate to one's credence that it is intrinsically valuable.

The same goes at the other end of the valuing scale. If one thinks that,
for example, there is a 5\% chance that purity is intrinsically
valuable, it doesn't follow that one needs to (intrinsically) value
purity at all. Nor does it follow that one needs to be motivated, at
all, by considerations of purity.

I'll call Option Three the rejection of all that's been said in the last
three paragraphs, and the insistence that changes in moral credences
must occasion changes in desires. The examples involving {Malai} and
involving purity make this option very unattractive.

Ultimately, I think this is the deepest problem for the symmetry view.
Factual uncertainty changes our actions, and it does so rationally
because it changes which factual uncertainty changes the expected value
of different actions. For moral uncertainty to have the same effect,
either we have to have a false view of the role of desire in action
(Option One), or have to reject the WMP (Option Two), or have to adopt
an implausible and unattractive view of how desires change when
credences change (Option Three). None of these are correct, so symmetry
fails.

\section{The Strong Motivation Principle
(SMP)}\label{thestrongmotivationprinciplesmp}

It is easy to imagine very good characters who are not motivated by the
good as such; instead they are directly motivated by things that are
actually good. Indeed, if one's motivations are fully in line with the
good, it isn't clear what extra there is to be gained by also being
motivated to be good. At worst, this motivation seems like either a
distraction, or impermissibly self-centered. As Michael Smith puts it,
people with this motivation ``seem precious, overly concerned with the
moral standing of their acts when they should instead be concerned with
the features in virtue of which their acts have the moral standing that
they have.'' ~(\citeproc{ref-Smith1996}{M. Smith 1996, 183})

There is something disturbing about a person who does not find the fact
that a certain act is, say, a torture of a child to be sufficient
motivation to not do it, and needs the extra motivation that it would be
wrong. And the same goes for any other wrong act. Nothing is wrong as a
matter of brute fact; there is always some explanation for why it is
wrong. And that explanation always provides a motivation that would
prevent a good person from doing the action. Anyone who needs some
further motivation is in some way deficient.

That is the intuitive argument for the SMP. And it seems to me
compelling. But we can say more to motivate, and justify, the SMP. I'll
start with a discussion of a central objection to the SMP; that it
doesn't allow a special role for moral reflection. Then I'll discuss
another reason to support the SMP; it avoids a certain kind of danger,
one that we see manifest in history. And I'll close with a sketch of
what a proponent of the SMP thinks the good person is like.

\subsection{How to Explain Reflection}\label{howtoexplainreflection}

We typically think the following kind of activity is good. A person is
faced with a difficult moral question, or with a question that she
thought was easy, but which it turns out people she respects take a
different view on. She reflects on what morality requires in such a
situation. Upon coming to believe that morality requires of her
something different than her current practices, she changes her
behaviour to match with her new moral beliefs.

Such a character seems to pose a problem for the SMP. At first glance,
it seems like a motivation to do good, or at least avoid doing bad,
plays a central role. It is, apparently, the agent's change in her moral
beliefs that triggers a change in action. And a change in a belief about
what is X can only make a difference in action if X enters into one's
motivational set in the right way. Since our agent seems to be a good
person, it seems like good people should have thin moral
motivations.\footnote{In the previous section I noted that the proponent
  of the WMP has an easy explanation of the appeal of moral reflection,
  since the agent who is motivated to engage in moral reflection is not
  in equilibrium. Since the SMP is not restricted to agents in
  equilibrium states, such an appeal will not work in defence of it.}

My response to this kind of case will be very similar to what Arpaly and
Schroeder (\citeproc{ref-ArpalySchroeder2014}{2014, 185ff}) say about
moral reflection. When our agent tries to figure out what morality
requires of her, she won't start with highly abstract theorising. She
will start with her concrete commitments concerning how she should
engage with the world around her, and work out how those commitments
apply to difficult or contested cases. As Michael Smith puts the point

\begin{quote}
{[}N{]}ot only is it a platitude that rightness is a property that we
can discover to be instantiated by engaging in rational argument, it is
also a platitude that such arguments have a certain characteristic
coherentist form. ~(\citeproc{ref-Smith1994}{M. Smith 1994, 40})
\end{quote}

When good people use thin moral concepts in their reasoning, it is not
because they are aiming at the good as such, but because these concepts
are useful tools to use in sorting and clarifying their commitments, and
making sure that they promote and respect the things they actually care
about. We see this in other walks of life too. A competitor in a
sporting event may steer their strategy towards moves that maximise
expected returns. That's not because they care about expected returns;
they want to win. It is because using the concept of an expected return
is a good way to manage your thoughts when you want to think about how
to win. And, in practice, this is often a very good way to manage your
thoughts, so good strategists will use the concept. Similarly, it may
turn out to be useful to use the concepts of goodness and rightness when
trying to promote and respect the things that really matter, and so it
isn't a surprise that we see good people using them.

\subsection{Against Motivation by
Morality}\label{againstmotivationbymorality}

If moral concepts are useful tools for good people to use in promoting
and respecting good aims, then we should expect that, like all tools,
they have their limits. And indeed those limits are not hard to find.
Moral reasoning is a kind of equilibrium reasoning. And equilibrium
reasoning has clear strengths and weaknesses. There are cases when it is
essential. Trying to work out the effect of a natural disaster on the
market for widgets is practically impossible without doing at least some
equilibrium reasoning. But there are also cases when it can go badly
awry if not used extremely carefully, and in which very small errors in
the inputs can lead to very large errors in the outputs. This is
particularly the case when there are large feedback effects around. It
is hard to use equilibrium reasoning to work out the effect of a rise in
the price of labour, because changing the price of labour changes the
demand curve for all goods, and hence raising the demand for labour.
This isn't an insuperable modelling difficulty; but it means that it
will take more than the back of a napkin to work out even approximately
what will happen when the price of labour changes. Similarly, weather
forecasting using equilibrium models is possible, but has to be done
very carefully because very small errors in the initial inputs can push
the modeller to an equilibrium that is far removed from reality.

We see the same problems when reasoning about morality. The method of
reflective equilibrium, that characteristic coherentist form of
reasoning, is the best method we've got for working out what is right
and wrong. And it is very powerful. But it is an equilibrium method, and
we are in a territory where there are very strong feedback effects.
Whether one things X's treatment of Y is right or wrong will depend a
lot on other moral judgments. If X is imprisoning Y, then that is
probably very seriously wrong, unless Y has themselves done something
seriously wrong, and X has been empowered (preferably by a good set of
institutions) to deal with that kind of wrongdoing. Given there are this
many feedback effects, we should expect that whether moral reflection
leads people closer to, or away from, the truth is in part a function of
how close they start to the moral truth. And this is, I think, what we
see. To the extent moral reflection strikes us as a basically good
practice, it is because we imagine it being used by people who have
basically good motivations to start with. But in those cases moral
reasoning will help smooth out the rough edges; it won't correct major
faults.

And this suggests a problem with having morality itself as one of one's
motivations: it is dangerous. Unless one starts with basically good
motivations, thinking about the good and aiming for it could very well
make things worse; perhaps catastrophically worse. We should acknowledge
that in the hands of good people, moral reasoning can be a useful tool.
The person who doesn't use that tool will almost certainly fail to
optimise unless they have the sentiments of a saint. But someone whose
aims include respect for others and their rights, freeing people from
deprivation, promoting friendship and education, and being honest in
their dealings, will usually act fairly well, even if they never engage
in moral reflection. They may get the balance between these aims wrong
from time to time, sometimes in ways that moral reflection would
prevent. But they will typically avoid moral disaster. The person who
aims for the good, as such, is more likely to land in disaster. One of
the most dangerous things in the world is a wrongdoer with the courage
of their convictions. Thinking about how and why equilibrium analyses
can fail reinforces how dangerous this trap is.

But it's not just theory that tells us this is dangerous. The fanatic
who thinks the individual is irrelevant, who will sacrifice any number
of individuals to an idea, who will destroy villages in order to save
them, is a recurring character in history. In some cases they are tragic
figures; people who really did start out with praiseworthy aims but who
refused to compromise when it turned out that those aims couldn't be
realised without much suffering. And sometimes they are self-centred
jerks, who feel empty unless they are trying to steer the whole world to
their vision, whatever the costs. But what all of them teach us is that
aiming for the good, and just the good, can go terribly, horribly,
wrong.

\subsection{Back to Symmetry, and Moral
Uncertainty}\label{backtosymmetryandmoraluncertainty}

Let's turn away from these ideologues, and towards a positive picture of
what a good but flawed person should look like. Our hero will mostly
desire things that are actually valuable, and by and large desire them
to the extent that they are actually valuable. They will have a
well-functioning belief-desire psychology, so they will act so as to
promote or respect those valuable things they desire. They will, from
time to time, think about what is good and what is valuable, and form
largely true beliefs about the good and the valuable. But since we are
not supposing they are perfect, we will not assume these beliefs are
inevitably true. And these moral beliefs, even the true ones, will not
necessarily lead to much change in their action, because they don't
connect up with any desire in the right kind of way. It is normal for a
mismatch between desires and moral beliefs to lead to some unease, and
to think that it might be wise to reform one's beliefs or one's desires.
But depending on how deep the disagreement is, this reform program need
not be a particularly high priority. And when it is carried out, there
is no guarantee that the two will be brought into line by changing
desires, as opposed to by changing beliefs. What there is a guarantee of
is that if the moral beliefs conflict with other first order desires
that the hero has, such as a desire that mass killings not happen, those
other first order desires will play a powerful role in stopping the
moral beliefs from taking control.

It is a thought almost as old as European philosophy that there is a
good analogy between the well functioning polis and the well functioning
mind. Although it is much less old, it is by now a venerable idea that
the well functioning polis includes a separation of powers. And one of
the virtues of such a separation of powers is that it limits the damage
that can be done by a sudden swing in opinion among the powers that be.
This is not a panacea; some states are rotten to the core, and no amount
of institutional design will help. But it will prevent, or at least
moderate, certain kinds of wrong. To put it in late 18th Century terms,
the Alien and Sedition Acts were bad; the Reign of Terror was worse.
It's worth thinking about what checks and balances in moral psychology
would be, and more generally what a Madisonian moral psychology would
look like.

My best guess is that competing desires, such as desires to promote
welfare and alleviate suffering, and desires to keep promises and
respect rights, are the appropriate kinds of balance to each other. But
for current purposes it doesn't matter exactly how one ought implement
checks and balances, only that it is good that there are some. Because
if moral uncertainty should be treated the same way as factual
uncertainty, then there will be no checks and balances at all. When we
firmly believe that some fact is true, then the thing to do is simply
act as if that's true. We only hedge against the possibility that
something is false when there is a possibility that it is false; not
when we are certain that it is true. The symmetry view says that we
should do the same with moral (un)certainty. But if that's the case,
then there is no space for any check or balance on our moral views at
all; when we are certain of them, they are guiding. That is wrong, and
dangerous, so the symmetry view is also wrong.

Sometimes good people get the moral facts wrong. Perhaps they get bad
advice, or bad evidence. Perhaps they start just a little wrong and
equilibrium reasoning takes them to a place that is very wrong. When
that happens, they have mechanisms to stop them acting seriously
wrongly. I've been arguing that the moral mistakes shouldn't have any
direct effect on action, because they won't aim at the good. But as I've
noted already, I don't need anything that strong for the main argument
of this book. What I need is that there should be some other forces that
prevent action from lining up perfectly with moral belief when moral
belief is seriously mistaken. A natural suggestion is that desires for
things that are actually good can be that force. But even if that
suggestion is wrong, as long as there should be some other force, then
the symmetry claim fails.

\section{Motivation Through Thick and
Thin}\label{motivationthroughthickandthin}

In this section I'm going to run through some interesting test cases for
WMP and SMP. I have two aims here. First, I want to strengthen the case
for WMP. Second, I want to raise some cases that are useful intuition
checks for testing the plausibility of the SMP. I know from talking to
many people about the cases that I have different views about them to
most people. So while I think the cases are evidence for a fairly strong
version of the SMP, I know that they won't strike many people that way.
Still, I hope the cases are useful ones for thinking about what's at
issue in debating the SMP, and in particular thinking about how we
should interpret the phrase `moderately thin' in it if we want the
principle to be plausible. But let's start with a case purely about
maximally thin moral properties.

{Milan} is torn between two theories, and two actions. He gives some
credence to an agent-neutral form of consequentialism, and some credence
to a Kantian ethical theory. And he is torn between making a moderate
donation to charity, one of 3\% of his income, and a much larger
donation to charity, one of 30\% of his income (which is all he can
reasonably afford). He thinks that if the Kantian theory is true, then
he isn't obliged to give more than 3\%, and really doesn't want to give
any more than he has to give. But he knows that if the consequentialist
theory is true, then he is obliged to give (at least) the much larger
amount.

Now {Milan} thinks most of the arguments favour the Kantian theory. But
he has one remaining worry. He knows that the theory relies on having a
workable notion of what it is for different people to do the same thing.
And he worries that we don't have such a workable notion, for reasons
familiar from philosophy ~(\citeproc{ref-Goodman1955}{Goodman 1955}) and
game theory ~(\citeproc{ref-ChoKreps1987}{Cho and Kreps 1987}). So he
sets out to do some philosophical research, reading about work on the
notion of same action, and thinking about whether any such notion can
generate a version of the categorical imperative that agrees with its
intuitive content, and is not trivial. As often happens when working
through a philosophical problem, his views on which side is stronger
changes frequently. All the time, he has a web browser open getting
ready to hit send on a donation. And as he changes his mind on whether
the grue paradox ultimately defeats Kant's theory, he keeps adding and
deleting a final zero from the amount in the box saying how much he will
donate.

The WMP says that moral agents are not obliged to be like {Milan}. They
don't have to have their charitable actions be sensitive to their
beliefs about technical problems for Kantian ethics. It is, I think,
reasonable to have one's credence in the correctness of Kantian ethics
turn on beliefs about relatively technical problems. (For what it's
worth, I think the kind of problem Milan is worrying about is a genuine
problem for some kinds of Kantian theory, particularly those that think
the formality of the theory is an important virtue of it.) But an agent
who is being epistemically reasonable need not have their actions be
sensitive to their technical worries. And that's because the agent need
not be motivated by rightness as such.

If we change the case a little, we get an interesting test for SMP.
Unlike {Milan}, {Torin} is convinced that some kind of Kantian theory is
true. He also thinks there are technical problems with getting the
formulation of the categorical imperative right. But he also thinks,
sensibly enough, that these kind of technical problems are challenges,
not reasons to reject the theory. Still, the way to solve the challenge
will be to formulate different versions of the categorical imperative,
and test them. And these different versions will have different
consequences for which actions are required in certain circumstances. Is
it reasonable for {Torin} to be differently motivated when he changes
his views about which is quite the right formulation of the categorical
imperative? I don't feel that it is, but I can imagine that different
people have different views here.

A slightly more natural case seems even trickier to come to a firm
judgment about.{Florentina} is trying to figure out what to do in a case
where there are competing reasons in favour of two incompatible actions.
She feels rather torn, but can't settle on a particular choice. Then she
notices something: one of the choices, but not the other, is
incompatible with the categorical imperative. Is it reasonable for her
to be now more motivated to do the one that is consistent? I think this
is a somewhat strange mindset, but I suspect many will disagree. What
makes this case tricky is that we have to distinguish two situations
that are rather hard to keep apart. We aren't interested in the case
where {Florentina} sees that a choice is incompatible with the
categorical imperative, and by seeing this sees that she had been
overvaluing its strengths or undervaluing its weaknesses. Rather, we are
interested in the case where this fact about the categorical imperative
is itself a new motivation, alongside all the old motivations, to not do
a particular action. To the extent I can keep a clear grip on the case,
I think this is not a reasonable stance for {Florentina} to take. And
that's why I think that it is wrong to be motivated by an action's
compatibility or otherwise with the categorical imperative. What is
reasonable is to see incompatibility with the categorical imperative as
a reason for thinking there is something else wrong with the action,
perhaps something we haven't yet seen.

{Florentina}'s case is interesting even if you think that basing a whole
moral theory around the categorical imperative is implausible. You can
think that such a theory is surely wrong, but also think that Kant was
nevertheless on to something important. Whether one could rationally
will that everyone does X could be a factor in determining whether X is
right or wrong, even if it is a long way from being a central factor. My
default view in first-order ethics is a kind of muddy pluralism, which
acknowledges that many distinct moral traditions have important insights
into the nature of rightness and goodness, but which rejects any claim
to comprehensiveness these theories may make. {Florentina}'s case
suggests that even if you have such a kind of pluralist view, you still
could reject the view that conformity with the categorical imperative is
a good motivation.

Let's move to some cases that seem a little easier. (I owe the following
case to discussions with Scott Hershowitz.) {Mercurius} is a professor
in a large university. As with most professorial positions, {Mercurius}
has a fair amount of control over how much work he does. Some of his
colleagues do more for the department than anyone could reasonably
require, some do less than anyone could think was reasonable.
{Mercurius} is a reasonable department citizen, handling a perfectly
fair share of the workload, but only just as much as fairness requires.
Today, as sometimes happens, a request comes around from the chair for
volunteers for an unexpected task. {Mercurius} does not find the task
intrinsically interesting, but he knows that none of his colleagues will
feel any differently. He knows he will feel a bit bad for whoever ends
up shouldering the task, but will feel worse if it ends up being him.
Still, he is worried he hasn't done his fair share of the work. This is
wrong, as I said he has done enough, but it isn't an irrational belief
since it is such a close call. So he volunteers, being motivated by a
desire to do his fair share of the collective work.

This strikes me, and most people I've spoken about the case with, as a
perfectly reasonable motivation. There is nothing objectionably
fetishistic about being motivated to do one's share of a task one
values. And {Mercurius} does value the good functioning of his
department, and knows that it requires that the members collectively
take on some unpleasant tasks. So he acquires a motivation to take on
this particular unpleasant task.

It isn't easy to classify {Mercurius}'s desire using the terminology we
discussed in the previous section. He certainly doesn't have an
intrinsic desire to do the unpleasant task. And it isn't strictly
speaking an instrumental desire. We can imagine that {Mercurius} knows
that one of the usual suspects, the people who already do more than
their fair share, will take on this unpleasant task if no one else does.
And we don't have to imagine that {Mercurius} values their time more
than his. Nor is it quite right to say that {Mercurius}'s desire to do
this job is a realizer desire of his desire that the department runs
well. After all, if he had just taken on a similar task the previous
week, he would not desire to take on this one, although its relationship
to the good functioning of the department would be unchanged. The best
thing to say is that {Mercurius} has an intrinsic desire to do his fair
share of collective projects that he has joined, and given his (false)
beliefs about his past actions, this creates a realizer desire to do
this unpleasant task.

So that puts an upper bound on the extension of `moderately thin' in
SMP. There isn't anything wrong with having a desire to do one's fair
share, i.e., being motivated by properties like fairness. But on the
other hand, thinking about these `fair share' or `good teammate'
motivations helps explain some otherwise tricky cases. Indeed, my
suspicion is that most intuitive counterexamples to the WMP, or even the
SMP, can be helpfully thought of as cases where the agent has some
independent motivation for joining a team or a project, and then a
desire to be a good member of that team or project.

That's what I want to say about, for example, this case from Hallvard
Lillehammer (\citeproc{ref-Lillehammer1997}{1997}).

\begin{quote}
Consider next the case of the father who discovers that his son is a
murderer, and who knows that if he does not go to the police the boy
will get away with it, whereas if he does go to the police the boy will
go to the gas-chamber. The father judges that it is right to go to the
police, and does so. In this case it is not a platitude that a desire to
do what is right, where this is read \emph{de re}, is the mark of moral
goodness. If what moves the father to inform on his son is a standing
desire to do what is right, where this is read \emph{de dicto}, then
this could be as much of a saving grace as a moral failing. Why should
it be an a priori demand that someone should have an underived desire to
send his son to death? ~(\citeproc{ref-Lillehammer1997}{Lillehammer
1997, 192})
\end{quote}

A well functioning justice system is a very valuable thing to have.
There is nothing at all fetishistic about desiring that one's state have
such a system, and that it be maintained. Yet a well functioning justice
system requires collective action, and this generates issues about
whether one is doing one's fair share. As noted above, it can be
reasonable, and not at all inconsistent with WMP, to desire to do one's
fair share of a group project. Here the father who informs on his son
should be motivated not be a desire to do what's right as such, but by a
desire to do one's fair share of maintaining a good justice system.

If that's the right analysis of the case, then the father should be less
motivated the less difference his informing will make to whether the
state has a well functioning justice system. We see this already in
Lillehammer's version of the case; the injustice of capital punishment
is a reason for thinking that informing is not really a way of doing
one's share in maintaining a system of justice. But similarly, if the
family lives in a state where justice is very much the exception, it's
reasonable to be less motivated to inform on one's son. By analogy, if
tasks like the one {Mercurius} is considering routinely go undone, so
there is no good functioning to maintain, that's a reason to be less
motivated to take on this task.

Finally, consider a case about welfare, which has interesting lessons
for moral motivations. {Xue} believes that human welfare is entirely
constituted by health, happiness and friendship. And she is strongly
motivated to promote her own health, happiness and friendships, which is
natural enough given that belief. She is also motivated to help
others--she is no moral monster--but for now we're just interested in
her prudential reasoning.

{Xue} is told that bushwalking is good for your welfare, though she
isn't told whether it makes you healthier, happier or have better
friendships. But the source of this information is very reliable, so
{Xue} forms a desire to do more bushwalking. And this seems reasonable
enough. Is this a case where {Xue} is motivated by welfare as such, and
reasonably so?

I think it isn't. We have to distinguish three possible states.

\begin{enumerate}
\def\labelenumi{\arabic{enumi}.}
\tightlist
\item
  {Xue} is motivated to do things that have the property \emph{promote
  my health}, and is motivated to do things that have the property
  \emph{promote my happiness}, and is motivated to do things that have
  the property \emph{promote my friendships}.
\item
  {Xue} is motivated to do things that have the disjunctive property
  \emph{either promote my health, or promote my happiness, or promote my
  friendships}.
\item
  {Xue} is motivated to do things that have the property \emph{promote
  my welfare}.
\end{enumerate}

Assuming fairly minimal coherence, we can't tell the difference between
1 and 2 by just looking at {Xue}'s actions. Whether 1 or 2 were correct,
she would do the same things in almost all circumstances. Perhaps she
would say different things if the issue of whether she had disjunctive
or non-disjunctive motivations arose in conversation. But we need not
assume she has any interests in such a question, or even a pre-existing
disposition as to how she would answer it. But that doesn't mean that
there is no difference between the states. It is, in general, better
practice to attribute non-disjunctive attitudes to agents rather than
disjunctive ones ~(\citeproc{ref-Lewis1994b}{Lewis 1994};
\citeproc{ref-Weatherson2013Lewis}{Weatherson 2013}). So we should think
that we are in state 1 rather than state 2.

Similarly, given her beliefs about the nature of welfare, there won't be
much difference between the actions she is motivated to perform in state
1 and in state 3. So the fact that she responds to the information that
bushwalking is good for her welfare by developing a desire for
bushwalking is no evidence that we are in state 3. It might just be that
we are in state 1. Since there is independent intutive reason to think
it would be unreasonable for her to be in state 3, and her desire for
bushwalking in this case is reasonable, we should think that we're
actually in state 1. In general, we should prefer to attribute a
plurality of underlying motivations to agents, rather than disjunctive
motivations (as in state 2), or higher-order motivations (as in state
3).

\section{Moller's Example}\label{mollersexample}

I'll end this chapter by discussing an analogy D. Moller
(\citeproc{ref-Moller2011}{2011}) offers to motivate something like
symmetry.\footnote{Though note that Moller's own position is more
  moderate than the genuinely symmetric position; he thinks moral risk
  should play a role in reasoning, but not necessarily as strong as
  non--moral risk plays. In contrast, I'm advocating what he calls the
  ``extreme view, {[}that{]} we never need to take moral risk into
  account; it is always permissible to take moral risks.'' (435).}

\begin{quote}
Suppose {Frank} is the dean of a large medical school. Because his work
often involves ethical complications touching on issues like medical
experimentation and intellectual property, {Frank} has an ethical
advisory committee consisting of 10 members that helps him make
difficult decisions. One day {Frank} must decide whether to pursue
important research for the company in one of two ways: plan A and plan B
would both accomplish the necessary research, and seem to differ only to
the trivial extent that plan A would involve slightly less paperwork for
{Frank}. But then {Frank} consults the ethics committee, which tells him
that although everyone on the committee is absolutely convinced that
plan B is morally permissible, a significant minority - four of the
members - feel that plan A is a moral catastrophe. So the majority of
the committee thinks that the evidence favors believing that both plans
are permissible, but a significant minority is confident that one of the
plans would be a moral abomination, and there are practically no costs
attached to avoiding that possibility. Let's assume that {Frank} himself
cannot investigate the moral issues involved - doing so would involve
neglecting his other responsibilities. Let's also assume that {Frank}
generally trusts the members of the committee and has no special reason
to disregard certain members' opinions. Suppose that {Frank} decides to
go ahead with plan A, which creates slightly less paperwork for him,
even though, as he acknowledges, there seems to be a pretty significant
chance that enacting that plan will result in doing something very
deeply wrong and he has a virtually cost-free alternative.
~(\citeproc{ref-Moller2011}{Moller 2011, 436})
\end{quote}

The intuitions are supposed to be that this is a very bad thing for
{Frank} to do, and that this illustrates that there's something very
wrong with ignoring moral risk. But once we fill in the details of the
case, this can't be the right diagnosis.

The first thing to note is that there is something special about
decision making as the head of an organization. {Frank} doesn't just
have a duty to do what he thinks is best. He has a duty to reflect his
school's policies and viewpoints. A dean is not a dictator, not even an
enlightened, benevolent one. Not considering an advisory committee's
report is bad practice qua dean of the medical school, whether or not
{Frank}'s own decisions should be guided by moral risk.

We aren't told whether A or B are moral catastrophes. If B is a moral
catastrophe, and A isn't, there's something good about what {Frank}
does. Of course, he does it for the wrong reasons, and that might
undercut our admiration of him. But it does seem relevant to our
assessment to know whether A or B are actually permissible.

Assuming that B is actually permissible, the most natural reading of the
case is that {Frank} shouldn't do A. Or, at least, that he shouldn't do
A for the reason he does. But that doesn't mean he should be sensitive
to moral risk. Unless the four members who think that A is a moral
catastrophe are crazy, there must be some non-moral facts that make A
morally risky. If {Frank} doesn't know what those facts are, then he
isn't just making a decision under moral risk, he's making a decision
involving physical risk. And that's clearly a bad thing to do.

If {Frank} does know why the committee members think that the plan is a
moral catastrophe, his action is worse. Authorising a particular kind of
medical experimentation, when you know what effects it will have on
people, and where intelligent people think this is morally
impermissible, on the basis of convenience seems to show a striking lack
of character and judgment. Even if {Frank} doesn't have the time to work
through all the ins and outs of the case, it doesn't follow that it is
permissible to make decisions based on convenience, rather than based on
some (probably incomplete) assessment of the costs and benefits of the
program. (I'll expand on this point in section 6.1, when I discuss in
more detail what a normative externalist should say about hypocrisy.)

But having said all that, there's one variant of this case, perhaps
somewhat implausible, where it doesn't seem that {Frank} should listen
to the committee at all. Assume that both {Frank} and the committee have
a fairly thick understanding of what's involved in doing A and B. They
know which actions maximise expected utility, they know that which acts
are consistent with the categorical imperative, they know which people
affected by the acts would be entitled to complain about our
performance, or non-performance, of each act, they know which acts are
such that everyone could rationally will it to be true that everyone
believes those acts to be morally permitted, and so on. What they
disagree about is what rightness and wrongness consist in. What's common
knowledge between {Frank}, the majority and the minority is that both A
and B pass all these tests, with one exception: A is not consistent with
the categorical imperative. And the minority members of the committee
are committed Kantians, who think that they have a response to the best
recent anti-Kantian arguments.

It seems to me, intuitively, that this shouldn't matter one whit. I'm
not resting the arguments of this book on the intuitiveness of my views.
That's in part due to doubts about the usefulness of intuition, but more
due to how unintuitive normative externalism often is. But it is worth
noting how counterintuitive the opposing internalist view is in this
extreme case. A moral agent making a practical deliberation simply won't
care what the latest journal articles have been saying about the pros
and cons of Kantianism. It's possible (though personally I doubt it),
that learning of an action that it violates the categorical imperative
would be relevant to one's motivations. It's not possible that learning
that some people you admire think the categorical imperative is central
to morality could change one's motivation to perform, or not perform,
actions one knew all along violated the categorical imperative. At least
that's not possible without falling into the bad kind of moral fetishism
that Smith rightly decries.

So here's my general response to analogies of this kind, one that should
not be surprising given the previous sections. Assuming the minority
committee members are rational, either they know some facts about the
impacts of A and B that {Frank} is unaware of, or they hold some
philosophical theory that {Frank} doesn't. If it's the former, {Frank}
should take their concerns into account; but that's not because he
should be sensitive to moral risk, it's because he should be sensitive
to non-moral risk. If it's the latter, {Frank} shouldn't take their
concerns into account; that would be moral fetishism.

\chapter{A Dilemma for Internalism}\label{adilemmaforinternalism}

In the previous chapter I argued against the idea that we should treat
factual uncertainty and normative uncertainty symmetrically. In this
chapter I'll assume for the sake of the argument that the arguments of
the previous chapter are unsuccessful. The upshot of that would be that
we prefer theories that respect this symmetry. But this preference
cannot be absolute. As with everything else in philosophy, we have to
ask what the cost of satisfying this preference would be.

And in this chapter I'll argue that the costs are not worth paying.
There are three kinds of theories that are possible. There are the
externalist theories that I favour, which unqualifiedly approve of doing
the right thing. There are theories that adopt an unqualified version of
symmetry, treating all uncertainty the same way. I'll argue that such
theories are implausibly subjective. And there are theories that adopt a
half-hearted version of symmetry. I'll argue that these theories are
under-motivated. There is no theoretical advantage, I'll argue, by
incorporating a half-hearted symmetry principle. And there is much to be
lost by giving up the idea that one should do the right thing.

The argument I'm offering here is based on a very similar argument that
Miriam Schoenfield (\citeproc{ref-Schoenfield2014}{2015}) offers against
various kinds of normative internalism in epistemology. The idea our
arguments share is that the more subjective an internalism gets, the
less plausible its verdicts about cases are, while the more objective it
gets, the less well it is motivated by symmetry. Schoenfield primarily
is interested in developing a problem for some forms of normative
internalism in epistemology, but as we'll see, the same dilemma arises
for internalism in ethics.

\section{Six Forms of Internalism}\label{sixformsofinternalism}

The following schema can be converted into one of six internalist theses
by picking one of the 3 options on the left and one of the 2 options on
the right.

\begin{itemize}
\tightlist
\item
  Rightness/Praiseworthiness/Rationality is choosing an action with the
  highest credal/evidential expected goodness.
\end{itemize}

In every case `goodness' is meant to be interpreted de dicto and not de
re. That is, what has highest \emph{credal expected goodness} is a
function of the agent's beliefs (or more precisely her credences) in
various hypotheses about goodness. And what has highest \emph{evidential
expected goodness} is a function of her evidence about is and is not
good. If we interpret `goodness' de re, then the principle is consistent
with various forms of externalism; the de dicto interpretation is what
makes these internalist theses.

The six theses we generate that way are all very strong. They all offer
both necessary and sufficient conditions for an interesting concept. In
the next two chapters, we'll look at internalist views that only offer
necessary, or only offer sufficient, conditions for one of these. But
it's helpful to start with the strong views to see what constraints
there are on a viable internalism.

And I really want the six theses to be understood in an even stronger
way. They should be understood to be explanatory in a right-to-left
direction. So the view in question is not just that rightness (say) is
co-extensive with maximising credal expected value, but that some act is
right because it maximises credal expected goodness. This is, I think,
implicit in the internalists that I'll cite below. And it makes sense
given the idea that factual and normative uncertainty should be treated
the same way. Orthodox decision theory doesn't just say that rational
action is co-extensive with expected utility maximisation, it says that
some act is rational because no alternative has higher expected utility.

It will help to have some abbreviations for the six theories. I'll use
abbreviations for all five of the possible choices, and concatenate them
to get abbreviations for the whole theory. I'll use \textbf{Ri} for
rightness, \textbf{Pr} for praiseworthiness, \textbf{Ra} for
rationality, \textbf{C} for credal and \textbf{E} for evidential. So,
for instance, here are two theses one can express using this
terminology.

\begin{itemize}
\tightlist
\item
  \textbf{RiE} - Rightness is doing the action with the highest
  evidential expected goodness.
\item
  \textbf{PrC} - Praiseworthiness is doing the action with the highest
  credal expected goodness.
\end{itemize}

I've picked these because they are close to two theses endorsed by
Michael Zimmerman (\citeproc{ref-Zimmerman2008}{2008}). They aren't
exactly what he endorses; he leaves it open whether agents should be
using expected value calculations, or some nearby variant. But they are
nice, clean theories, and for that reason useful for theorising about.
And Zimmerman is hardly the only theorist to endorse something in the
vicinity. Andrew Sepielli (\citeproc{ref-Sepielli2009}{2009}) endorses
something like \textbf{RaC}, and Michael (\citeproc{ref-Smith2006}{M.
Smith 2006}, \citeproc{ref-Smith2009}{2009}) endorses something like
\textbf{PrC} and \textbf{RaC}.

The short version of this chapter is that the following three theses are
both true and deeply problematic for any kind of internalism.

\begin{enumerate}
\def\labelenumi{\arabic{enumi}.}
\tightlist
\item
  Both \textbf{RiC} and \textbf{PrC} theories make false claims about
  cases of what Nomy Arpaly (\citeproc{ref-Arpaly2003}{2003, 10}) calls
  ``inadvertent virtue'' and ``misguided conscience''.
\item
  The \textbf{E} theories are unmotivated; they are a compromise between
  two extreme theories, but they inherit the vices and not the virtues
  of those extremes.
\item
  The \textbf{Ra} theories posit an asymmetry between cases of factual
  and normative uncertainty that undermines another kind of symmetry the
  internalist takes to be intuitive.
\end{enumerate}

So none of the 6 theories are true. But more than that, the way in which
the 6 theories collectively fail suggests that the problem won't be
solved by adding epicycles, or weakening the theories to deal with hard
cases. There is no version of normative internalism in ethics that is
both motivated and plausible.

Sections 4.3--4.5 will deal with each of these theses in order. But
first I need to say something about the assumptions behind the chapter.
In particular, I need to say something about which possible theses are
being set aside until the end of the chapter. And saying something about
why we're setting various views aside will help position this chapter in
the rest of the book.

\section{Two Difficult Cases}\label{twodifficultcases}

There are four ways one could try to motivate normative internalism: by
appeal to cases, by appeal to principles about coherence, by appeal to
principles about guidance, and by appeal to symmetry. The first two are
notably absent in the literature on normative internalism in ethics,
though they will play a major role when we turn to epistemology.

There are, to be sure, plenty of arguments that talk about cases where
agents have specified credences in theories \emph{T}\textsubscript{1} or
\emph{T}\textsubscript{2}, but typically, these arguments will not
specify what \emph{T}\textsubscript{1} and \emph{T}\textsubscript{2}
are. See, for example, Gustafsson and Torpman
(\citeproc{ref-GustafssonTorpman2014}{2014}) and the papers cited
therein, for instances of this phenomena.\footnote{And, for what it's
  worth, in the papers I've seen so far citing Gustafsson and Torpman
  (\citeproc{ref-GustafssonTorpman2014}{2014}), though that may change.}
I don't think these are really arguments from cases, since nothing like
a case that we can have intuitions about is specified until we are told
at least roughly what \emph{T}\textsubscript{1} and
\emph{T}\textsubscript{2} are. If we were told that, for example,
\emph{T}\textsubscript{1} is {Saint-Just}'s theory that the world has
been empty since the Romans, and \emph{T}\textsubscript{2} is Ayn Rand's
version of egoism, we would have an example that we could have
intuitions about.\footnote{I'm being flippant in reducing {Saint-Just}'s
  moral and political theory to his aphorism about the Romans, but the
  details aren't really that important for what's going on here. See
  Williams (\citeproc{ref-Williams1995}{1995}) for a more serious
  treatment of {Saint-Just}'s worldview, and the earlier references on
  {Robespierre} for more details on {Saint-Just}'s biography.} Lockhart
(\citeproc{ref-Lockhart2000}{2000}) does include some case studies where
he assigns credences to particular moral theories - including Rand's but
not as it turns out {Saint-Just}'s. But this isn't part of his defence
of internalism, it's in the service of arguing from his internalist
theory to various claims in applied ethics.

Now it isn't a bad thing that internalists don't argue from cases to
theories. Indeed, there has been much criticism in the literature on
philosophical methodology recently of philosophers' reliance on cases.
(See Nagel (\citeproc{ref-Nagel2013}{2013}) for a discussion of, and
reply to, some of that criticism.) But it does reduce how much we have
to discuss here.

It will also be best to leave pure coherence based arguments until we
get to epistemology. There is something intuitive about the following
argument. It is incoherent to think that X is the unique right thing to
do, but instead decide to do Y. Incoherence, in this sense, is a kind of
irrationality. So rationality requires an internal connection between
moral beliefs and action. Rather than discuss that argument directly,
I'll just note that it is no more powerful than the following argument.
It is incoherent to think that \emph{p} is the unique conclusion
supported by a body of evidence, but nevertheless believe \emph{q} on
the basis of that evidence. Incoherence, in this sense, is a kind of
irrationality. So rationality requires an internal connection between
epistemological beliefs and, well, beliefs. That looks like a pretty
good argument at first glance too. Indeed, it is hard to see why we
could accept the argument about moral coherence that I opened the
paragraph with and not accept this argument about epistemological
coherence. Now I'll deal with this epistemological argument at great
length in part II of this book, and argue that it doesn't work, so I'll
largely set the moral version of that argument aside for now.

But there is one version of the coherence argument that I want to more
explicitly set aside. Consider a theory that accepts all three of the
following principles. (See Markovits
(\citeproc{ref-Markovits2014}{2014}) for a sophisticated version of the
kind of theory I have in mind, but note that I'm simplifying a lot here
to make a methodological point.)

\begin{itemize}
\tightlist
\item
  One should always do the right thing, and one should do the right
  thing in virtue of the right-making features of those actions, not in
  virtue of one's moral beliefs.
\item
  Rationality requires that ones moral beliefs include all and only the
  true moral propositions.
\item
  Immoral action is irrational.
\end{itemize}

Such a theory might agree with something like \textbf{RaC}. At the very
least, it will say that rationality requires doing the action with the
highest credal expected goodness. But that's because rationality
requires both that one give credence 1 to the true claim about which
action is good to perform, and rationality requires performing the
action that is good to perform.

Is this theory internalist or externalist? I don't think it helps to try
to classify it. Just note that I'm setting it aside. More generally, I'm
setting aside theories that make moral omniscience the standard for
moral rationality. Rational people can make mistakes; at the very least
they can fail to believe some truths. That's true in science, it's true
in everyday life, and it's true, I'm assuming, in ethics and
epistemology.\footnote{This isn't an argument for this assumption, but
  perhaps a quick explanation for why the assumption seems plausible to
  me is in order. All arguments I've seen for the view that rationality
  requires moral omniscience have some kind of enkratic principle as a
  premise. And for reasons I will go over in Part II of the book, I
  don't think these enkratic principles are very plausible. Claire Field
  (\citeproc{ref-Field2017}{forthcoming}) has a very good critical
  discussion of the arguments for this assumption.}

I discussed the guidance arguments earlier in the book, and argued that
they only supported an implausibly subjectivist version of internalism.
Not coincidentally, that's going to be similar to what I say in this
chapter about the symmetry argument But you might think there is another
way to block the argument from symmetry to internalism. This chapter and
the last have been focussed on the following argument.

\begin{enumerate}
\def\labelenumi{\arabic{enumi}.}
\tightlist
\item
  Expected utility theory provides the correct treatment of decision
  making under factual uncertainty.
\item
  Factual uncertainty and normative uncertainty should be treated
  symmetrically.
\item
  So some kind of internalist theory provides the correct treatment of
  decision making under moral uncertainty.
\end{enumerate}

That's not valid, because a lot of the terms in it are rather vague. But
I'm not going to dispute the inference here; if the premises are both
true, then they will support some kind of theory that I want to reject.

I'm also going to assume, for now, that premise 1 of this argument is
basically correct. And this is a substantive assumption. There is one
very important moral theory that rejects premise 1 (under one important
disambiguation of it). That's the traditional consequentialist theory
that says that the moral status of an action is a function of the
consequences it actually has ~(\citeproc{ref-Sidgwick1874}{Sidgwick
1874}; \citeproc{ref-Smart1961}{Smart 1961}). I'm simply going to assume
that's false for now, and come back to it at the end of the chapter.
Note that I'm not assuming that modern consequentialist theories, like
the decision-theoretic consequentialism {Frank} Jackson
(\citeproc{ref-Jackson1991}{1991}) defends, are false. I'm just setting
aside views on which factual uncertainty is irrelevant to the moral
status of an action.

So to recap, we're making two large presuppositions at this stage of the
dialectic. The defence of these presuppositions is largely in earlier
chapters, but as noted above, some of it is to come. The presuppositions
are:

\begin{enumerate}
\def\labelenumi{\arabic{enumi}.}
\tightlist
\item
  The best argument for normative internalism is an argument from the
  symmetrical treatment of factual and normative uncertainty. This is an
  argument for a kind of internalism because (contra traditional
  consequentialism) factual uncertainty matters to the moral and
  rational status of actions.
\item
  Neither rationality nor morality requires moral omniscience, so if the
  morality or rationality of an action is sensitive to the actor's
  actual credence in moral propositions, or to the rational credence in
  those propositions given their evidence, then in some sense what they
  should do will differ from what the true (but unknown) moral or
  epistemological theory says they should do.
\end{enumerate}

\section{Inadvertent Virtue and Misguided
Conscience}\label{inadvertentvirtueandmisguidedconscience}

The next three sections will defend the three principles from the end of
4.1. So our aim here is to defend:

\begin{itemize}
\tightlist
\item
  Both \textbf{RiC} and \textbf{PrC} theories make false claims about
  cases of what Nomy Arpaly (\citeproc{ref-Arpaly2003}{2003, 10}) calls
  ``inadvertent virtue'' and ``misguided conscience''.
\end{itemize}

Arpaly's paradigm of inadvertent virtue is {Huck} Finn, so we'll start
with her description of his story.

\begin{quote}
At a key point in the story, {Huck}leberry's best judgment tells him
that he should not help Jim escape slavery but rather turn him in at the
first available opportunity. Yet when a golden opportunity comes to turn
Jim in, {Huck}leberry discovers that he just cannot do it and fails to
do what he takes to be his duty, deciding as a result that, what with
morality being so hard, he will just remain a bad boy (he does not,
therefore, reform his views: at the time of his narrative, he still
believes that the moral thing to do would have been to turn Jim in). If
one only takes actions in accordance with deliberation, or the faculty
of Reason or ego-syntonic actions {[}\ldots{]}, to be actions for which
the agent can be morally praised, {Huck}leberry's action is reduced to
the status accorded by Kant to acting on ``mere inclination'' or by
Aristotle to acting on ``natural virtue.'' He is no more morally
praiseworthy for helping Jim than a good seeing-eye dog is praiseworthy
for its helpful deeds. This is not, however, how Twain sees his
character. Twain takes {Huck}leberry to be an ignorant boy whose decency
and virtue exceed those of many older and more educated men, and his
failure to turn Jim in is portrayed not as a mere lucky accident of
temperament, a case of fortunate squeamishness, but as something quite
different. {Huck}leberry's long acquaintance with Jim makes him
gradually realize that Jim is a full-fledged human being, a realization
that expresses itself, for example, in {Huck}leberry's finding himself,
for the first time in his life, apologizing respectfully to a black man.
While {Huck}leberry does not conceptualize his realization, it is this
awareness of Jim's humanity that causes him to become emotionally
incapable of turning Jim in. To the extent that this is {Huck}leberry's
motive, Twain obviously sees him as praiseworthy in a way that he
wouldn't be if he were merely acting out of some atavistic mechanism or
if he were reluctant to turn Jim in out of a desire to spite Miss
Watson, Jim's owner. {Huck}leberry Finn is not treated by his creator as
if he were acting for a nonmoral motive, but rather as if he were acting
for a moral motive--\emph{without knowing} that it is a moral motive.
(9--10)
\end{quote}

Here are a few basic truths about {Huck}leberry's actions in helping Jim
remain free.

\begin{enumerate}
\def\labelenumi{\arabic{enumi}.}
\tightlist
\item
  {Huck}leberry does the right thing.
\item
  {Huck}leberry does not do the wrong thing.
\item
  {Huck}leberry is praiseworthy for helping Jim remain free.
\item
  {Huck}leberry is not blameworthy for helping Jim remain free.
\end{enumerate}

If a philosophical theory rejects any of those four claims, it is wrong.
Here are two more claims that I think are true, though I'm not going to
rest any argumentative weight on them, since I suspect they will strike
most readers as, at best, controversial.

\begin{enumerate}
\def\labelenumi{\arabic{enumi}.}
\setcounter{enumi}{4}
\tightlist
\item
  Huckleberry's upbringing, and in particular the testimony from his
  parents, friends and teachers, provides strong evidence for the false
  moral theory that he in fact believes, namely that morality requires
  him to turn Jim in, and Huckleberry's relationship with Jim does not
  provide strong enough counter-evidence to make that belief irrational.
\item
  Huckleberry is rational, and not irrational, to help Jim to remain
  free.
\end{enumerate}

If all of 1 through 6 are true, then all 6 of the theories we started
with are false. Turning in Jim maximises both credal and evidential
expected goodness. But helping Jim is right (1), praiseworthy (3) and
rational (6). So all six theories are false.

The argument of the last paragraph relies heavily on 5 and 6 though. If
5 is false, then the case does not show any of the \textbf{E} forms to
be false. And if 6 is false, the story does not show either of the
\textbf{Ra} versions to be false. So without relying on 5 and 6, and I'm
not going to rely on them, we can't argue against all forms of normative
internalism using just {Huck}leberry Finn. But we can argue against some
forms. Consider first \textbf{RiC} and \textbf{PrC}. The {Huck}leberry
Finn case shows these to be simply false. {Huck} does the right thing,
and is praiseworthy, although he clearly minimises credal expected
goodness (at least relative to the live choices).

{Huck}leberry is a case of what Arpaly calls `inadvertent virtue'. We
can also put pressure on internalism by looking at cases of what she
calls `misguided conscience'. I'll use some cases described by Elizabeth
E. Harman (\citeproc{ref-Harman2011a}{2011}), focussing on her examples
that involve currently contested moral issues. (As Harman notes, if you
don't find these examples forceful because you don't agree with the
underlying moral theory, you could easily `reverse' the cases to make a
similar point.)

\begin{quote}
Consider someone who believes abortion is wrong and who yells at women
outside abortion clinics. It is wrong to yell at women outside abortion
clinics: these women are already having a hard time and making their
difficult decision more psychologically painful is wrong. But this
person acts in a way that would be permissible if her moral views were
true. Another example is someone who believes abortion is wrong and who
kills an abortion doctor, in a part of the country where there is good
reason to think that this doctor's death will reduce the number of
abortions. This person believes that he ought to kill abortion doctors
if doing so would reduce the number of abortions that would be
performed. A third example is someone who believes homosexuality is
wrong who organizes a campaign against the legalization of gay marriage.
He believes he is doing something morally good in organizing the
campaign; in fact, in working to further oppression, he is acting
wrongly. (458)
\end{quote}

As it stands, the various versions of the \textbf{C} theories say that
these three actors are either acting rightly, or praiseworthily, or
rationally. And again, the first two of these evaluations are wrong, at
least if abortion and gay marriage really are morally permissible. Note
that I'm not here claiming that the false moral beliefs involved are
normatively irrelevant; it's consistent with what I say here that the
characters in Harman's stories are blameless without being praiseworthy.
I'm going to argue against that view in the next chapter, but I'll set
it aside for now. What we need to focus on first is whether their
mistaken moral belief suffices for their action being praiseworthy, and
it does not.

\section{Ethics and Epistemology}\label{ethicsandepistemology}

In the previous section we looked at arguments against \textbf{C}
theories; theories that linked normative statuses to the agent's own
credences. In this section we'll look at \textbf{E} theories, with the
aim being to defend this principle.

\begin{itemize}
\tightlist
\item
  The \textbf{E} theories are unmotivated; they are a compromise between
  two extreme theories, but they inherit the vices and not the virtues
  of those extremes.
\end{itemize}

I'm going to start by making the case against this, that the \textbf{E}
theories are in fact well motivated. That's partially because I think
most internalists in philosophy prefer these to the \textbf{C} theories.
And it's partially because the \textbf{E} theories are an interesting
attempt to solve a hard problem. But the problem they are trying to
solve is really not solvable; and the attempt just inherits the vices of
the positions it is trying to avoid without any offsetting virtues.

The debate will get very theoretical very quickly, so to try to keep
things a little grounded I'll start with a fairly familiar kind of case.
{Zaina} has been threatened by a group of determined pranksters. She is
told, convincingly, that unless she pranks one innocent person, the
group will prank that person and one hundred other people this week. But
if she does perform the prank, the group will perform no pranks this
week. And she knows that whatever happens this week will have no effect
on how many pranks the group performs after this week. The prank in
question is unpleasant for its victim; {Zaina} would not like to be the
victim of such a prank. And while it might be mildly amusing for
onlookers and perpetrators, {Zaina} knows that each performance of the
prank makes the world worse.

What {Zaina} doesn't know is what the correct moral theory is. She has
studied some philosophy as an undergraduate, and gives some credence to
a consequentialist moral theory, according to which she should perform
the prank so as to minimise prank performances, and the rest of her
credence to a deontological theory, according to which it would be wrong
of her to directly harm an innocent victim of her prank. And this is,
we'll assume, a perfectly reasonable reaction to the moral evidence she
has been presented. (If you don't believe this is possible, substitute
some other theories in which you do think a thoughtful undergraduate
could be unsure between after some kind of introductory philosophy
course, and which recommend different actions in a particular puzzle
case. It is a little unrealistic to think that Zaina could know that the
truth is in one of these two places, and that will matter a bit below.)

{Zaina} doesn't know what she should do. But she also doesn't know what
action will maximise expected goodness. She knows that according to the
consequentialist theory, performing the prank maximises goodness. She
knows that according to the deontological theory, not performing the
prank maximises goodness. But she needs to know a lot more than that to
work out what maximises expected goodness. She needs to fill in two
variables in the following table.

\begin{longtable}[]{@{}lcc@{}}
\toprule\noalign{}
& Consequentialist (Pr = \emph{p}) & Deontologist (Pr = 1-\emph{p}) \\
\midrule\noalign{}
\endhead
\bottomrule\noalign{}
\endlastfoot
Perform Prank & -1 & -\emph{v} \\
Don't Perform Prank & -101 & 0 \\
\end{longtable}

The expected value of not pranking is -101\emph{p}. The expected value
of pranking is -\emph{p}~-~\emph{v}(1-\emph{p}). Figuring out which of
these is larger requires solving two hard problems: exactly how likely
is it that the consequentialist theory is true, and how do you put the
violation of a deontological duty on the same scale as the difference
between better and worse consequences.

The latter problem is very hard, and we'll come back to it in chapter 6.
Ted Lockhart (\citeproc{ref-Lockhart2000}{2000}) had a nice idea on how
to make progress on it, but Andrew Sepielli
(\citeproc{ref-Sepielli2009}{2009}) shows that it doesn't work. Brian
Hedden (\citeproc{ref-Hedden2016}{2016b}) uses the difficulty of this
problem to argue against internalist theories generally. William
MacAskill (\citeproc{ref-MacAskill2016}{2016}) thinks that the problem
is hard enough that we should respond by not trying to maximise the
expected value of some random variable in cases of moral uncertainty,
but instead using tools from social choice theory such as voting
methods. I'm very sympathetic to MacAskill's approach, insofar as I
think that conditional on us wanting an internalist theory of action
under moral uncertainty, I think using tools from social choice theory
is more promising than trying to find a value for \emph{v}. But if we go
down this route, we've given up the symmetry between moral and factual
uncertainty, and as I argued at the start of this chapter, without that
symmetry it is very hard to motivate internalism. So I'll assume that
{Zaina} has to find out, or at least be sensitive to, the value of
\emph{v}.

Now the normative externalist has an easy thing to say about {Zaina}'s
case. If consequentialism is the true moral theory, then she should
perform to prank to spare the other 100. If the deontological theory is
true, then she should not perform the prank, since she should not commit
such an immoral act. And that's all there is to say about the case. It
might help {Zaina} to know what the right moral theory is, but it isn't
necessary. If she performs the prank out of care for the welfare of the
100 people she is saving then, if consequentialism is true, she does the
right thing for the right reasons. If she declines to perform the prank
because it would disrespect the victim of the prank, then, if the
deontological theory is true, she does the right thing for the right
reason. Neither of the last two sentences require that {Zaina} know that
she is doing the right thing or that her reasons are right - what's
needed at most is conformity between her motivations and the
right-making features of actions.

But the internalist tends to find this answer unsatisfactory for two
reasons. The reasons tend to pull in opposite directions. The first
reason is that it is in one respect too demanding. While it does not
require {Zaina} to know something she has insufficient reason to
believe, namely what the right thing to do here is, it does require her
to be sensitive to some fact she is unaware of. That fact is, simply,
what the right thing to do in this situation is. The second reason is
that it is in a different respect too weak. {Zaina} could be massively
incoherent, and the externalist would find nothing wrong with her.
Indeed, my preferred version of externalism says {Zaina} should be
incoherent in some respects. It says that if true moral theory says that
some factor is of no significance, then {Zaina} should give it no weight
in her calculation, even though she thinks, and \emph{should think} that
there is a decent probability this factor is very morally
important.\footnote{I try to offset the oddness of this result by
  adopting an extremely pluralist first-order moral theory, so very few
  things that are plausibly of moral significance turn out to be
  irrelevant. But I don't want my defence of normative externalism to
  turn on this pluralism.} And many philosophers seem to find it
extremely implausible that {Zaina} could be right, and rational, and
praiseworthy, all without qualification, while there is a serious
mismatch between her moral beliefs and her actions.

So let's try the opposite extreme, one suggested by our discussion of
Descartes in chapter one. (Though what we start with will not be the
view Descartes actually endorses.) What matters for morality is match
between credences and action. So as long as {Zaina} does what she thinks
is best, or perhaps what maximises expected goodness, she does the right
thing. In that case she acts rightly, is praiseworthy, and is rational.
While she needs to find values for \emph{p} and \emph{v}, she gets them
by introspecting her beliefs, not by hard looking into the external
world.\footnote{I'm setting aside, apart from in this footnote, two
  problems with this view. As Eric Schwitzgebel
  (\citeproc{ref-Schwitzgebel2008}{2008}) notes, we are often mistaken
  as what we believe. And thinking that {Zaina}`s beliefs settle the
  value of \emph{v} requires adopting a 'desire as belief' view that
  faces various technical problems ~(\citeproc{ref-Lewis1988b}{Lewis
  1988}, \citeproc{ref-Lewis1996b}{1996b};
  \citeproc{ref-RussellHawthorne2016}{J. S. Russell and Hawthorne
  2016}).} And the hero of this internalist Cartesian story is bound to
be coherent, at least in the sense of having their views about what to
do match up with the actions that are within their control.

But such a theory says some odd things about a different character,
{Antoine}, who was threatened by the pranksters just last week.
{Antoine} believes, rightly, that such a threat is a terrible affront to
his dignity as a free person. He further believes, wrongly, that the
only appropriate response to such an affront is to kill everyone who
makes the threat. Fortuitously, {Antoine} is as bad at figuring out how
to kill as he is at figuring out who to kill, so no one gets hurt. But
we shouldn't let this lucky break obscure the fact that what {Antoine}
does is seriously wrong. And yet, the Cartesian internalist has a
problem with this. {Antoine} does exactly what his conscience tells him
to do. He is as resolute a person as one could look for. And he is a
villain; someone to be loathed and avoided, not admired.

So there is an easy and natural way out of the problem {Antoine} poses.
Indeed, it is one that is entailed by the rest of what Descartes says in
philosophy. {Antoine} does believe that killing the pranksters is moral,
but this belief is extremely irrational. What he should be guided by is
not his actual worldview, which is abhorrent, but the moral evidence
that he has. And while we can't say for sure how that evidence would
resolve a problem like the pranksters, we know it would not endorse a
massacre.

And this is, I think, a natural motivation for the \textbf{E} theories.
There is something intuitively appealing about trying to find a middle
way between the externalist view that requires people to do the right
thing without saying what that is, and the kind of subjectivism that has
nothing plausible to say about {Antoine}.

But there are still problems. Indeed, the problems with this kind of
worldview were pointed out by Princess Elizabeth in her correspondence
with Descartes. The core problem is that this `way out' requires
treating ethics and epistemology very differently, and there is no
justification for this differential treatment.

{Antoine} doesn't just believe that the moral thing to do is to kill the
pranksters. He believes that his evidence supports that conclusion. If
we are to say that what he does is wrong in some respect, then we have
to insist that this does not matter. What he should believe is a
function of what the evidence actually supports, not what he thinks it
supports.

But now a version of the demandingness objection returns with a
vengeance. The internalist thought was that it really unfair to require
{Zaina} to be sensitive to a fact that she does not know - namely
whether a consequentialist or deontological moral theory is correct. The
proposed response now requires that she be sensitive to two facts that
she does not know, namely which values of \emph{p} and \emph{v} are best
supported by her evidence. And worse than that, we have replaced one
yes-no question with two quantitative questions. This does not feel like
progress.

I've skated over a division between ways the internalist might require
that {Zaina} be sensitive to her evidence. First, they might require
that she have the beliefs that are best supported by the evidence, and
then act as her beliefs maintain. This is the version of the view that
requires a fairly strong form of normative externalism in epistemology.
There is no guarantee that {Zaina} knows, or even could know, what the
rational credence in consequentialism given her evidence actually is. So
requiring her to have credences supported by her evidence is requiring
her to follow a norm that she does not, and could not, know. And
avoiding that was supposed to be a big payoff for internalism. So this
way of defending the \textbf{E} theories seems unmotivated.

But alternatively, the internalist here might just say that {Zaina} has
to be sensitive to her evidence, not that she must know what her
evidence supports. As far as it goes, the externalist agrees with this
point. The externalist view is that the following three things are in
principle separable. (In every case, read `believe' as meaning `fully or
partially believe'; this covers appropriate credences as well as
appropriate full beliefs.)

\begin{enumerate}
\def\labelenumi{\arabic{enumi}.}
\tightlist
\item
  What {Zaina} should do.
\item
  What {Zaina} should believe about what she should do.
\item
  What {Zaina} should believe about what she should believe about what
  she should do.
\end{enumerate}

The \textbf{E} theories say that while 2 and 3 might come apart, there
is a tight connection between 1 and 2. There's nothing incoherent about
that. But it is rather hard to motivate. The following situation is
possible. (And thinking through this situation is helpful for getting
clear on just what the \textbf{E} theories are saying.)

The true moral theory is deontological, so true morality requires that
{Zaina} not perform the prank. The rational values for \emph{p} and
\emph{v} given {Zaina}'s evidence are 0.2 and and 15. That is, violating
a deontological norm is (according to Zaina's evidence) as bad as the
consequentialist thinks letting 15 people be pranked is, but
consequentialism is fairly unlikely to be true. So given {Zaina}'s
evidence, it maximises expected goodness to perform the prank. But
{Zaina}'s credal distribution over possible values of \emph{p} and
\emph{v} is centred a little off those true values, centred on 0.15 and
20. And while this isn't right, her margin of error in assessing what
her evidence supports concerning \emph{p} and \emph{v} is great enough
that she can't know these are the wrong values. So given her credences,
she thinks her evidence supports not performing the prank.

Given all that, what is the sense in which she should perform the prank,
in which it would be more rational, or moral, or praiseworthy, to
perform the prank? It's true that if she were better in some respect -
in respect of having credences that actually tracked her evidence - then
she would perform the prank. But if she were fully moral, she would not
perform the prank. And if she maximised expected goodness given her
perspective, she would perform the prank only if she were a little
better epistemically, without being better morally. But what
philosophical significance could that counterfactual have?

While the case is artificial, it fits a natural enough pattern. Someone
makes a pair of mistakes. These are mistakes - they are irrational
things to do - but they are perfectly understandable since the task in
question is hard. Happily, the mistakes offset, so the person ends up
doing something they would do if they made neither mistake. But there is
some other option that the person would take if they fixed one
particular mistake. Does that fact mean that the `other option' is
something the person should do, or morally ought to do, or is
praiseworthy for doing, or is rational for doing? It doesn't seem like
it; it seems rather that all we can say about that option is this rather
technical claim that it has only one of two salient vices.

And that's the pattern for the \textbf{E} theories in general. They are
unhappy half-way houses. If we want people to follow standards that they
cannot know in full detail, those standards may as well be the standards
of true morality. If we don't want to require this of people, then what
their evidence supports is not determinative of what we can demand of
them. Just what the evidence supports is sometimes hidden too. But if we
start being too permissive, we end up saying nicer things than we really
want to say about {Antoine}. There are a lot of choice points here, but
none of them lead to a viable version of normative internalism.

\section{Rationality and Symmetry}\label{rationalityandsymmetry}

In the previous two sections, I argued against five of the six theories
we started with. All that is left is \textbf{RaC}, and that will be the
focus of this section. We'll start with a case modelled on an argument
that Nomy Arpaly gives in response to a theory of Michael Smith's
~(\citeproc{ref-Arpaly2003}{Arpaly 2003, 36--46}), then turn to the
difficulties the internalist could have in motivating \textbf{RaC} by
symmetry considerations.

Think again about {Huck}leberry Finn. I said it was rational of
{Huck}leberry to help his friend Jim. But that's obviously
controversial. One might think it is rational for {Huck}leberry to do
what he thinks is good or right. At least, doing what {Huck}leberry
believes to be bad and wrong seems like a kind of irrationality. If so,
{Huck}leberry is irrational, and this might lend some support to a
theory like \textbf{RaC}.

The last sentence of the previous paragraph is a non-sequiter. If
{Huck}leberry is rationally required to do what he thinks is good, it
does follow that what he does is irrational. But it doesn't follow that
turning Jim in would be rational, unless the requirement to do what one
believes is good is the only rational requirement there is. And that's
not true.

Let's leave {Huck}leberry for a second and think about a different
character, {Noah}. {Noah} has a friend, {Lachlan}, who he is thinking of
turning in as a runaway slave. He firmly believes that it is a moral
duty to turn in runaway slaves, and that {Lachlan} is such a runaway
slave. But both of these beliefs are absurd. {Noah} lives in Australia
in the early 21st century, and there is no slavery. And he has been
exposed to compelling reasons at school to believe that slavery is a
grave wrong, and that people who helped runaway slaves were moral
heroes. But {Noah} has somehow formed the implausible beliefs he has,
and is now deciding whether to act on them.

{Noah} is irrational. {Noah}'s beliefs that {Lachlan} is a runaway
slave, and that turning in runaway slaves is morally required, are both
irrational. If {Noah} attempts to turn {Lachlan} in though, would that
be rational? I doubt it. One might say that it would be irrational to
not attempt to turn {Lachlan} in, given {Noah}'s other beliefs. I rather
doubt this too, but we don't have to resolve the question. Even if not
attempting to turn {Lachlan} in would be irrational, it might also be
the case that attempting would also be irrational. There is no rule that
says anyone has a rational option in any situation, no matter how many
irrational things they have done to create the situation. Turning
{Lachlan} in is a manifestation of some extremely irrational beliefs; it
is irrational.

As Arpaly points out, the only way to motivate the idea that {Noah} is
rationally required to do what he believes is good is to impose very
strong coherence constraints on rational thought and action. We have to
say that rationality in action requires coherence between thought and
deed, even when that clashes with doing what one's evidence supports.
But turning {Lachlan} in would be bad even by the standards of
coherence. Such an action would not cohere at all well with the
mountains of evidence {Noah} has about slavery.

As I mentioned above, it is arguable that {Noah}'s case is a rational
dilemma. Perhaps {Noah} is irrational if he turns {Lachlan} in, since he
does something that he has no evidence is a good thing to do, and he is
irrational if he does not, since this actions do not cohere with his
judgments. But even saying that {Noah} faces a rational dilemma does not
help the internalist here. For if {Noah} is in a rational dilemma,
that's still a way of saying that rationality does not line up with
maximising expected goodness. After all, maximisation norms never, on
their own, lead to dilemmas.

We will have much more to say about the possibility of dilemmas in cases
like this in subsequent chapters. But perhaps it is useful to note here
that even if {Noah} is in a dilemma, it is an extremely asymmetric one.
Even if you think it is somewhat irrational to act against his best
judgment and fail to turn {Lachlan} in, it is much more irrational to
act on no evidence whatsoever, and actually turn him in. So \textbf{RaC}
doesn't even provide a way to track what is most rational, or least
irrational.

So \textbf{RaC} is false. It isn't only one's belief in what is good
that is relevant to what it is rational to do, one's basis for that
belief matters as well. But there's another reason to be suspicious of
the \textbf{Ra} versions of the principles. Recall {Cressida}, our
example of a reckless driver. What she does is irrational. But that's
not all that's true of her actions. What she does is blameworthy and
wrong. If we want to accept the internalist's symmetry principle, we
have to say that whatever is true of {Cressida} is true of {Huck}leberry
Finn. Saying that {Cressida} and {Huck} are alike in one respect, namely
that they are both irrational, isn't a way of endorsing symmetry.

In fact, thinking about the analogy with {Cressida} gives us a reason to
think that {Huck}leberry really is rational in what he does. Assume, for
reductio, that {Huck}`s action is irrational, in the way that
{Cressida}'s driving is irrational. {Cressida}, of course, also acts
wrongly. What is the relationship between the irrationality of
{Cressida}'s action, and its wrongness? If the irrationality wholly
explains the wrongness, then the irrationality of {Huck}'s action should
'explain' the wrongness of it. But that can't be right, since {Huck}'s
action isn't wrong. If the wrongness wholly explains the irrationality,
then there is no argument from symmetry for thinking {Huck}'s action is
irrational, since there is no underlying wrongness to explain the
irrationality. More likely, the wrongness of {Cressida}'s driving and
its irrationality are connected without one wholly explaining the other.
Now the externalist has a simple explanation of that connection;
{Cressida}'s knowledge of the risks imposed by driving as she does
explains both the irrationality and the wrongness. But that kind of
explanation clearly does not generalise to {Huck}'s case. {Huck}'s
evidence clearly does not explain both the wrongness and the
irrationality of his action, since it isn't in fact wrong.

Put another way, the defender of \textbf{RaC} can either try and defend
their view with a narrowly tailored symmetry thesis, one that just
applies to rationality, or with the broader symmetry thesis that would
apply to rightness and praiseworthiness too. If we use the broader
symmetry thesis, then {Cressida}'s and {Huck}leberry's actions are alike
in rationality iff they are alike in rightness. But they are not alike
in rightness, so they are not alike in rationality. So \textbf{RaC}
fails, since they are clearly alike in rationality according to
\textbf{RaC}. So the defender of \textbf{RaC} is forced to use a narrow
symmetry thesis. But it is hard to see the motivation for the narrowly
tailored thesis. Once we allow that people can be wrong about normative
facts, and so can violate a norm while believing they are following it,
it seems plausible that one could be wrong about rationality norms, and
so could be irrational while believing one is rational.

\section{Conclusion}\label{conclusion}

So far I have argued against six forms of internalism. As I noted at the
start, internalism is not committed to the disjunction of these six
forms, so there is yet no full argument against internalism. So it might
be hoped that some form of internalism can be found that is not
committed to any of the six theses that have so far been undermined.

Hopefully though, it should be clear why the argument so far generalises
to other forms of internalism. If the motivations for internalism can be
used to support anything, they can be used to support a kind of radical
subjectivism. According to this radical subjectivism, rightness,
praiseworthiness and rationality are all matters of conformity to one's
own views. And conformity, in the relevant sense, is also to be
understood subjectively; to conform to one's views in the relevant sense
is to meet one's own standards for conformity. Such a view has to say
implausible things about cases of misguided conscience like {Antoine},
so can be seen to be false.

This completes the arc of chapters 2 through 4. In chapter 2 I discussed
some reasons for thinking that our theory should treat moral uncertainty
the same way that it treats factual uncertainty, and how this idea has
motivated a number of recent versions of normative internalism about
ethics. In chapter 3, I argued that this symmetry idea was not as
intuitively plausible as it first seemed, and that there were in
principle reasons to think that moral uncertainty, and constitutive
uncertainty more generally, should be treated differently to the way we
treat factual uncertainty. In this chapter, I argued that even if those
arguments worked, and a symmetric treatment of factual and moral
uncertainty is a theoretical desideratum, we should reject symmetry
because it leads to implausible subjectivism. The only way to really
respect symmetry is to have a radical subjectivism, and that is
implausible.

This gets to the heart of what I find unsettling about internalism. We
start out with three classes of facts:

\begin{itemize}
\tightlist
\item
  Moral facts, e.g., genocide is wrong.
\item
  Epistemic facts, e.g., it is irrational, given current evidence, to
  have a low credence that carbon emissions from human activity are
  causing global warming.
\item
  Coherence facts, e.g., it is incoherent to prefer A to B, and D to C
  in the main example in Allais (\citeproc{ref-Allais1953}{1953}).
\end{itemize}

It is easy to feel that one should have something to say about agents
who are unaware of all the moral facts. And that can push one towards a
theory where the moral facts themselves don't play a substantial role in
evaluating agents, rather something that is more accessible plays that
role. But what could that be? If we say it is evidential probabilities
of moral claims, then we are left saying that some facts that are beyond
some agents' ken, i.e., facts about what is evidence for what, are
evaluatively significant. Moreover, this kind of view will have strange
things to say about cases of inadvertent virtue in agents whose
credences track their evidence. So we might want to say something else.
If we say it is not evidence but credence that matters, we are left
saying that our most important criteria of evaluation turn solely on the
agent's coherence. And again, we can ask whether by `coherence' here we
mean actual coherence, or coherence as it strikes the agent. Facts about
coherence are not obvious. It is incoherent to believe the naive
comprehension axiom. It is incoherent to have the usual preferences in
the Allais paradox. Some people think it is incoherent to will something
that one could not will to be universally endorsed. Some people think it
is incoherent to believe there are discontinuous functions on the reals.
If we judge agents by how well their actions, beliefs and evidence
actually cohere, then we are judging them by a standard that could well
be beyond their knowledge. If we judge agents by how well the think
their actions, beliefs and evidence cohere, we'll be back to saying that
{Antoine} is a hero. Assuming we want to avoid that, we have to apply
some standards beyond what the agent accepts, and probably beyond what
they could rationally accept.

The business we're in here is trying to work out how to evaluate agents
and their actions. To evaluate is to impose a standard on the agent, one
that they may not accept, and may even lack good reason to accept.
That's the crucial externalist insight. We don't escape that conclusion
by making the standard epistemic rather than moral. Agents can disagree
with their evaluators about epistemic matters. And we don't escape that
conclusion by making the standard simply one of internal coherence.
Agents can disagree with their evaluators about what is and is not
coherent. That the correct standards of coherence are arguably a priori
knowable isn't relevant here; arguably the correct standards in ethics
and epistemology are a priori knowable too. It is plausible that the
correct standards of coherence are somehow true in virtue of their form,
but it isn't at all clear what the normative significance of that is.
Disputes about whether there can be discontinuous functions or
contingently existing objects turn on principles that are true (if true)
in virtue of their form, but nothing follows from that about whether one
could rationally have anything other than a firm true belief concerning
the correct resolution of such a dispute.

It is natural to think that we should try to find something relatively
easy to use as our initial evaluation of agents. If one thought ethics
is hard, but epistemology is easy, it would be natural to think that we
should use epistemic considerations as our starting point. But
epistemology isn't easy. Or, at least, it isn't the case that all
epistemic questions are easier than all ethical questions. If one
thought coherence questions were easy while ethical questions were hard,
it would be natural to think that we should use coherence considerations
as our starting point. But coherence questions aren't easy either. It's
epistemically worse to believe that torturing babies is morally good
than it is to believe naive comprehension. Normative internalism is a
search for what Williamson (\citeproc{ref-Williamson2000}{2000}) calls a
cognitive home, but no such home exists.

There is one loose end to tidy up. Perhaps there is another way to
respect symmetry. We could respect symmetry by having a much more
radical objectivism. If we agreed with classical consequentialists such
as Sidgwick (\citeproc{ref-Sidgwick1874}{1874}) and Smart
(\citeproc{ref-Smart1961}{1961}) that the right thing to do is what
produces the best consequences, irrespective of the agent's evidence or
beliefs, we could respect symmetry without getting {Huck}leberry Finn's
case wrong. I'm not going to have anything original to say about this
kind of consequentialism, but I wanted to briefly rehearse the reasons I
don't think this is a good way to save symmetry. (For much more, see
Slote (\citeproc{ref-Slote1992}{1992}, Ch. 15).)

This kind of actualist consequentialism gets the case of {Cressida} the
reckless driver wrong. And the moves that consequentialists make in
response to {Cressida}'s case do not seem particularly helpful in
Jackson cases, as Jackson himself emphasises.
~(\citeproc{ref-Jackson1991}{Jackson 1991}) And actualist
consequentialism combined with symmetry can't handle the cases of
{Prasad} and {Archie}. That combination implies that the parents should
have the same attitude towards their past actions, and they should not.

None of what I have to say about actualist consequentialism is at all
original, which is why I've left it to the end. And of course this view
is externalist, even more externalist in a sense than my own view.
That's why the objections of this chapter do not really touch it. The
reason the normative externalist is not forced into actualist
consequentialism is that symmetry fails, as was shown in the previous
chapter. It's true that if the arguments of that chapter fail
completely, then a new argument could open up against normative
externalism, as follows.

\begin{enumerate}
\def\labelenumi{\arabic{enumi}.}
\tightlist
\item
  If normative externalism is true, then actualist consequentialism is
  true.
\item
  Actualist consequentialism is not true.
\item
  So, normative externalism is not true.
\end{enumerate}

But give what we saw in the previous chapter, we should already reject
premise 1. And the arguments of this chapter, showing that symmetry will
have one or other kind of implausible consequence, provides another
reason to reject premise 1.

Most discussions of normative internalism in the ethics literature to
date have revolved around symmetry. But there are considerations other
than symmetry that may seem to motivate a variety of internalism, and in
the next two chapters I'll discuss them.

\chapter{Blame and Moral Ignorance}\label{blameandmoralignorance}

If an argument from premises concerning symmetry to a conclusion about
internalism worked, we would get a very strong conclusion. It would turn
out that true morality is irrelevant to our judgment of actions and
persons. All we should use, when judging someone's actions, is the moral
compass that they have. Or, perhaps, the moral compass that they should
have given their evidence. As I have stressed though, the arguments for
symmetry might undermine the availability of this fallback. It really
looks like that if symmetry proves anything, it proves that the only
moral standard (and indeed only epistemic standard) is (perceived)
consistency with one's own values.

Put so baldly, it perhaps it isn't surprising that symmetry-based
internalism as it was developed in the last three chapters ended up
looking like a hopeless project. So it is time to look at alternative
motivations for internalism, ones that suggest somewhat weaker forms of
internalism. The views we looked at so far said that being true to one's
own self (in one way or another) was both necessary and sufficient for
the applicability of some key moral concept. Over the next two chapters,
we'll look at views that drop one or other direction of that connection.
So in this chapter, we'll look at views which say that conformity to
one's own values is a sufficient condition for blamelessness. And in the
next chapter, we'll look at views which say that conformity to one's own
values is a necessary condition for avoiding all vices.

\section{Does Moral Ignorance Excuse?}\label{doesmoralignoranceexcuse}

In recent work on moral responsibility, many philosophers have argued
that that blameless ignorance of what's right and wrong is exculpatory.
Something is exculpatory if it provides a full excuse; it makes an agent
not blameworthy for a wrong action they perform. So slightly more
precisely, what these philosophers have argued for is a version of the
following view.

\begin{quote}
\textbf{Moral Ignorance Excuses (MIE)}\\
If agent S does act \emph{X}, even if \emph{X} is wrong, S is not
blameworthy for this if:

1. S believes that \emph{X} is not wrong; and\\
2. This belief of S's is not itself blameworthy; and\\
3. The belief is tied in the right way to the performance of \emph{X}.
\end{quote}

The second and third clauses are vague, and getting clear on whether
there is a way of resolving the vagueness that makes the theory viable
is going to be a big part of the story to follow. But the vagueness also
makes it plausible to attribute MIE to a lot of philosophers. A classic
statement of MIE, that I'll return to at some length below, is in
``Reproach and Responsibility'' by Cheshire Calhoun
(\citeproc{ref-Calhoun1989}{1989}). But the view has been more recently
defended by Gideon Rosen (\citeproc{ref-Rosen2003}{2003},
\citeproc{ref-Rosen2004}{2004}), Michael Zimmerman
(\citeproc{ref-Zimmerman2008}{2008}) and Neil Levy
(\citeproc{ref-Levy2009}{2009}). And I plan to argue against all of
these views.

Although I'll focus on philosophers such as Rosen and Zimmerman who have
openly defended MIE, I've crafted the definition of MIE so that it is
endorsed by many of their critics. In taking on MIE, then, I'm taking on
a much broader range of philosophers than those who describe themselves
as holding that moral ignorance excuses.

MIE is not a reductive account of blameworthiness; it uses the notion of
blameworthy belief right there in the second clause. And through the
1990s and 2000s, much of the debate around MIE turned on how to
understand that clause. Rosen and Zimmerman use MIE to argue for a very
strong view. They think that mistaken moral beliefs are very rarely
blameworthy, so they think MIE implies that people are rarely
blameworthy. Or, at least, there are very few cases where we can be
confident someone is blameworthy. This view is rejected by, for example,
{Alex}ander Guerrero (\citeproc{ref-Guerrero2007}{2007}) and William
FitzPatrick (\citeproc{ref-Fitzpatrick2008}{2008}). Looking back to the
earlier debate, Michelle Moody-Adams
(\citeproc{ref-MoodyAdams1994}{1994}) similarly rejects some of the
practical conclusions that Calhoun (\citeproc{ref-Calhoun1989}{1989})
draws. But all of these rejections are accompanied by acceptance of
something like MIE. The complaint these philosophers are making is not
that MIE fails, but that philosophers have been too generous in their
application of clause 2. I'm arguing for the stronger claim, that MIE
itself fails.

In recent years more philosophers have adopted the more radical view
that MIE itself is false. Elizabeth Harman
(\citeproc{ref-Harman2011a}{2011}, \citeproc{ref-Harman2014}{2015}) has
argued against the view that moral mistakes can be exculpatory. Since
Harman thinks that moral mistakes are themselves blameworthy, her view
is in a technical sense consistent wth MIE. But that's only because she
thinks that strictly speaking, it never applies. And the broader view on
responsibility I'm adopting draws on work by Nomy Arpaly
(\citeproc{ref-Arpaly2003}{2003}), Angela A. M. Smith
(\citeproc{ref-AngelaSmith2005}{2005}) and Julia Markovits
(\citeproc{ref-Markovits2010}{2010}).

Harman notes that the debate has been misnamed. When we talk about moral
ignorance excusing, what we really mean is that moral \emph{mistakes}
might excuse. If someone is extremely confident that \emph{X} is wrong,
but not quite confident enough to know it, few philosophers would say
that mental state is exculpatory when \emph{X} is done. If they have a
justified true belief that \emph{X} is wrong, but don't strictly know
this because their belief is in some other way defective, no one takes
their lack of knowledge to be exculpatory. What matters are cases of
moral mistake; cases where an agent firmly and reasonably has a moral
belief that's simply false. Harman notes that this point is at least
implicit in Guerrero's response to Rosen
~(\citeproc{ref-Guerrero2007}{Guerrero 2007}), and a similar point is
made by Rik Peels (\citeproc{ref-Peels2010}{2010}).

In order to keep terminological consistency with most of the debate,
while avoiding getting caught up on the point of the last paragraph,
I'll make some more terminological stipulations. Say that a person is
\textbf{thoroughly ignorant} of a truth \emph{p} iff she believes
¬\emph{p}. And then the live issue is whether thorough moral ignorance
excuses. I'll assume that when other writers hypothesise that moral
ignorance excuses, the term `thorough' has been elided. And I will join
them in this way of writing.

\section{Why Believe MIE?}\label{whybelievemie}

There are three main classes of arguments that have been given for MIE.
The first is what I'll call the Argument from Symmetry, defended by
Rosen (\citeproc{ref-Rosen2003}{2003, 64}) and Zimmerman
(\citeproc{ref-Zimmerman2008}{2008, 192}).

\begin{enumerate}
\def\labelenumi{\arabic{enumi}.}
\tightlist
\item
  Cases like Adelajdra's and {Billie}'s show us that mistakes about
  matters of fact can excuse wrongdoing.
\item
  Moral mistakes and non-moral mistakes should be treated the same way.
\item
  So moral mistakes can excuse wrongdoing.
\end{enumerate}

I've in effect already offered two responses to this argument. Adelajdra
and {Billie} don't do anything wrong, so they don't need an excuse, so
they aren't reasons to think that non-moral mistakes are excuses. And in
general non-moral mistakes are not excuses. Borrowing some terms from
jurisprudence, mistakes of fact are defences, not excuses; they are
reasons to find someone not guilty, rather than reasons to not punish
them despite their guilt. And chapters 3 and 4 are long arguments for
thinking that premise 2 of this argument is wrong.

The second is an argument from motivation, which we could put as
follows.

\begin{enumerate}
\def\labelenumi{\arabic{enumi}.}
\tightlist
\item
  It is good, or at least blameless, to do \emph{X} because one thinks
  \emph{X} is the thing to do.
\item
  If an action is blameworthy, this blame must be traceable to some
  stage that led to the production of the action.
\item
  So if the belief that \emph{X} was the thing to do is blameless, then
  so is the performance of \emph{X}.
\end{enumerate}

The long argument that Michael Zimmerman gives for a version of MIE is,
I think, a version of an argument from motivation
~(\citeproc{ref-Zimmerman2008}{Zimmerman 2008, 175ff}). And the argument
plays a central role in Gideon Rosen's discussion. He describes a
character{Bonnie} who is, as he puts it, an ``unreconstructed selfish
creep'' (77) {Bonnie} cuts in front of a father waiting in the rain for
a cab with his family, for no good reason other than she wanted to get
uptown in more of a hurry. It turns out later that {Bonnie} has been
suffering from a virus, and one effect of this virus is that she ceases
to view considerations involving others as giving her reasons for
action. But it did so, remarkably, in a way that left {Bonnie} in a
relatively coherent state. She reflectively endorses her self-centred
behaviour, and dismisses the importance of traditional moral
considerations. Indeed, she apparently can hold her own in philosophical
argumentation when confronted with the standard arguments against the
kind of nihilism or egoism (it isn't exactly clear which) she now
espouses. Rosen argues it is reasonable for her to take arguments for
conventional morality seriously, but she does that, so to get her act
well we will need her to do more.

\begin{quote}
But is it reasonable to expect more? Here is {Bonnie}. She blamelessly
thinks that she has most reason to steal the cab. What do you expect her
to do? To set that judgment aside? To act on what she blamelessly takes
to be the weaker reason? To expect this is to expect her to act
unreasonably by her own lights. This is certainly a possibility, but is
it fair to expect it or demand it? Is it reasonable to subject an agent
to sanctions for failing to exhibit akrasia in this sense? When these
questions are raised explicitly, the answers can seem self-evident. No,
it is not reasonable to expect a person to do what she blamelessly
thinks she has less reason to do. No, it's not fair to subject someone
to sanctions for `pursuing the apparent good' when it is clear that she
is blameless for the good's appearing as it does.
~(\citeproc{ref-Rosen2003}{Rosen 2003, 79--80})
\end{quote}

Given what I've said about moral fetishism, I have to think premise 1 of
the motivation argument fails. It is not good to be motivated by the
good as such. What is good, or at least what is best, is to be motivated
by that which is good. We expect {Bonnie} to be motivated by good
reasons, even if she falsely takes them to be bad reasons. And premise 2
is, as Manuel Vargas (\citeproc{ref-Vargas2005}{2005}) argues, far from
obvious. Maybe every blameworthy act is downstream from a move that is
blameworthy in isolation, but it isn't obvious why we should assume
this.\footnote{When Guildernstern says ``There must have been a moment
  at the beginning, where we could have said---no. Somehow we missed
  it.'' ~(\citeproc{ref-Stoppard1967}{Stoppard 1967/1994, 125}) he is
  voicing something like premise 2. And it's not obvious that
  Guildernstern is right.}

Now I should note that there is an exception I noted in chapter 3 to the
general rule that it is best to not act on the basis of thin moral
beliefs. And that exception provides a possible way to defend something
like MIE in a very narrow range of cases. I'll come back to this in the
discussion below of Calhoun's view.

The third argument for MIE comes from cases where an agent acts from
moral ignorance, and apparently it is intuitive that they are blameless.
Many of these cases do not elicit anything like clear intuitions. And in
the cases that do elicit relatively clear intuitions, it is a further
step to say the correct explanation of that intuition is that the moral
ignorance explains the blameworthiness.

For example, consider {JoJo}, as described by Susan Wolf
(\citeproc{ref-Wolf1987}{1987}). {JoJo} is the son of Jo, a vicious
dictator. Jo rose through the ranks to become dictator in a coup, and we
aren't supposed to feel any hesitation in blaming him for his misdeeds.
But {JoJo} was born in the palace, and raised to be ruler. He hasn't
known any life other than the life of the vicious dictator, and has
never been exposed to other moral systems. Many philosophers intuit that
when Jo ascends to the throne, and continuous the family business of
being vicious dictators, he is less blameworthy than Jo.

But to get to MIE, we need two stronger assumptions, neither of which we
can get from raw intuition. One is that {JoJo} is not just less
blameworthy than his father, but that he is blameless. The other is that
{JoJo} is blameless because he is morally ignorant. I'm going to argue
that neither of these extra assumptions is correct. When we return to
{JoJo} at the end of the chapter, I will draw on some insightful things
that Elinor Mason (\citeproc{ref-Mason2015}{2015}) says about cases like
{JoJo}'s to argue that MIE is the wrong conclusion to draw cases from
like his.

\section{Chapter Plan}\label{chapterplan}

The argument here is going to be a little more roundabout than in other
chapters, so let's have a road-map to see where we're going.

\begin{itemize}
\tightlist
\item
  In sections 4 and 5, I'll set out some very general features of blame
  that I'll be taking mostly for granted in the discussion that follows.
\item
  Sections 6 through 9 will discuss the idea that moral ignorance only
  excuses if it is connected to action in the right way. I'll argue that
  most of the ways we might try to make this vague notion of `connected
  in the right way' precise lead to implausible theories. The only
  version of MIE that survives being precisified is very weak.
\item
  Sections 10 and 11 will discuss two kinds of cases that this weak
  version of MIE might still cover: wrong actions that are done
  habitually, and wrong actions by people in very different moral
  cultures. In each case, I'll argue that there are better explanations
  than MIE for why blame might be eliminated on reduced in these cases.
\item
  Finally, I'll return to the picture of blame that results from these
  discussions, and go over how it handles a number of difficult cases.
\end{itemize}

\section{Blame and Desire}\label{blameanddesire}

The main aim of the chapter is to argue that MIE is false. It posits a
necessary connection between (rationally) believing actions have a
certain moral property, and actions actually having some similar
property. Since I'm in the business of rejecting all such necessary
connections, it's my job to argue against it.

I think MIE fails for a fairly systematic reason. It makes beliefs
central to whether someone is blameworthy, but blameworthiness is a
matter of having the wrong desires. Here I'm following a version of the
view defended by Arpaly and Schroeder
(\citeproc{ref-ArpalySchroeder2014}{2014}). Blameworthy people either
desire things that are bad, or fail to (sufficiently) desire things that
are good. Actions are blameworthy (or praiseworthy) to the extent that
they manifest bad (or good) desires to have. While I'm directly
following Arpaly and Schroeder, what I say here owes a lot to the
broader tradition on moral responsibility that traces back to Strawson
(\citeproc{ref-Strawson1962}{1962}).

One might think that a desire-based view of blame would immediately rule
out MIE: it says that beliefs are relevant to blame, but in fact only
desires are relevant to blame. And beliefs and desires are distinct
existences, so beliefs can't be relevant to blame. But that's too quick.
After all, which desires an action manifests is a property of the
beliefs of the actor. So it could be that moral beliefs matter because
they affect which desires are manifest by an action.

Here's one (implausible) way that moral beliefs could matter to blame.
Assume that the good person has one and only one desire: to do the right
thing. Then if a person thinks that what they are doing is right, it
shows that they are manifesting the one and only desire that a good
person has, so they are blameless. That is a way to make MIE compatible
with the desire-based view of blame. But it's implausible twice over.
For reasons we've discussed already, it's not true that the \emph{only}
thing a good person wants is to do what's right. And this view would say
that acts of misguided conscience are not just blameless, they are
positively praiseworthy, since they manifest the one and only good
desire.

We don't need such an implausible view to get something like MIE though,
within a broadly Strawsonian view on blame. Even if good people have
more than one desire, we might suppose that one of their desires is to
do the right thing. Assume someone has some bad desires, and perform an
act that manifests those bad desires, but also desires to do the right
thing, and this very action also manifests that desire. Then a plausible
version of the desire-based view of blame is that their bad action will
be less blameworthy than a similar bad action by someone who only has
the bad desires. To that extent, the false moral belief will be
something that reduces blameworthiness. And if we call anything that
reduces blameworthiness an excuse, then the false moral belief will be
an excuse. It won't be a full excuse, which is what most defenders of
the MIE want, but it will be excusing.

There is another, even more plausible, way that false moral beliefs
could be related to blame.\footnote{I only realised the importance of
  these cases in discussions with Claire Field about her work on blame
  and normative ignorance. I'll return to these cases in section 10.}
Assume that false moral beliefs are somehow connected to false beliefs
about practical reasoning. (`Connected' here is deliberately vague, and
the vagueness will matter in what follows.) People who make mistakes in
or about practical reasoning tend to manifest different desires than we
might have thought they did. And that could turn out to matter.

We can see this with non-moral examples. {Abbott} and {Costello} are
each offered a deal. If they take the deal, it will gain them \$10
straight away, and then lose \$1 every day for the next 30 days. Both of
them take the deal. But they do so for different reasons. {Abbott} has a
really steep discount function. He values \$10 now much more than the
loss of \$30 over the next month. {Costello} is practically irrational;
the deal makes him worse off by his own lights, but he does not realise
this. I've stipulated that {Abbott} and {Costello} value the deal
differently, but in practice we can often detect these values without
stipulation. Imagine we point out to {Abbott} and {Costello} that taking
the deal will leave them \$20 worse off at the end of the month.
{Abbott} will say, ``Who cares? I want the money right now.'' {Costello}
will say, ``Huh, I hadn't realised that. I wonder if I can back out of
the deal.'' On a desire based view of blame, {Abbott} is to blame for
his own misfortune when in a month's time he is \$20 worse off than he
might have been. But {Costello} is not blameworthy, since his desires
are not defective.

In general, people who are practically irrational might be blameless for
what seem like wrong acts, because the act does not reflect their
underlying desires. This isn't exactly the same thing as saying that
normative ignorance excuses, but it is very close. The following two
groups are not identical as a matter of conceptual necessity, but they
have a huge overlap.

\begin{itemize}
\tightlist
\item
  People whose actions do not reflect their desires.
\item
  People who do not know what actions are best given their desires.
\end{itemize}

Indeed, many people who are in both groups would cease being in the
first group as soon as they ceased being in the second. The view I'm
going to be defending is that whether someone is in the second group
does not directly matter to how blameworthy they are. That is, MIE is
false. But whether someone is in the first group matters a lot, and
whether someone is in the first group might in practice depend on
whether they are in the second group. So normative ignorance will in
many real world cases be indirectly relevant to blame.

\section{Blame, Agents and Time}\label{blameagentsandtime}

I'm not going to try to present a full theory of blameworthiness, and
then derive results about ignorance and blameworthiness from it. But I
will record with two important general points about blame that will
matter in what follows.

The first is that it is agents, not actions or outcomes, which are the
primary subjects of praise and blame. I will still say, and have already
said, that agents can be blameworthy for actions. (Peter A. Graham
(\citeproc{ref-Graham2012}{2014}), in the course of offering a plausible
general theory of blame, denies even this.) But it is the agent, not the
act, that is the focus of blame.

The second point, which has not received sufficient attention in the
recent literature, is that blameworthiness is time sensitive. It seems
very bizarre to say that a particular action, performed at
\emph{t}\textsubscript{1}, is wrong at \emph{t}\textsubscript{2} but not
wrong at \emph{t}\textsubscript{3}. Perhaps that is even contradictory.
But it is certainly not contradictory to say that the agent of that
action is blameworthy for the action at \emph{t}\textsubscript{2} but
not at \emph{t}\textsubscript{3}. Indeed, such claims are often true, as
in the following case.

{Glyn} is a twelve year old boy. He steals {Mehdi}'s expensive new
jacket. {Glyn} does not need a new jacket, he is not suffering from any
kind of duress or compulsion, and he knows it is wrong to steal. But he
wants the jacket, so he steals it. At the time he steals it, he is
blameworthy for the theft.

Fast forward forty years, and {Glyn} is now a middle aged man. He has
not gone onto a life of crime. He is no moral saint, but an ordinary
mostly moral-law-abiding member of society. It would be wrong to still
blame him for the theft. Indeed, it is overdetermined that {Glyn} is no
longer blameworthy. Typically, adults are not blameworthy for the wrongs
committed by their juvenile selves. And typically, people are not
blameworthy for the wrongs they committed in the distant past. We can
test this by varying {Glyn}'s case in different ways. If {Glyn} turns
into a decent 19 year old, it seems wrong to blame him for the actions
of his 12 year old self. And if he steals the jacket at 22, it seems
wrong to blame his 52 year old self for the theft.

The law backs up many of these intuitions. Except for cases of severe
wrongdoing, we typically give people a clean slate when they become
adults. Records of juvenile wrongdoing are sealed, so as to prevent past
misdeeds being held against someone. In the UK, this principle is taken
further. The \emph{Rehabilitation of Offenders Act 1974} makes it the
case that after a certain length of time, even adult convictions for
minor to moderately serious offences are \emph{spent}. It can be
defamatory to describe someone as a convicted criminal, if their
conviction is spent. The law recognises that after a while, people are
not responsible for the misdeeds of their earlier selves. More
generally, whether someone is blameworthy for an action might change
over time.

In some cases, I suspect this change of status can happen rather
quickly. Change {Glyn}'s case so that a few weeks later, he has a change
of heart. He sheepishly returns the jacket to {Mehdi}, and apologises.
And, crucially, {Mehdi} accepts the apology. Now {Glyn} is no longer
blameworthy for the theft. He was blameworthy, but in a case where the
misdeed was not too excessive, where only one person was harmed, and
that person has accepted an apology, the period of moral responsibility
has passed. {Glyn} was blameworthy for the theft, but he is no longer.

What's crucial is that blameworthiness can be time limited, not anything
in particular I've said about apologies, or even about juvenile
wrongdoing. We should reject the `branding' model of blameworthiness,
that once a person is blameworthy for something, they are branded with a
moral cross, and must carry this mark for eternity. Rather,
blameworthiness can ebb and, occasionally, flow.

This matters for one of the arguments Rosen gives concerning {Bonnie}.
{Bonnie} is an ``unreconstructed selfish creep''
~(\citeproc{ref-Rosen2003}{Rosen 2003, 77}), who nevertheless is
internally coherent. So far, so bad. {Bonnie} seems like an appalling
person, even if it is rather sad that she has become an appalling
person. But a few weeks later, the virus wears off, and she regrets
having the views that she previously had, and of course acting on them.
Is she now blameworthy for the terrible things she did while suffering
the virus?

I think one could go either way on this. But it seems that question is
separate from the question of whether she was then blameworthy for what
she did. If you think {Bonnie} should not now be blamed for what she did
while suffering the virus, because you think that in some sense she
isn't the same person as the one who committed those misdeeds, then your
willingness to let {Bonnie} off the hook now isn't even evidence that
what she did wasn't then blameworthy.

Rosen actually notes that time might matter to {Bonnie}'s case, but
dismisses this consideration too quickly. He writes,

\begin{quote}
You may think that blame is no longer appropriate, not because the act
was not blameworthy when it was committed, but rather because time has
passed and it is time for you to let it go. The judgment that
forgiveness is now mandatory is not the judgment that it was unfair to
blame {Bonnie} in the first place. It is the judgment that further blame
would be unfair given the severity of the transgression. Since we want
to focus on whether the act was blameworthy when committed, we need to
set this thought aside. So let's stipulate that the offence was recent
enough and serious enough that if {Bonnie} was indeed responsible, you
are not yet required to forgive her. ~(\citeproc{ref-Rosen2003}{Rosen
2003, 81})
\end{quote}

But the last point is exactly what can't be stipulated. It isn't just
passage of time or forgiveness of victims that makes blameworthiness go
away. Sufficient change of character can too. That's why it doesn't take
too long for juvenile wrongdoing to be morally expunged. Commit the
misdeed at the right time, and it might be legally expunged in a few
days. Morality doesn't use the same hard cutoffs the law uses, but the
principle is the same. {Bonnie}'s change of character is much quicker,
but not completely unrealistic. (Compare the case discussed by Burns and
Swerdlow (\citeproc{ref-BurnsSwedlow2003}{2003}).) And we should have
the same verdict; her actions were blameworthy, but she is no longer to
blame for them.

\section{Acting In Ignorance is No
Excuse}\label{actinginignoranceisnoexcuse}

The next three sections are on clause 3 of MIE. The clause is ambiguous,
and on the most natural interpretations, MIE is clearly false. So we'll
look at whether there is any interpretation that makes MIE plausibly
true. It will take a while to cover the possibilities, but at the end
the only version of MIE left viable will be one that is very weak.

In the \emph{Nichomachean Ethics}, Aristotle distinguishes between
acting in ignorance of the wrongness of one's actions, and acting from
that ignorance. Getting clear on just what this means is not easy. But
doing so is crucial to finding a version of MIE that is plausible.

Consider first an extremely contented carnivore. We'll assume she's in a
world where meat-eating is wrong. And we'll assume she is ignorant of
that fact, as she chews away happily on a hamburger. But this ignorance
plays no role in bringing about her eating. She certainly does not think
to herself ``It's a good thing this is permissible,'' as she eats away.
She is not disposed to order different foods on learning that
meat-eating is wrong. She would not eat differently were she to have
different views about meat-eating. She regards the coincidence between
her wants and what is, by her lights, morally permissible as a happy but
irrelevant accident. She eats a hamburger because she wants a hamburger,
and that settles things as far as she is concerned.

The ignorance that our carnivore shows does not excuse her. It is true
that she is ignorant of the wrongness of her action. But she doesn't eat
because she is so ignorant. So it does not affect the moral status of
her action. The general point is that moral ignorance that merely
accompanies a wrongful act doesn't excuse the act. The ignorance must in
some way make a difference to the act.

There are two natural ways to think that moral mistakes could be
relevant to actions in ways that are excusing. First, the action might
be counterfactually dependent on the mistake. If the agent wasn't making
the mistake, they wouldn't have performed the action. Second, the action
might be motivated by the mistake. That is, the reasons the agent had
for the action might have included the mistaken belief. In many cases,
these two will go together. But this actually makes things tricky for
the idea that action from ignorance can excuse. The next subsection will
show that adding a condition that the action was counterfactually
dependent on the mistake does not provide a sufficient condition for
blamelessness. And the following subsection will show that if ignorance
ever does excuse, it isn't necessary that the ignorance is motivating.
Indeed, it is sometimes necessary that the ignorance is not motivating.
The space of cases in which ignorance excuses is, if not empty,
exceedingly small.

\section{Against Counterfactual Interpretations of Acting From
Ignorance}\label{againstcounterfactualinterpretationsofactingfromignorance}

The mere presence of a blameless moral mistake does not excuse. It is a
little more plausible to think that actions that are in some way
traceable to a blameless moral mistake are excusable. Here's a version
of MIE that makes that idea rigorous.

\begin{quote}
For any agent S, proposition \emph{p} and action \emph{X}, if

\begin{enumerate}
\def\labelenumi{\arabic{enumi}.}
\tightlist
\item
  S blamelessly believes \emph{p}; and
\item
  \emph{p} is false; and
\item
  If S had not believed \emph{p}, S would not have done \emph{X}, then
\end{enumerate}

S is not blameworthy for doing \emph{X}, since her ignorance of \emph{X}
is an excuse.
\end{quote}

This proposal won't work for a reason Gideon Rosen notes
~(\citeproc{ref-Rosen2003}{Rosen 2003, 63n4}). {Pasco} has read online
that his football team has lost. The website he reads this on is, as he
knows, extremely reliable. But the website is wrong on this occasion.
Since {Pasco} knows the website is generally reliable, he is blameless
for believing his team lost. {Pasco} reacts to the report of the loss by
throwing a brick through his neighbour's window. He wouldn't have done
this had his team won. So all three conditions are satisfied, and yet
{Pasco}'s ignorance of the result of the football match does not provide
an excuse for the brick-throwing.

We need to at least supplement the simple theory. Rosen suggests the
following fourth condition.

\begin{quote}
If \emph{p} had been true, then S's action would not have been
blameworthy. ~(\citeproc{ref-Rosen2003}{Rosen 2003, 63n4})
\end{quote}

I'm not sure why Rosen uses `blameworthy' here, rather than `wrong'. It
seems unintuitive to say that a false belief that would have offered a
mere excuse if true could actually furnish an excuse. But I won't press
the point, since it doesn't matter for the larger debate. Nothing like
this condition can work. Indeed, it seems very unlikely that we can hold
onto the idea that actions done from moral ignorance excuse, while
understanding the concept of acting from moral ignorance in terms of
conjunctions of counterfactuals.

To see this, add another assumption to the example: {Pasco} is a moral
nihilist. That is, he thinks that nothing is good or bad, right or
wrong, blameworthy or praiseworthy. This doesn't affect what he does
very much (unless you're unfortunate enough to be stuck in a
philosophical discussion with him). It certainly doesn't affect whether
he reacts to bad football news by quietly cursing that overpaid forward,
or by tossing bricks around. And assume that this belief in moral
nihilism is blameless; it is a natural enough reaction to the strange
diet of philosophical reading he has had.

Now let \emph{p} be the proposition \emph{Pasco's football team lost,
and moral nihilism is true}. {Pasco} believes that. It is false; doubly
so since both conjuncts are false. If he did not believe \emph{p}, he
would have not thrown the brick through his neighbour's window. I'm
making an extra assumption here, but it's a plausible addition to the
case. The assumption is that possible worlds in which either the website
reports the results correctly, or {Pasco} reads some other website to
get the football score, are much more like reality than the world where
he sees the error of his nihilist thinking. That is, if he were to see
that \emph{p} is false, it would be because he saw the first conjunct is
false, not because he saw the second conjunct is false. Finally, if
\emph{p} were true, what {Pasco} did would not be bad, or wrong, or
blameworthy.

The last point is a little delicate, in a way that I don't think helps
Rosen's case. Moral nihilism is necessarily false. False global moral
theories are, typically, necessarily false. So evaluating
counterfactuals about what would happen were one of them true require
thinking about counterfactuals with necessarily false antecedents. Such
counterfactuals are, to put it mildly, not well behaved. I'm a little
inclined to think, following Lewis (\citeproc{ref-Lewis1973a}{1973})
that they are all trivially true.\footnote{But wait! Haven't I been
  talking all book about examples in which some things are true that
  are, in fact, necessarily false? Yes, I have, but there's no
  contradiction here. I think, following Ichikawa and Jarvis
  (\citeproc{ref-IchikawaJarvis2009}{2009}) that we should understand
  philosophical examples as little fictions. And, contra Lewis
  (\citeproc{ref-Lewis1978b}{1978}), I think there are good reasons to
  not understand truth in fiction in terms of counterfactuals. It would
  take us too far afield to go into these reasons, but see Gendler
  (\citeproc{ref-Gendler2000}{2000}) for discussion of some of the
  relevant considerations. If I'm right about both of those claims, then
  there are non--trivial truths about about what is true in a
  necessarily false thought experiment, but not about what would happen
  if a necessary falsehood were true.} And I suspect this will make
trouble for any attempt to spell out the idea of acting from ignorance
in the way Rosen suggests. But set that aside, because the issues are
very hard, and because we don't need to address them. Any theory of
counterfactuals should say that it is true that if moral nihilism were
true, then {Pasco}'s action would not be bad, or wrong, or blameworthy.
And that's all we need to make trouble for Rosen's view.

So on Rosen's view, {Pasco}'s false belief in the conjunction \emph{My
football team lost and moral nihilism is true} excuses the brick
throwing. And that's implausible. To see how implausible, note that the
belief on its own that moral nihilism is true is not exculpatory.
Imagine first a person just like {Pasco}, except that this person
planned to throw the brick either in anger or celebration, whether the
team lost or won. He shares {Pasco}'s false belief, but he doesn't have
an excuse. Next consider a person who is like {Pasco} except he has
correct moral beliefs, and knows he is acting immorally when he throws
the brick. He too has no excuse. It is only the strange combination of
views and dispositions that {Pasco} has that are excusing. And that's
very implausible, even if one thinks that false moral beliefs could in
principle excuse.

One possible move that could be made here is to restrict the quantifier
in our principle about excuse. Perhaps we should say that a false belief
is excusing only if it is a false belief in a proposition about
morality, and satisfies these conditions. But such a move would be
undermotivated twice over. For one thing, a core motivation for MIE is
the argument from symmetry, and making this move is to insist on a huge
asymmetry. For another thing, we need some special story about why such
a restricted theory should be true, and the literature is not exactly
forthcoming with such stories. Indeed, if we had such a restriction in
place, it isn't clear we would even need Rosen's fourth condition. But,
as Rosen acknowledged, we do need such a condition to get a plausible
thesis about mistake and excuse.

\section{Against Motivational Interpretations of Acting From
Ignorance}\label{againstmotivationalinterpretationsofactingfromignorance}

Still, there is a natural enough fix to MIE. {Pasco}'s false moral
belief, either on its own or in conjunction with false factual beliefs,
doesn't excuse because it doesn't play the right kind of role in his
deliberations. For a false belief to excuse, it isn't sufficient for an
agent's actions to be counterfactually sensitive to the presence of the
belief. Rather, the belief must play some kind of affirmative role in
the agent's motivations, not just the kind of regulative role that is
implied by counterfactuals like the one Rosen uses. The action must not
just be sensitive to the presence of the belief, but in some way brought
about by the belief.

That's intuitively why {Pasco}'s false belief in the conjunction
\emph{p} is not exculpatory. Although he would not have acted had he not
believed \emph{p}, his belief that \emph{p} doesn't play any role in
bringing about the wrong action. So adding a requirement that the
ignorance be motivating avoids that counterexample. But it introduces
new problems. I'm going to discuss two counterexamples to the new
version of MIE.

First consider {Gusto}. {Gusto} normally has little interest in
morality. But he is interested in girls, and right now he is interested
in {Irene}. She says that she will only date him if he does nothing
immoral for a week. So for this reason, and this reason only, he
develops a keen interest in morality. Sadly, he gets some things wrong.
So even though he thinks it is morally acceptable to break a particular
promise he made to {Oleg} in the service of a greater good, it is not,
and he acts immorally when he breaks the promise. {Oleg} can blame him
for breaking the promise, despite {Gusto}'s instrumental desire to act
morally. It is very strange to think that his promise breaking ceases to
be blameworthy merely because it is driven by his desire to date
{Irene}.

For another example, consider {Sebastian} and {Belle} who are,
blamelessly, committed consequentialists. That is, they do the actions
they think will have the best consequences, understood in a completely
neutral manner. They are also siblings. But there is a difference
between them. When faced with any choice of any importance whatsoever,
{Sebastian} will first think to himself, ``What will produce the most
utility?''. Having convinced himself that a particular action is utility
maximizing, he will perform that action just because it is utility
maximizing. {Belle} has simply adjusted her values and dispositions in
such a way that she sees actions in terms of their utility, and is
directly motivated to do the thing that is in fact utility maximising.

One day, their mother is sick in hospital. It isn't life threatening,
but it is a bit scary, and she would be helped by a visit from her
children. But neither of them visit. They are both volunteering at a
soup kitchen, and don't want to leave their posts. {Sebastian}
deliberates about what to do. He thinks ``If I leave, some people will
go hungry. That will produce more disutility than my mother's sadness.
And it is bad to produce more disutility, and I don't want to do what is
bad. So I'll stay here.'' {Belle} simply is moved by the plight of the
hungry in front of her, and stays without deliberating.

Assume, for the sake of the argument, that the beliefs {Sebastian} and
{Belle} have are blameless. And assume that the impersonal
consequentialism they believe is wrong - they should go to visit their
sick mother. Finally, note that they would not have ignored their
mother's needs had they not had their false belief in consequentialism.
As Rosen's proposal stands, both of them are blameless for their action,
since their false belief in consequentialism excuses. But Rosen's
proposal is false, as the example of {Pasco} shows. And the natural way
to fix it puts a gap between {Sebastian} and {Belle}. Since
{Sebastian}'s false belief in consequentialism does motivate his
decision to stay, but {Belle}'s false belief does not motivate her
decision to stay, {Sebastian} has an excuse but {Belle} does not.

This is the wrong way around. {Sebastian} is worse than {Belle}. The
kind of hyper-moralised thinking that {Sebastian} engages in is exactly
the kind of `one thought too many' thinking that Bernard Williams
(\citeproc{ref-Williams1981}{1981}) accuses consequentialists of. I
think, following Railton (\citeproc{ref-Railton1984}{1984}), that
Williams's complaint against consequentialism misses the mark. {Belle}
is a perfectly good consequentialist, but can't be accused of having too
many thoughts. But I do think Williams is right that having one thought
too many is a bad thing,. And we should not reward {Sebastian} for
having one thought too many by excusing his lack of filial piety. Even
if you think Williams's idea in general is too strong, it seems
extremely appropriate here. By pausing to deliberate, and check for his
own moral rectitude, {Sebastian} helps nobody. Deliberation takes time,
and it is time he could have spent helping others. Indeed, {Belle} does
spend that time helping others. It seems extremely odd to say that she
is blameworthy because she kept on serving food rather than stopping for
a bit of a think then, quite predictably, gone on serving the food.

Perhaps we can avoid both kinds of counterexamples just described if we
modify the idea that false moral belief can excuse even further. For a
false moral belief to excuse it must:

\begin{itemize}
\tightlist
\item
  Be blamelessly held; and
\item
  Be relied on in action guidance; and
\item
  Be blamelessly relied on in action guidance.
\end{itemize}

Note that I say `action guidance' not `deliberation' here, because I
take it we want to say that someone can be guided by their beliefs
without using them in deliberation. When I descend from a balcony by the
stairs rather than jumping over the railing, I'm guided by my belief
that jumping over the railing will result in injury, even if I don't
deliberate using that belief. Typically, I don't deliberate at all
before descending via stairs rather than via jumping, so I don't use any
beliefs in deliberation. Now given what we said about {Pasco}, we'll
need some notion of action guidance that is stronger than counterfactual
dependence, and that will be no small challenge. I'll return to that
problem in the next section, because first I need to note some points
about the restriction to reliance that is blameless.

First, this restriction gives us at least a chance of getting the cases
of {Belle} and {Gusto} right. A philosopher who thinks that false moral
beliefs can excuse should, I suspect, say that {Belle} blamelessly
relies on her consequentialism. She isn't directly motivated by it;
indeed it feels rather forced to say she is motivated by it at all.
She's motivated by the needs of her clients at the soup kitchen. But she
relies, in some sense, on consequentialism. On the other hand, it is
plausible that {Gusto} doesn't get off the hook so easily. It is,
perhaps, blameworthy to have a merely instrumental motivation to act
morally. So while {Gusto}'s false moral beliefs may be blamelessly held,
and may be relied on in action, they are not blamelessly relied on in
action.

Second, it is a commonplace that blameworthy acts have to be traceable
to something blameworthy. Indeed, something like this is a key premise
in an important argument for MIE. (Though recall that I expressed some
scepticism about that argument.) Now note that someone who did \emph{X}
being motivated by \emph{p}, where \emph{p} was both a false belief
blamelessly held, and a completely terrible reason to \emph{X}, would
still be blameworthy. And a natural story about why that's true is that
relying on an irrelevant consideration when deliberating about whether
to do something wrong is blameworthy.

But once we remember Williams's point about too many thoughts, we should
see that this is a very tight restriction. In a lot of cases, it is
wrong to directly bring considerations of the morality of the action
into one's deliberation. Rather, actions should be guided by the facts
in virtue of which the action is right or wrong. A false belief about
morality should generally be behaviourally inert, and so is not clear it
can ever be blamelessly relied upon.

\section{Adopting a Decision Procedure and Acting on
It}\label{adoptingadecisionprocedureandactingonit}

But, says the objector, it is not always wrong to think about right and
wrong and use this to guide one's actions. Indeed, this is what one
should do when faced with novel, hard cases. The objector I'm imagining
here is making a point similar to something Sigrun Svavarsdóttir
(\citeproc{ref-Svavarsdottir1999}{1999}) says in response to Michael M.
Smith (\citeproc{ref-Smith1994}{1994}). Smith argued that good agents
would never be motivated by right and wrong as such, but things that
made actions right and wrong. Svavarsdóttir argued, in effect, that this
should hold only in equilibrium. (See, for instance, her example of Mike
on page 209. I also discussed this example in section 3.5, making a
somewhat related point.) When an agent first reaches a momentous moral
decision, it is fine that they are moved to act by it. In the long run,
they should be able to be moved by the forces behind the moral truth.
That is, in the long run they should reach an equilibrium between their
moral beliefs and their motivations; things they believe to be good
should become directly motivating. But it is too much to require one's
motivations to turn on a dime, the instant belief changes, especially if
the decision is one where there are weighty interests on either side of
the scale.

This doesn't make any trouble for the {Sebastian} and {Belle} case from
the previous section, for we can easily add to the case that they have
been consequentialists for long enough that they should have by now
reached this kind of equilibrium. But it does mean that there could be
some cases where someone actually is moved by a moral belief, and is not
thereby blameworthy. It is easiest to see that happening in novel cases
where there is a lot at stake, morally speaking. In some of these cases,
we might think, virtue requires both careful moral deliberation, and
perhaps even acting on the result of that deliberation in advance of
one's motives lining up with one's resulting view of the good.

But it turns out there is a distinct problem these cases pose for the
view that moral ignorance is exculpatory. The problem is one raised by
Alex Guerrero (\citeproc{ref-Guerrero2007}{2007}), though I'm going to
put his point in a slightly different way. The worry is that defenders
of the view that moral ignorance excuses haven't been sensitive enough
to time. If any kind of moral mistake matters, it is a mistake at the
time of action. But it is all too easy, when thinking about cases, to
focus on mistakes at the time of belief formation. If there can be cases
where a belief is blamelessly formed, but the persistence of that belief
is blameworthy, these will come apart. And that's just what happens,
Guerrero argues, in some of the cases that we've just said looked most
promising for the view that moral ignorance excuses.

Consider the example, used by both Rosen and Guerrero, of the ancient
slaveowner. Rosen says that such people were often blamelessly wrong
about the morality of slaveowning. They were blameless because they
simply absorbed the prevailing morality of the day. No one around them
questioned whether slaveowning was right or wrong, so they were under no
obligation to do so either.

But, says Guerrero, look at things from the slaveowner's perspective. He
sees families being torn apart. He sees people being cast into chains
and thrown into dungeons. When faced with such appalling cruelty, it is
callous in the extreme to not wonder for a moment whether this is all an
acceptable way to treat people. Perhaps it is blameless to simply absorb
moral standards in childhood the way one absorbs a language. But
retaining those beliefs, not subjecting them to question when faced with
the misery one sees every day in the institution of slavery, is a very
different matter. (In general, slaveowning is a pretty terrible case for
the proponents of the view that ignorance is exculpatory, as argued by
by Michelle Moody-Adams (\citeproc{ref-MoodyAdams1994}{1994}).)

Guerrero puts forward these considerations in service of what he calls
`moral contextualism'. I agree with his reasoning, and his conclusion,
but not the name he gives to the view. What he defends isn't analogous
to the epistemic contextualism of Cohen
(\citeproc{ref-Cohen1986}{1986}), DeRose
(\citeproc{ref-DeRose1995}{1995}) and Lewis
(\citeproc{ref-Lewis1996b}{1996b}), but to the interest-relative
invariantism of Stanley (\citeproc{ref-Stanley2005}{2005}), and Fantl
and McGrath (\citeproc{ref-FantlMcGrath2009}{2009}). The closest
epistemological equivalents to his view came after his paper, in the
theories developed by Ganson (\citeproc{ref-Ganson2008}{2008}) and
Weatherson (\citeproc{ref-Weatherson2012}{2012}) that make belief
relative to the agent's interests, stakes and deliberations.

When an agent is abstractly deliberating the morality of slavery, or
even mindlessly absorbing the prevailing wisdom, the stakes are not so
high. But when they head to the auction block, or commission a
slave-catching party, the stakes are about as high as can be. Taking a
belief formed in such a low-stakes setting, and acting on it without
further consideration in a high-stakes setting, is blameworthy. Not
reconsidering the belief in light of the change in stakes is itself
blameworthy. So even if the formation of the belief that slavery is
permissible is blameless, the retention of it through the course of
deciding to acquire and retain slaves, need not be.

I've set up Guerrero's argument in terms of interest-relative theories
of belief. But his conclusion need not rest on anything quite so
controversial. Ross and Schroeder
(\citeproc{ref-SchroederRoss2014}{2014}) object to these
interest-relative theories of belief. One key problem they raise is that
such theories make change of belief without change of evidence too easy.
Ross and Schroeder propose instead that belief should be constituted by
defeasible dispositions to use propositions in inquiry. In high-stakes
settings, we retain the belief, but the disposition to use the
proposition in inquiry is defeated. It should be clear this is no help
to the proponent of the view that moral ignorance excuses. On Ross and
Schroeder's view, the retention of the belief that slaveowning is
permissible is not blameworthy. Indeed, it may be wrong to change that
belief on the basis of familiar evidence. But when one is actually
deciding to enslave other people, one should lose the disposition to act
on the belief. Just having the belief is no guarantee that one can, or
should, use it. And in such a high stakes case, one should not.

So we have, after a long detour, an answer to some of the rhetorical
questions Rosen posed earlier. {Bonnie} believes that she has most
reason to steal the cab; what do we expect her to do? On the
Ganson-Weatherson view, we expect her to lose the belief in light of the
stakes. On the Ross and Schroeder view, we expect her to lose the
disposition to act on the belief in light of the stakes. Either way,
there's no excuse for simply harming others on the basis of a prior
belief that one would be blameless in so doing.

\section{Calhoun on Blame and
Blameworthiness}\label{calhounonblameandblameworthiness}

The considerations raised so far suggest that there will be very few
cases of wrongdoing that are excused for the reasons that Rosen and
Zimmerman raise. The wrongdoing must be counterfactually sensitive to
the mistaken belief, and be motivated, or at least guided, by the
mistaken belief, and both of these things must be blameless, and the
belief itself, both in formation and retention, must be blameless, along
with the use of that belief in deliberation. And it turns out these
exceptions to the excuse condition are complementary, so between them
they cover a vast range of cases. Indeed, at this stage it would be
reasonable to speculate that there are no cases at all that satisfy all
of these constraints.

But while the constraints are tight, there are some interesting cases
that might comply with all of them. These are the cases that are at the
centre of the classic treatment of normative ignorance, Cheshire
Calhoun's ``Responsibility and Reproach''. Calhoun's position is more
complex than most of the contemporary views, and that complexity
reflects a sensitivity to where the really tricky cases are.

Calhoun thinks that blameless ignorance can excuse. But she also thinks
it is hard to be blamelessly ignorant of the wrongfulness of your
actions when the society you're in knows they are wrong. Blameless
ignorance will, in almost all cases, require social ignorance. (This
theme is echoed in more recent work by Miranda Fricker
(\citeproc{ref-Fricker2010}{2010}).) But in cases of social ignorance,
if we want to bring about social change, we may have no alternative but
to blame wrongdoers who we think, when engaged in philosophical
reflection, are blameless for their wrongdoing. I'm not going to engage
with that last point, as interesting as it is, save to note that it
might go some way to assuaging the intuitions of those who find the idea
that ignorance can excuse highly counter-intuitive.

Following Calhoun, I'll spend some time on cases that are relevant to
the way that structures of sexist oppression are maintained by actions
that are hardly oppressive in themselves. These acts are thoughtless,
but on their own they are almost harmless. The problem, of course, is
that these actions are not done on their own. There are a lot of sexist
actions that are much worse than thoughtless, but we'll set those aside
for now. So don't focus for now on the pimps and the pornographers, or
even on the fathers who go out of their way to provide more for their
sons than their daughters. Instead focus on casual, everyday sexism of
ordinary men in sexist societies. (`Microaggressions' in current
terminology.) Calhoun certainly wants to say that the things these
ordinary men do are wrong. Indeed, they are collectively extremely
wrong, as they collectively maintain a structure of oppression. But, she
says, the ordinary men involved are not blameworthy for their misdeeds.
There are three grounds for that claim in her paper.

First, blaming everyone would make us massively revise our views of the
virtue of many people around us.

\begin{quote}
If we assume, as we often do, that only morally flawed individuals could
act oppressively, the we will have to conclude that the number of
morally flawed individuals is more vast than we had dreamed and includes
individuals whom we would otherwise rank high on scales of moral virtue
and goodwill. The oddity of this conclusion forces serious questions
about the possibility of morally unflawed individuals committing serious
wrongdoing. ~(\citeproc{ref-Calhoun1989}{Calhoun 1989, 389})
\end{quote}

This conclusion shouldn't strike us as odd. The world is full of people
who have good features alongside serious character flaws. Indeed, it is
common in psychological research and in classic literature and in
history to see basically good people easily led down a path of moral
corruption. (Robespierre was one of the most morally decent people in
the French Revolution until he wasn't.) If a theory implied that
basically half the population fell under the description \emph{largely
good but with a big moral flaw} I wouldn't see that as a fatal flaw in
the theory.

In the case of casual sexism, the argument that we couldn't conclude
that everyone is flawed seems particularly strange.Let's say we don't
want to blame the ordinary man for the part he plays in oppression by
everyday acts like using demeaning terms like `girls' to refer to women.
Still, most of these ordinary men provided considerably more resources
for their sons than their daughters. And of those that did not, most
were `off the hook' solely because they didn't have both sons and
daughters. And that unfair distribution, or at least disposition to
distribute unfairly, isn't the kind of individually minor wrong that
Calhoun wants to excuse. It isn't that surprising to think that most men
in sexist societies are at least somewhat blameworthy.

Second, Calhoun notes that we should not simply assume that causing harm
is ``the same as being responsible for the harm''
~(\citeproc{ref-Calhoun1989}{Calhoun 1989, 392}). This is surely right -
and I want to come back to it below. It is important that we respect the
conceptual difference between wrong and blameworthy action, and return
to that distinction.

But the biggest consideration, one that runs through Calhoun's paper, is
that there are cases where the wrongdoer has no reason to reconsider
their false moral belief. Calhoun shows that many false moral beliefs
are not be like that. If society disagrees with one's false moral
beliefs, then one has frequent occasion to reconsider the belief. So the
false moral beliefs one has no reason to reconsider will only be one's
shared by the community. For many other false moral beliefs, the thing
that makes actions wrong will be a reason to reconsider the belief at
the moment of action. (Guerrero's response to Rosen turns on a similar
point.) But maybe that isn't true for all false moral beliefs. Maybe it
is only true if the wrongness rises above a certain threshold. Not every
potential harm is a ground for reconsidering one's views.

Let's consider a very different kind of example to Calhoun's in order to
see the possibility I have in mind. Assume that {Inka} lives in a city
with a busy underground train system. It is possible to delay a train's
departure from a station for a few seconds by standing in a doorway
preventing a train leaving. {Inka} believes, as do most people around
her, that it is acceptable to so delay a train in order to let a friend
running for the train catch it. Indeed, this is widely taken to be a
requirement of friendship. It is also a false belief. Costing the other
600 people on the train 10 seconds each of extra travel time is a very
bad thing to do in order to save one friend the five minutes they would
have to wait for the next train. It is, in {Inka}'s world, as morally
bad as trapping one person in a train for 100 minutes.\footnote{I think
  holding a train for a friend to catch it is also wrong in \emph{our}
  world, for just this reason. But the argument doesn't turn on this
  assumption.}

Now {Inka} is in a position where she can help a friend catch a train by
blocking the doorway, and letting her friend run the last few steps to
catch it. She does just this. It's not true that what {Inka} should have
done instead was stop and think about her action. Any deliberation and
the moment for action will have passed. Even if {Inka} did have time to
contemplate action, it isn't clear she has any reason to do so. There is
no person who she is harming so severely that this harm makes it
compulsory for her to reconsider her actions. After all, each of them is
only being delayed in the train for a few seconds more, and people get
delayed in subways for seconds all the time. And {Inka} presumably
thinks that small delays like this are of no moral significance.

Putting these points together, we can sketch a way in which moral
ignorance might excuse. Assume that all of the following conditions are
met.

\begin{enumerate}
\def\labelenumi{\arabic{enumi}.}
\tightlist
\item
  S is, as a matter of policy, disposed to do \emph{X} in circumstances
  C.
\item
  S has this policy for the same reasons that she has the false moral
  belief that doing \emph{X} in circumstances C is permissible. This
  means that she would (eventually) lose the policy if she lost the
  belief, and that the adoption and maintenance of the policy is part of
  a general disposition to adopt policies she regards as permissible.
\item
  It is commonly held in S's community that it is permissible to do
  \emph{X} in circumstances C.
\item
  S is in circumstances C.
\item
  If S does \emph{X} right now, no one will be greatly harmed.
\item
  S has no time or reason to reconsider her policy of doing \emph{X} in
  C before it becomes time to act.
\item
  S does \emph{X}.
\end{enumerate}

None of the arguments I've offered so far refute the idea that S is less
blameworthy in these cases than is a typical person who does \emph{X}.
Even without having necessary and sufficient conditions for `acting from
ignorance', it is plausible that S acts from ignorance and not just in
ignorance. And it's not true that the situation S finds herself in calls
for reflection and deliberation rather than action. So nothing I've said
so far rules out S's actions being excused. I'm going to defend three
claims about this case.

\begin{itemize}
\tightlist
\item
  We need to know more about S to know how much her situation excuses
  her behaviour.
\item
  Even when S has a partial excuse, it isn't the fact that she is
  morally ignorant that excuses. But the moral ignorance is relevant;
  the reasons that she is morally ignorant will typically be the reasons
  that she is excused.
\item
  In any case, there are no such things as full excuses for wrong
  action, so S couldn't have a full excuse for wrong action.
\end{itemize}

I'm going to defend the first and second claims in this section, and the
third claim at the end of the chapter. The defences will turn on
considerations arising from cases of practical irrationality.
\footnote{As I mentioned above when introducing the cases of {Abbott}
  and {Costello}, I only saw the importance of these cases in
  discussions with Claire Field. Her work in progress has a different,
  and interesting, take on the cases I'm about to describe.}

Imagine that you have the following views. You think that the
conclusions of ``Famine, Affluence and Morality''
~(\citeproc{ref-SingerFAM}{Singer 1972}) are basically correct. And
you're a Strawsonian about moral responsibility. And you know of
someone, call him {Gloucester}, who doesn't give a lot to charity. But
you also know that {Gloucester} is disposed to massively increase his
charitable giving were he to simply be presented with Singer's drowning
child argument. If he were presented with that argument, he would have
an ``A Ha!'' moment, and see that he was required to give much more to
charity. What should you say about {Gloucester}'s responsibility for his
current insufficient charitable givng?

I think you should say that {Gloucester} is largely blameless. He would
be completely blameless after he reads Singer and starts donating. But
reading the paper does not change his fundamental desires. And the
Strawsonian picture is that these fundamental desires are what makes him
praiseworthy or blameworthy. So he isn't particularly blameworthy now.

Does this mean {Gloucester} is a case where moral ignorance is excusing?
I don't think it does. {Gloucester} is practically irrational. Right
now, before reading Singer, he values charitable donation over spending
money on himself. But he doesn't act on those values. He also doesn't
realise that those are his values. And both the failure to act and the
failure to realise have a common cause - his practical irrationality.

Now {Inka} might be just like {Gloucester}. It might be that as soon as
you present her the simple argument for why holding the train doors is
wrong, she has an ``A ha!'' moment, and sees that what she is doing is
wrong. That is, she sees that by her own lights, it was better to not
hold the train door. And what matters is not that she sees this, but
that it was true all along that she desires implied that train doors
should not be held in these cases. She presumably also had other
desires, inconsistent with those, that implied that the doors should be
held open. And she is somewhat blameworthy for having those desires; but
the existence of the correct desires mitigates her blameworthiness.

But {Inka} might be different. It might be that she needs to change her
desires to change her actions. She might be like someone who gives more
to charity after visually seeing the suffering of the impoverished. Such
people, I think, change their desires upon seeing the suffering. And
they are blameworthy for not doing more in the first place. (At least if
the conclusion of Singer (\citeproc{ref-SingerFAM}{1972}) is correct,
and I'm not assuming it is for anything more than the duration of this
argument.)

So that's my overall conclusion about the people who commit minor wrongs
out of habit, while not believing they are wrong, and not having social
pressure to change. Some of these people all along had desires that the
wrongs not be committed. They were practically irrational in committing
the wrongs. They have a partial excuse. They are also morally ignorant.
But the moral ignorance does not explain the excuse. Rather something
else, their practical irrationality, explains both the excuse and the
ignorance. I don't want to take a stand on whether this is rejecting
MIE, since it says ignorance never makes one have an excuse, or
endorsing a weak version of MIE, since it says that ignorance goes along
with having an excuse in some cases. That would require more careful
parsing of the words of MIE-defenders than seems useful to do here.

Instead I'll turn to one other aspect of Calhoun's examples. {Inka} is a
case of habitual minor wrongdoing, in cases where habitual rather than
reflective action is called for. Now we'll spend a bit of time on
wrongdoing that is culturally approved.

\section{Moral Mistakes and Moral
Strangers}\label{moralmistakesandmoralstrangers}

Nothing I've said so far explains why {JoJo} is less blameworthy than
his father. And many philosophers hold it to be very intuitive that he
is, and that something like MIE explains why. My preferred account of
{JoJo} relies on work by Elinor Mason (\citeproc{ref-Mason2015}{2015}).

Mason argues thinks that the debate about moral ignorance has been
oversimplified in a number of ways. She argues that there are really two
kinds of blame, what she calls `ordinary blame' and `objective blame'.
And MIE is completely wrong about objective blame. But it is correct for
ordinary blame, provided we are careful to restrict clause 2 to full
beliefs of the agent. This restriction excludes where the agent knows,
or even suspects, deep down that they are doing something wrong. And she
is much more willing than Rosen or Zimmerman to say that even when
someone does fully believe that what they are doing is right, this could
be due to a blameworthy kind of motivated reasoning. But all that said,
she does think that a suitable version of MIE could hold for ordinary
blame. Here is an important part of what she means by `ordinary blame',
and why she thinks MIE is correct about it.

\begin{quote}
Normally, we blame each other for what we deliberately do. And if we
find out that some piece of behavior was not deliberate, we let the
agent off the hook. This is ordinary everyday blameworthiness. Ordinary
blameworthiness is based on subjective wrongdoing. When ordinary people
behave badly, they are usually, at some level (and this need not be the
fully conscious level that Rosen and Zimmerman require), aware that they
are doing it. Tony Blair did many wrong things during his time as Prime
Minister, and it seems plausible that he knew, at some level, that these
actions were wrong. He is not outside of our moral community: he did not
seem to have the wrong end of the stick about what morality required.
Rather, he was too easily swayed by the wrong sorts of reason. He did
not try hard enough. Much of his ignorance, both factual and moral, was
motivated ignorance or affected ignorance. He was (and is) thus
blameworthy in the ordinary way. When we blame people for their akratic
acts, we take it that they have the capacities and moral knowledge that
we have: they are part of our moral community in that they share the
basic standards that we hold ourselves to.
~(\citeproc{ref-Mason2015}{Mason 2015, 12})
\end{quote}

There are (at least) three interesting claims being made here.

\begin{enumerate}
\def\labelenumi{\arabic{enumi}.}
\tightlist
\item
  Only people who are in our moral community are subject to ordinary
  blame. We might have contempt, or disdain, or disrepect for people in
  alien moral communities. Indeed, we might hold them subject to
  objective blame. Indeed, having contempt, etc for them might be a way
  of objectively blaming them.
\item
  When an ordinary person in our moral community does something wrong,
  they know that what they are doing is wrong, at least at some,
  possibly sub-conscious, level, or they are engaged in blameworthy
  motivated reasoning.
\item
  Only people who know that what they are doing is wrong, at least at
  some, possibly sub-conscious, level, are subject to ordinary blame,
  unless they are engaged in blameworthy motivated reasoning.
\end{enumerate}

I'm not sure whether Mason intended to argue from 1 and 2 to 3; the text
doesn't make that appear to be the central strand on her reasoning.
(Cases like JoJo's are much more important.) But as I've set this up,
there is a valid argument from 1 and 2 to 3. And it's a pretty
interesting argument for MIE.

Premise 2 in this argument is clearly an empirical claim. We saw a
version of it in Calhoun's picture that blameless ignorance typically
requires societal ignorance. People whose society generally disapproves
of their moral theories have sufficient reason to temper those theories
at least enough so they should not be acted on. And the general picture
is continuous with what John Kenneth Galbraith was articulating when he
described modern conservatism as ``the search for a superior
justification for selfishness'' ~(\citeproc{ref-Galbraith1964}{Galbraith
1964, 16}). The picture is that deep down, people who do wrong know that
they are being selfish, or discriminatory, or in some way immoral, but
they attempt to, or perhaps successfully motivate themselves to, justify
this behaviour in moral language.

But unless the claim about moral community is taken to, by stipulation,
include only people who don't have any thorough-going, unmotivated,
false moral beliefs, I don't see why we should accept this claim. Here
are three very broad classes of exceptions.

{Hacker} is an anti-abortion activist. He knows that many people think
abortion is permissible, and for this reason he refrains from using
violence in support of his anti-abortion crusade. But he does campaign
for regulations that are used to close abortion clinics. And he hacks
into the computer systems of abortion clinics to render their systems
inoperative, and make it harder for them to carry out abortions. He
thinks this work is morally mandatory, so he certainly thinks it is
permissible. I doesn't strike me as plausible that he believes, deep
down, that he is doing something wrong. We can stipulate that he knows
that what he is doing reduces the autonomy of women seeking abortions.
But since he regards abortion as morally equivalent to murder, he
doesn't think that reducing the autonomy of would be murderers is a bad
thing.

{Guy} is a regular American, with a regular American diet. This includes
generous helpings of factory farmed meat. To the extent that he worries
about this, it is just because of the health consequences. He doesn't
think animals bred for food have any moral standing. (Though he does
think people should be jailed for promoting dog fighting.) The interests
of humans, he think, come first, and factory farming is justified
because it lowers the cost of meat. It seems consistent to think that he
is very badly wrong about this, and yet he doesn't have any internal
sense, even deep down, that this is so.

Finally,{Rush} is in a hurry to get home. It's an emergency. Well, it's
not an emergency really, but his child is hungry and is screaming, so it
feels like one. So {Rush} drives somewhat aggressively, and somewhat
impolitely, and in so doing wrongs other road users. He actually does
know that what he is doing is usually wrong. But he also knows that what
he's doing is acceptable, and perhaps mandatory, in an emergency.
(Assume that his driving, though inconvenient for other road users,
would be exactly the right thing to do if he needed to race his child to
hospital. So Rush really knows that there is an exception here; he's
just mistaken about its scope.) And he is, like most of us, rather too
willing to classify his own challenges as falling into the scope of that
exception for emergency. Note that he is making a moral mistake here,
not a factual one. The mistake is not about how to describe what it's
like in his car. He knows that he has a hungry but not actually
endangered baby who is screaming. What he's wrong about is whether this
is enough to trigger an exception to the normal moral rules about
carefulness and politeness on the roads. Maybe some real people who are
like {Rush} know that there is some special pleading here. But I don't
think they all do; some people make moral mistakes that are convenient
without being at any level motivated.

{Hacker}, {Guy} and {Rush} all do something they think is permissible.
And they are all wrong. And they are all in our moral community, at
least as I'd understand the expression `moral community'. And in none of
those cases should their mistaken moral beliefs lessen their
responsibility. That's a striking contrast with {JoJo}.

And this all suggests a simple explanation for the intuitions about
{JoJo}. We intuit that there is a kind of blameworthiness, or perhaps a
degree of blameworthiness, that only applies when the agent is part of
our moral community. I'm not endorsing this intuition; I'm just trying
to explain why we think what we do about {JoJo}. What I do think is that
we intuit that {JoJo} is, in virtue of his depraved upbringing, outside
our moral community in a way that his father might not be. This idea,
that intuitions about blameworthiness track membership in a moral
community rather than moral ignorance, seems like a better explanation
of our intuitions about {JoJo}.

This view is similar to what Miranda Fricker
(\citeproc{ref-Fricker2010}{2010, 152}) calls the `relativism of blame'.
She holds that

\begin{quote}
Blame is inappropriate if the relevant action or omission is owing to a
structurally caused inability to form the requisite moral thought.
~(\citeproc{ref-Fricker2010}{Fricker 2010, 167})
\end{quote}

I don't think this can be quite right, because moral thoughts are never
requisite for an action. Fricker talks about a schoolmaster a few
decades back who canes students, but could not be expected to realise
that this common practice was immoral. Well, maybe he couldn't, but he
doesn't have to in order to not beat students. He just has to not beat
them. {Huck} Finn didn't have to realise it was wrong to turn in
fugitive slaves in order to not turn in Jim; he just had to not turn Jim
in. But set this point aside, and assume the issue is not `requisite'
moral thoughts, but simply corresponding ones.

And I don't think we should necessarily insist that we are dealing with
structurally caused thoughts. A young child with a weakly developed
sense of morality may be excused for their wrong actions. And that's
because they are not fully part of our moral community. But it's a
stretch to call this a structurally caused inability to form moral
thoughts. Better to just say that the mid-Century schoolmaster, the
ancient slaveholder and the young child are not full members of our
moral community, and that's why blame is inappropriate.

(Joseph Shin pointed out in a seminar at Michigan that it is an
attractive feature of MIE that is offers an explanation of why children
are typically exempt from moral blame. He's right about this, and this
strikes me as a point that future proponents and opponents of MIE should
engage with.)

But all that said, there is something plausible about Fricker's
relativism of blame. To blame someone is to stand in a relationship to
them that is most natural in the context of some pre-existing
relationship, or at least some pre-existing commonality. People who are
outside our moral community are not so much excused for their wrong
actions as exempt from blame.

Moral communities are informal entities, quite unlike countries.
Membership of a common moral community could be a matter of degree. So
we could make a gradational version of Fricker's relativism of blame. A
wrong-doer is only blameworthy to the extent they are a member of our
moral community. As a corrollorary they are only fully blameworthy if
they are fully a member of the community.

This approach agrees with intuitions about cases. {JoJo}'s upbringing is
so strange that he's outside our community in many respects. So he's
exempt from certain kinds of blame. {Hacker}, {Guy} and {Rush} are in
our moral community, although they make moral mistakes. Those mistakes
don't exempt them from blame. On this way of thinking, the argument from
cases starts with plausible premises, but overgeneralises. What is
relevant is not that {JoJo} thinks the token acts he performs as a
vicious dictator are permissible, but the broader moral system he is a
part of. MIE goes wrong by focussing on local beliefs of the wrong-doer;
we should be looking at structural features of their belief system.

I've said this is a plausible explanation of why we have certain
intuitions, but that's a long way from saying those intuitions are true.
I don't have to take a stand on that question, and I don't know what the
right thing to say is. My best guess is that figuring out what to say
about {JoJo} requires settling some very big picture questions about the
role of blame in a moral theory. One possible consequence of the ideas
I've just been sketching is that we shouldn't \emph{blame} {JoJo}, but
we should have some other negative person-level evaluation of him.
That's to say, we should still in some good sense hold him responsible
for his actions - though maybe not in the way that we hold people
responsible when we blame them. It is not uncommon to see philosophers
identify moral responsibility with susceptibility to praise and blame,
and if the picture I've been building to here is right, that
identification must be wrong. We could treat {JoJo} as responsible by,
for example, being angry at him, or having contempt for him, even if we
don't blame him, or think blame would be the right kind of attitude to
hold. While I'm not going to settle any questions that big, I will end
with some other points about blame that help make sense of the views
I've defended in this chapter.

\section{Two Approaches to Blame}\label{twoapproachestoblame}

Neil Levy (\citeproc{ref-Levy2005}{2005}) helpfully distinguishes two
approaches to responsibility.

\begin{quote}
There are accounts that hold that an agent is responsible for something
(an act, omission, attitude, and so on) just in case that agent has --
directly or indirectly -- chosen that thing, and there are accounts that
hold that an agent is responsible for something just in case that thing
is appropriately attributable to her. \ldots{} Call these accounts
volitionist and attributionist accounts of moral responsibility.
~(\citeproc{ref-Levy2005}{Levy 2005, 2})
\end{quote}

The view of responsibility I'm taking here is very much in the spirit of
the views that Levy calls attributionist. In particular, I've relied
heavily on the idea that someone can be responsible for not
reconsidering their moral views. Yet we rarely choose to deliberate.
Indeed, the notion of choosing to deliberate is of dubious coherence.
Once we are thinking about whether to look more closely at, say, our
belief that \emph{p}, we already are to some extent deliberating about
\emph{p}. So some things that we are responsible for, namely failures to
deliberate, are not choices. And that's contrary to the view that Levy
calls volitionist.

The argument of the last paragraph is not novel; it draws heavily on the
arguments for attributionism by Angela A. M. Smith
(\citeproc{ref-AngelaSmith2005}{2005}). It is sometimes thought that
agents are never really responsible for certain failures, such as
failures to deliberate. What they are responsible for are the actions
that produce a disposition to deliberate or not in the appropriate
circumstances. This seems rather implausible to me, for reasons set out
by Manuel Vargas (\citeproc{ref-Vargas2005}{2005}). But perhaps this
kind of consideration can be used to produce a form of normative
externalism that is compatible with volitionism. After all, normative
externalism doesn't require rejecting volitionism, at least as Levy has
defined it here. Consider again the ancient slaveowner in the example
that Rosen and Guerrero discuss. The slaveowner does choose to own
slaves, and it is the slaveowning for which he is blameworthy. He
doesn't choose to do the wrong thing as such. But it is a very stringent
condition on responsibility that one choose to do the wrong thing as
such. Someone could be a volitionist and a normative externalist
provided they reject that stringent condition.

Levy argues that one problem for the attributionist view is that it
can't distinguish between the wrong and the blameworthy. As we saw, this
idea is also behind one of Calhoun's arguments. She thinks that some
arguments that men are blameworthy for their part in maintaining an
oppressive society turn on conflating wrong and blameworthy acts. And
intuitively this is a conflation between two distinct concepts. As Levy
notes, attributionists can find some difficulty in making sense of the
distinction. Indeed, he quotes two prominent attributionists, Robert
Adams (\citeproc{ref-Adams1985}{1985}) and Gary Watson
(\citeproc{ref-Watson1996}{1996}) explicitly saying that thinking
something is wrong is, to some extent, blaming the wrongdoer. To
conclude this discussion of responsibility, I want to note that there is
a possible view that holds that blameworthiness and responsibility are
conceptually distinct, even though any wrong act is blameworthy. This
view is particularly congenial to the normative externalist.

On the view I have in mind, wrongfulness and blameworthiness differ in
three respects.

\begin{enumerate}
\def\labelenumi{\arabic{enumi}.}
\tightlist
\item
  They have different targets. It is, in the first instance, actions
  that are wrong, but agents who are blameworthy.
\item
  They differ with respect to time. As noted above, an action can become
  less blameworthy over time, but does not become less wrong.
\item
  They frequently differ with respect to degree.
\end{enumerate}

The last point is true because there are such things as partial excuses.
Indeed, it is arguable that all excuses are partial excuses. By a
partial excuse, I mean something that reduces an agent's
blameworthiness, without fully absolving the agent. It is helpful to
think of a familiar analogy from the criminal law. Sometimes, special
circumstances can provide an agent with a defence; the circumstances
mean that the agent was not guilty of any crime even though they
fulfilled all the elements of the crime. (Defence of another is often
thought of this way.) In other cases, circumstances mitigate an agent's
guilt. They don't provide a reason for finding the agent not guilty, but
they provide a reason for imposing a lesser punishment. In this context,
a full excuse would be something that meant there was no punishment at
all that was appropriate, but which did not provide a reason for finding
the person not guilty. This is a strange combination. Indeed, it may be
incoherent. The finding of guilt is itself a punishment. The same thing
is true in the moral case. Excuses typically mitigate responsibility.
But things that absolve an agent from responsibility are usually
defences, which imply the agent didn't do anything wrong. Holding that
the agent has no defence for what they did, but they are fully excused,
is an unstable position.

The reasoning of the last paragraph suggests that the following
principle is plausible:

\begin{itemize}
\tightlist
\item
  If S's action \emph{X} at \emph{t} is wrong, then S is to some extent
  blameworthy at \emph{t} for \emph{X}.
\end{itemize}

This principle does not imply that S's action is blameworthy, only that
S herself is. And the principle does not imply anything about how
blameworthy S is at later times. And it does not imply that S is
blameworthy in strict proportion to the wrongness of her action. Indeed,
none of these three claims is plausibly true. So here we have three
important conceptual distinctions between wrongfulness and
blameworthiness. But the principle does amount to a kind of
attributionism, one that is very friendly to normative externalism. So
the normative externalist, and the attributionist, need not be guilty of
any conceptual confusion.

Here is another way to defend the principle. It is sufficient to count
as blaming someone for an action that you in some way harm them, or
sanction them, for performing the action, on non-consequentialist
grounds. To believe that someone has done something wrong is to harm
them. To improperly believe someone has done something wrong is indeed
to wrong them ~(\citeproc{ref-BasuSchroeder2018}{Basu and Schroeder
forthcoming}). It's not a wrong if the belief is well-grounded, but it
is still a harm. And, like most beliefs, it isn't held on
consequentialist grounds. So to believe that someone has done something
is wrong is already to blame them - at least a little. When philosophers
say that some wrong actions are not blameworthy, I think it would be
better to say that no further blame, beyond believing the person
performed the wrong actions, is fitting. In cases like {Inka}'s, or
{Gloucester}'s, that might be the right thing to say.

\chapter{Double Standards}\label{doublestandards}

This chapter wraps up three loose ends. First, I discuss whether
hypocrisy is a vice. I'm going to argue that it isn't, and say why this
matters to the broader issue of normative externalism. Second, I'm going
to say why I haven't relied on arguments about inter-theoretic value
comparison to argue for normative externalism. Roughly, I think those
arguments over-generalise, so it is a mistake to use them here. And
finally, I'll say something about which aspects of normative externalism
are central to the view, and which are peripheral. This will serve as a
summing up of this part of the book, and help bring into focus how the
different parts fit together.

\section{Hypocrites}\label{hypocrites}

{Janus} has the following odd set of views. He has basically correct
views about the physiology of animals that are used for livestock. On
the basis of these views, and some straightforward philosophical
reflection, he has concluded that meat eating is almost certainly
impermissible. Given the philosophical evidence available to him, this
isn't a particularly irrational view to have, but it is false in the
world he is in. And this is good luck for {Janus}, since he eats meat at
every opportunity.

Here's a natural objection to the simple externalism I have so far
defended.

\begin{enumerate}
\def\labelenumi{\arabic{enumi}.}
\tightlist
\item
  {Janus} is, in some way or other, criticisable.
\item
  According to simple forms of normative externalism, he is not
  criticisable, since what he does is not wrong.
\item
  So simple forms of normative externalism are wrong.
\end{enumerate}

I'm going to primarily push back against premise 1 here. While there is
some intuitive pull to the idea that {Janus} is criticisable, I will
argue that intuition can easily be explained away. But first, I'll start
with a couple of clarifications of the case.

\subsection{Why hypocrisy?}\label{whyhypocrisy}

Some readers may be wondering why I'm talking about {Janus}`s hypocrisy,
rather than his akrasia. After all, 'akrasia' is the standard
philosophers' term for a person who acts against their better judgment.
And hypocrisy is attached to doing other than what one says is right, as
much as doing what one doesn't think is right.

On that last point, I think hypocrisy applies to more people than just
the character who says one thing and does another. It seems fine to me
to describe Robinson Crusoe as hypocritical if he comes to a firm
opinion that some action is wrong, and then goes and does it. Perhaps we
can understand this case as Crusoe making a speech to himself, and then
being hypocritical for acting against his (inner) speech. But that feels
at best like a forced reading of the case. It is more natural to say
that actions can be hypocritical even if they don't conflict with any
prior speech of the agent, if they do conflict in some way with her
judgments.

Still, why not describe this as akrasia? Well, for one thing the term
`akrasia' is barely a term of English. There is the English expression
`weakness of will', but this has very little to do with the phenomenon
we're considering here ~(\citeproc{ref-Holton1999}{Holton 1999}). To the
extent we understand what it is to be akratic, we understand it
stipulatively. But as we'll see below, there are some tricky borderline
cases of hypocrisy. I would like to use our intuitive judgments to help
clarify those cases. But I can hardly ask the reader to share intuitions
about akrasia, since it is a notion introduced by stipulation.

\subsection{The Hypocrite and the
Rationaliser}\label{thehypocriteandtherationaliser}

I've known the occasional person like {Janus}, but in many ways he seems
like a rather foreign character. A much more common character, at least
in the circles I typically move in, is the person who comes up with
rationalizations for their particular behaviour. (I'm drawing heavily in
this subsection on what Eric Schwitzgebel
(\citeproc{ref-Schwitzgebel2011}{2011}) says about rationalizations.) It
is not obvious why one would think that that making these
rationalizations is a moral improvement. Indeed, I suspect I prefer the
character who faces up to their own moral failures to the one who
constantly finds a spurious reason to justify their own behaviour.

I only bring this up to note that if you're like me, it might be a
little hard to make firm judgments about {Janus}. After all, {Janus} is
so foreign, so Other, that he doesn't attract our normal sympathies, and
without those sympathies we aren't great at moral judgments. I'm not
going to lean heavily on this point, but I do think that it is a reason
to suspect that we might not be the best judge of people like {Janus}.

\subsection{Recklessness and Character}\label{recklessnessandcharacter}

My preferred thing to say about {Janus} is that his action is not in
itself criticisable, but that actions like this are revealing of a
character flaw that is worrying. (This diagnosis borrows from some
suggestions Hawthorne and Srinivasan
(\citeproc{ref-HawthorneSrinivasan2013}{2013}) make about how to analyse
a parallel case in epistemology.) The character flaw is not taking the
interests of others as seriously as one should, especially in comparison
to one's own interests. One way to do that is adopt theories or
standards that are helpful to oneself. Another is simply to ignore what
one thinks are the appropriate standards in one's actions.

Now {Janus} is not, by hypothesis, doing anything wrong. He puts his own
desire for meat above the interests of the animals who are killed to
provide the meat. And by hypothesis that's fine, since his interests are
sufficiently more significant. But humans are notoriously bad at
balancing their own interests with the interests of others. Someone who
acts so as to promote their own pleasure over the interests of others,
even in cases where they have judged the others' interests to be more
significant, seems very prone to selfish action. We want people to act
against their own interests, when the interests of others are
sufficiently strong. It is true that {Janus} does not violate this
desideratum. But since he thinks he violates it, it is likely that he
will actually violate it next time he gets a chance. It is reasonable to
worry about the character of such a person, even if the particular thing
they are doing is not objectionably selfish.

To support this interpretation of the case, let's compare {Janus} to
someone whose actions against their moral judgments are not
self-serving. {Yori} is both a parent and an academic. It's hiring
season, and {Yori} has a bunch of job applications to read. He has read
them all closely, but worries that he really should go back and look at
a few a bit more closely before tomorrow's meeting. But he doesn't have
time to do that and attend his child's soccer game, and he knows his
presence at the game will mean a lot to his child. {Yori} thinks that
his professional obligations are stronger than his parental obligations,
so he should re-read the files. But he can't bring himself to disappoint
his child in this way, so he goes to the soccer game. He doesn't get any
pleasure from this. He finds the soccer deathly boring. And while he
would feel guilty if he skipped the game, and this feeling would not be
pleasurable, as it is he feels equally bad about the files. Now it turns
out {Yori} has a bad theory of duty. Given the work he has already put
in, his parental duties are stronger than his professional duties, so he
does the right thing. And he even does the right thing for basically the
right reason, being motivated by his child's feelings. (I'm assuming
here that being unable to bring oneself to disappoint a child is a
perfectly acceptable way to be moved by a child's feelings. If you don't
agree, you may have to change the story, but for what it's worth I think
that assumption is true.)

While we might criticise {Yori} for his false moral theory, we should
not criticise his action in any way. He does the right thing, and does
it for the right reason, even if he falsely believes that this very
reason is not a strong enough reason. {Yori} is, just like {Huck}leberry
Finn, a case of inadvertent virtue.

{Yori} is like {Janus} in one respect; he acts against his judgment of
what is best to do. And he is unlike {Janus} in a different respect; he
does not act selfishly. The appropriate attitudes to take towards {Yori}
and {Janus} are very different; the kind of negative attitude that is
natural to take towards {Janus} is uncalled for when it comes to {Yori}.

And this suggests that the explanation for that negative attitude
towards {Janus} comes from the respect in which he differs from {Yori},
not from the respect in which the two of them are alike. That is to say,
{Janus} is criticisable, to the extent he is, because he acts selfishly,
not because he acts against his best judgment.

Selfish action is not always wrong. Sometimes, you should put yourself
first. Happily, {Janus} is in such a situation. So why do we criticise
him? It is not because he does something wrong, but because he reveals
bad character. If he does something wrong, it is something that {Yori}
too does wrong; yet {Yori} does nothing wrong. {Janus}'s actions reveal
a worryingly selfish personality. Even if this very action wasn't
selfish, it's a good bet that he will soon act in a way that's
objectionably selfish. That's not true about {Yori}.

And this is all evidence that hypocrisy isn't in itself a vice. {Yori}
is hypocritical; he thinks he be reading job applications, but instead
finds himself at a children's soccer game. But this isn't something bad
about his actions, or even his character. Indeed, it would have been
worse to act in accord with his false views about duty.

And that is the key thing for normative externalists to say about
hypocrisy. Often, the hypocrite does something bad, and that should be
criticised. In many other cases, the hypocrite reveals a character flaw
that will, quite probably, lead to bad actions in the near future. That
too should be criticised, at least with the aim of preventing the bad
actions from happening in the near future. But sometimes the hypocrite
simply does the right thing, and ignores their false moral views. That
isn't bad, and isn't even a character flaw. There is nothing wrong with
simply doing the right thing, even if one doesn't recognise it.

\section{Value Comparisons}\label{valuecomparisons}

There is a prominent argument against normative internalism that I have
not discussed here. This is the problem of inter-theoretic value
comparisons ~(\citeproc{ref-Sepielli2009}{Sepielli 2009};
\citeproc{ref-Hedden2015}{Hedden 2016a}). I haven't discussed it because
I don't think it is as big a problem for internalism as some of my
fellow externalists do. But it is an interesting problem, and thinking
about how it could be solved shows some constraints on the form of a
viable internalism.

{Ulysses} is trying to decide between two problematic forms of action.
If he does action A, he will break a promise to his dear wife,
{Penelope}, but he will also improve the welfare of hundreds of people
on the island he is visiting. If he does action B, he will be able to
keep the promise, but he will lose the opportunity to help the people
around him. {Ulysses} is also torn between two moral theories. One is a
welfare consequentialist theory that says he should do action A, and the
other is a deontological theory that says he should do action B. Let's
assume that he is reasonable in being so torn. (This is a huge
simplification, but the problem doesn't change with fewer
simplifications, it just becomes harder to state.) What should he do?

The externalist says that we need to know whether the consequentialist
or the deontological theory is correct, and that will determine what
{Ulysses} should do. But some internalists don't like this answer. They
note, correctly, that it is really hard to work out what the right moral
theory is. And they think that it shouldn't be so hard to work out what
to do. So {Ulysses} must be able to do something with just the knowledge
he has. (Or so say the internalists. I obviously disagree, but we'll set
aside my disagreement for the moment.)

What options does {Ulysses} have? If he knew one of the moral theories
was more likely than the other, perhaps he could just do the thing
recommended by the more likely moral theory. But we've assumed that
{Ulysses} knows no such thing. So perhaps the thing to do is to maximise
expected moral value. To do that, we just need to know whether
\emph{x}~\textgreater~\emph{y}, where \emph{x} and \emph{y} are defined
as follows:

\begin{itemize}
\tightlist
\item
  \emph{x} = the amount which action A is better than action B,
  according to the version of consequentialism {Ulysses} takes
  seriously.
\item
  \(y\) = the amount which action B is better than action A, according
  to the deontological moral theory that {Ulysses} takes seriously.
\end{itemize}

The problem is, how are we going to find out whether
\emph{x}~\textgreater~\emph{y}? We can't look to either of the moral
theories that {Ulysses} takes seriously. They can only answer questions
internal to themselves; they can't say how to compare something that's
wrong by their lights to a kind of wrongness taken seriously by a rival
theory. What we need is a comparison of wrongness across theories. That
is, we need an inter-theoretic comparison of wrongness. Or, as it is
sometimes put, we need an inter-theoretic value comparison. (It
sometimes seems to me that there is an implicit consequentialism built
into this way of putting the problem, but set that worry aside.)

Now there are a number of moves that have been proposed for how to get
around this impasse, and a number of criticisms of each of them. What
I'm interested in here is the following argument.

\begin{enumerate}
\def\labelenumi{\arabic{enumi}.}
\tightlist
\item
  If normative internalism is true, there is a solution to the problem
  of inter-theoretic value comparison.
\item
  There is no solution to the problem of inter-theoretic value
  comparison.
\item
  So, normative internalism is false.
\end{enumerate}

Premise 2 of this argument is false. I don't say that because I know the
solution, or because I have an argument in favour of a particular
solution. What I do have is an argument that a solution must exist. The
argument turns on considerations about democracy and representation.

{Saraswati} is a good democratic representative. She currently faces a
tricky decision between two options. One option will maximise welfare,
but breach some moral principles that are often held to be important.
The other option will do neither of these things. Now as it turns out,
the true moral theory in {Saraswati}'s world is a kind of pluralism that
says it is morally permissible to make either choice when acting for
oneself, and making this kind of choice. But {Saraswati} isn't acting
for herself, she is a representative. And representatives have a duty to
represent, at least in cases where the people want them to act in
morally permissible ways. And it turns out {Saraswati}'s constituents
are torn. Half of them are committed welfarist consequentialists, the
other half are deontologists. Assume further that {Saraswati} has not
promised, either implicitly or explicitly, to make one choice rather
than the other in this kind of situation.

Given all those assumptions, what {Saraswati} should do turns on the
correct answer to the problem of inter-theoretic value comparison. What
she should do depends, at least in part, on whether the welfare loss
matters more to her welfarist constituents than the principle violation
matters to her principled constituents. That is to say, what she should
do turns on exactly the same kind of question that {Ulysses} faced when
he was deciding whether \emph{x}~\textgreater~\emph{y}.

Now as an externalist, I don't think {Saraswati} has to in any sense
solve the problem of inter-theoretic value comparison. She just has to
do the right thing for the right reasons. And one can do the right thing
for the right reasons without knowing they are the right reasons, or
even without having any general disposition to act rightly in similar
cases. But I do think that we as theorists need to solve the problem in
order to say anything evaluative about {Saraswati}'s actions. And even
if we can't do that, if we believe there is a fact of the matter about
whether {Saraswati} did the right thing, then we are committed to
thinking that there is a solution to the problem of inter-theoretic
value comparisons.

So premise 2 in the externalist argument above is not true. There is a
way, somehow, of solving the problem of inter-theoretic value
comparisons. That's not to say it will be easy. Personally, I suspect it
is one of the hardest problems in all of ethics. None of the remotely
viable solutions to it seem either obvious to the lay actor, or easy to
implement.

The difficulty of solving the problem does not show that normative
internalism is false. But this difficulty does undermine a popular
motivation for internalism. If the idea behind internalism is that there
should be a sense of `should' in which ordinary people can usually tell
what they should do, that can't be a sense of `should' which is
sensitive to the correct solution to the problem of inter-theoretic
value comparisons. So it is incoherent to motivate internalism by saying
that externalism makes it too hard to know what to do, and then develop
a theory of right action that requires a solution to the problem of
inter-theoretic value comparisons.

As we've seen, this isn't the only way to motivate internalism. Some
theorists motivate internalism by an analogy to the wrongness of
reckless action. Those theorists often need there to be a solution to
the problem of inter-theoretic value comparison, but they don't need
this solution to be in any way transparent. And reflection on cases like
{Saraswati}'s makes me think that the externalist must concede that such
a solution must exist, and so cannot rely on its non-existence in
arguing against internalists.

It's worth noting just how strong a conclusion we could draw from the
inter-theoretic value comparisons argument. Assume, for reductio, that
we really couldn't make sense of any kind of inter-theoretic value
comparison. It would follow that there is no way to define hypocrisy in
probabilistic terms. Someone could only be counted as a hypocrite if
they fully believed that what they were doing was wrong. But this
doesn't seem right.

Imagine that someone faces a choice about whether to betray a
confidence. The betrayal would be extremely disrespectful, but they
think there would be a small gain to the welfare of the world if they
did so. And while they mostly think respect is central to morality, then
have a non-zero credence that welfare consequentialism is the correct
moral theory. They break the confidence. Are they hypocritical? I think
they probably are, even if we can't give an algorithm for weighing the
downside of the betrayal on the moral theory they think is probably
right against the welfarist upside on the theory they give a small
credence to. If we think the problem of inter-theoretic welfare
comparisons is not solvable in principle, then we can't even define a
notion of hypocrisy that applies in cases like this. And that would be a
very strong result. I don't think the arguments that hypocrisy is no
vice are nearly as strong as the arguments against more systematic forms
of internalism in chapters 2--4. So I suspect the argument from
inter-theoretic value comparisons proves too much. It doesn't just rule
out views like \emph{The best thing to do is maximise expected
goodness}, it also rules out views like \emph{It's at least a minor vice
to not live up to your own principles}. And that feels like too much
weight for the argument to bear.

\section{The Externalist's
Commitments}\label{theexternalistscommitments}

I'm going to finish up this part by saying a bit about what I take to be
the more and less central parts of normative externalism. Like any -ism,
the view not only makes many different commitments, those commitments
differ greatly in strength. Setting out these commitments serves a few
useful functions. It helps us see how the different parts of the view
hang together. And it is good practice to say ahead of subsequent
refutations what retreats would be minor setbacks, and what would amount
to fleeing the field\footnote{I'm assuming here that there will indeed
  be subsequent refutations, but this follows from a version of the
  pessimistic induction.}. It's very tempting when a part of one's view
is shown to be flawed to insist that it was only a peripheral aspect of
the view to begin with. Writing down which commitments are central and
which are peripheral before the flaws are made visible is a way to avoid
this temptation.

The core idea is that moral norms are independent of both what one
thinks the moral norms are, and what one should think the moral norms
are. Here are a few ways that could fail that would threaten the
periphery of the view; we'll then move to seeing what a catastrophic
failure would look like.

First, there could be some one-way dependencies between moral norms and
(rational) beliefs about moral norms. For instance, a view that said
being true to oneself was one moral requirement among many would violate
one direction of the externalist's independence constraint. It would say
that believing that something is wrong is sufficient, but not necessary,
to make performing the action wrong. And the view discussed in chapter
five, where believing that an action is not wrong excuses it, violated
the other direction. It says that believing that something is wrong is
necessary, but not sufficient, for the performance to be blameworthy.

Second, there could be some dependencies that concern minor aspects of
morality. The most natural versions of this possibility combine it with
the one-way dependencies of the previous paragraph. Consider a view that
said that hypocrisy is a minor vice. Such a view might say that it's bad
to do what you think is wrong, but unless the belief is true, this is
not a major vice. Or consider a view where false moral beliefs are
partial excuses. These are paradigms of the `peripheral' failures I was
talking about above. What I want ultimately to argue for is that
morality is about respect, welfare, rights and so on, and not about
conformity to one's own principles. A view that says that morality is
almost entirely about respect, welfare, rights and so on, but conformity
to one's own principles has a small role too, is inconsistent with my
preferred view, but the differences are minor.

The third kind of peripheral failure takes a little more setup. Consider
again Descartes' view of the good person, and compare it with Kant's
view. (I'm simplifying both thinkers here, but the caricatures are
useful for setting out the philosophical point.) Both of them think that
the good person will do what they think is right. But Descartes thinks
this because he thinks that resoluteness is one of the supreme virtues.
Kant, on the other hand, thinks this because he thinks that the nature
of the moral law is visible to good people. So because the moral law is
the way it is, the good person will both act a certain way, and have
correlated beliefs about morality. For Descartes, the fact that the
person beliefs that they should do X explains why it is good that they
do X. For Kant, the fact that the moral law is the way it is explains
both why it is good that the person does X, and good that they believe
that X is good to do. In Descartes' case, but not Kant's, the moral
beliefs explain the moral status of the action.

More generally, we can distinguish amongst views that say there is a
connection between morality law and what one (reasonably) believes about
morality. Some such views say that (reasonable) beliefs about morality
explain why actions have the moral status they do. Other views say that
the moral status of the actions explain why certain beliefs are actual
or, more likely, reasonable. Yet other views say that some third thing
explains both the moral status of the actions and the actual or
reasonable beliefs about their moral status. What I'm really committed
to denying is the first of these options, where actual or reasonable
beliefs about morality explain the moral status of actions.

We can put all this in terms of a checklist. Ideally, from the point of
view of normative externalism, there would be no necessary connections
between moral properties, on the one hand, and actual or reasonable
moral beliefs on the other hand. If, however, there is such a
connection, we can ask three questions about it.

\begin{enumerate}
\def\labelenumi{\arabic{enumi}.}
\tightlist
\item
  Is the connection two-way, as opposed to moral beliefs providing
  merely a necessary or a sufficient condition for the moral property?
\item
  Is the moral feature morally central, as opposed to being, say, a
  minor vice or virtue?
\item
  Does the (actual or reasonable) moral belief explain why the moral
  property is instantiated, as opposed to the explanation going the
  other way, or some third factor explaining the connection?
\end{enumerate}

The more `yes' answers we give, the worse things are for normative
externalism. The view I want to defend is that there are no necessary
connections between moral belief and morality. But if there are such
connections, I want to defend the view that these are one-way, or are
minor, or that the moral belief does not explain the moral property.
Being true to yourself is not part of morality. But if it is, it is a
small part, and actions that are true to yourself aren't good because
they are true to yourself.

This last possibility, the one about surprising orders of explanation,
is a useful segue into epistemology. Here's another way that normative
externalism could strictly speaking fail, without threatening the core
commitments of the theory. There has been a pronounced `factive turn' in
recent epistemology. Many epistemologists think that our most important
epistemological concepts are factive. The most important ways for
beliefs to be good are such that if a belief is good in that way, it
must be true. One way to implement the `factive turn' is to make
knowledge central to epistemology. But another way, not inconsistent
with the first, is to argue that other epistemological notions are
factive. And that turns out to have consequences for normative
externalism.

Let's say that one thought, on quite general grounds, that only true
beliefs could be rational, or that only truths could be well supported
by evidence. I don't think either of those things, and I'll say a little
more in the next part as to why, but for now I just want the view on the
table. That would imply there is a necessary connection between rational
moral beliefs and morality. If one rationally believes that lying is
wrong, then it must be that lying is wrong. But that's not because the
rationality of the belief explains the wrongness, or that the having of
the belief explains the wrongness. It's because the wrongness of the
lying is a necessary precondition of rationally believing that lying is
wrong.

This kind of view would not show us anything special about morality. On
such a view, if one rationally believes that lying is common, then it
must be that lying is common. And it doesn't threaten the central
commitments of normative externalism. But it does mean there is a
necessary connection between morality and rational moral belief. So it's
a small defeat, but one we can absorb without too much distress. When we
turn to epistemology, we'll have to pay more attention to this kind of
possibility.

\part{Epistemology}

\chapter{Level-Crossing Principles}\label{level-crossingprinciples}

\section{First-Order and Second-Order
Epistemology}\label{first-orderandsecond-orderepistemology}

In the previous part I argued that morality is independent of both what
one thinks about morality, and what one should think about morality. In
this part I want to argue the same thing for epistemology. But we have
to be a bit careful setting up the independence thesis. Informally, a
thesis of the previous part was that morality and epistemology are
distinct existences. Arguing `the same thing' for epistemology would
amount to arguing that epistemology and epistemology are distinct
existences. That doesn't sound particularly plausible. So to state my
intended conclusion a bit more carefully, and a bit more plausibly, we
need one bit of terminology.

Say that a claim that either describes or evaluates a particular belief
of a person is \textbf{first-order} when that very belief is not itself
a description or evaluation of a particular belief. And say that a claim
that either describes or evaluates a particular belief of a person is
\textbf{second-order} when the belief in question is a description or
evaluation of another belief. So here are some examples of first-order
claims.

\begin{itemize}
\tightlist
\item
  {Baba} believes that his keys are missing.
\item
  {Baba} should believe that his keys are missing.
\end{itemize}

And here are some examples of second-order claims.

\begin{itemize}
\tightlist
\item
  {Baba} believes that he believes that his keys are missing.
\item
  {Baba} should believe that he believes that his keys are missing.
\item
  {Baba} believes that he should believe that his keys are missing.
\item
  {Baba} should believe that he should believe that his keys are
  missing.
\end{itemize}

And we can replace `should' in any of these claims with any other kind
of epistemic norm. So here are some more first order claims.

\begin{itemize}
\tightlist
\item
  {Baba}'s evidence supports the belief that his keys are missing.
\item
  {Baba}'s belief that his keys are missing is justified.
\item
  {Baba} rationally believes that his keys are missing.
\end{itemize}

And here are a sample of some more second-order claims.

\begin{itemize}
\tightlist
\item
  {Baba} should believe that he rationally believes his keys are
  missing.
\item
  {Baba}'s evidence supports the belief that his belief that his keys
  are missing is justified.
\item
  {Baba}'s belief that he believes that his keys are missing is
  justified.
\end{itemize}

The core thesis of this part of the book, the core thesis of normative
externalism in epistemology, is that first-order and second-order claims
are independent. There are no true \textbf{level-crossing} principles,
describing necessary connections between first-order and second-order
claims.

Just like in part one, there are fall-back positions I will adopt if
this strong claim (no necessary connections at all) turns out to be
false. If there are necessary connections, they are one-way, or they are
about less central concepts in epistemology, or the explanation of the
claim does not go from the second-order claim to the first-order claim.
But I'd rather not retreat even to there, and instead to argue that
there are no true level-crossing principles at all.

I'm interested in level-crossing principles for a few reasons. For one
thing, I find them intrinsically interesting. For another, they have
consequences for a bunch of epistemological debates. I'm going to
discuss at the end of the book the consequences they have for disputes
about how to best respond to peer disagreement. But they also matter for
a bunch of other debates. And they matter because to the extent they are
true, they push us towards a certain kind of coherentism, and away from
a certain kind of foundationalism. Just which kind will depend on just
which level-crossing principles are true. But the general idea is that
if rationality requires conformity to one's own beliefs about the
rational, then rationality is more of a coherence concept than we might
have thought it was.

\section{Change Evidentialism}\label{changeevidentialism}

It isn't just the principles that push away from foundationalism. The
examples that are used to motivate level-crossing principles are also
taken to mitigate against a fairly weak form of foundationalist
evidentialism that I'll call Change Evidentialism.

\begin{description}
\tightlist
\item[Change Evidentialism]
A person with a rational attitude towards \emph{p} is under no rational
obligation to change that attitude unless their evidence for or against
\emph{p} changes.
\end{description}

I think Change Evidentialism is true. Indeed, I think a much stronger
form of foundationalist evidentialism, one that says the rational status
of a mental state supervenes on the evidence it is based on, is true. It
is far beyond the scope of this book to defend the stronger claim. I
will, in effect, be defending evidentialism against a class of attacks,
but that defence will not be the focus.

Change Evidentialism is related to theses level-crossing principles
because some cases that motivate the principles also appear to undermine
Change Evidentialism. Here is one such case, due to David Christensen.

\begin{quote}
I'm a medical resident who diagnoses patients and prescribed appropriate
treatment. After diagnosing a particular patient's condition and
prescribing certain medications, I'm informed by a nurse that I've been
awake for 36 hours. Knowing what I do about people's propensities to
make cognitive errors when sleep-deprived (or perhaps even knowing my
own poor diagnostic track-record under such circumstances), I reduce my
confidence in my diagnosis and prescription, pending a careful recheck
of my thinking. ~(\citeproc{ref-Christensen2010a}{Christensen 2010a,
186}).
\end{quote}

We might naturally reason about the case as follows. (Note this isn't
Christensen's own considered take on the case.) When the resident learns
he has been awake 36 hours, he does not get evidence against the
diagnosis. That a particular resident has been awake awhile seems
evidentially irrelevant to whether a particular patient has, let's say,
dengue fever. But it is rational, indeed it is rationally required, for
the resident to change his attitude towards the diagnosis on learning
how long he's been awake. That's a counterexample to Change
Evidentialism. And the explanation for why rationality requires a change
is, we might conjecture, that principle 1 is true. The resident does
have evidence excellent that he's making irrational diagnoses. So he
can't rationally believe that he rationally believes the diagnosis. So,
by the contrapositive of 1, he can't rationally believe the diagnosis.

I'm going to argue that the previous paragraph is all false. What the
resident should do depends a lot on the details of the case. On some
ways of filling in the case, the resident's evidence changes
substantially, so Change Evidentialism is consistent with the resident
rationally changing their view. Indeed, the explanation of the change of
view in terms of change of evidence is preference to the explanation in
terms of a level crossing principle like 1. That's because in other
versions of the case, where the resident's evidence does not change, the
belief in the diagnosis should not change either.

So I'm going to argue that cases like Christensen's resident not only
fail to challenge Change Evidentialism, they end up supporting it. And
because the cases support it, they don't support 1--4. There are also
direct counterexamples to 1--4. The example in the next chapter of
{Roshni} is one such counterexample.

\section{Motivations for
Level-Crossing}\label{motivationsforlevel-crossing}

The rest of this book will be devoted to investigating three recent
motivations for level-crossing principles. The first concerns
higher-order evidence, the second akrasia, and the third peer
disagreement.

\subsection{Higher-Order Evidence}\label{higher-orderevidence}

As well as evidence that bears on a question, agents can have evidence
that bears on the rationality of their verdicts about the question.
Christensen's example involving the medical resident is one such case.
Elsewhere, Christensen has provided several other examples along similar
lines, e.g., ~(\citeproc{ref-Christensen2007a}{Christensen 2007a, 8}),
~(\citeproc{ref-Christensen2010b}{Christensen 2010b, 126}) and
~(\citeproc{ref-Christensen2011}{Christensen 2011, 5--6}). Similar
examples have also been proposed by Adam Elga
(\citeproc{ref-Elga2008}{2008}), Thomas Kelly
(\citeproc{ref-Kelly2010}{2010, 140}), Joshua Schechter
(\citeproc{ref-Schechter2013}{2013, 443--44}) and Sophie Horowitz
(\citeproc{ref-Horowitz2014}{2014, 719}). The examples suggest something
like the following argument against Change Evidentialism.

\begin{enumerate}
\def\labelenumi{\arabic{enumi}.}
\tightlist
\item
  It is irrational for the resident in this case to stick with the
  original prescription without making some kind of cross-check.
\item
  The best explanation of why it is irrational to stick with the
  original prescription is that it is irrational to stick with the
  original diagnosis, i.e., the original belief.
\item
  The information the nurse provides is not evidence one way or the
  other about whether the patient has the disease originally diagnosed.
\item
  It was rational, before the information the nurse provides about how
  long the resident has been awake, to believe in the original
  diagnosis.
\item
  So this is a case where the rationality of a belief changes without
  any change in the evidence.
\end{enumerate}

The last line follows from what came before, so the issue is whether the
first four claims are true. I'm going to raise doubts about every one of
those steps. But this is, I think, the most pressing challenge to Change
Evidentialism.

\subsection{Akrasia}\label{akrasia}

Assume, for reductio, that the level-crossing principles are false. And
assume that in any field, it is possible to have evidence that supports
being extremely confident in something that is, as a matter of fact,
false. Then there should be cases where one's evidence strongly supports
\emph{p}, but one's evidence also strongly supports the falsehood that
one has very poor evidence for \emph{p}. If one follows the evidence
where it leads, one should be very confident in is the conjunction
\emph{p, and I have very weak evidence for p}. Assuming one believes
(correctly!) that it is rational to follow the evidence where it leads,
one should believe the conjunction: \emph{p, and it is irrational for me
to be confident in p}. But it is absurd to think that one can rationally
be confident in either of these conjunctions; they are instances of
epistemic akrasia, and akrasia is paradigmatically irrational.

I'm going to come back to this argument in chapter 10. The main response
will be that the apparent absurdity is really not that absurd. Indeed,
the intuition that it is absurd can be shown to be highly unreliable; it
supports the `absurdity' of many things that are plainly true. For now,
note the connection between intuitions about akrasia and intuitions
about Christensen's resident case. If the resident follows the evidence
where it leads, he'll believe that the diagnosis is correct, and this
belief is irrational. It looks like Evidentialism, and perhaps just
Change Evidentialism, implies that the resident should be akratic.
Unlike many philosophers, I won't take this to be a decisive objection
to Change Evidentialism.

\subsection{Disagreement}\label{disagreement}

It seems possible for people who are known to have equally good track
records, and who in some sense have the same evidence, to come to
different conclusions. When they do, there is something intuitively
plausible about each moving their beliefs in the direction of the other.
Here is one such case.

\begin{quote}
{Ankita} and {Bojan} have known each other for a long time, and know
each other to be equally reliable, and equally reasonable, when it comes
to arithmetic problems about as complex as multiplying two two-digit
numbers. For some practical purpose they need to know what 22 times 18
is. They each do the multiplication quickly in their head. {Ankita}
announces that she got 396, while {Bojan} announces that he got 386.
(Compare a similar case in Christensen
(\citeproc{ref-Christensen2007c}{2007b, 193}).)
\end{quote}

Again, we can use the case to construct an argument against Change
Evidentialism, as follows.

\begin{enumerate}
\def\labelenumi{\arabic{enumi}.}
\tightlist
\item
  {Ankita}'s original evidence provides her strong reason to believe
  that 22 times 18 is 396.
\item
  {Bojan}'s announcement is no evidence against the claim that 22 times
  18 is 396.
\item
  Yet, on hearing {Bojan}'s announcement, and respecting the fact that
  the two of them have equally good track records, {Ankita} should be
  unsure which of them is right, and which wrong, on this occasion.
\item
  Since {Ankita} knows what each of them announced, the only way she can
  consistently be unsure which of them is right is to be unsure whether
  22 times 18 is 396.
\item
  So although {Bojan}'s announcement does not change her evidence that
  bears on whether 22 times 18 is 396, it does change whether it is
  rational for her to fully believe that this is true.
\end{enumerate}

Again, this looks like a reasonably intuitive argument against Change
Evidentialism. And again, I'm going to raise doubts about every premise.
The focus of chapter 12 will be the picture of disagreement behind
premise 3. This is the view that has come to be called
\emph{conciliationism}. But what I say about evidence over the next few
chapters will also raise concerns about the first two premises of the
argument.

\section{The Plan for the Rest of the
Book}\label{theplanfortherestofthebook}

In what follows, the even-numbered chapters will deal with the three big
arguments for level-crossing principles, and against Change
Evidentialism, that I just discussed. In chapter 8, I'll discuss
higher-order evidence; in chapter 10, I'll discuss akrasia principles,
and in chapter 12, I'll discuss disagreement. In between I'll address
two big issues that arise out of those discussions.

In chapter 9, I'll talk about what it means for some reasoning to be
problematically circular. This turns out to matter to our purposes
because of a potential bit of circular reasoning that is, according to
my view, perfectly acceptable. In particular, in some cases where there
is reason to believe the agent is incapable of correct reasoning, I
think it is possible for the agent to simply do some correct reasoning,
notice that it is correct, and infer that they are, after all, capable.
This can feel worryingly circular. But it turns out to be incredibly
hard to find an anti-circularity principle that is both true, and
violated by this reasoning.

In chapter 11, I discuss how level-crossing principles lead to nasty
regresses. More precisely, I argue that level-crossing principles are
only motivated if one accepts a particular assumption concerning
evidential screening. And that assumption, I argue, leads to nasty
regresses. The regress arguments here are similar to the regress
arguments in chapter 2 against very strong level-crossing principles in
ethics.

And in chapter 13, I briefly summarise the lessons of both the
epistemology part, and of the book as a whole.

But before we get to all that, I need to do a little ground-clearing.
The rest of this chapter contains two fairly self-contained sections on
things I wanted to get out of the way before defending Change
Evidentialism against level-crossing principles. The next section
concerns the relationship between state-level evaluations, like the
rationality of a belief, and agent-level evaluations, like the wisdom of
a believer. And then I argue, against most orthodox wisdom in
epistemology, that we acquire evidence while doing mathematical
investigation.

These two sections are helpful for understanding the rest of the book.
But they are not essential. And someone who is impatient to get on to
higher-order evidence, akrasia or disagreement could skip ahead to any
one of those chapters.

\section{Evidence, Rationality and
Wisdom}\label{evidencerationalityandwisdom}

Change Evidentialism is a claim about the rationality of beliefs and
other doxastic attitudes. The level-crossing principles I reject are
principles about evidence, and about rationality. The focus here, as you
may have gathered, is on rationality and on evidence. There are other
concepts in the area that I don't have as much to say about, and which
may not be systematically related to those concepts.

I don't want to assume that a belief is rational if and only if it is
justified. It might be that only true beliefs are justified
~(\citeproc{ref-Littlejohn2012}{Littlejohn 2012}), but it is very
unlikely that only true beliefs are rational.\footnote{That false
  beliefs can be rational seems more plausible to me than the premises
  of any argument I could give for it. But here is one independent way
  to make the case. Arbitrarily high credences in false propositions can
  be rational. Indeed, false propositions can have arbitrarily high
  objective chances, consistent with those chances being known. In such
  cases the only rational credence matches the chance. The best theories
  of the relationship between credence and chance do not require
  credence 1 for belief simpliciter
  ~(\citeproc{ref-Weatherson2014}{Weatherson 2014a}). And if a high
  credence constitutes a belief, and the credence is rational, the
  belief is rational. So some false beliefs can be rational.} In any
case, there is something a little artificial about talking about
justified beliefs. In everyday English, it is typically actions that are
justified or not. The justification of belief seems a somewhat
derivative notion. So I'll stick to rationality.

I'm also going to set aside, for the most part, a discussion of wisdom.
Just as in the discussion of ethics, it is very important to keep
evaluations of agents apart from evaluations of acts or states. It is
attitudes or states that are in the first instance rational or
irrational. We can talk about rational or irrational agents, but such
notions are derivative. Rational agents are those generally disposed to
have rational attitudes, and to be in rational states. Wisdom, on the
other hand, is in the first instance a property of agents. Again, we can
generalise the term to attitudes or states. A wise decision, for
instance, is one that a wise person would make. But the wisdom of agents
is explanatorily and analytically prior to the wisdom of their acts,
judgments, decisions and attitudes.

I think that everything I said in the last paragraph is true if we use
`wise' and `rational' and their cognates with their ordinary meaning.
But I'm not committed to that, and it doesn't matter if I'm wrong. You
can read me as stipulating that `rational' is to be used as a term that
in the first instance applies to states, and `wise' is to be used as a
term that in the first instance applies to agents, and little will be
lost.

Change Evidentialism is not a claim about wise agents, it is a claim
about the rationality of various beliefs and belief transitions. Perhaps
a wise agent is one who always has rational attitudes. If so, then
Change Evidentialism will have some implications for what wise agents
are like. But it is far from obvious that wisdom and rationality are
this tightly linked. Indeed, at the end of chapter 11, I'll come back to
a reason to question the connection. For all I've said, it may well be
wise to change one's beliefs in some situations where one's evidence
does not change. That is consistent with Change Evidentialism, provided
we understand those situations as being ones where it is unwise to have
rational attitudes.

I am leaning heavily her on work on the connection between rationality
and wisdom by Maria (\citeproc{ref-Lasonen-Aarnio2010b}{Lasonen-Aarnio
2010b}, \citeproc{ref-Lasonen-Aarnio2014}{2014a}). I agree with almost
everything she says about the connection. The biggest difference between
us is terminological. She uses `reasonable' and `reasonableness' where I
use `wise' and `wisdom'. In my idiolect, I find it too easy to confuse
`rational' and `reasonable'. So I'm using a different term, and one
that, to me at least, more strongly suggests a focus on agents not
states. But this is a small point, and everything I say about the
distinction draws heavily on Lasonen-Aarnio's work.

\section{Evidence, Thought and
Mathematics}\label{evidencethoughtandmathematics}

The picture of evidence behind the version of evidentialism that I'm
presenting here differs from a natural picture that many epistemologists
have. In particular, I draw the line between acquiring evidence and
processing evidence at a very different place than many others do. I'm
going to motivate this re-drawing by working through some examples
involving mathematics. Much of what I say about these examples follows
closely the arguments that Paul Boghossian
(\citeproc{ref-Boghossian2003}{2003}) made against simple forms of
reliabilism and internalism about logic and mathematics. But these
arguments of Boghossian's are worth rehearsing, because their
significance for recent epistemological debates has not always been
appreciated.

A young mathematics student, {Tamati}, starts thinking about primes. He
notices the gaps between primes get larger, and starts to wonder whether
there is a largest prime. He is struck by a sudden strong conviction
that there is no largest prime, and so forms the belief that there is no
largest prime. Now {Tamati} is not usually prone to forming beliefs on
the basis of spontaneous convinctions like this. Apart from this time,
he only does this for very simpile arithmetic claims, like that seven
plus five is twelve. But nor is he a mathematical savant. He couldn't
produce any reason for the claim that there is no largest prime. He
hasn't seen, even implicitly, anything like the argument that if
\emph{n} is the largest prime, then \emph{n} + 1 would be both prime and
not prime. It's just an immediate convinction for him.

{Tamati} does not know that there is no largest prime. This fact,
assuming it is a fact, needs explaining. The evidentialist has a natural
explanation. In a normal case, when someone comes to learn by proof that
there is no largest prime, there are two extra facts they learn. The
first is that if \emph{n} is the largest prime, then \emph{n}!~+~1 is
prime; the second is that if \emph{n} is the largest prime, then
\emph{n}!~+~1 is not prime. These in turn aren't immediately obvious; to
be known they must be figured out on the basis of other things. Those
extra pieces of knowledge are extra evidence\footnote{Is this consistent
  with my earlier note that we would, for the sake of discussion,
  identify evidence with non-inferential knowledge? It isn't obvious
  that it is, and it is more than a little tricky to say just what
  evidence {Tamati} would gain if he worked through the problem
  carefully. It would work to defend normative externalism if we
  identified evidence with all knowledge, as Williamson suggests, but I
  would rather not make that identification on other grounds. I hope to
  return to the question of just how we should conceptualise evidence,
  both in mathematical and empirical investigations, in subsequent work.}
It could be that the extra evidence would just be the premises that he
would use, not the conclusions he draws from them. Or it could be that
we need to give up the idea that evidence is non-inferential. I don't
have a worked. It is with that evidence that a normal student can come
to know, by proof, that there is no largest prime. Alternatively, the
student may learn that some teacher, or some book, says that there is no
largest prime, and that teacher, or book, is reliable. Those things are
the extra evidence. That case is clearly different to {Tamati}'s,
because it relies on engagement with the outside world. But even the
student who thinks through the case themselves acquires evidence, namely
the above facts about the relationship between \emph{n} and
\emph{n}!~+~1.

So the evidentialist has a nice explanation of what is going on in
{Tamati}'s case. Other explanations look less promising.

We could try to explain {Tamati}`s case in strictly reliabilist terms.
But note that {Tamati}'s convictions are perfectly reliable. The method
'trust my convinctions' gets him arithmetic knowledge every day, and the
true belief that there is no largest prime. In no case does it go wrong.
So the reliabilist has no explanation of why this use of {Tamati}'s
convictions does not yield knowledge. The reliabilist could try to argue
that methods have to be individuated more finely than this; it is
different to trust one's convictions about simple matters as compared to
more complex matters. But this assumes we have some grasp on the idea
that saying there is no largest prime is a complex matter. It isn't
clear why this should be so. It isn't hard to state the proposition that
there is no largest prime. It is a little hard to prove it. The
evidentialist has an explanation of why how hard it is to prove the
theorem matters to whether {Tamati} can spontaneously know it. But it
seems very hard to motivate the idea that proof complexity should define
the relevant reference class. It seems to use the very thing we were
trying to give a reliabilist explanation of. In any case, even if we
restrict the reference class to things that are hard to prove,
{Tamati}'s convictions are still reliable. He sensibly declines to form
beliefs about most things in this class, while forming one true belief.
So he's got a perfect success rate, so is reliable!

Alternatively, we could say that {Tamati} doesn't ``appreciate'' the
evidence for the absence of a largest prime. (The idea that appreciating
the evidence is important to mathematical knowledge comes from Richard
Fumerton (\citeproc{ref-Fumerton2010}{2010}), though he doesn't use it
for quite this purpose, and shouldn't be thought responsible for the
view I'm about to criticise.) The thought would be that {Tamati} has
some evidence about primes, but doesn't stand in the special
relationship to it needed to ground knowledge. This is obviously an
anti-evidentialist position, since it says that rationality depends not
just on what evidence one has, but on some further relationship that one
may or may not stand in to evidence.

But depending on how we understand `appreciate', the view will be too
strong or too weak. If appreciation means understanding how and why the
evidence supports the conclusion, and appreciation is required for
knowledge, then very few people will know very much. Before they take a
logic class, introductory students can come to know \emph{Ga} by
inferring it from \emph{Fa} and ∀\emph{x}(\emph{Fx}~→~\emph{Gx}). But
they don't need to know how or why their evidence supports \emph{Ga}.
Indeed, they can be radically mistaken about the nature of logical
implication, as many students are, and still know \emph{Ga} on that
basis. On the other hand, if appreciation means having a true belief
that the evidence supports the conclusion, it won't rule out {Tamati}
knowing that there is no largest prime. We can assume that {Tamati} is
sophisticated enough to know that mathematical truths are entailed by
any proposition. So if he believes there is no largest prime, he can
immediately (and correctly) infer that the fact that his coffee has gone
cold entails there is no largest prime. But that isn't enough for him to
know there is no largest prime, not even if he knows that his coffee has
gone cold. If we insist that appreciation means knowing that the
evidence supports the conclusion, then we are back where we started,
needing to explain why {Tamati} doesn't know that there is no largest
prime.

So the best explanation of {Tamati}'s ignorance is that he lacks
sufficient evidence to know that there is no largest prime. If he worked
through the problem slowly, he would acquire evidence for that
conclusion. And that's the general case. Thinking through a mathematical
problem involves acquiring mathematical evidence. Similarly, when one
has to do some mathematical reasoning to get from empirical data to
empirical conclusion, that reasoning doesn't just involve processing the
empirical evidence, it involves acquiring new, mathematical evidence.

This way of thinking about mathematics is hardly radical. It is a
commonplace in mathematical discussions that one can get evidence for or
against mathematical propositions. Philosophers too often think that
evidence that entails a conclusion is maximally strong evidence. This
assumption is even encoded into probabilistic models of evidential
support. But it isn't true. Facts about Andrew Wiles's diet are terrible
evidence that Fermat's Last Theorem is true, although they entail it.
Fact about what he wrote in his notebooks, on the other hand, are
excellent evidence that it is true. Thinking that entailing reasons are
maximally strong reasons is just another way to confuse inference with
implication ~(\citeproc{ref-Harman1986}{G. Harman 1986}).

This attitude, of thinking that entailing reasons are maximally strong
reasons, goes along with another bad attitude that it is easy to adopt.
That is the attitude that when \emph{p} is a mathematical proposition,
our evidence supports either a maximally strong belief in \emph{p} or a
maximally strong belief in ¬\emph{p}. There are numerous counterexamples
to this view. Sanjoy Mahajan (\citeproc{ref-Mahajan2010}{2010})
describes heuristics that can be used to quickly refute various
mathematical hypotheses. The heuristics involve, for example, checking
whether the `dimensions' of a proposed identity are correct, and
checking limit cases. So consider the hypothesis that the area of an
ellipse is π\emph{ab}, where \emph{a} is the distance from the centre of
the nearest point on the ellipse, and \emph{b} is the distance from the
centre to the furthest point. After going through a number of other
proposals and showing how they can be quickly refued, Mahajan says this
about the proposal that the area is π\emph{ab},

\begin{quote}
This candidate passes all three tests \ldots{} With every test that a
candidate passes, confidence in it increases. So you can be confident in
this candidate. And indeed it is correct.
~(\citeproc{ref-Mahajan2010}{Mahajan 2010, 21})
\end{quote}

It might be worried that the position I'm adopting here, that we often
need evidence of a connection between premises and conclusion in order
to reasonably infer the conclusion from the premises, even when the
premises entail the conclusion, risks running into the regresses
described by Lewis Carroll (\citeproc{ref-Carroll1895}{1895}). It
certainly would be bad if my view implies that to infer \emph{q} from
\emph{p} and \emph{p} → \emph{q}, and agent needed to know (\emph{p} ∧
(\emph{p} → \emph{q})) → \emph{q}. That way lies regress, and perhaps
madness. But that's not what my view implies. The claim is just that for
non-obvious entailments, the agent needs extra knowledge to infer from
premises to conclusions. It is consistent with this to say that
immediate entailments, like modus ponens, can justify immediate
inferences. And that's enough to stop the regress.

The idea that we accumulate evidence when working through philosophical
or mathematical puzzles will matter quite a bit for debates about
disagreement. It is agreed on all sides that when the parties to a
disagreement do not have the same evidence, then the existence of the
disagreement is a reason for each to move their attitudes. (Assuming, of
course, that the other person is not irrational, or known to be bad at
processing this kind of evidence.) If we allow that there is
philosophical evidence, then it will be incredibly rare that each party
to a debate has the same evidence. It will be vanishingly rare that each
party knows that each party has the same evidence. This means that any
case where the parties know about the evidence the other parties have
will be a fair way removed from the kind of real-world case where we
have reliable intuitions. It also means that in practice learning about
the existence of a people who disagree with you is often evidence that
there is evidence against your view that they have and you lack.

The main claim I'll need in what follows is that thinking through a case
sometimes gives you evidence. But it's independently interesting to
think how far this extends; to think about how much reasoning is a form
of evidence acquisition. And examples with the same structure as
{Tamati}'s can be used to motivate the thought that very often reasoning
involves evidence acquisition.

A, B and C are trying to figure out how many socks are in the drawer.
They each know there are seven green socks, and five blue socks, and
that that's all the socks, and that no sock is both green and blue. From
this information, they all infer, and come to know, that there are seven
plus five socks in the drawer. A is an adult with statistically normal
arithmetic skills, so she quickly infers that there are twelve socks in
the drawer. B is a three year old child, who is completely unreliable at
arithmetic. She guesses that there are twelve. At this stage, we can say
that A knows there are twelve socks in the drawer, and B does not know
it. But C, who is four years old, is a more subtle case. She says to
herself, ``I think it's twelve, but I better check.'' That's a good
reaction; like B she isn't so reliable that she can know without
checking. So she uses a method for doing addition; she starts counting
from seven, putting one finger up at each count. So she says ``eight''
and raises her thumb, ``nine'' and raises her index finger, and so on
through saying ``twelve'' and raising her little finger. She looks at
her hand, sees that she has five fingers raised, and concludes the
answer is twelve.

At the end of this process, but not before, she knows that there are
twelve socks in the drawer. Indeed, it is only at the end of the process
that she is in a position to know that there are twelve socks in the
drawer. It is because she has come to know that seven plus five is
twelve that she has sufficient evidence to know there are twelve socks
in the drawer. Previously, this was not part of her evidence, now it is,
and now she can know there are twelve socks in the drawer. That suggests
it is because A knows that seven plus five is twelve that she can know
there are twelve socks in the drawer to. She might not have consciously
said to herself that seven plus five is twelve, but if she didn't have
that as part of her evidence, she wouldn't have been in a position to
know that there were twelve socks in the drawer. C also knows this,
because she acquired this evidence. Indeed, she acquired it a
posteriori; it in part relied on seeing that she had five fingers
raised.

So even reasoning that relies on simple arithmetical identities relies
on those identities being in evidence. In these cases, the only rule of
implication that really seems to do double duty as a rule of inference
is the transitivity of identity. An agent who knows that \emph{x} equals
seven plus five, and knows that seven plus five equals twelve, is in a
position to infer that \emph{x} equals twelve. They don't, it seems,
need to know that identity is a transitive relationship. Whether we
grant C knowledge that there are twelve socks in the drawer does not, it
seems, depend on whether we grant her knowledge of the fully general
principle that identity is transitive.

What's special about that last step is that the general principle that
might be relevant is considerably more complicated to state, and to
believe, than the general principle in arithmetic cases. It is easier to
know that seven plus five is twelve than it is to know exactly what rule
about identity that A, B and C need to use to figure out how many socks
there are in the drawer. It is harder to know the general principle of
disjunctive syllogism than it is to use it on an occasion. So it might
be that there are more kinds of simple inferential steps that track
simple implicative rules than there are kinds of simple inferential
steps that track simple arithmetic identities. For all I've said here,
it might be that all arithmetic inferences just involve the transitivity
of identity, plus knowledge of a lot of arithmetic facts. It isn't so
plausible that all logical inferences just involve one rule, such as
modus ponens, plus a lot of logical facts.

\chapter{Higher-Order Evidence}\label{higher-order-evidence}

\section{Varieties of Higher-Order
Examples}\label{varietiesofhigher-orderexamples}

Higher-order evidence is evidence about one's own evidence, or
reliability, or rationality. Several examples have been proposed which
are often taken to show that rationality requires adjusting one's
confidence in certain propositions to higher-order evidence. And the
best explanation of that phenomena may well be that some level-crossing
principle or other is true. Since it's my task to argue against
level-crossing principles, I need to say something about these examples.

The examples that have been proposed thus far all have a similar
structure. The hero starts out with a firm belief, and the belief would
licence a decisive action. Something happens that would, in normal
cases, cause a person to question both the belief and the wisdom of
taking decisive action. The suggested explanation is that a
level-crossing principle is true, and explains the normal person's
hesitation. But the structure of the level-crossing principles has
nothing to do with hesitation, either in belief or action. If the
principles were true, there should be cases where higher-order evidence,
evidence about the nature of one's evidence or capacity, licences
decisive belief or action that is not licensed by the first-order
evidence. And once we see what such a licensing looks like in practice,
the level-crossing principles look less attractive. So my main aim here
is to expand the diet of examples that we have, and judge explanations
by how well they handle all the examples in this class.

I already introduced one of the proposed examples in the previous
chapter: David Christensen's example of the medical resident. I'm going
to argue that the details of the case are underspecified in important
ways. Once we fill in those details, it becomes clear that there are
ways to respond to the case without thinking that they provide any
support for level-crossing principles. Since we'll discuss the example
at some length, it's worth repeating it here.

\begin{quote}
I'm a medical resident who diagnoses patients and prescribed appropriate
treatment. After diagnosing a particular patient's condition and
prescribing certain medications, I'm informed by a nurse that I've been
awake for 36 hours. Knowing what I do about people's propensities to
make cognitive errors when sleep-deprived (or perhaps even knowing my
own poor diagnostic track-record under such circumstances), I reduce my
confidence in my diagnosis and prescription, pending a careful recheck
of my thinking. ~(\citeproc{ref-Christensen2010a}{Christensen 2010a,
186}).
\end{quote}

First, a relatively trivial point. Many of the examples in the
literature to date are written as either first-personal narratives, as
this one is, or as second-personal narratives. It's not particularly
easy to write commentary on such narratives. How, exactly, should we
refer to the protagonist of the story? Should we call him David? That
seems informal, and incorrect. I've been using the clumsy `the narrator'
or `the resident', but those aren't the easiest phrases to track,
especially over time. So it's better to give the protagonist a name. For
similar reasons, it is better to say what exactly the diagnosis is, so
we can easily refer back to it directly. There are two scope ambiguities
in \emph{David doubts that his diagnosis is supported by his evidence},
and those ambiguities can be cleared up if we specify what the diagnosis
is, and what the evidence for it is.

While there are these general reasons to eschew first-personal
narratives, there is an extra reason for concern here. The externalist
thinks that is very important to distinguish evaluation of states from
evaluations of agents, and to distinguish both of these from advice.
We're interested here in what it would be rational for the resident to
believe. That's distinct, at least in principle, from what a wise
resident would believe in the circumstances. And both of those are
distinct, again at least in principle, from what would be advisable for
the resident to believe; i.e., from what advice we should give the
resident about how to deal with such situations. Using first-personal,
or second-personal, narratives in philosophical examples encourages
conflation of rationality, wisdom and advisability. And we're wading
into territory where it is important to remember those can come apart.

Returning to this example, Christensen does not make clear whether the
doubts that have been raised are focussed in the first instance on the
rationality of the resident, or on the reliability of the resident.
(Indeed, the parenthetical remark seems to point in the opposite
direction to the main text on just this point.) This distinction may be
important.\footnote{Indeed, in later work Christensen
  (\citeproc{ref-Christensen2014}{2016}) himself is very clear on the
  importance of this distinction, and what I say here draws on that
  later work.} That is, it may be that the rational response to learning
that one is prone to irrationality is very different to the rational
response to learning that one is prone to unreliability. Maybe that
won't be so, but at the beginning of inquiry there is little reason to
think these two responses are certain to go together. So let's keep them
apart in the examples we introduce.

I'm going to spend a lot of time on these three cases. All of them have
a similar structure to Christensen's case, but with many more details
filled in.

\begin{quote}
{Raisa} is a medical resident with a new patient. He came in complaining
of a burning sensation in his scalp and a nasty smell that he can't
explain. {Raisa} looks at him and sees his hair is on fire. She decides
that this is the cause of his symptoms, and starts to put the fire out.
She is then told that she has been on duty for 36 hours, and that
residents who have been on duty that long are typically over-confident
in their diagnoses and prescriptions. What should she believe and do?
\end{quote}

\begin{quote}
{Regina} is a medical resident with a new patient. The whites of his
eyes are yellow, and he is lethargic. {Regina} was taught in medical
school that literally every lethargic patient with yellow eyes is
jaundiced. (This is, we'll assume, actually true in Regina's world,
though I'm sure it is actually false.) And she was taught, correctly,
that every jaundiced patient should be treated with quinine. In her
world, quinine cures all cases of jaundice and is, unlike every other
medicine, free of all adverse side-effects. (Remember this is a
fictional example!) So {Regina} prescribes quinine, recalling these
facts from her medical training. But she is then told that she has been
on duty 36 hours, and that residents who have been on duty that long are
typically over-confident in their diagnoses. What should she believe and
do?
\end{quote}

\begin{quote}
{Riika} is a medical resident with a new patient. He has a fever,
headache, muscle and joint pains, and a rash that blanches when pressed.
And he has recently returned from a trip to Louisiana. It seems to
{Riika} that her patient has dengue fever, and that he should be treated
with paracetemol and intravenous hydration. This is right; {Riika}'s
patient does actually have dengue fever, and it's rational to make that
diagnosis after correctly processing the available evidence. But then
{Riika} is told that she has been on duty for 36 hours, and that
residents who have been on duty that long are typically over-confident
in their diagnoses. What should she believe and do?
\end{quote}

My judgment on these cases is that {Raisa} should keep trying to put out
the fire, {Riika} should get a second opinion, and hold off on the
treatment if it seems at all safe to do so, and that {Regina}'s case is
rather hard. That is, the details of what the symptoms are, and what the
diagnosis and prescription are, matter to the judgment about what they
should believe and do.

Now note that this doesn't immediately get Change Evidentialism off the
hook. All it takes to refute Change Evidentialism is one case, and
{Riika}'s case may be enough to get the job done. But {Raisa}'s case,
and {Regina}'s too, are important. Our best theory should explain what's
true about those cases, and explain why the cases are different from
{Riika}'s. (If, indeed, Regina's case is different.) Ideally, they would
even explain why {Regina}'s case is a hard case, though maybe that's too
much to ask of a philosophical theory
~(\citeproc{ref-Ichikawa2009}{Ichikawa 2009}).

As you may have guessed, I'm going to argue that Change Evidentialism
does the best job at discharging these explanatory burdens. Before I
start showing that, we need one more case. Christensen's example is one
where the higher-order evidence seems to push in the direction of being
more uncertain. All of the cases from the literature that I cited
earlier have the same feature. But in principle we can imagine cases
that go the other way.

\begin{quote}
{Roshni} is a medical resident with a new patient. His symptoms are
similar to those of {Riika}'s patient, but his rash does not blanch when
pressed, and indeed is light enough that it doesn't have the distinctive
visual characteristics of the rash produced by dengue fever. Given his
symptoms and history, {Roshni} thinks he probably has dengue fever,
though the oddity of the symptoms means that she thinks other diagnoses
are possible. So she wants to run more tests before commiting to any
course of treatment. One reason for her to run more tests is that she
remembers there are some other illnesses going around that display
similar symptoms to what her patient displays. {Roshni} is then told
that she has been on duty for 13 hours and (and this is actually true in
the world of the story) that residents who have been on duty between 12
and 14 hours are typically over cautious in their diagnoses. If such a
resident thinks probably \emph{p}, then \emph{p} is almost always true,
and the resident should simply have come to believe \emph{p}. Now as it
turns out {Roshni} is an exception to this rule; she really doesn't have
strong enough evidence to conclude that her patient has dengue fever,
and she's right to stop at the conclusion that he probably has dengue.
But she has no independent reason to believe that she is an exception.
So what should she believe and do?
\end{quote}

It would be wrong for {Roshni} to reason as follows.

\begin{quote}
When someone in my circumstance concludes probably \emph{p}, then there
is almost always sufficient evidence to conclude definitely \emph{p}.
I've concluded he probably has dengue fever. So he definitely has dengue
fever. So I'll stop running tests and start the prescribed treatment for
dengue fever.
\end{quote}

{Roshni} can't rule out other possible diagnoses simply on the basis of
general characteristics of residents in situations like her. If her
patient has some other disease, and {Roshni} treated him for dengue on
the basis of higher-order considerations, she'd be guilty of
malpractice.

So now we have another task for our theory to perform. It must explain
why there is, to use a term Stewart Cohen suggested to me,
\emph{epistemic gravity}. {Riika}'s case shows that, at least sometimes,
intuition wants agents to lower confidence when they learn they are in a
situation where people are often over-confident. But {Roshni}'s case
shows that the converse is not always true. Higher order evidence can,
according to intuition, make confidence go down but not up. And that's
especially true if one had judged correctly to begin with.

I'm going to argue that a theory that rejects level-crossing principles,
and accepts Change Evidentialism, is best placed to explain these four
cases.

\section{Diagnoses and Alternatives}\label{diagnosesandalternatives}

It is easy to see why one might think {Riika}'s case is a problem for
Change Evidentialism. Imagine that {Riika}'s twin sister is also a
medical resident, and looks at the same public data about {Riika}'s
patient. And she, like {Riika}, concludes that the patient has dengue
fever. Now the residents are both told that {Riika} (but not her sister)
has been awake for 36 hours, and hence a member of a class that is
systematically over-confident in their diagnoses. This seems like a
reason for {Riika}, but not her sister, to reduce their confidence that
the patient has dengue fever. And that's a problem for Change
Evidentialism. That {Riika} has been awake for 36 hours either is, or is
not, evidence against the hypothesis that the patient has dengue fever.
If it is, then both sisters should become less confident. If, more
plausibly, it is not, then if {Riika} should change, that violates
Change Evidentialism.

There is a purely technical solution to this problem that I mention
largely to set aside. The argument of the previous paragraph assumed
that when the nurse told {Riika} how long she'd been awake, the evidence
{Riika} received was a proposition like \emph{Riika has been awake for
36 hours}. That's evidence that {Riika} can get, and that her sister can
get. And intuitively learning that has a different effect on the two of
them. But we could conceptualise {Riika}'s evidence differently. We
could think her evidence is a centered world proposition, in the sense
popularised by David Lewis (\citeproc{ref-Lewis1979}{1979}). On this
picture, {Riika}'s evidence is \emph{I have been awake for 36 hours},
while her sister's evidence is \emph{My sister has been awake for 36
hours}. So they get different evidence. So there is no argument that
Change Evidentialism fails.

This feels a bit like a cheat, at best. After all, we can imagine that
the nurse explicitly says to the pair of them, ``{Riika} has been awake
for 36 hours''. In that case it would feel extremely artificial to say
that the evidence is really this first-personal claim about {Riika}. But
while this technical attempt to save the letter of Change Evidentialism
isn't attractive, it tells us something useful. The information about
{Riika}'s sleep (or lack thereof) matters to {Riika} because of what it
tells her about her mind, i.e., about the very mind she is both using to
think about the patient, and thinking about. And an explanation of what
goes on in the case should be sensitive to this fact.

It is important that {Riika} and her sister are medical residents. The
patient in the next bed can't reasonably believe that {Riika}'s patient
has dengue fever on the basis of the data. Or at least he can't unless
he has medical training. Should we think this is a case where different
people with the same evidence can draw different conclusions? No,
because this data about the patient does not exhaust the evidence. The
evidence also includes everything relevant that {Riika} learned in her
medical training. That's evidence she has in common with her sister, but
not with the patient in the next bed.

The evidence provided by training, and background information, has to
play two roles. First, it has to make it plausible that the patient has
dengue fever. It does that by including facts about the symptoms the
patient displays, and facts about what symptoms patients with dengue
fever typically display. But it must also play a second role. In making
a diagnosis and a prescription, {Riika} isn't just saying that the
patient has dengue fever. She is also saying that dengue fever is the
cause of the symptoms. And that requires excluding a lot of other
possible diseases, either on the basis that they are inconsistent with
the symptoms displayed, or because they are initially implausible and
the evidence does not sufficiently raise their likelihood to make them
worth taking seriously. If the patient has dengue fever and some other
equally serious disease that causes some of the symptoms, then to
diagnose dengue fever is to some extent to mis-diagnose the patient. And
to start the treatments for dengue fever is, in such a case, to
mis-treat the patient. In these respects, forming a diagnosis of dengue
fever is importantly different, and stronger, to forming a belief that
the patient has dengue fever.

This exclusion of alternative diseases must be prior to the diagnosis of
dengue fever. Imagine how strange it would sound for {Riika} to have
this conversation with her supervisor:

\begin{quote}
Supervisor: Why do you think that the patient does not have West Nile?\\
{Riika}: Well, the patient has a fever, headache, a rash etc.\\
Supervisor: Yes, those are all consistent with West Nile.\\
{Riika}: Ah, but you see, from those symptoms we can conclude that the
patient has dengue fever.\\
Supervisor: Yes, and?\\
{Riika}: So the symptoms have been fully explained, so there is no
reason to believe the patient has West Nile.
\end{quote}

That's not good reasoning. It would be perfectly good to reason that the
symptoms aren't consistent with West Nile, so the patient doesn't have
West Nile. Or that West Nile is very rare among people with the
patient's background, so it is better to conclude that he has a disease
that is (much) more prevalent in areas he has been. But it isn't good to
first diagnose the patient with dengue fever, and use that to conclude
they don't have West Nile.

So a good diagnosis draws on lots of background information. So that
information must in some sense be available to the doctor. I don't mean
that the information has to be accessible in the sense that she could
recite it off hand. But she must be able to base her diagnosis on the
background information. And if she's been awake for 36 hours, then that
information is probably not available, even in this weak sense. As I
will discuss in section 8.4, there are hard questions about just when it
is that evidence previously acquired can still be used. But it is
plausible that the relevant information that excludes other diagnoses is
not something {Riika} can use in her tired state.

There is another complication to consider here. {Riika} has to rule out
particular alternatives like West Nile before she can diagnose the
patient with dengue fever. But she also has to rule out, collectively,
alternative explanations she hasn't thought of, or may have forgotten.
It's not enough that the alternative explanations simply fail to exist.
If one knows the patient has yellow eyes, and as a matter of fact the
only possible explanation for this is that they are jaundiced, it
doesn't follow that one is in a position to rationally conclude the
patient is jaundiced. One must know that only jaundice causes yellow
eyes, or at least that it's the only plausible cause. And the same holds
for all other diagnoses.

It is here that concerns about one's own alertness become particularly
pressing. At least in my own case, the most worrying consequence of
excessive tiredness is that I overlook alternative explanations of
phenomena. When that happens, my abductive inferences to particular
explanations are unreasonable because I should have looked harder for
alternatives before settling on one explanation. So let's spend some
time thinking about how this might affect the reasonableness of
{Riika}'s diagnosis.

\section{Tiredness and Abduction}\label{tirednessandabduction}

We'd like to show that NR is true, and even better, that LNR is true,
without positing any kind of level-crossing principle.

\begin{description}
\tightlist
\item[NR]
It is Not Reasonable for {Riika} to believe that her patient has dengue
fever.
\item[LNR]
When she Learned that she had been awake for 36 hours, it became Not
Reasonable for {Riika} to believe that her patient has dengue fever.
\end{description}

Since we're not using level crossing principles, we can't reason as
follows.

\begin{enumerate}
\def\labelenumi{\arabic{enumi}.}
\tightlist
\item
  {Riika} has been awake for 36 hours, and she knows this.
\item
  So it is reasonable for her to believe that her diagnoses are
  unreasonable.
\item
  Whenever it is reasonable to believe that some mental state is
  unreasonable, it is unreasonable to maintain that mental state.
\item
  It was not unreasonable to believe that she'd made a reasonable
  diagnosis before learning how long she'd been awake.
\item
  So, from 3 and 4, LNR is true.
\end{enumerate}

If we want to reject level-crossing principles, then we have to reject
step 3 of that purported explanation. We need to find something to put
in its place. I'm going to offer three explanations. The first two are
probably flawed. But I'm offering them in part because they aren't
obviously wrong, and would solve the problem without appeal to
level-crossing principles. And, more importantly, thinking through
what's wrong with these explanations helps us see what's right about the
correct explanation of {Riika}'s case. Here is the first of these
probably flawed explanations.

\begin{enumerate}
\def\labelenumi{\arabic{enumi}.}
\tightlist
\item
  To reasonably conclude that \emph{p} by abductive inference, {Riika}
  needs to antecedently, reasonably believe that other explanations of
  the data fail.
\item
  Her best evidence is that other explanations of the data fail is that
  (a) it seems to her that no other explanation works, and (b) she is a
  reliable judge of when alternative explanations are available.
\item
  When she learns she has been awake for 36 hours, she is no longer in a
  position to reasonably use part (b) of that evidence.
\item
  So LNR is true; once she learns that she has been awake for 36 hours,
  she can no longer reasonably make the abductive inference from the
  data to the diagnosis of dengue fever.
\end{enumerate}

I suspect there are two, related, mistakes in this explanation. It
relies on a `psychologised' conception of evidence, and Timothy
Williamson (\citeproc{ref-Williamson2007}{2007}) has argued convincingly
against that conception of evidence. It isn't at all obvious that
{Riika} has to reason from how things seem to her to conclusions about
the world in order to form medical diagnoses.

And it isn't obvious that {Riika}'s has to form a reasonable belief that
there are no alternative explanations, and that she has to do so before
forming the diagnosis. It might be that an abductive inference is
reasonable if one's evidence rules out alternative explanations of the
data, and one is reliably disposed to consider alternative explanations
when they are not ruled out. In other words, an abductive inference
might be good (in part) in virtue of being based in a skill in
considering explanations, and that skill may be manifest when the
abductive conclusion is drawn, not antecedently to it being drawn.

Even if all that is true, there is still a skill that is needed. That
skill needs to reliably rule out alternative explanations. And {Riika}
is really tired; maybe she can't exercise that skill while so tired.
This idea leads to our second (probably mistaken) explanation.

\begin{enumerate}
\def\labelenumi{\arabic{enumi}.}
\tightlist
\item
  To reasonably conclude that \emph{p} by abductive inference, {Riika}
  has to be able to reliably rule out alternative explanations as
  unreasonable.
\item
  Since she's been awake for 36 hours, {Riika} cannot reliably rule out
  alternative explanations of the symptoms as unreasonable.
\item
  So NR is true; {Riika} cannot make the diagosis reasonably because she
  cannot reliably rule out alternatives.
\end{enumerate}

One shortcoming of this explanation is that it doesn't explain LNR.
Indeed, if the premises here are true, then LNR is in fact false. It is
the fact that {Riika} has been awake for 36 hours that makes her
diagnosis unreasonable, not her learning that she's been awake that
long. To the extent that we think LNR is true, that's a reason to
dislike the explanation.

A bigger problem for this explanation is that we don't really know that
premise 2 is true. What we know is that folks in general who have been
awake as long as {Riika} as not reliable. But perhaps she is an
exception. Indeed, the setup of the example suggests she may well be an
exception. The fact that other people in her position are unreliable
does not entail that she is unreliable. Or, at least, it doesn't entail
this without some strong assumptions about the reference class that is
relevant to {Riika}'s reliability. So let's try a different explanation.

The alternative explanation starts with the observation that the
reliability of a mechanism is not normally enough for it to produce
reasonable, or rational, beliefs. If a scale is working, but there is
excellent testimonial evidence that it is not working, it is
unreasonable to believe what the scale says. This applies to internal
mechanisms too. If one is reliably told that one is in an environment
full of visual illusions, it is unreasonable to believe what one sees,
even if one's eyesight is reliable.

A similar story holds true for skills. To learn that the patient has
dengue fever, {Riika} has to exercise her skill at reliably ruling out
alternative explanations of the data. And while she has such a skill,
she has no reason to believe that she has it. Indeed, she has a positive
reason to believe that she lacks it, since she has been awake so long,
and people who have been awake that long typically lack the skill. So
she should not rely on the skill. Here, then, is my preferred
explanation for what's going on in {Riika}'s case. I'll call that
explanation the \emph{evidentialist explanation} in what follows, since
it makes key use of how evidence does (or does not) change in explaining
changes in what states it is rational to hold.

\begin{enumerate}
\def\labelenumi{\arabic{enumi}.}
\tightlist
\item
  To reasonably conclude \emph{p} by abductive inference, {Riika} must
  reasonably rely on her skill at excluding alternative explanations of
  the data.
\item
  It is not reasonable to rely on a skill if one has excellent,
  undefeated, evidence that one does not currently possess the skill.
\item
  So, once {Riika} learns she has been awake 36 hours, she cannot
  reasonably infer from the observed data to the conclusion that the
  patient has dengue fever.
\end{enumerate}

If this explanation is correct, the case is not a counterexample to
Change Evidentialism, and we do not need to appeal to level-crossing
principles. {Riika} had to rely on her sensitivity to explanations she
had not considered in order to have a justified diagnosis. Even though
she is, in the circumstances, sufficiently sensitive to alternative
explanations, she could not reasonably rely on that sensitivity when she
has such good evidence that her skills are temporarily diminished. So
her belief that the patient has dengue fever is unjustified.

That is our explanation of why {Riika} loses knowledge, and loses
reasonable belief, when she learns that she has been awake for 36 hours.
But it isn't the only possible explanation. There are, for example,
explanations that appeal to level-crossing principles. Why should we
prefer the explanation I just offered? As I'll argue in the next
section, the answer is that only this explanation in terms of skill can
generalise to cover all of the cases.

\section{Explaining all Four Cases}\label{explainingallfourcases}

Let's start with Raina. Unlike {Riika}, Raina needs neither specialist
background information, nor expert insight, to form a diagnosis. There's
a guy with his hair on fire, and she comes to the belief that his hair
is on fire. She perhaps needs the background information that burning
hair burns and smells, and has a distinctive fiery appearance, but most
adults will have that information ready to hand in case of emergency. So
the kinds of evidence that are threatened by fatigue are not needed to
form the judgment in Raina's case. So she still knows, even in her
fatigued state, that her patient's hair is on fire. Since judging that
the patient's hair is on fire doesn't require any particular skill, it
doesn't matter that her skills are diminished.

Unlike {Riika}, {Roshni} didn't have enough public information to
conclude her patient had dengue fever. She needed the extra step that
there are no other plausible explanations of the data. Since there are
other plausible explanations of the data, she can't know there are none.
Hence it cannot be part of her evidence that there are none. Being
fatigued might explain why one's `insights' do not really constitute
evidence. But it can't turn non-insights, and non-facts, into evidence.
So even in her semi-fatigued state, {Roshni} still lacks sufficient
evidence to diagnose her patient with dengue fever. So she still doesn't
know her patient has dengue fever, as we hoped to explain.

We'll spend much more time on this in chapter 11, but for now note one
quick reason to suspect that {Roshni}`s credence that her patient has
dengue fever should not move at all. Assume that she learns not just
that residents who have been on duty 12--14 hours are systematically
under-confident in their diagnoses, but that they remain so after making
their best efforts to incorporate this information about their own
under-confidence. And assume that {Roshni} should, on learning that she
is part of a group that is systematically under-confidence, increase her
confidence in her preferred diagnosis. Now we have a perpetual
confidence increasing machine. Even once she has increased her
confidence in light of the information about herself, she has reason to
increase it again, since she is still in a group that systematically is
too cautious in their judgments. And this fact persists no matter how
hard she tries. But perpetual confidence increasing machines, like
perpetual motion machines, are absurd. The best place to stop this
machine is at the very start. So {Roshni} should not increase her
confidence at all. (I think this is intuitively the right thing to say
about her case, but this argument is offered to those who don't share
the intuition.) And that in turn provides reason to not just believe the
evidentialist explanation of {Riika}'s case, but to believe the
'non-psychologised' version of that explanation.

The really tricky case, from this perspective, is {Regina}. She doesn't
need any skill in identifying possible alternative explanations of the
data. She just needs to remember some facts from her medical training,
make some straightforward observations, and perform a very simple
logical deduction. Her tiredness does not affect her ability to make the
observations or, I suspect, to do this deduction. A tired person may
struggle to draw complicated consequences from data, but going from
\emph{All Fs are Gs} and \emph{This is F} to \emph{This is G} does not
require particular skill.

The big question is whether {Regina} can really rely on her memory when
she is tired. It is helpful to think about this case by comparing it to
the Shangri-La example developed by {Frank} Arntzenius
(\citeproc{ref-Arntzenius2003}{2003}). Here is the slightly simplified
version of the case that Michael Titelbaum sets out.

\begin{quote}
You have reached a fork in the road to Shangri La. The guardians of the
tower will flip a fair coin to determine your path. If it comes up
heads, you will travel the Path by the Mountains; if it comes up tails,
you will travel the Path by the Sea. Once you reach Shangri La, if you
have traveled the Path by the Sea the guardians will alter your memory
so you remember having traveled the Path by the Mountains. If you travel
the Path by the Mountains they will leave your memory intact. Either
way, once in Shangri La you will remember having traveled the Path by
the Mountains. The guardians explain this entire arrangement to you, you
believe their words with certainty, they flip the coin, and you follow
your path. What does ideal rationality require of your degree of belief
in heads once you reach Shangri La.
~(\citeproc{ref-Titelbaum2014}{Titelbaum 2014, 120})
\end{quote}

The name of the person Titelbaum's narrator is addressing isn't given,
so we'll call him {Hugh}. And we'll focus on the case where {Hugh}
actually travels by the Mountains.

There is something very puzzling about {Hugh}'s case. On the one hand
many philosophers (including Arntzenius and Titelbaum) report a strong
intuition that once in Shangri La, {Hugh} should have equal confidence
that he came by the mountains as that he came by the sea. On the other
hand, it's hard to tell a dynamic story that makes sense of that. When
he is on the Path by the Mountans, {Hugh} clearly knows that he is on
that path. It isn't part of the story that the paths are so confusingly
marked that it is hard to tell which one one is on. Then {Hugh} gets to
Shangri La and, well, nothing happens. The most straightforward dynamic
story about {Hugh}'s credences would suggest that, unless something
happens, he should simply retain his certainty that he was on the Path
by the Mountains.

Resolving the tension here requires offering a theory of the
epistemology of memory. And I have no desire to do that, any more than I
had a desire in the ethics part of the book to offer a first-order
ethical theory. What I am going to do is say why hard questions within
the epistemology of memory are relevant to what we should say about
{Hugh}'s case, and by extension {Regina}'s case.

Some theories of memory are synchronic. Whether the agent's mental state
at time \emph{t} makes it rational for her to believe that \emph{p}, on
the basis of her (apparent) memories, solely depends on the the
properties she possesses at \emph{t}. There are two natural ways to fill
in the synchronic theory. First, we could say that the agent's faculty
of memory outputs propositions that become, if it is a reliable faculty,
evidence for the agent. (It's presumably a gross oversimplification of
the best cognitive and neural theories of how memory works in humans to
describe it as a faculty, but we'll have to work with such
simplifications to get a broad enough view of the philosophical
landscape.) Second, we could say that the apparent memories the agent
has provide her evidence, and she can then reason using either what she
knows about herself, or perhaps some default entitlements to trust
herself that she possesses, to the truth of the contents of those
memories.

On either kind of synchronic theory, {Hugh} won't know that he came to
Shangri La via the mountains. If memory provides evidence directly, it
does so only when it is reliable. And on this question, it is
unreliable, since in nearby worlds it produces mistaken outputs. It's
true that there is nothing funky about the causal chain leading to
{Hugh}'s memory. But on a synchronic theory of memory, the nature of the
chain is not relevant; all that is relevant is the reliability of the
output. And the output is not reliable. If, on the other hand, the
evidence is something like the apparent memory {Hugh} has, then things
are even worse. He knows that he can't reason from his apparent memory
to any claim about how he got to Shangri La, because in very nearby
worlds his apparent memories are badly mistaken.

Arntzenius argues that {Hugh} should have a credence of 0.5 that he came
by the mountains as follows. (Assume Arntzenius is talking to Hugh here,
so `you' picks out Hugh.)

\begin{quote}
For you will know that he would have had the memories that you have
either way, and hence you know that the only relevant information that
you have is that the coin was fair.
~(\citeproc{ref-Arntzenius2003}{Arntzenius 2003, 356})
\end{quote}

That argument seems to presuppose that we are using the second,
psychologised, version of the synchronic theory of memory. If we
understand memories to be not just phenomenal appearances, but traces of
lived experiences, then {Hugh} would very much not have the memories
that he has either way. He might think that he had the same memories had
he come by the sea, but he'd be wrong. Still, Arntzenius's argument
doesn't seem to rely on this feature of memory. What it does seem to
rely on is that in an important sense, {Hugh} would be the same right
now however he had arrived at Shangri La. That is, it relies on a
synchronic theory of memory. Sarah Moss (\citeproc{ref-Moss2012}{2012})
makes a similar claim about the case. (Again, her narration is addressed
to Hugh.)

\begin{quote}
Intuitively, even if you travel on the mountain path, you should have .5
credence when you gets to Shangri La that the coin landed heads. This is
a case of abnormal updating: once you arrive in Shangri La, you can no
longer be sure that you traveled on the mountain path, because you can
no longer trust your apparent memory. ~(\citeproc{ref-Moss2012}{Moss
2012, 241--42})
\end{quote}

Again, the presupposition is not just that we have a synchronic
epistemology of memory, but that the evidence memory provides comes from
appearances. And, once again, the second presupposition does not seem to
really matter. We would get the same result if we took memory to provide
evidence directly, but only when it was reliable. What matters, that is,
is the synchronic epistemology of memory.

In recent work, Moss (\citeproc{ref-Moss2015}{2015}) has developed a
systematic defence of synchronic epistemology, what she usefully calls
`time-slice epistemology'. And while she makes a good case for it, there
is also a good case for a diachronic epistemology. Richard
(\citeproc{ref-Holton1999}{Holton 1999},
\citeproc{ref-Holton2014}{2014}) has argued for diachronic norms of
intention, and for understanding belief as being in important ways like
intention. From these premises he concludes that there are diachronic
norms on belief. David James Barnett (\citeproc{ref-Barnett2015}{2015})
has offered more direct arguments for adopting a diachronic epistemology
of memory. So we should work through what happens in cases like Shangri
La on a diachronic approach.

It turns out that we quickly face another choice point. The cases we are
interested in are ones where an agent knows \emph{p} at an earlier time
\emph{t}\textsubscript{1}, and then this belief is preserved from
\emph{t}\textsubscript{1} to a later time \emph{t}\textsubscript{2}. The
theoretical choice to make is, is this sufficient for the agent to know
\emph{p} at \emph{t}\textsubscript{2}, or could the knowledge be
defeated by things that happen in the interim? If the knowledge could
not be defeated, then {Hugh} knows he came by the mountains, for the
obvious reason that he once knew this and has never forgotten it. If it
can be defeated, then on any of the most obvious ways to incorporate
defeat into the theory, {Hugh}'s claim to knowledge will be defeated. He
is, after all, part of a group (explorers who arrive at Shangri La) who
have very unreliable memories, and he knows that.

Whatever we say about defeat here can be made consistent with Change
Evidentialism\footnote{Note that the key notion in the statement of
  Change Evidentialism is \emph{change} of evidence, not accrual of
  evidence. Losing evidence matters too.}. Since we're developing a
diachronic epistemology, we should allow that evidence can be accrued
over time. On the version of the theory where memories are indefeasible,
{Hugh}'s evidence that he came via the mountains is his perception of
the mountain path. This perception can be his evidence well into the
future, as long as his memory does its job of preserving the visual
evidence. (He could of course forget how he got to Shangri La, but we're
only discussing cases where beliefs are preserved throughout the
relevant time period.) If memories can be defeated, the Change
Evidentialist should say that the defeaters prevent the past perceptions
from being current evidence. (In general, I think the evidentialist
should say that defeaters prevent propositions becoming part of one's
evidence. But defending that claim would take us too far afield.) If his
evidence does include the contents of his perceptions while on the path,
then he now knows that he came via the mountains, if it does not he does
not. Either way, it is the change or lack of change of evidence (and not
merely his worries about his own reliability) that explain why he knows
what he does.

I've described four theories of memory, two synchronic and two
diachronic. On three of the four theories, {Hugh} does not know, indeed
does not even have reason to be particularly confident, that he came by
the mountain. On the fourth he does know this. I think that's a
reasonable stopping point; it's left as a somewhat difficult
philosophical question whether {Hugh} knows that he came via the
mountains. But it's not one we have to settle the big picture views I've
been defending, since either answer to the philosophical question about
memory is consistent with those views.

And what we say about {Hugh} carries over to {Regina}'s case. The big
issue is whether she (still) has the following two propositions as
evidence.

\begin{enumerate}
\def\labelenumi{\arabic{enumi}.}
\tightlist
\item
  All lethargic patients with yellow eyes are jaundiced.
\item
  All jaundiced patients should be treated with quinine.
\end{enumerate}

If she has 1 and 2, then she should treat her patient with quinine. This
isn't, or at least isn't just, because 1 and 2 entail that she should
treat her patient with quinine. It's rather because these pieces of
evidence provide strong and immediate support for the claim that she
should treat her patient with quinine.

Does she (still) have those propositions as evidence, or as something
she can derive and use as evidence? On either synchronic theory of
memory, she does not. Her apparent memory of 1 and 2 cannot ground an
inference to the truth of 1 and 2, since she knows that she is
unreliable given her fatigue. Alternatively, if memory delivers
propositions like 1 and 2 directly, the fact that she is so fatigued
right now will defeat memory's claim to being a source of evidence. If
we adopt a diachronic theory of memory, then what matters is whether we
allow for (anything like) defeaters. If we do, her current fatigue is,
probably, a defeater, so she again doesn't know that her patient should
be treated with quinine. But on the (not totally implausible!)
diachronic theory that rejects defeaters, we get that she does know. I
think this is the right result; {Regina}'s case is not as clear as
{Riika}'s, and it is right that it turns on hard philosophical
questions.\footnote{In a recent paper
  ~(\citeproc{ref-Weatherson2015}{Weatherson 2015}) I take a stand on
  some of these questions about memory in ways that go beyond what is
  necessary for rejecting level-crossing.}

If we explain {Riika}'s case using level-crossing principles, then we
should say that {Regina}'s case does not turn on hard philosophical
questions. On this approach, {Regina}'s case is easy. She can't
rationally believe that she rationally believes that the patient is
jaundiced, so she can't rationally believe that the patient is
jaundiced. Now this seems to me to be the wrong result in {Regina}'s
case. It's wrong twice over; it says the wrong thing about {Regina}, and
it says the case is easy when in fact it is hard. But because the
question is hard, I don't want to lean any argumentative weight on it.
And I doubt that we should ever put much argumentative weight on
intuitions about whether cases are hard or easy. Instead I'll argue
against the application of level-crossing principles to {Riika}'s case
by comparing {Riika}'s case with Raina's and {Roshni}'s.

The level-crossing explanation of {Riika}`s case provides no resources
to distinguish between {Riika}'s case and Raina's. Both of them have
reasonably responded to the evidence that is available. Both of them
then get evidence that they are (temporarily) unlikely to be responding
correctly to evidence. These facts are, in {Riika}'s case, held to be
sufficient to explain why she should change her view. But they are
features of {Riika}'s case that are shared with Raina's case. Since
Raina should not change her view on being told she has been awake for 36
hours, we need either something more, or something else. An explanation
of {Riika}'s case based on level-crossing principles will
over-generalise; it will 'explain' why Raina should change her mind too.

{Roshni} is even more of a challenge for explanations that rely on
level-crossing principles. Let \emph{p} be the proposition \emph{The
patient might not have dengue fever}. At the start of the story,
{Roshni} believes that, and rationally so. But then she gets evidence
that she cannot rationally form beliefs like that given her state. So,
if the level-crossing principle is true, then she should lose the belief
in \emph{p}. But if she thinks that the patient's having dengue fever is
at least very likely, and does not believe that it might be false, that
sounds to me like she believes it. That is, the only way to comply with
the level-crossing principles is to believe the patient does have dengue
fever. And that conclusion is absurd.

So {Roshni} is a counterexample to a lot of level-crossing principles.
The following claims about her are true:

\begin{itemize}
\tightlist
\item
  {Roshni} rationally believes that \emph{p}.
\item
  {Roshni} could not rationally believe that she rationally believes
  that \emph{p}.
\item
  {Roshni} should believe that her evidence does not support rational
  belief in \emph{p}.
\end{itemize}

And level-crossing principles are meant to rule out just those
combinations. So {Roshni}'s case does not just undermine an abductive
argument for level-crossing principles, it provides direct evidence that
those principles are mistaken.

\section{Against Bracketing}\label{againstbracketing}

David Christensen draws a different response to these puzzles involving
higher-order evidence. His theory is that higher-order evidence requires
us to `bracket' first-order evidence. Here is how he introduces the
idea. (The background is that he is discussing a case where he did a
logic problem, got the right answer, and then was told he took a drug
that distorts most people's logical abilities.)

\begin{quote}
It seems to me that the answer comes to something like this: In
accounting for the HOE (higher order evidence) about the drug, I must in
some sense, and to at least some extent, \emph{put aside} or
\emph{bracket} my original reasons for my answer. In a sense, I am
barred from giving a certain part of my evidence its due. After all, if
I could give all my evidence its due, it would be rational for me to be
extremely confident of my answer, even knowing that I'd been drugged. In
fact, it seems that I would even have to be rational in having high
confidence that I was immune to the drug. By assumption, the drug will
very likely cause me to reach the wrong answer to the puzzle if I'm
susceptible to it, and I'm highly confident that my answer is correct.
Yet it seems intuitively that it would be highly irrational for me to be
confident in this case that I was one of the lucky immune ones. \ldots{}
Thus it seems to me that although I have conclusive evidence for the
correctness of my answer, I must (at least to some extent) bracket the
reasons this evidence provides, if I am to react reasonably to the
evidence that I've been drugged.
~(\citeproc{ref-Christensen2010a}{Christensen 2010a, 194--95}, emphasis
in original)
\end{quote}

There are a few different arguments here that we need to tease apart.

There is an argument that bracketing is needed because otherwise the
narrator will have `conclusive' evidence for the answer to the logic
problem. This isn't right; or at least it is misleading. In a sense
seeing my coffee cup on my desk is conclusive evidence for the truth of
any mathematical proposition. It does entail it. But it's a terrible
reason to believe, for example, Fermat's Last Theorem. There is another
sense of conclusive that is more relevant; whether some evidence
provides epistemically conclusive reason to believe a conclusion. And
mere entailment does not suffice for that.

There is an argument I think implicit in Christensen's remarks that if
we allowed the first order evidence to stand, we'd be licencing some
improperly circular reasoning. That's an interesting observation, and
I'll discuss it at more length in the next chapter.

But what we're interested in is the conclusion, that the original
evidence must be bracketed or set aside in cases where higher order
evidence suggests we are likely to be making a mistake. And that
conclusion, we can now see, can't be right. It can't be right because of
Raina's case and {Roshni}'s case. If Raina brackets her first order
evidence, she won't have reason to put out the fire in her patient's
hair. But she has excellent, indeed compelling, reason to do that. And
if {Roshni} brackets her first order evidence, she will have sufficient
reason to believe that her patient has dengue fever, and to start
treating him. But she does not have sufficient reason to do that.

These cases aren't isolated incidents. They point to two general
problems with the bracketing picture. It doesn't distinguish between
cases where evidence immediately supports a conclusion, and cases where
the evidence supports the conclusion more indirectly. The latter cases,
ones where the agent must use the initial evidence to derive more
evidence, and then use the larger evidence set to support the
conclusion, are cases where higher order evidence matters. But the
reason higher-order evidence matters in those cases is that higher-order
evidence blocks those intermediate steps. Cases like Raina's are
different, but the bracketing story does not distinguish them. And the
bracketing story can't explain the existence of epistemic gravity, while
the evidentialist explanation I've offered can.

There are other cases that, while not as clear, seem to me cases the
bracketing story cannot handle correctly. The following case is inspired
by some examples presented Jonathan Weisberg
(\citeproc{ref-Weisberg2010}{2010}).

{Jaga} has been taking some medication. She knows that she has taken the
medication for 22 days, and that she has taken 18 pills each day. She
then learns some very worrying news. The medication is being withdrawn
from sale because it has a striking effect on anyone who takes 400 or
more pills; it makes them incredibly bad at arithmetic for several
weeks. The effect is surprisingly sharp in its effect; anyone who has
taken 399 or fewer is unaffected, but once one has taken the 400th pill,
it kicks in with full force. (Yes, this is a very unrealistic case, but
more realistic cases are possible, and would simply be more complicated
to discuss.)

Now {Jaga} is very worried. She knows that she has taken 22 times 18
pills. But she is unsure what 22 times 18 is. That's not unreasonable;
most of us wouldn't know what it is off the top of our heads either,
without doing the calculation. And one of the things that worries {Jaga}
is that before doing the calculation, it seems pretty likely to her that
it is greater than 400. And that isn't unreasonable either. It's wrong,
but well within the reasonable range of error.

So {Jaga} does the calculation. She works out that 22 times 18 is 20
times 18 plus 2 times 18, so it is 360 plus 36, so it is 396. Wonderful,
she thinks, I haven't taken too many pills. So I can do arithmetic well,
as indeed I just did. That's exactly the right attitude for {Jaga} to
have. Her evidence does not actually show that she is bad at arithmetic.
Before she sat down to do the calculations, she should have worried that
she was bad at arithmetic. But now that she's done the calculations, she
knows better.

But note this isn't what a defender of the bracketing view can say about
{Jaga}'s case. There is a serious doubt about whether she is good at
arithmetic, and relatedly about whether she has taken 400 or more pills.
She can't resolve that by appeal to her first order evidence about
whether she has taken 400 or more pills, since whether her calculations
provide her with reason to believe that she's taken 400 or more pills is
exactly what is at issue. More formally, let \emph{p} be the proposition
that she's taken less than 400, and \emph{q} be the proposition that
she's good at arithmetic. The intuition behind the bracketing view is
that one can't come to believe \emph{q} by doing some arithmetic and
trusting your answers. Yet that is exactly what {Jaga} has done,
admittedly via the roundabout route of coming to believe \emph{p}, and
antecedently knowing that \emph{q} is true iff \emph{p} is true.

The point of {Jaga}'s case is that bracketing has implications not just
in cases where an agent gets evidence that does suggest she is
irrational or unreliable, but also in cases where she gets evidence that
might suggest that. And those implications are much less plausible than
they are in the cases where the force and direction of the higher-order
evidence is clearer. We'll return to such cases extensively in chapter
10. The next priority, however, is to deal with the circularity worry.
If we reject level-crossing principles, and accept Change Evidentialism,
are we committed to accepting what are in fact bad kinds of circular
reasoning?

\chapter{Circles, Epistemic and Benign}\label{circlesepistemicandbenign}

\section{Normative Externalism and
Circularity}\label{normativeexternalismandcircularity}

Some of the views that I'm opposing are motivated by anti-circularity
considerations. Consider, for instance, the principle David Christensen
calls Independence, which is a version of the bracketing principle that
was the focus of the previous section. I'm quoting it here with the
argument for it that immediately follows.

\begin{quote}
\begin{description}
\item[\textbf{Independence}]
In evaluating the epistemic credentials of another's expressed belief
about P, in order to determine how (or whether) to modify my own belief
about P, I should do so in a way that doesn't rely on the reasoning
behind my initial belief about P.
\end{description}

The motivation behind the principle is obvious: it's intended to prevent
blatantly question-begging dismissals of the evidence provided by the
disagreement of others. It attempts to capture what would be wrong with
a P-believer saying, e.g., ``Well, so-and-so disagrees with me about P.
But since P is true, she's wrong about P. So however reliable she may
generally be, I needn't take her disagreement about P as any reason at
all to question my belief.''
~(\citeproc{ref-Christensen2011}{Christensen 2011, 1--2})
\end{quote}

To my eyes, this argument seems to involve a category mistake. Moves in
a dialectic can be question-begging or not. But here Christensen seems
to want to put restrictions on rational judgments on the grounds that
the alternative would be question-begging. That seems like the wrong way
to get the desired end. If we want to stop ``blatantly question-begging
dismissals'' we can just remind people not to be rude.

I think the problem Christensen is highlighting is not to do with
question-begging, but to do with circularity. The problem is that if we
violate \textbf{Independence}, we can use our reasoning to conclude that
our reasoning is reliable, and that's circular. Or, to be more accurate,
it has a whiff of circularity about it. Trying to turn this into an
argument for \textbf{Independence} though will be difficult.

Part of the difficulty is that it isn't easy to say exactly what the
circularity involved is. Consider the following little example, where
{Chiyoko} and {Aspasia} are discussing arithmetic. They know that
exactly one of them has taken a drug that makes people bad at simple
arithmetic. {Chiyoko} does some sums in her head, listens to {Aspasia},
and reasons as follows.

\begin{enumerate}
\def\labelenumi{\arabic{enumi}.}
\tightlist
\item
  2+2=4, and 3+3=6, and 4+5=9, and 7+9=16.
\item
  {Aspasia} believes that 2+2=5, and 3+3=7, and 4+5=8, and 7+9=15, while
  I believe that 2+2=4, and 3+3=6, and 4+5=9, and 7+9=16.
\item
  So, she got those four sums wrong, and I got them right.
\item
  It is likely that I would get at least one of them wrong if I'd taken
  the drug, and unlikely that she would get all four wrong unless she'd
  taken the drug.
\item
  So, probably, I have not taken the drug, and she has.
\item
  So I should not modify my beliefs about arithmetic in light of what
  {Aspasia} says; she has taken a drug that makes her unreliable.
\end{enumerate}

It isn't clear to me just which step is meant to be circular. If
{Chiyoko} had reasoned as follows, I could see how we might take her
reasoning to be circular.

\begin{enumerate}
\def\labelenumi{\arabic{enumi}.}
\tightlist
\item
  It seems to me that 2+2=4, and 3+3=6, and 4+5=9, and 7+9=16.
\item
  It seems to {Aspasia} that 2+2=5, and 3+3=7, and 4+5=8, and 7+9=15.
\item
  From 1, it's true that 2+2=4, and 3+3=6, and 4+5=9, and 7+9=16.
\item
  From 1, 2 and 3, my arithmetic seemings are reliable, and {Aspasia}'s
  are not.
\item
  So, probably, I have not taken the drug, and she has.
\item
  So I should not modify my beliefs about arithmetic in light of what
  {Aspasia} says; she has taken a drug that makes her unreliable.
\end{enumerate}

If {Chiyoko} reasons this way, the only reason for thinking she is right
and {Aspasia} is wrong is her own judgment, which is exactly what is at
issue in 6. But that isn't at all how people usually reason. Nor is it a
sensible rational reconstruction of their reasoning. Rather, the first
version of the inference is much more like the way normal human beings
do, and should, reason. And in this case the symmetry of the dispute
between {Chiyoko} and {Aspasia} is broken by a fact recorded at line 1,
namely that 2 plus 2 really is 4, 3 plus 3 really is 6, and so on. And
while {Chiyoko} uses her mathematical competence to come ot know that
fact, she doesn't learn it by reasoning about her mathematical
competence. If she did, it would be a posteriori knowledge, whereas in
fact it is a priori knowledge. So if there is some circular reasoning
going on in the first inference, the circularity is fairly subtle, and
it won't be easy to say just what it is.\footnote{David James Barnett
  (\citeproc{ref-Barnett2014}{2014}) also notes that it is important to
  distinguish the case where Chiyoko uses her mental faculties from the
  case where she reasons about them. He thinks, and I agree, that once
  we attend to this distinction, it is far from clear that there is
  anything problematically circular about what Chiyoko does.}

Still, there is some vague feeling of circularity that goes along with
even that first inference. And in principle we shouldn't say that some
reasoning is acceptable just because we can't precisely articulate the
sin it commits. Compare: We shouldn't say that the Dharmottara cases
described by Jennifer Nagel (\citeproc{ref-Nagel2014}{2014, 57}) are
cases of knowledge just because it is hard to say exactly what makes
them not knowledge.\footnote{Cases with the same structure as
  Dharmottara's became the focus of some discussion in the Anglophone
  philosophical tradition after they were independently discovered by
  Edmund Gettier (\citeproc{ref-Gettier1963}{1963}).} Call this the
`whiff of circularity' objection to normative externalism, since
normative externalism arguably licences the first form of reasoning, but
there is a whiff of circularity about it. The aim of this chapter is to
respond to the whiff of circularity objection. Much of our time will be
spent trying to make the objection more precise. (As David Lewis almost
said, I cannot reply to a whiff.) We'll start with the worry that the
objection trades on a fundamental confusion between inference and
implication.

\section{Inference, Implication and
Transmission}\label{inferenceimplicationandtransmission}

As Gilbert G. Harman (\citeproc{ref-Harman1986}{1986}) has pointed out,
it is very important to separate the theory of implication, i.e., logic,
from the theory of inference, which sits in the intersection between
psychology and epistemology. The following argument is perfectly valid,
even though following it would make a lousy inference.

\begin{enumerate}
\def\labelenumi{\arabic{enumi}.}
\tightlist
\item
  The Eiffel Tower is large.
\item
  The Eiffel Tower is not large.
\item
  So, London is pretty.
\end{enumerate}

Using terminology drawn from work by Crispin Wright
(\citeproc{ref-Wright2000}{2000}, \citeproc{ref-Wright2002}{2002}), we
might say this is a case where warrant does not transmit from the
premises to the conclusion. An agent could not gain warrant for the
conclusion of this argument by gaining warrant for its premises. But
that does not tell against the validity of the argument. Whenever the
premises are true, so is the conclusion. Any proof of the premises can
be converted into a proof of the conclusion. And so we have excellent
reason to believe the argument is valid, even though it does not ground
any good inference.

Rather than using Wright's slightly technical term `warrant', we'll
focus on the class of \textbf{Potential Teaching Arguments}, or PTAs.
These are arguments where an agent could come to learn the conclusion by
first learning the premises, and then reasoning from them to the
conclusion. The modal term `could' there is context-sensitive, and
vague. The context sensitivity comes from the fact that whether an
argument is a PTA might depend on which agent we are focussing on, and
on how that agent came to know the premises. Imagine, for example, that
{Marie} is a scientist who is working on a machine to measure the
relative radioactivity of two substances. The machine is, it turns out,
very accurate, but it is also the first of its kind, and the theory
behind it is somewhat speculative. Now consider this argument.

\begin{enumerate}
\def\labelenumi{\arabic{enumi}.}
\tightlist
\item
  {Marie}'s machine says that \emph{a} is more radioactive than
  \emph{b}.
\item
  In fact, \emph{a} is more radioactive than \emph{b}.
\item
  So, {Marie}'s machine is accurate about \emph{a} and \emph{b}.
\end{enumerate}

That's a valid argument, but it isn't a PTA. At least, it isn'a PTA for
{Marie} while she is in the process of building and testing her machine,
if her evidence for 2 is simply that 1 is true. She can't learn that the
machine is accurate by simply trusting its readings. That's true even if
it is, in fact, reliably accurate. Jonathan Vogel
(\citeproc{ref-Vogel2000}{2000}) has argued that this is a problem for
many forms of reliabilism. Stewart Cohen
(\citeproc{ref-Cohen2002}{2002}, \citeproc{ref-Cohen2005}{2005}) has
offered a generalisation of Vogel's argument that threatens normative
externalism plus evidentialism, and we'll return to Cohen's argument
later in this chapter. But for now we just need to note that this
argument is not a PTA for {Marie}, using her new machine, while it might
be for other agents. A historian of science a century after {Marie},
trying to retrospectively figure out how accurate {Marie}'s innovative
machine was, could use this argument in their inquiry.

So when we say that an argument is, or is not, a PTA, we mean to be
talking about a particular, contextually supplied, agent, using
something like the methods for learning the premises that they actually
use. The phrase `something like' is obviously rather vague, but the
vagueness shouldn't worry us overly, as it won't compromise the
discussion to come.

We have already seen some valid arguments that are not PTAs. The
argument from the Eiffel Tower to London might not be a PTA for anyone
in any possible world. There is a radical version of the view that
inference and implication must be kept separate which says that there
are literally no valid PTAs. On this view, we never learn by following
arguments from premises to conclusions, and thinking we do is a sign one
has not properly appreciated the inference/implication distinction. I
doubt this view is right. It is worth being sceptical about how often we
use valid arguments in inference, but do seem to be some cases where we
do. This schema, for instance, seems to be one we can easily use.

\begin{enumerate}
\def\labelenumi{\arabic{enumi}.}
\tightlist
\item
  \emph{a}\textsubscript{1} is the most recent \emph{F}, and it is
  \emph{G}.
\item
  \emph{a}\textsubscript{2} is the second most recent \emph{F}, and it
  is \emph{G}.
\item
  \emph{a}\textsubscript{3} is the third most recent \emph{F}, and it is
  \emph{G}.
\item
  So the last three \emph{F}s are \emph{G}.
\end{enumerate}

For a concrete instance of this, let \emph{F} be \emph{President of the
USA}, \emph{G} be \emph{is left-handed}, and the
\emph{a\textsubscript{i}} be Barack Obama, George W. Bush and Bill
Clinton, and imagine someone considering the argument in 2009. More
generally, consider cases where \emph{G} is a coincidental property of
the last 3 (or more) \emph{F}s, and we see that the last few \emph{F}s
have this property by simply working through the cases. The result is a
conclusion that we learn simply by remembering the premises, and then
doing a very simple deduction. So there are some PTAs, even if not every
valid argument is a PTA.

The clearest example of a valid argument that is not a PTA, for any
agent, is \emph{A, therefore A}. By definition, a PTA is one where the
agent could first learn the premise, and then, in virtue of that, later
come to learn the conclusion. But one cannot first learn the premise of
\emph{A, therefore A}, and later come to believe the conclusion. For
similar reasons, it will be rare that \emph{A and B, therefore A}, could
be a PTA for an agent, though perhaps there are some possible instances
of this schema, and some possible agents, for whom this is a PTA.

Why isn't the argument about radioactivity a PTA for {Marie}? In some
sense, we might say that it is because it is circular. {Marie} can't use
her new machine to learn that one of the premises is true, then use the
argument to learn that the machine is reliable. And then, presumably, go
on to use the fact that the machine is reliable to defend the second
premise of the argument. Something looks to have gone wrong.

It is tempting now to generalise from {Marie}'s case to the principle
that no argument whose conclusion is that a particular method or tool is
reliable, and whose premises were based on that method or tool, could be
a PTA. But this is too quick. Or at least, as I'll argue in the next
section, those of us who are not sceptics should think it is too quick.

\section{Liberalism, Defeaters and
Circles}\label{liberalismdefeatersandcircles}

In this section I discuss the following argument.

\begin{enumerate}
\def\labelenumi{\arabic{enumi}.}
\tightlist
\item
  Normative externalism says that some arguments that exemplify defeater
  circularity are PTAs.
\item
  No argument that exemplifies defeater circularity is a PTA.
\item
  So, normative externalism is false.
\end{enumerate}

I'm going to spend a bit of time setting up what defeater circularity
is. But the basic idea behind premise 2 is that the principle suggested
at the end of the previous section is true. And the idea behind premise
1 is that if we reject level-crossing and accept normative externalism,
we end up committed to violations of that principle. I will mostly be
concerned to argue against premise 2, though I'll note that there ways
we could push back against premise 1 as well. The ideas of this section
draw heavily on work by James Pryor (\citeproc{ref-Pryor2004}{2004}) and
we'll start with an important distinction he draws.

Pryor distinguishes three different approaches epistemological theorists
might take towards different epistemological methods. He offers labels
for two of these approaches; I've added a label for the third that
naturally extends his metaphor. In every case, we assume agent S used
method \emph{M} to get a belief in proposition \emph{p}. And we'll say
the proposition \emph{M works} is the conjunction of every proposition
of the form (\emph{M represents that q}) → \emph{q} for every salient
\emph{q}, where → is material implication. Then we have the following
three views.\footnote{I'm modifying Pryor's views a bit to make these
  attitudes towards methods, rather than towards propositions; this
  makes everything a touch clearer I think. But I'm following Pryor, and
  the literature that has built up around his work, in focussing on
  justification rather than rationality. For reasons that I discussed in
  chapter 7, I would rather focus on rationality. I think the difference
  between the two concepts is not significant to this part of the
  discussion.}

\begin{description}
\tightlist
\item[Conservatism]
S gets a justified belief in \emph{p} only if she antecedently has a
justified belief that \emph{M works}.
\item[Liberalism]
S can in some circumstances get a justified belief in \emph{p} without
having an antecedently justified belief that \emph{M works}, but in some
other circumstances she can properly use \emph{M} and not get a
justified belief in \emph{p}, because her prior evidence defeats the
support that \emph{M} provides for \emph{p}.
\item[Radicalism]
As long as S uses \emph{M} correctly, and \emph{M} genuinely says that
\emph{p}, and \emph{M} actually works, then no matter what evidence S
has against \emph{M works}, she gets a justified belief in \emph{p}.
\end{description}

Whether conservatism, liberalism or radicalism is the most intuitive
initial view will vary depending on which particular method we are
considering.

Scientific advances naturally produce a lot of methods that we should
treat conservatively. This is what we saw in the case of {Marie} and her
machine; she couldn't learn things about how radioactive some things are
until and unless she knew the machine worked. And that's true in general
of new methods we develop. But it isn't true, isn't even intuitively
true, of all methods.

Arguably we should be radicals about our most fundamental methods, such
as introspection. A child doesn't antecedently need to know that
introspection is reliable to come to have introspective knowledge that
she's in pain. As long as introspection works, it isn't clear this is
defeasible. If as the child grows up, she hears from some fancy
philosophers that there is no such thing as pain, she might get some
reasons to doubt that introspection works. But when she introspectively
(and perhaps involuntarily) forms the belief that she's in pain, she
knows she is in pain.

It is a little trickier to say which methods we should be liberals
about. Pryor (\citeproc{ref-Pryor2000}{2000}) suggests that we should be
liberals about perception. Many epistemologists, following C. A. J.
Coady (\citeproc{ref-Coady1995}{1995}) are liberals about testimony.
They deny that we need antecedent reason to believe that a particular
speaker is reliable, i.e., that that person's testimony work's before
getting testimonial knowledge. But we shouldn't just believe everything
we hear, so testimonial justification is defeasible.

Conservatism and radicalism are fairly well defined views. That is, the
class of conservative views all share a strong family resemblance to
each other, as do the class of radical views. The main thing we need to
say about distinguishing different types of conservatism is that some
conservatives have supplementary views that greatly alter the effect of
their conservatism. For instance, the Cartesian sceptic is a
conservative about perception who denies that we can believe perception
works without having perceptual beliefs. But some other philosophers are
conservatives about perception who also believe that it is a priori that
perception works. Those positions will be radically anti-sceptical. So
conservatism may have rather different effects elsewhere in
epistemology, depending on what it is combined with. But the basic idea
that one can use \emph{M} iff one has prior justification for believing
\emph{M works} gets us a fairly well defined region of philosophical
space, as does the view that one can use \emph{M} under any
circumstances at all.

In contrast to conservativism and radicalism, liberalism covers a wide
variety of fairly disparate theories. The liberal essentially makes a
negative claim, antecedent justification for believing that \emph{M
works} is not needed for getting a justified belief that \emph{p}, and
an existential claim, there is some way of blocking the support \emph{M}
provides to \emph{p}. Different liberals may have very different views
about when that existential claim is instantiated.

A conservative-leaning liberal thinks that there are a lot of ways to
block the support that \emph{M} provides to \emph{p}. One way to be a
conservative-leaning liberal is to say that whenever S has any reason to
doubt that \emph{M works}, the use of \emph{M} does not justify belief
in \emph{p}. Pryor's own view on perception is that this kind of
conservative-leaning liberalism is true about perception. If any kind of
liberalism about testimony is correct, then presumably it is a very
conservative-leaning liberalism, since it is easy to block the support
that testimony that \emph{p} provides to \emph{p}.

A radical-leaning liberal thinks that there are very few ways to block
the support that \emph{M} provides to \emph{p}, even if in principle
there are some. One natural way to be a radical-leaning liberal is that
the support is blocked only if S believes, or is rational in believing,
that \emph{M works} is false. An even more radical view says that the
support is blocked only if S knows that \emph{M works} is false. A
fairly radical form of liberalism seems intuitively plausible for
memory; we are entitled to trust memories unless we have good reason to
doubt them. It's worth keeping these radical forms of liberalism in mind
when thinking about whether pure radicalism is ever true.

Pryor also notes an interesting way in which arguments can seem to be
circular. He doesn't give this a name, but we'll call it defeater
circularity.\footnote{I'm assuming throughout this chapter that it makes
  sense to talk about defeaters for beliefs. I actually don't want to
  commit to that being true. But the assumption is safe nevertheless.
  Dialectically, the situation is this. I'm trying to respond to the
  best arguments I know of that normative externalism licences a
  problematic form of circular reasoning. If the whole ideology of
  defeaters is misguided, there isn't any danger that a defeater based
  argument will threaten work. But I'm not going to have the defence of
  normative externalism rest on that ideological claim.}

\begin{description}
\tightlist
\item[Defeater Circularity]
An argument exemplifies defeater circularity iff evidence against the
conclusion would (to at least some degree) undermine the justification
the agent has for the premises. This is Pryor's Type 4 dependence; see
Pryor (\citeproc{ref-Pryor2004}{2004, 359}).
\end{description}

It is important that Pryor uses `undermine' here rather than something
more general, like `defeat'. Any valid one premise argument will be such
that evidence against the conclusion will rebut, at least to some
degree, the justification for the premises. But it won't be necessary
that this evidence undermines that justification. If one reasons \emph{X
is in Ann Arbor, so X is in Michigan}, then evidence against the
conclusion will rebut whatever evidence one had that X is in Ann Arbor.
But that might not undermine the support the premise provides to the
conclusion, or that the evidence supplies to the premise. If one thought
X was in Ann Arbor because a friend said that they just saw X, the
counter-evidence need not impugn the friend's reliability in general. It
might just mean the friend got this one wrong.

It is not preposterous to think that arguments which exemplify defeater
circularity are defective in some way. Indeed, it is not preposterous to
think that they are not PTAs. If the falsity of the conclusion would
undermine the premises, then the premises rely, in some intuitive sense,
on the conclusion being true. And that suggests the argument is
circular. And circular arguments are not PTAs. Or at least so we might
intuitively reason.

Pryor argues that some arguments which exemplify defeater circularity
are, in the language being used here, PTAs. He gives two arguments for
this conclusion. First, he offers direct examples of arguments that he
says exemplify defeater circularity, but which could, it seems, be used
to form justified beliefs in their conclusions. As he notes, however,
the intuitive force of these examples is not strong. His second argument
is that defeater circularity arguments suffer from some other vice, such
as a dialectical vice, and we confuse this for their not being sources
of justification.

Most forms of liberalism imply that there will be good arguments that
exemplify defeater circularity. If liberalism about \emph{M} is true,
and S can sometimes observe that she is using \emph{M}, then she should
be able to make the following argument, which we'll call the \emph{M}
argument.

\begin{enumerate}
\def\labelenumi{\arabic{enumi}.}
\tightlist
\item
  \emph{p}.
\item
  \emph{M} says that \emph{p}.
\item
  So \emph{M} got this one right.
\end{enumerate}

By hypothesis, this could be a way that S comes to know that \emph{M} is
working. Since liberalism about \emph{M} is true, she doesn't need to
know that antecedently to using \emph{p} to get the first premise. But
the conclusion is obviously entailed by the premises. So it looks like
it could be learned by learning the premises an doing a little
reasoning. A kind of liberalism that says that whenever S recognises
which method she is using, that method is blocked from providing
support, would not licence this reasoning. But that's a kind of
liberalism that doesn't seem particularly plausible.

But the \emph{M} argument does exemplify defeater circularity, at least
if we're assuming a not-too-radical form of liberalism about \emph{M}.
If S got evidence against the conclusion, that would trigger the clause
saying that evidence that \emph{M} does not work blocks the support that
the agent gets for \emph{p} by using \emph{M}. That is, in the presence
of such evidence, the first premise would not be supported. So we have
the conditions needed for defeater circularity. So if some
not-too-radical form of liberalism is true, then some arguments that
exemplify defeater circularity can generate knowledge, and are in that
sense not viciously circular.

This is all relevant to us because it is plausible that defeater
circularity is the kind of circularity that's at issue in debates over
\textbf{Independence}. Return again to {Chiyoko} and {Aspasia}, and
recall the reasoning {Chiyoko} does.

\begin{enumerate}
\def\labelenumi{\arabic{enumi}.}
\tightlist
\item
  2+2=4, and 3+3=6, and 4+5=9, and 7+9=16.
\item
  {Aspasia} believes that 2+2=5, and 3+3=7, and 4+5=8, and 7+9=15, while
  I believe that 2+2=4, and 3+3=6, and 4+5=9, and 7+9=16.
\item
  So, she got those four sums wrong, and I got them right.
\item
  It is likely that I would get at least one of them wrong if I'd taken
  the drug, and unlikely that she would get all four wrong unless she'd
  taken the drug.
\item
  So, probably, I have not taken the drug, and she has.
\item
  So I should not modify my beliefs about arithmetic in light of what
  {Aspasia} says; she has taken a drug that makes her unreliable.
\end{enumerate}

This violates \textbf{Independence}. {Chiyoko} believes that {Aspasia}
is the unreliable one because she calculated some sums, and realises
that {Aspasia} got them wrong. And she uses this to conclude that
disagreement with {Chiyoko} should not move her.

But where has {Chiyoko} gone wrong? If, as the defender of
\textbf{Independence} insists, she should not have ended up where she
did, where was her first mistake? All parties agree that statements like
premise 2 are usable in debates. And step 3 follows from steps 1 and 2,
presumably in a way that {Chiyoko} can realise. Step 4 is true, and
isn't anything she has any reason to doubt. Step 5 follows from 4 in a
simple way, so {Chiyoko} can sensibly go from 4 to 5. And step 6 follows
from 5 on any plusible theory of disagreement. One shouldn't modify
one's beliefs in light of disagreement with someone who has taken
accuracy-destroying drugs.

So the problem must be with step 1. Now it isn't immediately obvious
what is problematic about step 1. But perhaps we can see the problem if
we think about things in terms of defeater circularity. The argument
from step 1 to step 5 does, plausibly, exemplify defeater circularity.
If {Chiyoko} had reason to believe that step 5 was false, she would have
arguably have a defeater for step 1. So here we have an argument that
the normative externalist thinks is a perfectly good argument, indeed a
PTA, and a kind of circularity that it exemplifies. I suspect this is
probably the best case for the claim that normative externalists are
committed to a dubious kind of circularity.

There is a tricky dialectical point here. The normative externalist need
not themselves agree that {Chiyoko}'s argument exemplifies defeater
circularity. After all, they think that {Chiyoko} can reason well about
arithmetic even if she has misleading evidence that she has been
drugged. But it would be good to not have to rely on this aspect of the
theory in order to defend the theory. So let's just note that the
objection does to some extent rely on a premise the normative
externalist may wish to question, and move on.

My main reply to the objection is that exemplifying defeater circularity
cannot, in general, prevent arguments being PTAs. And that's because
there is a general argument that there must be at least some PTAs that
exemplify defeater circularity. Here's the argument for that conclusion.

\begin{enumerate}
\def\labelenumi{\arabic{enumi}.}
\tightlist
\item
  Liberalism is true about some method of forming beliefs or other,
  though we aren't necessarily in a position to know which method it is.
\item
  If liberalism is true about some method of forming beliefs or other,
  then some PTAs exemplify defeater circularity.
\item
  So, some PTAs exemplify defeater circularity.
\end{enumerate}

I think this argument can be found in Pryor
(\citeproc{ref-Pryor2004}{2004}), though he spends more time on arguing
that particular exemplifications of defeater circularity are PTAs than
directly defending the existential claim.

I've already argued for premise 2, in the discussion of liberalism. And
the argument is valid. So the important thing is to argue for premise 1.
The main argument here is a scepticism-avoidance argument. I'm going to
make an argument very similar to one found in reecnt work by David
Alexander (\citeproc{ref-Alexander2011}{2011}) and Matthias Steup
(\citeproc{ref-Steup2013}{2013}). They both argue, and I agree, that
otherwise plausible anti-circularity principles lead to intolerably
sceptical conclusions. My version of this argument goes via Pryor's
notions of conservatism, liberalism and radicalism.

Call someone an \emph{extremist} if they are anti-liberal about all
methods. One way to be an extremist is to be a global conservative. The
Pyrrhonian sceptics we will meet soon are global conservatives, and
that's why they reach such implausibly sceptical conclusions. But there
are more extremists than that. Someone who thought that for any method,
either radicalism or conservatism is true of that method is an extremist
in my sense.

It actually isn't too hard to motivate extremism. I suspect many
philosophers would find the following argument at least somewhat
plausible.

\begin{enumerate}
\def\labelenumi{\arabic{enumi}.}
\tightlist
\item
  For any method of forming beliefs, either it is a priori knowable that
  it works, or it is not.
\item
  We should be radicals about any method of forming beliefs such that it
  is a priori that the method works.
\item
  We should be conservatives about any method of forming beliefs such
  that it is not a priori that the method works.
\item
  So we should be extremists about all methods.
\end{enumerate}

For what it's worth, I think both premises 2 and 3 are false. But
starting with this connection to the a priori helps bring out the
connection between the argument against extremism and what I've written
elsewhere about Humean scepticism
~(\citeproc{ref-Weatherson2005}{Weatherson 2005},
\citeproc{ref-Weatherson2014-ProbScept}{2014b}). The problem with
extremism is that it implies external world scepticism, and we should
not be external world sceptics.

Why think that extremism implies external world scepticism? One strong
reason is the long-running failure of anyone to come up with a plausible
extremist response to sceptical doubts. To my mind, there is only one
such response that even seems remotely plausible. This is the view that
says we should be radicals about inference to the best explanation and
introspection, plus the premise that the best explanation of our
introspected phenomenology is that the external world exists. This kind
of approach is defended, though not exactly in these terms, by Bertrand
B. Russell (\citeproc{ref-Russell1912}{1912/1997}, ch.~2), {Frank}
Jackson (\citeproc{ref-Jackson1977}{1977}), Jonathan Vogel
(\citeproc{ref-Vogel1990}{1990}), Laurence BonJour
(\citeproc{ref-BonjourSosa}{2003}), and other internalists.

Perhaps you think this kind of view can be made to work; my hopes for
this project are dim. Let's just note one problem, one boldly conceded
by BonJour. Since most humans have not justified their use of
perception, etc by inference to the best explanation, it follows that
most people do not have (doxastically) justified beliefs. That's
implausible on its face, and it's symptomatic of a deeper problem.
Figuring out, or even being sensitive to, the quality of different
explanations of the way the world appears is cognitively downstream from
the kind of simple engagement with the world that we get in perception.
So it is impossible to use inference to the best explanation to justify
our belief that perception is reliable, at least if conservatism about
perception is correct, because we need perception to make plausible
judgments about the quality of explanations.

If that's all correct, then liberalism must be true about some methods.
And that implies exemplifying defeater circularity cannot always be a
bad-making feature of arguments. So the fact that normative externalists
are committed to the goodness of arguments that exemplify defeater
circularity cannot be, on its own, an argument against normative
externalism.

And there is even more that the normative externalist can say. Assume
I'm wrong in the last few paragraphs, and actually extremism is correct.
Then we have a further question to ask: Is global conservatism correct
or not? If not, some kinds of radicalism are correct. And if some kinds
of radicalism are correct, then a strong form of normative externalism
is true, at least with respect to beliefs formed by some methods. That's
because radicalism implies that certain belief-forming methods are
immune to all kinds of defeat, including belief that they don't work, or
evidence that they don't work, or even knowledge that they don't work.
That's a very strong form of normative externalism! Now it's true that
what we get here isn't normative externalism in general, because all we
get here is that for some belief-forming methods, higher order evidence
is irrelevant. That's consistent with higher order evidence mattering
sometimes, in a way that normative externalists deny. But if the
position I'm imagining here - that higher order evidence is relevant to
beliefs formed by certain methods - is correct, then general objections
to normative externalism, ones that are insensitive to the methods by
which people form beliefs, must be wrong.

On the other hand, if radicalism is never true, and extremism is true,
then global conservatism is true. And global conservatism is a very
implausible doctrine. To see how implausible, it's worth working through
some varieties of sceptical argument.

\section{Pyrrhonian Scepticism and Normative
Externalism}\label{pyrrhonianscepticismandnormativeexternalism}

In the previous section I argued that the principle that no PTA
exemplifies defeater circularity leads to external world scepticm. But
perhaps that was understating the case. Perhaps it really leads to
Pyrrhonian scepticism, and Pyrrhhonian scepticism is a kind of reason
scepticism. (The next few paragraphs draw on a discussion of scepticism
by Peter Klein (\citeproc{ref-KleinSEP}{2015}).)

Pyrrhonian scepticism starts with reflection on the problem of the
criterion. Any knowledge we get must be via some method or other. But,
says the Pyrrhonian sceptic, we can't use a method to gain knowledge
unless we antecedently know that it is a knowledge-producing method. And
plausibly that implies knowing it is reliable, since methods that are
unreliable do not produce knowledge. So the Pyrrhonian is a global
conservative, in the terminology of the previous section. Now knowing
that a method is reliable is a piece of knowledge. So to know anything,
there is something we need to know before we can know anything. That's
impossible, so we know nothing.

The problem of the criterion is potentially a very strong argument.
After all, the conclusion of the last paragraph was not that we know
nothing about the unobservable, or about the external world, or even
about contingent matters. It is that we know nothing at all. That even
extends to philosophical knowledge. So the problem of the criterion is
naturally an argument for Pyrrhonian scepticism, the view that we cannot
know anything, even the truth of philosophical claims like Pyrrhonian
scepticism.

For much the same reason, the view looks so strong as to be
self-defeating. You might think that by the lights of the Pyrrhonian
sceptic, we can't even assert Pyrrhonian scepticism, since we can't know
it to be true. That's too quick, since it assumes as a premise that
\emph{Only assert what you know} is a valid rule, and that's both false
in academic contexts, and easily denied by the Pyrrhonian. But still, a
view that says we can't know that we exist, we can't know that we are
thinking, we can't know that ¬(0 = 1), and so on is just absurd.

And worse still, it is an argument for an absurd conclusion with really
only one key premise, namely global conservatism. Sometimes arguments
can have absurd conclusions, but at least they present us with a
challenge to identify where things have gone wrong. Not here! The
mistake is obviously the global conservatism, since that's the only
premise there is. I'm assuming here that we are reading `method' so
weakly that it is uncontroversial that any knowledge is gained by some
method or other.

And so most epistemologists do indeed reject that premise. Reliabilists
say that any reliable method can produce knowledge, whatever the user of
that method knows about the method's reliability. Other philosophers
might say that we can use induction in advance of knowing that induction
is reliable, and hence in advance of knowing it is knowledge-producing.
Or perhaps we can, as Descartes suggests, use clear and distinct
perception before we know it is reliable. One way or the other, the
overwhelming majority of epistemologists reject global conservatism
somewhere.\footnote{The regress argument I've given here requires that
  the conservative view be stated a little carefully. It matters that
  the conservative says that \emph{M} only provides justification if the
  subject antecedently believes, with justification, that \emph{M
  works}. A view that says that \emph{M} provides justification as long
  as \emph{M works} was antecedently justifiably believable is not
  conservative as I'm carving up the space of views.}

If global conservatism is false, then either liberalism is true
somewhere, or radicalism is true somewhere. And we have already seen
that either of these conclusions would be very bad news for circularity
based objections to normative externalism. They certainly suggest that
the argument from defeater circularity against normative externalism
fails. If liberalism is true somewhere, then some PTAs exemplify
defeater circularity, contra premise 2 of the argument. And if
radicalism is true somewhere, then it is possible to be a normative
externalist without committing to the view that the problematic
arguments exemplify defeater circularity, contra premise 1 of the
argument.

\section{Easy Knowledge}\label{easyknowledge}

The normative externalist looks like they will be subject to what
Stewart (\citeproc{ref-Cohen2002}{Cohen 2002},
\citeproc{ref-Cohen2005}{2005}) calls ``The Problem of Easy Knowledge''.
This might be a better way to cash out the intuition that normative
externalism leads to problematic kinds of circular reasoning.

The problem of easy knowledge arises for any theory that says an agent
can use a method to gain knowledge without knowing that it is
knowledge-producing. Say \emph{M} is one such method, and S one such
agent. And assume, at least for now, that S can identify how and when
she is using \emph{M}. That is, when she forms a belief that \emph{p}
using \emph{M}, she at least often knows that she is doing so. Say that
she forms beliefs \emph{p}\textsubscript{1}, \ldots,
\emph{p\textsubscript{n}} this way, and each of these beliefs amount to
knowledge. Then she can reason as follows.

\begin{enumerate}
\def\labelenumi{\arabic{enumi}.}
\tightlist
\item
  \emph{p}\textsubscript{1} ∧ \ldots{} ∧ \emph{p\textsubscript{n}}
\item
  \emph{M} said that \emph{p}\textsubscript{1} ∧ \ldots{} ∧
  \emph{p\textsubscript{n}}
\item
  So, \emph{M} is fairly reliable.
\end{enumerate}

What could be wrong with this argument? We've assumed that the agent
knows premise 1 and premise 2, so as long as she can use whatever she
knows in an argument, she is in a position to run the argument. The
argument is not deductive, but it seems like a decent inductive
argument. Perhaps it could fail if there were external defeaters, but we
can assume there are no such defeaters in S's situation. And if the
sample size strikes you as too small for the inductive inference, we can
increase the size of \emph{n}.

So given some weak assumptions, it looks like S can use this argument to
gather inductive support for the claim that \emph{M} is fairly reliable.
That is to say, she can use \emph{M} itself to gather inductive support
for the claim that \emph{M} is fairly reliable. And that has struck many
philosophers as absurd. This is, in essence, is the Problem of Easy
Knowledge. Here are a few quotes from Cohen setting out what he takes
the Problem to be. (The `evidentialist foundationalist' in these quotes
is the theorist who thinks that an agent can gain knowledge by drawing
appropriate conclusions from evidence in advance of knowing that
evidence reliably correlates with the appropriate conclusion. This is a
form of normative externalism, and it's at least arguable that if
Cohen's arguments work against the evidentialist foundationalist, they
will generalise to all forms of normative externalism.)

\begin{quote}
For example, if I know the table is red on the basis of its looking red,
then it follows by the closure principle that I can know that it's not
the case that the table is white but illuminated by red lights.
Presumably, I cannot know that it's not the case that the table is white
but illuminated by red lights, on the basis of the table's looking red.
So the evidentialist foundationalist will have to treat this case
analogously to the global deception case: I can know the table is red on
the basis of its looking red, and once I know the table is red, I can
infer and come to know that it is not white but illuminated by red
lights. But, it seems very implausible to say I could in this way come
to know that I'm not seeing a white table illuminated by red lights.
~(\citeproc{ref-Cohen2002}{Cohen 2002, 313})
\end{quote}

\begin{quote}
It's counterintuitive to say we could in this way know the falsity of
even the \emph{alternative} that the table is white but illuminated by
red lights. Suppose my son wants to buy a red table for his room. We go
in the store and I say, ``That table is red. I'll buy it for you.''
Having inherited his father's obsessive personality, he worries,
``Daddy, what if it's white with red lights shining on it?'' I reply,
``Don't worry--you see, it looks red, so it is red, so it's not white
but illuminated by red lights.'' Surely he should not be satisfied with
this response. Moreover I don't think it would help to add, ``Now I'm
not claiming that there are no red lights shining on the table, all I'm
claiming is that the table is not white with red lights shining on it''.
But if evidentialist foundationalism is correct, there is no basis for
criticizing the reasoning.~(\citeproc{ref-Cohen2002}{Cohen 2002, 314})
\end{quote}

\begin{quote}
Imagine again my 7 year old son asking me if my color-vision is
reliable. I say, ``Let's check it out.'' I set up a slide show in which
the screen will change colors every few seconds. I observe, ``The screen
is red and I believe it's red. Got it right that time. Now it's blue
and, look at that, I believe its blue. Two for two\ldots{}'' I trust
that no one thinks that whereas I previously did not have any evidence
for the reliability of my color vision, I am now actually acquiring
evidence for the reliability of my color vision. But if Reliabilism were
true, that's exactly what my situation would be. We can call this the
problem of ``easy evidence''. ~(\citeproc{ref-Cohen2002}{Cohen 2002,
317})
\end{quote}

Cohen thinks that the lessons to draw from these cases is that we must
distinguish between \textbf{KR} and \textbf{PKR}.

\begin{description}
\tightlist
\item[KR]
A potential knowledge source \emph{K} can yield knowledge for S, only if
S knows that \emph{K} is reliable.
\item[PKR]
A potential knowledge source \emph{K} can yield knowledge for S, only if
S has prior knowledge that \emph{K} is reliable.
\end{description}

\textbf{PKR} is the problematic global conservatism. It leads to
implausibly sceptical results. But, thinks Cohen, this is no argument
against \textbf{KR}. Nothing in the discussion so far shows that there
is anything absurd with a sweeping form of coherentism that says that S
can to know simultaneously, and for the same reasons, all of the
following propositions.

\begin{enumerate}
\def\labelenumi{\arabic{enumi}.}
\tightlist
\item
  ¬(0=1).
\item
  I used a knowledge generating method to form the belief that ¬(0=1).
\item
  I used a knowledge generating method to form the belief that I used a
  knowledge generating method to form the belief that ¬(0=1).
\item
  I used a knowledge generating method to form the belief that I used a
  knowledge generating method to form the belief that I used a knowledge
  generating method to form the belief that ¬(0=1).
\end{enumerate}

And so on. Cohen's opponents are the anti-coherentists who think it is
possible to know ¬(0=1) prior to having this infinite chain of
knowledge. Such anti-coherentists can, and do, disagree substantially
about what exactly is required for one to know ¬(0=1). Let's start by
considering just one opponent, a reliabilist who says that a method can
produce basic knowledge if the following two conditions are met:

\begin{itemize}
\tightlist
\item
  The method is in fact reliable; and
\item
  The agent has no reason to doubt that the method is reliable.
\end{itemize}

This is a somewhat simplified version of the reliabilism defended by
Alvin Goldman (\citeproc{ref-Goldman1986}{1986, 111--12}), and similar
in form (though not in its externalist commitments) to Pryor's dogmatism
~(\citeproc{ref-Pryor2000}{Pryor 2000}). And it is very much the kind of
view that Cohen takes his arguments to be targeted against. He makes
three observations about this kind of theory.

First, the theory allows for a fairly simple response to doubts grounded
in sceptical possibilities. If something appears to be a red table, and
so we come to know that it is is a table, we can simply deduce that we
are not in a tableless room but deceived by an evil demon to think there
is a table. This looks too quick, but as Cohen concedes, any response to
scepticism will have some odd feature.

Second, the theory allows for a fairly simple response to more everyday
doubts. This is the core of Cohen's objection to basic knowledge views.
For instance, he notes that the kind of foundationalism that he
considers would allow an agent to easily infer that they are not looking
at a white table illuminated by red lights simply on the basis of the
appearance of a red table. And this he thinks is absurd. This is the
upshot of the first of the imagined conversations with his (then) 7
year-old son.

Third, the theory seems to allow for a fairly simple generation of
grounds for an absurd inductive argument. Assume that the agent is
living in a world where appearances do in fact reliably correlate with
facts about the external world. So whenever something appears φ, the
agent can know that it is φ, for any φ. So she can easily test the
accuracy of her appearances just by looking. And the test will be passed
every time, with flying colours! So she will have grounds for an
inductive argument that appearances are an accurate guide to reality.
This is the conclusion of the argument containing the second imagined
conversion.

For now, let's assume that the intuitions about these cases are correct,
and start with a question about the cases' significance. After bringing
up intuitions about these few cases, Cohen makes some rather sweeping
generalisations about the impossibility of a plausible theory of basic
knowledge. And that generalisation isn't supported by these cases.

Adding a defeasibility clause to foundationalism already avoids the
worst of the problems. Cohen carefully distinguishes between inferences
from everyday propositions to the falsity of outlandish sceptical
claims, and inferences from everyday propositions (like \emph{That's a
red table}) to the falsity of other everyday-ish propositions (like
\emph{That's not a white table illuminated by red lights}). His reason
for doing this is that it is the latter inferences that are especially
implausible, since the necessity and difficulty of responding to the
sceptic makes some otherwise counter-intuitive moves plausible. But once
the defeasibility clause is in place, it isn't clear that the everyday
cases are really problems. After all, if white tables illuminated by red
lights are everyday occurrences, then the defeasibility clause will be
triggered. And if they are not, we are back in the realm of sceptical
doubts.

In other words, once the basic knowledge theorist adds a defeasibility
clause, I don't think Cohen can avoid considering the kind of sceptical
scenarios that he grants intuitions are unreliable about. It might be
that the only things we can know by basic means are relatively simple
anti-sceptical propositions, since we have reason to doubt everything
else. Put another way, it's arguable that the unintuitiveness of Cohen's
example is due to the fact that we have reason to doubt that the
lighting is normal in a lot of examples. So my preferred foundationalist
externalist will think it is not a case of basic knowledge. And anything
they do think is basic knowledge won't be subject to these doubts.

To make this point more dramatically, consider the theorist (such as
perhaps Descartes) who thinks that introspection is a form of basic
knowledge. It is not unintuitive that we can see, by introspection, that
introspection is reliable. We can introspect that \emph{p} and
introspect that we are introspecting that \emph{p}, and so deduce that
introspection worked on that occasion. At the very least, this isn't
obviously wrong. For example, we mostly take our pain appearances to be
reliable indicators of actually being in pain. They may or may not be
reliable indicators of bodily damage, but they are reliable indicators
of being in pain. We have no non-introspective evidence about this
reliability. So we must, at some level, assume that introspection is
good evidence that introspection is reliable.

Let's take stock. The big question is whether the Problem of Easy
Knowledge helps us isolate a class of circular reasoning that is not
acceptable. Cohen has demonstrated that several epistemological theories
are committed to some reasoning that looks circular, like the reasoning
involved in the imaginary conversations with his 7 year old son. Cohen
himself takes those to be arguments against these epistemological
theories, and by extension against a lot of circular reasoning. But it
isn't clear that Cohen's arguments generalise as far as he intends;
their intuitive force may turn on some special features about colour
perception. So let's look more closely at the intuitions behind Cohen's
examples.

\section{What's Wrong with Easy
Knowledge?}\label{whatswrongwitheasyknowledge}

It's hard to put one's finger on just what is supposed to be wrong with
easy knowledge. Cohen usually just relies on the intuitive
implausibility of the methods he is discussing being knowledge
producing. But it is hard to generalise from particular cases since
intuitions about any given case might be based on particular features of
that case. An explanation of the intuition would avoid that problem. So
I'll go over a bunch of possible explanations of the intuitions Cohen is
relying on, with the hope that once we know why these intuitions are
true (when they are), we'll know how far they generalise.

Note that one simple explanation of intuitions in the cases Cohen gives
is simply that radicalism, or even radical leaning liberalism, is wrong
about colour perception. That would tell us something interesting about
the epistemology of colour, but not something more general about
knowledge and circular arguments. And it wouldn't be any kind of problem
for the normative externalist, since the normative externalist as such
has no commitments at all about the epistemology of colour perception.

The worry is that there is something more general behind Cohen's cases,
something that will be general enough to raise a problem for normative
externalism. I deny there is, but I don't think there is any way to back
up this denial except to work through all the principles we might think
are supported by Cohen's cases. So that's the game plan for this
section. I'll set things up as a dialogue between an objector, who uses
reasoning inspired by Cohen's cases to put forward views that are
inconsistent with Change Evidentialism, and my responses to the
objector. I'll generally leave off the arguments that the objector's
positions are actually in conflict with Change Evidentialism, but mostly
they are. There is one exception, where I make a fuss about this in the
reply. The objector assumes that we are radicals, or at least radical
leaning liberals, about perception in general. We could resist that,
while holding on to Change Evidentialism, but I'd rather acquiesce in
this assumption.

\subsection{Sensitivity}\label{sensitivity}

\begin{quote}
\emph{Objection}:\\
If you use perception to test perception, then you'll come to believe
perception is accurate whether it is or not. So if it weren't accurate,
you would still believe it is. So your belief that it is accurate will
be insensitive, in the sense of Nozick
(\citeproc{ref-Nozick1981}{1981}). And insensitive beliefs cannot
constitute knowledge.
\end{quote}

The obvious reply to this is that the last sentence is false. As has
been argued at great length, e.g.~in Williamson
(\citeproc{ref-Williamson2000}{2000}, ch.~7), sensitivity is not a
constraint on knowledge. We can even see this by considering other cases
of testing.

Assume a scientist is trying to figure out whether Acme machines are
accurate at testing concrete density. She has ten Acme machines in her
lab, and proceeds to test each of them in turn by the standard methods.
That is, she gets various samples of concrete of known density, and gets
the machine being tested to report on its density. For each of the first
nine machines, she finds that it is surprisingly accurate, getting the
correct answer under a very wide variety of testing conditions. She
concludes that Acme is very good at making machines to measure concrete
density, and that hence the tenth machine is accurate as well.

We'll return briefly to the question of whether this is a good way to
test the tenth machine below. It seems that the scientist has good
inductive grounds for knowing that the tenth machine is accurate. Yet
the nearest world in which it is not accurate is one in which there were
some slipups made in its manufacture, and so it is not accurate even
though Acme is generally a good manufacturer. In that world, she'll
still believe the tenth machine is accurate. So her belief in its
accuracy is insensitive, although she knows it is accurate. So whatever
is wrong with testing a machine (or a person) against their own outputs,
if the problem is just that the resulting beliefs are insensitive, then
that problem does not preclude knowing those outputs are accurate.

\subsection{One-Sidedness}\label{one-sidedness}

\begin{quote}
\emph{Objection}:\\
If you use perception to test perception, then you can only come to one
conclusion; namely that perception is accurate. Indeed, the test can't
even give you any reason to believe that perception is inaccurate. But
any test that can only come to one conclusion, and cannot give you a
reason to believe the negation of that conclusion, cannot produce
knowledge.
\end{quote}

Again, the problem here is that the last step of the reasoning is
mistaken. There are plenty of tests that can only produce knowledge in
one direction only. Here are four such examples.

First example. The agent is an intuitionist, so she does not believe
that instances of excluded middle are always true. She does, however,
know that they can never be false. She is unsure whether \emph{Fa} is
decidable, so she does not believe \emph{Fa} ∨ ¬\emph{Fa}. She observes
\emph{a} closely, and observes it is \emph{F}. So she infers \emph{Fa} ∨
¬\emph{Fa}. Her test could not have given her a reason to believe
¬(\emph{Fa} ∨ ¬\emph{Fa}), but it does ground knowledge that \emph{Fa} ∨
¬\emph{Fa}.

Second example. The agent is trying to figure out which sentences are
theorems of a particular modal logic she is investigating. She knows
that the logic is not decidable, but she also knows that a particular
proof-evaluator does not validate invalid proofs. She sets the evaluator
to test whether random strings of characters are proofs. After running
overnight, the proof-evaluator says that there is a proof of some
particular sentence \emph{S}\textsubscript{0} in the logic. The agent
comes to know that \emph{S}\textsubscript{0} is a theorem of the logic,
even though the failure of the proof-evaluatory to output that
\emph{S}\textsubscript{0} has a proof would not have given her any
reason to believe it is not a theorem.

Third example. {Ada} has a large box of Turing machines. She knows that
each of the machines in the box has a name, and that its name is an
English word. She also knows that when any machine halts, it says its
name, and that it says nothing otherwise. She does not know, however,
which machines are in the box, or how many machines are in the box. She
listens for a while, and hears the words `Scarlatina', `Aforetime' and
`Overinhibit' come out of the box. She comes to believe, indeed know,
that Scarlatina, Aforetime and Overinhibit are Turing machines that
halt. Had those machines not halted, she would not have been in the
right kind of causal contact with those machines to have singular
thoughts about them, so she could not have believed that they are not
halting machines. So listening for what words come out of the box is
one-sided in the sense described above; for many propositions, it can
deliver knowledge that \emph{p}, but could not deliver knowledge that
¬\emph{p}.

Fourth example. {Kylie} is a Red Sox fan in Australia in the
pre-internet era. Her only access to game scores are from one-line score
reports in the daily newspaper. She doesn't know how often the Red Sox
play. She notices that some days there are 2 games reported, some days
there is 1 game reported, and on many days there are no games reported.
She also knows that the paper's editor is also a Red Sox fan, and only
prints the score when the Red Sox win. When she opens the newspaper and
sees a report of a Red Sox win (i.e.~a line score like ``Red Sox 7,
Royals 3'') she comes to believe that the Red Sox won that game. But
when she doesn't see a score, she has little reason to believe that the
Red Sox lost any particular game. After all, she has little reason to
believe that any particular game even exists, or was played, let alone
that it was lost. So the newspaper gives her reasons to believe that the
Red Sox win games, but never reason to believe that the Red Sox didn't
win a particular game.

So we have four counterexamples to the principle that you can only know
\emph{p} if you use a test that could give you evidence that ¬\emph{p}.
The reader might notice that many of the examples involve cases from
logic, or cases involving singular propositions. Both of those kinds of
cases are difficult to model using orthodox Bayesian machinery. That's
not a coincidence. There's a well known Bayesian argument in favour of
the principle I'm objecting to, namely that getting evidence for
\emph{p} presupposes the possibility of getting evidence for ¬\emph{p}.
The argument turns on the fact that this is a valid argument, for any
values of \emph{E}, \emph{H}, \emph{x} you like.

\begin{enumerate}
\def\labelenumi{\arabic{enumi}.}
\tightlist
\item
  Pr(\emph{H}) \textless{} \emph{x}
\item
  Pr(\emph{E}) \textgreater{} 0
\item
  Pr(\emph{H} \textbar{} \emph{E}) ⩾ \emph{x}
\item
  So, Pr(\emph{H} \textbar{} ¬\emph{E}) \textless{} Pr(\emph{H})
\end{enumerate}

Intuitively, we might read this as saying that if \emph{E} raises the
probability of \emph{H} above any threshold \emph{x}, then ¬\emph{E}
would be evidence against \emph{H}. I haven't discussed that objection
here, because it's irrelevant. When dealing with foundational matters,
like logical inference, Bayesian modelling is inappropriate. We can see
that by noting that in any field where Bayesian modelling is
appropriate, the objection currently being considered works. What's not
so clear, in fact what is most likely false, is that we can model the
above four examples in a Bayesian framework. Bayesianism just isn't that
good at modelling logical uncertainty, or changes in which singular
propositions are accessible to the agent. But that's what matters to
these examples.

\subsection{Generality}\label{generality}

\begin{quote}
\emph{Objection}:\\
Assume we can use perception to come to know on a particular occasion
that perception is reliable. Since we can do this in arbitrary
situations where perception is working, anyone whose perception is
working can come to know, by induction on a number of successful cases,
that their perception is generally reliable. And this is absurd.
\end{quote}

I'm not sure that this really is absurd, but the cases already discussed
should make it clear that it isn't a consequence of Change
Evidentialism. It is easily possible to routinely get knowledge that a
particular \emph{F} is \emph{G}, never get knowledge that any \emph{F}
is not \emph{G}, and no way be in a position to infer, or even regard as
probable, that all \emph{F}s are \emph{G}s.

For instance, if we let \emph{F} be \emph{is a Turing machine in the box
Ada is holding}, and \emph{G} be \emph{halts}, then for any particular
\emph{F} {Ada} comes to know about, it is \emph{G}. But it would be
absurd for her to infer that every \emph{F} is a \emph{G}. Similarly,
for any Red Sox game that {Kylie} comes to know about, the Red Sox win.
But it would be absurd for her to come to believe on that basis that
they win every game.

There's a general point here, namely that whenever we can only come to
know about an \emph{F} only if it is a \emph{G}, then we are never in a
position to infer inductively that every \emph{F} is \emph{G}, or even
that most of them are. Since even the foundationalist externalist
doesn't think we can come to know by perception that perception is not
working on an occasion, this means we can never know, by simple
induction on perceptual knowledge, that perception is generally
reliable.

\subsection{A Priority}\label{apriority}

\begin{quote}
\emph{Objection}:\\
Assume it is possible to come to know that perception is reliable by
using perception. Then before we even perceive anything, we can see in
advance that this method will work. So we can see in advance that
perception is reliable. That means we don't \emph{come} to know that
perception is reliable using perception, we could have known it all
along. In other words, it is a priori knowable that perception is
reliable. (This objection is related to an argument by Roger White
(\citeproc{ref-White2006}{2006}), though note his argument is directed
against a slightly different target.)
\end{quote}

This objection misstates the consequences of the view that perception
provides evidence when it works. If perception is working, then we get
evidence for this every time we perceive something, and reflect on what
we perceive. But if perception is not working well, we don't get any
such evidence. The point is not merely that if perception is unreliable,
then we can't possibly know that perception is unreliable since
knowledge is factive. Rather, the point is that if perception is
unreliable, then using perception doesn't give us any evidence at all
about anything at all. So it doesn't give us evidence that perception is
reliable. Since we don't know antecedently whether perception is
reliable, we don't know if we'll get any evidence about its reliability
prior to using perception, so we can't do the kind of a priori reasoning
imagined by the objector.

This response relies heavily on an externalist treatment of evidence. A
first order internalist is perhaps vulnerable to this kind of objection.
As I've argued elsewhere ~(\citeproc{ref-Weatherson2005}{Weatherson
2005}), first-order internalists have strong reasons to think we can
know a priori that foundational methods are reliable. Some may think
that this is a reductio of this first-order internalism. (I don't.) But
the argument crucially relies on first-order internalism, not just on
foundationalism.

\subsection{Testing}\label{testing}

\begin{quote}
\emph{Objection}:\\
It's bad to test a belief forming method using that very method. The
only way to learn that a method is working is to properly test it. So we
can't learn that perception is reliable using perception.
\end{quote}

This objection is, to me, the most interesting of the lot. It is
interesting because the first premise, i.e.~the first sentence in it, is
true. Testing perception using perception is bad. What's surprising is
that the second premise is false. The short version of my reply is that
in testing, we aim for more than knowledge. In particular, we aim for
sensitive knowledge. A test can be bad because it doesn't deliver
sensitive knowledge. And that implies that a bad test can deliver
knowledge, at least assuming that not all knowledge is sensitive
knowledge. Defending these claims is the point of the next section.

\subsection{Circularity}\label{circularity}

\begin{quote}
\emph{Objection}:\\
Even if we haven't put our finger yet exactly on the problem, the
reasoning involved in getting easy knowledge is in some way circular,
and we should be suspicious of it.
\end{quote}

By this stage of the chapter, it should be clear what's wrong with this
objection. The hope was that we would find some way of making the
anti-circularity intuition more precise by investigating easy knowledge.
But all we've ended up with is the view that easy knowledge is bad
because it is in some vague sense circular. If this is the intuition
behind the Problem of Easy Knowledge, we're back in the territory of the
`whiff of circularity' objection.

\subsection{Multiple Properties}\label{multipleproperties}

\begin{quote}
\emph{Objection}:\\
Let's say we grant that each of the six properties you mentioned so far
is individually compatible with knowledge. That doesn't show that every
combination of them is compatible with knowledge. In general, ◇\emph{p}
and ◇\emph{q} don't entail ◇(\emph{p}~∧~\emph{q}). So you haven't shown
easy knowledge is possible.
\end{quote}

I don't quite know what to think about this objection. It strikes me as
completely wrong-headed. The `no easy knowledge' intuition seems, to me
at least, to rest on an overlapping set of plausible but ultimately
mistaken judgments about the relationship between knowledge, evidence
and rationality/justifiability. I've argued that any possible reason one
could have to support the intuition that easy knowledge is not knowledge
is false, or not strong enough to support that conclusion. Could it be
that the reasons work collectively when they don't work singularly? It's
logically possible, but I don't see any reason at all to suspect it is
true.

In short, there isn't any one reason to believe that the intuitions
behind the most general form Problem of Easy Knowledge are correct. It
could be that no one of them is correct, yet the intuitions are right
because of some combination, or because of some extra factor. But at
this stage, the best thing to do is to treat the intuitions as suspect.
That means they can't form the basis for any objection to normative
externalism, or any other theory.

\section{Coda: Testing}\label{coda:testing}

In response to the `testing' argument for the intuition that easy
knowledge is no knowledge at all, I suggested that we should distinguish
between a test being in general good and a test being the kind of thing
which can ground knowledge. I think that's true because tests also aim
at sensitive belief. A test can fail in this aim, but still produce
knowledge, because sensitivity isn't necessary for knowledge. Here's a
simplified version of a real-life situation that makes that position
somewhat intuitive.

\begin{quote}
\textbf{Inspection}\\
In a certain state, the inspection of scales used by food vendors has
two components. Every two years, the scales are inspected by an official
and a certificate of accuracy issued. On top of that, there are random
inspections, where each day an inspector must inspect a vendor whose
biennial inspection is not yet due. Today one inspector, call her {Ins},
has to inspect a store run by a shopkeeper called {Sho}. It turns out
{Sho}'s store was inspected just last week, and passed with flying
colours. Since {Sho} has a good reputation as an honest shopkeeper,
{Ins} knows that his scales will be working correctly.
\end{quote}

{Ins} turns up and before she does her inspection watches several people
ordering caviar, which in {Sho}'s shop goes for \$1000 per kilogram. The
first customer's purchase gets weighed, and it comes to 242g, so she
hands over \$242. The second customer's purchase gets weighed, and it
comes to 317g, so she hands over \$317. And this goes on for a while.
Then {Ins} announces that she's there for the inspection. {Sho} is happy
to let her inspect his scales, but one of the customers, call him {Cus},
wonders why it is necessary. ``Look,'' he says, ``you saw that the
machine said my purchase weighed 78g, and we know it did weigh 78g since
we know it's a good machine.'' At this point the customer points to the
certificate authorising the machine that was issued just last week.
``And that's been going on for a while. Now all you're going to do is
put some weights on the scale and see that it gets the correct reading.
But we've done that several times. So your work here is done.''

There is something deeply wrong with {Cus}'s conclusion, but it is
surprisingly hard to see just where the argument fails. Let's lay out
his argument a little more carefully.

\begin{enumerate}
\def\labelenumi{\arabic{enumi}.}
\tightlist
\item
  The machine said my caviar weighed 78g, and we know this, since we
  could all see the display.
\item
  My caviar did weigh 78g, and we know this, since we all know the
  machine is working correctly.
\item
  So we know that the machine weighed my caviar correctly. (From 1, 2)
\item
  By similar reasoning we can show that the machine has weighed
  everyone's caviar correctly. (Generalising 3)
\item
  All we do in testing a machine is see that it weighs various weights
  correctly.
\item
  So just by watching the machine all morning we get just as much
  knowledge as we get from a test. (From 4, 5)
\item
  So there's no point in running {Ins}'s tests. (From 6)
\end{enumerate}

{Cus}'s summary of how testing scales works is obviously a bit crude but
we can imagine that the spot test {Ins} plans to do isn't actually any
more demanding than what the scale has been put through while she's been
standing there. So we'll let premise 5 pass. (If you'd prefer more
realism in the testing methodology, at the cost of less realism in the
purchasing pattern of customers, imagine that the purchases exactly
follow the pattern of weights that a calibrator following the guidelines
of the officially approved methods of calibration.) If 3 is true, it
does seem 4 follows, since {Cus} can simply repeat his reasoning to get
the relevant conclusions. And if 4 and 5 are true, then it does seem 6
follows. To finish up our survey of the uncontroversial steps in {Cus}'s
argument, it seems there isn't any serious dispute about step 1.

So the contentious steps are:

\begin{itemize}
\tightlist
\item
  Step 2 - we may deny that everyone gets knowledge of the caviar's
  weight from the machine.
\item
  Step 3 - we may deny that the relevant closure principle that {Cus} is
  assuming here.
\item
  Step 7 - we may deny that the aim of the test is (merely) to know that
  the machine is working.
\end{itemize}

One way to deny step 2 is to just be an inductive sceptic, and say that
no one can know that the machine is working merely given that it worked,
or at least appeared to work, last week. But that doesn't seem very
promising. It seems that the customers do know, given that the testing
regime is a good one, and that the machine was properly tested, that the
machine is working. And the inspector has all of the evidence available
to the customers, and is in an even better position to know that the
testing regime is good, so as step 2 says, she gets knowledge of the
caviar's weight from the machine.

In recent years there has been a flood of work by philosophers denying
that what we know is closed under either single-premise closure, e.g.,
Dretske (\citeproc{ref-Dretske2005}{2005}), or multi-premise closure,
e.g., Christensen (\citeproc{ref-Christensen2005}{2005}). But it is hard
to see how that kind of anti-closure view could help here. We aren't
inferring some kind of heavyweight proposition like that there is an
external world. And Dretske's kind of view is motivated by avoidance of
that kind of inference. And Christensen's view is that knowledge of a
conjunction might fail when the amount of risk involved in each conjunct
is barely enough to sustain knowledge. But we can imagine that our
knowledge of both 1 and 2 is far from the borderline.

A more plausible position is that the argument from 1 and 2 to 3 is not
a PTA. But that just means that {Ins}, or {Cus}, can't get an initial
warrant, or extra warrant, for believing the machine is working by going
through this reasoning. And {Cus} doesn't claim that you can. His
argument turns entirely on the thought that we already know that the
machine is reliable. Given that background, the inference to 3 seems
pretty uncontroversial.

That leaves step 7 as the only weak link. I want to conclude that
{Cus}'s inference here fails; even if {Ins} knows that the machine is
working, it is still good for her to test it. But I imagine many people
will think that if we've got this far, i.e., if we've agreed with
{Cus}'s argument up to step 6, then we must also agree with step 7. I'm
going to offer two arguments against that, and claim that step 7 might
fail, indeed does fail in the story I've told, even if what {Cus} says
is true up through step 6.

First, even if {Ins} won't get extra knowledge through running the tests
on this occasion, it is still true that this kind of randomised testing
program is an epistemic good. We have more knowledge through having
randomised checks of machines than we would get from just having
biennial tests. So there is still a benefit to conducting the tests even
in cases where the outcome is not in serious doubt. The benefit is
simply that the program, which is a good program, is not
compromised.\footnote{The arguments of the next few paragraphs are
  obviously close to the arguments in Hawthorne and Srinivasan
  (\citeproc{ref-HawthorneSrinivasan2013}{2013}).}

We can compare this reason {Ins} has for running the tests to reasons we
have for persisting in practices that will, in general, maximise
welfare. Imagine a driver, called {Dri}, is stopped at a red light in a
quiet part of town in the middle of the night. {Dri} can see that there
is no other traffic around, and that there are no police or cameras who
will fine her for running the red light. But it is wise to stay stopped
at the light. The practice of always stopping at red lights is a better
practice than any alternative practice that {Dri} could implement. I
assume she, like most drivers, could not successfully implement the
practice \emph{Stay stopped at red lights unless you know no harm will
come from running the light}. In reality, a driver who tries to
occasionally slip through red lights will get careless, and one day run
a serious risk of injury to themselves or others. The best practice is
simply to stay stopped. So on this particular occasion {Dri} has a good
reason to stay stopped at the red light: that's the only way to carry
out a practice which it is good for her to continue.

Now {Ins}'s interest is not primarily in welfare, it is in epistemic
goods. She cares about those epistemic goods because they are related to
welfare, but her primary interest is in epistemic goods. But we can make
the same kind of point. There are epistemic practices which are optimal
for us to follow given what we can plausibly do. And this kind of
testing regime may be the best way to maximise our epistemic access to
facts about scale reliability, even if on this occasion it doesn't lead
to more knowledge. Indeed, it seems to me that this is quite a good
testing regime, and it is a good thing, an epistemically good thing, for
{Ins} to do her part in maintaining the practice of randomised testing
that is part of the regime.

The second reason is more important. The aims of the test are, I claim,
not exhausted by the aim of getting knowledge that the machine is
working. We also want a sensitive belief that the machine is working.
Indeed, we may want a sensitive belief that the machine has not stopped
working since its last inspection. That would be an epistemic good. Our
epistemic standing improves if our belief that the machine has not
stopped working since its last inspection becomes sensitive to the
facts. Before {Ins} runs the test, we know that the machine will work.
If we didn't know that, we shouldn't be engaged in high-stakes
transactions (like the caviar sales) that rely on the accuracy of the
machine. But our belief that the machine will work is not sensitive to
one not completely outlandish possibility, namely that the machine has
recently stopped working. After the test, we are sensitive to that
possibility.

This idea, that tests aim for sensitivity, is hardly a radical one. It
is a very natural idea that good tests produce results that are
correlated with the attribute being tested. And `correlation' here is a
counterfactual notion. For variables \emph{X} and \emph{Y} to correlate
in the releavnt sense just means that if \emph{X} had been different,
then \emph{Y} would have been different, and the ways \emph{Y} would
have been different had \emph{X} been different are arranged in a
systematic way. When we look at the actual tests endorsed in manuals on
how to calibrate balances, producing this kind of correlation looks to
be a central aim. If a machine weren't working, and it were run through
these tests, the tests would issue a different outcome than if the
machine were working. But `testing' the machine by using its own
readings cannot produce results that are correlated with the accuracy of
the machine. If the machine is perfectly accurate, the test will say it
is perfectly accurate. If the machine is somewhat accurate, the test
will say it is perfectly accurate. And if the machine is quite
inaccurate, the test will say that it is perfectly accurate. The test
{Ins} plans to run, as opposed to the `test' that {Cus} suggests, is
sensitive to the machine's accuracy. Since it's good to have sensitive
beliefs, it is good for {Ins} to run her tests.

So I conclude that step 7 in {Cus}'s argument fails. There are reasons,
both in terms of the practice {Ins} is part of, and in terms of what
epistemic goods she'll gain on this occasion by running the test, for
{Ins} to test the machine. That's true even if she knows that the
machine is working. The epistemic goods we get from running tests are
not restricted to knowledge. That's why it is a bad idea to infer from
the badness of testing our eyes, say, using our eyes that we cannot get
knowledge that way. The aims of tests don't perfectly match up with the
requirements of getting knowledge.

\chapter{Akrasia}\label{akrasia-1}

The normative externalist seems to be committed to the following
possibility. An agent, we'll call her {Aki}, has been given excellent
arguments in favour of a false sceptical thesis. For concreteness, we'll
assume the scepticism in question is testimonial scepticism. Nothing
turns on the particular choice of sceptical thesis. But something does
turn on whether there can be excellent arguments for any false sceptical
thesis, and we'll return to this assumption below. For now we'll assume
that {Aki} is confident that one cannot get reasons to believe
propositions on the basis of testimony. And she is rational to be
confident in this; it's what her philosophical evidence supports. But,
we'll also assume, testimonial scepticism is false.

{Aki} now learns the proposition that a long-time friend, who has not
lied to her in the past, said that \emph{p}. She has weak probabilistic
reasons to have greater credence in ¬\emph{p} than \emph{p}, but these
are the kinds of background reasons that are routinely overturned by
testimony. The details don't matter, but if it helps to make the case
concrete, imagine that \emph{p} is the proposition that the home team
won last night's baseball game, when it was known in advance that the
away team was stronger, and was favoured to win. Upsets happen all the
time in baseball, so a friend's testimony that the home team won should
be only mildly surprising, and cause one to believe that the home team
won. Since in this case the friend's testimony was caused by the fact
that the home team did indeed win, it is doubly true that one should
believe the friend.

And this is what {Aki} does. Despite her philosophical leanings, she
can't bring herself to not believe what her friend says. That she can't
follow her own views in this way shouldn't be surprising. The ancient
sceptical texts are filled with both arguments for scepticism, and
techniques for putting one's sceptical conclusions into practice. It was
never assumed that mere belief in a sceptical view would suffice for
control over one's mental states ~(\citeproc{ref-Morison2014}{Morison
2014}). {Aki} is just like the people that the ancient sceptics were
writing for; people who believed their views but could not put them into
practice.

And of course, it's a good thing {Aki} does not have her theoretical
doubts govern her beliefs. She gets a well-confirmed, and true, belief
by trusting the testimony. Does she get knowledge? That's a hard
question, turning on whether one thinks that knowledge is incompatible
with this kind of mistake by one's own lights. I'm going to set that
aside, and just focus on the fact that she gets a well-supported true
belief. I think, though this is controversial, she gets a rational
belief. So {Aki} is an epistemological case of what Arpaly calls
inadvertent virtue. She forms the right belief, for the right reasons,
while thinking these are bad reasons.

Normative externalism, of the kind I prefer, says that {Aki} is doing as
well as she can in the circumstances. She is believing what her evidence
supports. She violates a level-crossing principle, but since I'm arguing
against level-crossing principles, I don't take this to be a problem.
Good for {Aki}, a paragon of rationality!

This take on {Aki}'s situation strikes many philosophers as implausible.
Some philosophers go so far as to say that {Aki}'s situation is
literally impossible; we cannot truly believe of {Aki} that she both
believes \emph{p} and believes that this is an irrational belief
~(\citeproc{ref-Hurley1989}{Hurley 1989}; \citeproc{ref-Owens2002}{Owens
2002}; \citeproc{ref-Adler2002}{Adler 2002}). Many others think that
{Aki} is possible but irrational; rationality requires that {Aki} keep
her first-order and higher-order beliefs coherent, so if she has this
combination of beliefs, she is irrational
~(\citeproc{ref-Hookway2001}{Hookway 2001};
\citeproc{ref-Ribeiro2011}{Ribeiro 2011};
\citeproc{ref-Smithies2012}{Smithies 2012};
\citeproc{ref-Greco2014}{Greco 2014};
\citeproc{ref-Horowitz2014}{Horowitz 2014};
\citeproc{ref-Titelbaum2015}{Titelbaum 2015};
\citeproc{ref-Littlejohn2015}{Littlejohn 2018}).

So we get the following argument.

\begin{enumerate}
\def\labelenumi{\arabic{enumi}.}
\tightlist
\item
  If normative externalism is true, then some akratic attitudes are
  rational.
\item
  No akratic attitudes are rational.
\item
  So, normative externalism is false.
\end{enumerate}

The short version of my response is that there is no understanding of
`akratic' that makes this argument plausible. We have to have a fairly
expansive understanding of what akrasia is for premise 1 to be true. And
on that understanding, premise 2 is implausible.

Note I'm using `attitude' in a fairly expansive sense here. If one
believes \emph{p} and believes that it is irrational to believe \emph{p}
in one's situation, I'll call that combination an akratic attitude. This
is perhaps non-standard - maybe we should say that's a pair of attitudes
that only become a single attitude if one forms the conjunctive belief
that \emph{p} is true and irrational to believe. But distinguishing
belief in a conjunction and belief in each conjunct would be needlessly
distracting in this context. Put in other terminology, the best version
of premise 2 will be a `wide-scope' principle, saying that it is
irrational to both believe \emph{p} and believe that this very belief is
irrational or otherwise defective.

\section{The Possibility of Akrasia}\label{thepossibilityofakrasia}

I'm going to mostly assume that it is at least possible to, as {Aki}
does, hold a belief while believing that very belief is in some way
improper. I've tacitly given the argument for that assumption already.
It draws on a very similar argument by Brian Ribeiro
(\citeproc{ref-Ribeiro2011}{2011}). In practice there is a gap between,
on the one hand, coming to accept a sceptical argument and being
motivated to adjust one's mental life around it, and making those
adjustments effectively. The very existence of Pyrrhonian techniques for
resisting belief in propositions that one's theory says one should not
believe is evidence of this gap. Anyone who falls into that gap, like
{Aki}, will be akratic.

Could it be said that {Aki} doesn't really believe that sceptical
arguments work? As David Owens (\citeproc{ref-Owens2002}{2002}) points
out, we don't want to just rely on {Aki}'s firm avowals that she
endorses testimonial scepticism; it takes more than talk to form a
belief. But if {Aki} says that she endorses the sceptical arguments, and
she tries to convince others of them, and she, for example, carefully
studies Sextus Empiricus for strategies for putting her testimonial
scepticism into effect, it seems plausible that she really does believe
in testimonial scepticism. And that's true even if she lacks whatever it
would take to put this sceptical doubt into full practice.

Is {Aki}, so described, akratic? Owens says that she is not, because she
does not freely and deliberately choose to believe that the home team
won last night, against her better judgment. Most other authors say, or
perhaps just assume, that epistemic akrasia does not require freely and
deliberately choosing one's beliefs. I'm not going to take a stand on
the substantive question here. If we're trying to find a plausible
version of the anti-externalist argument, it is best to not use
`akrasia' the way Owens does. That's because given Owens's usage,
premise 1 is clearly false. Normative externalism makes no commitments
at all concerning what it is rational to freely and deliberately
believe. So let's assume we're working with a notion of akrasia that is
not so demanding, and in particular that `akrasia' applies to all cases
where an agent believes against their better judgment.

\section{Three Level-Crossing
Principles}\label{threelevel-crossingprinciples}

But even that characterisation is unclear on a key point. Here are three
formulations of anti-akrasia principles that you could read as
precisifications of the idea.

\begin{itemize}
\tightlist
\item
  ``No situation rationally permits any overall state containing both an
  attitude A and the belief that A is rationally forbidden in one's
  current situation.'' ~(\citeproc{ref-Titelbaum2015}{Titelbaum 2015,
  261})
\item
  ``It can never be rational to have high confidence in something like
  \emph{P, but my evidence doesn't support P}.''
  ~(\citeproc{ref-Horowitz2014}{Horowitz 2014, 718})
\item
  ``If we use \emph{Cr} for an agent's credences and Pr for the
  credences that would be maximally rational for someone in that agent's
  epistemic situation {[}then{]}
  \emph{Cr}(\emph{A}~\textbar~Pr(\emph{A})~=~\emph{n})~=~\emph{n}''
  ~(\citeproc{ref-Christensen2010b}{Christensen 2010b, 122})
\end{itemize}

Titelbaum calls the principle he puts forward the `Akratic Principle'. I
don't want to use that name because part of what we're discussing is
whether it is the most helpful way to understand akrasia. So I'll just
call it Titelbaum's principle. Horowitz calls her principle the
`Non-Akrasia Constraint'. For similar reasons, I'll instead call it
Horowitz's principle. The principle Christensen puts forward is commonly
called \emph{Rational Reflection}, and I'll follow that usage.

Rational Reflection is, in practice, considerably stronger than
Titelbaum's principle. Imagine that {Aki} is having doubts about her
testimonial scepticism. She doesn't fully endorse it. But she is still
pretty confident in it; her credence in testimonial scepticism is 0.9.
And she thinks that if testimonial scepticism is right, then the
rational credence in the proposition that the home team won last night
is below one-half. But she still has a very high confidence that the
home team won, while thinking this is most likely irrational. This is a
violation of Rational Reflection, but not of Titelbaum's principle.
After all, there is no attitude that {Aki} both has and believes that it
is irrational to have.

That doesn't show that Rational Reflection is logically stronger than
Titelbaum's principle. Maybe there are states that violate Titelbaum's
principle but not Rational Reflection. Whether this is so turns out to
turn on difficult questions about the relationship between credence and
belief. I'm not going to get into those questions here, in part because
I have rather idiosyncratic views on them. On almost all theories about
that relationship, however, it is impossible to violate Titelbaum's
principle without violating Rational Reflection. That's what I mean by
saying that in practice, Rational Reflection is a stronger principle.

Whether Rational Reflection is also stronger than Horowitz's principle
is a little less clear. At first glance, it seems like it must be.
Imagine someone whose credences are given by the following table:

\begin{longtable}[]{@{}cc@{}}
\toprule\noalign{}
Proposition & Credence \\
\midrule\noalign{}
\endhead
\bottomrule\noalign{}
\endlastfoot
\emph{p} & 0.7 \\
The rational credence for me to have in \emph{p} is 0.7 & 0.9 \\
The rational credence for me to have in \emph{p} is 0 & 0.1 \\
\end{longtable}

Such an agent violates Rational Reflection. Rational Reflection implies
that an agent's credence in a proposition equals their expectation of
the rational credence in that proposition. And the agent's rational
expectation of the rational credence in \emph{p} is, from the last two
lines of the table, 0.63. But on the face of it, it doesn't look like
they violate Horowitz's principle. There is no proposition they are both
confident in, and confident their evidence does not support. So it looks
like Rational Reflection is stronger than Horowitz's principle too. But
the arguments below concerning iterated cases may cause us to doubt
whether that's ultimately the case.

My view is that all three of these principles are false. It's a little
trickier to say exactly which of the principles are inconsistent with
normative externalism, and so must be rejected by anyone who accepts
normative externalism. The simplest thing to say here uses the framework
developed at the end of Part I, concerning core and peripheral
commitments of normative externalism.

It is a core commitment of normative externalism that Rational
Reflection is false. Rational Reflection offers a bidirectional link
between what it is rational to believe, and what one believes about what
it is rational to believe. And, at least as I read the proponents of the
principle, the direction of explanation goes (at least in part) from the
subject's beliefs about what is rational to facts about what is
rational.

Just what to say about Horowitz's principle and normative externalism is
less clear, because we need to see exactly how it applies in some tricky
cases to get a sense of its scope. We'll return to this below.

On the other hand, it is a relatively peripheral commitment that
Titelbaum's principle is false. Titelbaum's principle is only a one way
connection. And it is at least possible to endorse it while thinking the
order of explanation goes in an externalist friendly way. One might
think that if one believes A, it is irrational to believe that it is
irrational to believe A in part in virtue of having that very first
order belief. So there is at least a version of Titelbaum's principle
for which the answers to all of the questions posed at the end of Part I
is ``No'', and that makes it an extremely peripheral violation.

We get this very externalist friendly version of Titelbaum's principle
if we think that rational beliefs must be true, at least when the belief
is about the normative. Why might we think that? One way to motivate
that view is to start with the arguments given by Clayton Littlejohn
(\citeproc{ref-Littlejohn2012}{2012}) that only true beliefs can be
justified, and try to either reason from there to the conclusion that
only true beliefs are rational, or to amend the arguments so as that
conclusion falls out. But another way is to argue that there is
something special to normative beliefs. While descriptive beliefs can be
false and rational, normative beliefs cannot. That is the lesson
Titelbaum draws from his principle (which remember he calls the `Akratic
Principle').

\begin{quote}
Ultimately, we need a story that squares the Akratic Principle with
standard principles about belief support and justification. How is the
justificatory map arranged such that one is never all-things-considered
justified in both an attitude A and the belief that A is rationally
forbidden in one's current situation? The most obvious answer is that
every agent possesses a priori, propositional justification for true
beliefs about the requirements of rationality in her current situation.
An agent can reflect on her situation and come to recognize facts about
what that situation rationally requires. Not only can this reflection
justify her in believing those facts; the resulting justification is
also empirically indefeasible. ~(\citeproc{ref-Titelbaum2015}{Titelbaum
2015, 276})
\end{quote}

But even if Titelbaum's principle were true, it wouldn't support a
conclusion nearly that strong. The inference here is of the form: Agents
can't rationally form false beliefs about a particular topic, so agents
have a priori justification for all possible true beliefs about that
topic. And there are all sorts of ways to block that. We could say that
all rational beliefs are true, as noted. Or we could simply say that for
this topic, the truth of a proposition is a reason to believe it that is
always strong enough to defeat rational justification to fully believe
its negation. There are a lot of spaces between the claim that a
proposition has a priori justification that can never be overridden, and
the claim that that proposition can never be rationally believed to be
false.

The upshot is that there are two distinct ways out, for the externalist,
from the challenge posed by akrasia. One could adopt an extremely
externalist epistemology of normative beliefs, as Titelbaum does. That
will accept that akrasia is irrational, but deny that the core
commitments of externalism entail that akrasia may be rational. Or one
could accept that some forms of akrasia, such as violations of Rational
Reflection, are rationally possible, and deny they are problematic. I'm
going to take this second path. That's in part because it gives us a
stronger form of externalism and I want to show how a strong form of
externalism may be defended. And it's in part because that's the path I
think is correct. Let's turn, then, to reasons that have been given for
thinking that all forms of epistemic akrasia are problematic.

\section{Why Not Be Akratic}\label{whynotbeakratic}

I'm going to briefly discuss a simple, but bad, argument for thinking
that all akratic agents are irrational. I'll call this the
\textbf{Argument from the Ideal}. I don't think anyone in the current
literature endorses this argument, so it should be uncontroversial that
it fails. Indeed, I suspect it is relatively uncontroversial why it
fails. But working through the argument will be helpful for getting to
our main task, discussing the \textbf{Argument from Weirdness}. This
argument turns on the following premise.

\begin{description}
\tightlist
\item[Weirdness is Irrational]
Akratic agents will say or do weird things, and only irrational agents
would say or do those weird things.
\end{description}

I think Weirdness is Irrational is false, but the following similar
principle is true.

\begin{description}
\tightlist
\item[Weirdness is Non-Ideal]
Akratic agents will say or do weird things, and no ideal agent would say
or do those weird things.
\end{description}

Different forms of the argument from weirdness will occupy the rest of
the chapter, and in every case my reply will have this form. Akratic
agents do some odd things, weird things even, but this is evidence of
their not being ideal, not of their being irrational.

But let's start with the Argument from the Ideal. Imagine a perfect
agent, who is all knowing and perfectly good. For convenience, call this
agent God. God will never be akratic. That's because God only believes
things that are strongly supported by His evidence, and only believes
truths, so He believes (truthfully!) that everything He believes is
supported by His evidence. This suggests a simple argument.

\begin{enumerate}
\def\labelenumi{\arabic{enumi}.}
\tightlist
\item
  God is not akratic.
\item
  Rational people will, so far as they can, replicate God's properties.
\item
  So rational people will not be akratic.
\end{enumerate}

The problem is that premise 2 has any number of counterexamples. As well
as not being akratic, God is \emph{opinionated}. By this, I mean that
for any \emph{p}, God will either believe \emph{p} or believe ¬\emph{p}.
(I'm assuming here that if God exists then a kind of realism is true.)
Does it follow that all rational people are opinionated? No, of course
not. I don't know what the weather is like where you, dear reader, are.
In many cases, I don't even know who you are, or when you are reading.
So far, we might think this is just a failure of omniscience. But it
doesn't mean that rationality requires that I be opinionated about who
you are, where you are, when you are, or what the weather is like there
then. Indeed, rationality requires that I not be opinionated about these
questions. And that's true even though I know if I were ideal, I would
be opinionated.

The point is not just that premise 2 of the Argument from the Ideal is
false. It's that once we have the distinction between what would be
ideal, and what would be rational in non-ideal circumstances, we can see
how a lot of other arguments fail too. So let's start working through
some of the Arguments from Weirdness with this distinction in mind.

It is plausible that in {Aki}'s situation, where she believes \emph{p}
and believes the evidence does not support it, that she should say
\emph{p, but my evidence does not support p}. And this kind of
Moore-paradoxical utterance is absurd, say some philosophers
~(\citeproc{ref-Smithies2012}{Smithies 2012};
\citeproc{ref-Greco2014}{Greco 2014}); it's not something a rational
person could say. And it's certainly weird, and non-ideal. But we can
see that it could be rational by working through some other non-ideal
cases.

{Bulan} isn't sure who she is. She is highly confident that {Bulan}'s
evidence is \emph{E\textsubscript{B}}. This is rational, though not
quite right. She knows that \emph{E\textsubscript{B}} is weak evidence
for \emph{q}, and that her evidence is \emph{E\textsubscript{A}}, and
that \emph{E\textsubscript{A}} is good evidence for \emph{q}, and that
\emph{q} is true. And that's all good, because all of those things are
true. She says \emph{q, but Bulan's evidence does not support q}. It's
hard to see what's wrong with that claim, and indeed even opponents of
epistemic akrasia should not say it is irrational. It's only the
distinctively first-personal claim, the one that we get when {Bulan}
thinks her attitude is mistaken under a first-personal mode of
presentation, that is problematic. That's interesting in itself; the
Argument from Weirdness seems to rely on a view about the
distinctiveness of first-personal thought and talk. So there is a
potential line of defence for the normative externalist that denies the
critic's assumption that first-personal belief is special
~(\citeproc{ref-CappelenDever2014}{Cappelen and Dever 2014}). But let's
grant the assumption that first-person thought and talk is special, and
see what other ways we can raise problems for the Argument.

Imagine that {Bulan} now learns who she is. Since she can't hold on to
all of the claims that she is {Bulan}, that her evidence is
\emph{E\textsubscript{A}}, and that {Bulan}'s evidence is
\emph{E\textsubscript{B}}, she drops the middle claim. She instead holds
on to the first and third claim, and infers that her evidence is
\emph{E\textsubscript{B}}. Since she knows that
\emph{E\textsubscript{B}} is weak evidence for \emph{q}, she now
believes that her evidence for \emph{q} is weak. But since the fact that
she is {Bulan} is no evidence against \emph{q}, she also holds onto her
belief that \emph{q}. So now she thinks that \emph{q, but my evidence
does not support q}. And this is meant to be problematic, at least
according to some opponents of epistemic akrasia. But it isn't at all
clear which step was mistaken. I think that proponents of the Argument
from Weirdness have to say that at the last step, one of two things must
happen. Either {Bulan} must not resolve the tension in her beliefs by
dropping the belief that her evidence is \emph{E\textsubscript{A}}, or
she must take the fact that she is {Bulan} to be a reason to lose her
belief in \emph{q}, although her identity is probabilistically
independent of whether \emph{q} is true. Neither option seems appealing,
and it's striking that proponents of the argument are forced into it.

Let's go back to the question of just what {Aki} (or Bulan) should say
given their beliefs. Even if epistemic akrasia is possible, it doesn't
immediately follow that rational agents will make these weird
utterances. If it is only appropriate to say things if one knows them,
as Williamson (\citeproc{ref-Williamson2000}{2000}) argues, and one can
only know something if one's evidence supports it, then it can never be
appropriate to say \emph{p, but my evidence does not support p}. If one
knows one evidence does not support \emph{p}, then by the factivity of
knowledge, one's evidence does not support \emph{p}, so one does not
know \emph{p}, so one should not assert it. On this view, {Aki}
shouldn't say \emph{My evidence does not support p}, even if that
proposition is supported by her evidence.

We don't need anything as strong as the rule \emph{Only say what you
know} to make the argument of the last paragraph work. Assume that for
descriptive claims, the rule is \emph{Only say what your evidence
supports}, and for normative claims the rule is \emph{Only say what is
true}. Then if \emph{p} is descriptive, it won't be permissible for
{Aki} to say \emph{p, but my evidence does not support p}. She will be
able to say this if \emph{p} is itself a normative claim. But the
evidence that her assertion would be absurd in such cases is weak; there
seem to be cases where this is exactly the right thing for her to say
~(\citeproc{ref-WeathersonMaitra2010}{Maitra and Weatherson 2010}).

Horowitz (\citeproc{ref-Horowitz2014}{2014}) carefully designs her
akratic principle so as to ensure the arguments for it can't be so
easily deflected. Imagine that {Aki} is more careful to not commit to
anything that might be false. So she says \emph{I'm confident that p,
and I'm confident my evidence does not support p}. It is not plausible
to say that one should only be confident in a proposition, or should
only announce one's confidence in that proposition, if one knows the
proposition to be true. For every lottery ticket in a large, fair
lottery, I'm confident it will lose, yet I can't know each ticket will
lose. (Perhaps I can't know any ticket will lose.) Horowitz argues that
even this qualified utterance of {Aki}'s is defective.

Notably, she doesn't just argue for this on the basis of intuitions
about how weird the assertion itself sounds. There is a good dialectical
reason for her to reason this way. The anti-akratic thinks that it is
wrong to both be confident in \emph{p} and in the proposition that the
evidence for \emph{p} is not strong, no matter which proposition
\emph{p} is, and no matter what the agent's background. It's hard to see
how getting intuitions going about a few token utterances could support
a universal generalisation that sweeping. So Horowitz offers some more
careful arguments, ones that have at least the potential to generalise
in the needed way.

Horowitz argues that {Aki} should be in a position to conclude, on the
basis of her evidence, that her evidence is misleading, and that she was
lucky to become so confident in the truth. And this, Horowitz thinks, is
wrong. One needs independent reason to think that one's evidence is
misleading, so it's wrong for {Aki} to conclude that on the basis of
this very evidence. But that last premise seems too strong. Sometimes
parts of one's evidence can be sufficient ground for thinking one's
overall evidence is misleading. That's indeed what happens in {Aki}'s
case. There is no one part of her evidence that is both grounds for
something and (complete) grounds for thinking those very grounds are
misleading. The internal relations between the different parts of her
evidence provide all the independent support we need for a reasonable
judgment that other parts are misleading.

Horowitz has another argument that {Aki} will be in an untenable
position. Imagine she is offered a bet that wins a small amount if
\emph{p} is true, and loses a larger amount if it is false. {Aki} takes
the bet, as she should given that she has excellent reason to believe
\emph{p} is true. But she is then asked why she is doing this, she'll
say that she should not be doing it; she has no good reason to believe
the bet will win. Is this, doing something while saying one should not
be doing it, problematic? Once we've seen other cases of inadvertent
virtue, we can see why the answer is no. {Huck} Finn should help Jim
escape, and should say he's doing the wrong thing while doing so.
{Aki}'s predicament is no worse.

Recently, Clayton Littlejohn (\citeproc{ref-Littlejohn2015}{2018})
argued for an anti-akrasia view by suggesting that {Aki} would end up
with a distinct kind of untenable attitude. He imagines a conversation
between {Aki} and her epistemic conscience with the following punchline.
(Note in Littlejohn's example, the first order evidence supports not
believing in \emph{p}, and the higher-order evidence supports belief in
\emph{p}. This is the reverse of the case I've started with, but that
doesn't matter much. What matters is that the levels diverge, and Aki
follows the first-order evidence.)

\begin{quote}
EC: You agree that it's irrational for you not to believe \emph{p}. You
agree that it's rational for you to agree on this point. You acknowledge
that you don't believe \emph{p}. You just don't yet see that this calls
for any sort of change.\\
{Aki}: Right. ~(\citeproc{ref-Littlejohn2015}{Littlejohn 2018, 12},
reference to preprint)
\end{quote}

And this last statement of {Aki}'s is untenable, thinks Littlejohn. And
I suspect he is right about that. But it doesn't matter, because that's
not what {Aki} should be saying. She should say that there is a ``call
for change'', and she should think that there is such a call. After all,
she thinks that she is not following her evidence, and that one should
in general follow one's evidence. At the very least, that seems like
reason to stop and have a think about how one got into this situation,
and see if there wasn't some big mistake made along the way.

If {Aki} doesn't stop and reflect on her odd situation, that would be
somewhat strange behaviour. But even the normative externalist can say
that she should stop and reflect. It's true that she she isn't doing
anything wrong. But whether one should stop and reflect is not entirely
a function of whether one is doing anything particularly wrong. If one's
cognitions or activities (or the conjunction of these) resemble those of
people who are making mistakes, one has a reason to be think through
what one has done. Of course, if {Aki} were ideal, she wouldn't need to
stop and reflect, since she would know she is responding optimally to
being in a strange situation. But if she were ideal, i.e., if she were
God, she wouldn't be in that situation in the first place.

So we still haven't seen anything that {Aki} should do or say, given
normative externalism, that is weird in a way that is inconsistent with
rationality. She should perhaps say one thing and do another, just like
{Huck} Finn. And she should say that {Aki}'s evidence doesn't support
what she herself believes, just like {Bulan} (in the original case)
should say that {Bulan}'s evidence doesn't support what she herself
believes. But {Huck} Finn, and {Bulan}, aren't problematic. And the
attempts to get {Aki} to say weirder things so far haven't worked;
they've got her making assertions that violate norms of assertion even
by the externalist's lights.

\section{Self-Awareness and Rational
Reflection}\label{self-awarenessandrationalreflection}

In the previous section I argued that there was nothing distinctively
weird about akratic agents. They say and do weird things that other
non-ideal but rational agents do. In this section I'll continue the
argument a little, with more focus on two particular principles. In
particular, I'll argue for these two claims:

\begin{enumerate}
\def\labelenumi{\arabic{enumi}.}
\tightlist
\item
  Cases where agents do not know exactly what their situation is
  generate counterexamples to Rational Reflection, and to Horowitz's
  principle.
\item
  There is no reason to believe that these principles hold in cases
  where agents do know what their evidence is, since there is no reason
  to think that violations of the principles are more problematic in
  cases where agents do know what their evidence is.
\end{enumerate}

I'll start with two relatively plausible assumptions:

\begin{enumerate}
\def\labelenumi{\arabic{enumi}.}
\tightlist
\item
  What attitudes it is rational for an agent to have depend on features
  of her situation that vary from agent to agent and time to time.
\item
  The features that are relevant in point 1 are not luminous; agents
  might possess them without knowing that they do.
\end{enumerate}

My view is that the `features' in assumption 1 are just the agent's
evidence, but I'm not assuming that. I'm just assuming that what's
rational depends on the circumstances.

Premise 2 follows from the anti-luminosity arguments introduced by
Williamson (\citeproc{ref-Williamson2000}{2000}), and defended recently
by Hawthorne and Magidor (\citeproc{ref-HawthorneMagidor2009}{2009},
\citeproc{ref-HawthorneMagidor2011}{2011}) and by Srinivasan
(\citeproc{ref-Srinivasan2015}{2015a}). I don't need the full blown
anti-luminosity principle to complete the argument. All I need is that
luminosity fails for some of the features that are relevant to rational
belief. So if there are some luminous states, as I've argued elsewhere
~(\citeproc{ref-Weatherson2004}{Weatherson 2004}), that won't matter
unless all features relevant to rationality are luminous. And that's not
particularly plausible.

Even if all rational agents know exactly what is rationally required in
all possible situations, as Titelbaum argues they do, there will still
be failures of Rational Reflection. That is because an agent need not
know what situation they are actually in. It is possible for an agent to
have perfect knowledge of the function from situations to the rational
status of states in such a situation, and not know what is rational for
them. If rather extreme rational states are only permissible in rare
situations, and the agent is in such a rare situation, then Rational
Reflection will fail.

The abstract possibility described in the previous sentence is realized
in Williamson's case of the unmarked clock
~(\citeproc{ref-Williamson2011}{Williamson 2011},
\citeproc{ref-Williamson2014}{2014}). I'll work through Horowitz's
variant, her case of the unmarked dartboard, because it provides a
useful platform for setting up Horowitz's criticisms of the example, and
my reply.

A dart is thrown at a dartboard that is infinite in height and width.
The dartboard has gridlines on it running up-down and left-right. Due to
magnets in the dart and the board, we know in advance that it will land
on the intersection of two gridlines. The agent, we'll call her {Siiri},
can almost, but not quite, make out where it lands, and she knows in
advance this will be the case.

Say that the `distance' between two grid points,
⟨\emph{x}\textsubscript{1},~\emph{y}\textsubscript{1}⟩ and
⟨\emph{x}\textsubscript{2},~\emph{y}\textsubscript{2}⟩ is
\textbar{}\emph{x}\textsubscript{1} -
\emph{x}\textsubscript{2}\textbar{} +
\textbar{}\emph{y}\textsubscript{1} - \emph{y}\textsubscript{2}\textbar.
This is not the straight-line distance between the points; it is the
shortest path between them on gridlines. {Siiri} knows in advance that
if the dart lands on ⟨\emph{x},~\emph{y}⟩, then she'll know it is on
⟨\emph{x},~\emph{y}⟩ or one of the four points distance 1 away from it.
And she knows in that situation it will be rational to have equal
credence that it is on each of those five points.

Assume the dart lands on ⟨8,~3⟩, and consider her credence in the
proposition that it is on ⟨7,~3⟩, ⟨8,~4⟩, ⟨9,~3⟩ or ⟨8,~2⟩. Call that
proposition \emph{p}. After getting visual evidence of where the dart
is, her credence in \emph{p} should be 0.8. But she should have credence
0.8 in \emph{p} iff the dart is on ⟨8,~3⟩, and credence 0.2 in \emph{p}
if the dart is on any of the other four squares she thinks it might be
on. So given her situation, the expected rational credence in \emph{p}
is 0.32. So Rational Reflection fails, even though {Siiri} knows exactly
the function from situations to rational credences.

Horowitz argues that this is a special case. She thinks that a
restricted version of Rational Reflection can be crafted that is immune
to such a counterexample. There is something odd about the example.
We're interested in a proposition \emph{p} that is in a very odd class.
Consider all propositions of the form \emph{the dart lands distance 1
from point} ⟨x, y⟩. {Siiri} knows in advance that she will be very
confident in such a proposition iff it is false. And that is odd. Here
is how Horowitz puts the point. (Note that I've adjusted the terminology
slightly to match what's here, and what she calls `akrasia' is being
highly confident in \emph{p, but my evidence doesn't support p}.)

\begin{quote}
In Dartboard, however, the evidence is \emph{not} truth-guiding, at
least with respect to propositions like \emph{p}. Instead, it is
\emph{falsity}-guiding. It supports high confidence in \emph{p} when
\emph{p} is false---that is, when the dart landed at ⟨8,~3⟩. And it
supports low confidence in \emph{p} when \emph{p} is true---that is,
when the dart landed at ⟨7,~3⟩, ⟨8,~4⟩, ⟨9,~3⟩ or ⟨8,~2⟩. This is an
unusual feature of Dartboard. And it is only because of this unusual
feature that epistemic akrasia seems rational in Dartboard. You should
think that you should have low confidence in \emph{p} precisely
\emph{because} you should think \emph{p} is probably true---and because
your evidence is falsity-guiding with respect to \emph{p}. Epistemic
akrasia is rational precisely because we should take into account
background expectations about whether the evidence is likely to be
truth-guiding or falsity-guiding. ~(\citeproc{ref-Horowitz2014}{Horowitz
2014, 738}, notation altered, emphasis in original)
\end{quote}

Surprisingly, it isn't essential to the example that the evidence is
falsity-guiding in Horowitz's sense. This feature of the case is a
byproduct of its simplicity; more complicated cases don't have this
feature.

Imagine instead that when the dart lands at a particular spot
⟨\emph{x},~\emph{y}⟩, all spots whose distance from ⟨\emph{x},~\emph{y}⟩
is 10 or less are open epistemic possibilities for {Siiri}. But they are
not equal possibilities; her probability distribution is peaked at
⟨\emph{x},~\emph{y}⟩ itself. For any grid point distance \emph{d} from
⟨\emph{x},~\emph{y}⟩, her posterior probability that it landed there is:

\[
\frac{4^{10-d}}{2,912,692}
\]

The denominator there is just what's needed to make the probabilities
add to 1. The intuitive idea is for each step further away from the
center we get, the probability of being in that particular cell falls by
a factor of 4. Now assume again the dart lands on ⟨8,~3⟩, though of
course {Siiri} does not know this, and let \emph{q} be the proposition
that the distance between the dart and ⟨8,~3⟩ is either 0 or 3.

The evidence is not falsity-guiding with respect to \emph{q}. Given what
we said about {Siiri}, then among the worlds that are epistemically
possible for her, her credence in \emph{q} would be higher if \emph{q}
were true than if it were false. More precisely, her credence in
\emph{q} would somewhere between 0.413 and 0.44 if she were in one of
the worlds that made \emph{q}, and at most 0.307 if she were in one of
the worlds that made \emph{q} false. (The calculations to confirm the
facts I'll run through about the example are tedious, but trivial, to
verify with a computer.) The evidence supports higher confidence in
\emph{q} when \emph{q} is true than than when \emph{q} is false. That's
unlike the original example. But this case also generates violations of
Rational Reflection. {Siiri}'s credence in \emph{q} is about 0.4275, but
her expectation of the rational credence in it is about 0.3127.

Now you might think that's not a huge difference. Perhaps this is a
counter-example to Rational Reflection, but not to Horowitz's principle
that it is irrational to be highly confident in a proposition while also
being highly confident that one is irrational to be so confident. But if
we iterate the example, we get a counterexample to that principle too.

Imagine {Siiri} starts off (rationally) certain that repeated throws at
the board are independent. And imagine that the dart is removed after
each throw, so she can't see that successive darts land at the same
spot. And imagine that her ability to detect where it lands doesn't
improve, indeed doesn't change, over repeated throws. Finally imagine
(somewhat improbably!) that repeated throws keep landing on ⟨8,~3⟩. Let
\emph{r} be the proposition that at least 35 percent of throws are
either distance 0 or distance 3 from ⟨8,~3⟩. As the number of throws
increases, she should get more and more confident that is true, and get
more and more confident that it is irrational to think that it is true.
After 100 throws, for example, her credence in \emph{r} should be over
0.95, but her expectation of the rational credence in \emph{r} should be
under 0.25. This kind of iteration of examples can be used to turn any
dartboard-like counterexample to Rational Reflection into a
counterexample to Horowitz's principle.

\section{Akrasia and Odd Statements}\label{akrasiaandoddstatements}

So Horowitz's explanation of why cases like {Siiri}'s are special, that
they are cases where agents know evidence is not truth-conducive,
doesn't work. And that raises doubts for any attempt to separate {Aki}'s
case from {Siiri}'s.

A large part of the motivation for thinking {Aki}'s state is irrational
is that {Aki} says weird things, like \emph{p is true, although my
evidence supports p being false}. But {Siiri} says similar things, and
they are the right things for {Siiri} to say. So the very fact that
{Aki} says them can't show that her position is incoherent; she is, in
this respect, just like the perfectly coherent (if unfortunate) {Siiri}.

{Siiri} might regard it as a lucky break that she has a true belief
despite not following her evidence. Of course, {Aki} could feel the same
way. She should think that the home team won, think that her evidence
doesn't support this, and from those claims think it is lucky that she
has a correct belief despite not following the evidence. But {Siiri}
will think something structurally similar. Horowitz argues that {Siiri}
doesn't have to regard herself as implausibly lucky. In the original
version of the case {Siiri} knows the evidence is not truth-conducive,
so it isn't a lucky break that not following the evidence (as it seems)
leads to truth. But in the revised case, {Siiri} has to think she's just
as lucky as {Aki}. And if it is reasonable for {Siiri} to think she is
lucky, it is also reasonably for {Aki} to think she is.

Let's take stock. {Siiri}'s case shows that Rational Reflection fails,
and that it can be rational to be confident in something while also
being confident that one's evidence does not support this view. It does
not show that it can be rational to be confident in a falsehood about
what rationality itself requires, as opposed to what one's situation is.
That is, one could be certain about all the truths about what
rationality requires in each situation, and still end up like {Siiri}.
Indeed, we assumed she was certain about all the truths about what
rationality requires in each situation, and still got a strange result
falling out. So {Siiri}'s case does not directly tell against the most
plausible version of Titelbaum's principle.

But the arguments for Titelbaum's principle (or anything like it) are
all Arguments from Weirdness. And {Siiri}'s case does undermine the
force of those arguments. For she says a lot of weird things too, and
they are the right thing to say. So the fact that violations of
Titelbaum's principle will lead to people saying weird, akratic things
is no reason to think that Titelbaum's principle is a requirement of
rationality. In weird situations, rational people are weird. Ideal
people aren't weird, but that's only because they know things about
their situation that are hidden from normal, rational, people. Normative
externalism does imply that rational people will be akratic, and be
weird, and be non-ideal. But none of that is surprising; the kinds of
weirdness and non-idealness we see are just what we should independently
expect in rational, but non-ideal, people.

\section{Desire as Belief (Reprise)}\label{desireasbeliefreprise}

The dartboard example is relevant to more than debates over akrasia. It
also helps illustrate a point I alluded to frequently in part one,
without ever setting out in detail. Proponents of the idea that moral
uncertainty matters to rational decision making seem to be committed to
a kind of `desire as belief' thesis. David
(\citeproc{ref-Lewis1988b}{Lewis 1988},
\citeproc{ref-Lewis1996a}{1996a}) raised some technical problems for
such theories, and recently those problems have been expanded by J. S.
Russell and Hawthorne (\citeproc{ref-RussellHawthorne2016}{2016}). I'm
not going to add anything to the arguments they have offered. But I
think it might be helpful to translate those arguments into the idioms
that are more familiar in the moral uncertainty debates, since
participants in that debate have not always appreciated the significance
of these formal results. The only philosopher I know who has connected
the moral uncertainty debates with the desire as belief debates is Ittay
Nissan-Rozen (\citeproc{ref-NissanRozen2015}{2015}), and he takes an
externalist position on moral uncertainty. My focus will be on the
argument Russell and Hawthorne give, because it would be too much of a
digression to investigate whether the `desire by necessity' response
that Huw Price (\citeproc{ref-Price1989}{1989}) gives to Lewis's
arguments is successful.

Let's assume that we want moral uncertainty to play an important role in
decision making. We should be able to provide some kind of semantics for
claims about moral uncertainty. In particular, we would like a semantics
for claims of the form \emph{A is better than B} that satisfies the
following four constraints.

\begin{enumerate}
\def\labelenumi{\arabic{enumi}.}
\tightlist
\item
  Claims like \emph{A is better than B} should be the kind of thing that
  can be believed, and that one can have higher or lower credences in.
  So that claim should be associated with a set of worlds, or a set of
  n-tuples, where the first member of that tuple is a world. (The latter
  disjunction is relevant if one thinks, perhaps following Lewis
  (\citeproc{ref-Lewis1979}{1979}), that the objects of belief are
  something like centred worlds.)
\item
  These attitudes in moral `propositions' (or whatever else is picked
  out by \emph{A is better than B}) should be updated in the way that
  credal attitudes are usually updated. Ideally that would be by
  conditionalisation, or by some other update rule that can be given
  independent motivation.
\item
  The semantics should associate with \emph{A is better than B} a set of
  worlds (or tuples or whatever) that at least roughly corresponds with
  what those words ordinarily mean in English.
\item
  The claim should be action guiding, so (perhaps barring exceptional
  circumstances) conditional on \emph{A is better than B},~\emph{A}
  should be more choice-worthy than \emph{B}.
\end{enumerate}

And it turns out to be incredibly hard to find a semantics that
satisfies these four constraints. In fact, there are principled reasons
to think that no such semantics is possible.

There is one technical complication that we need to address first.
Whether \emph{A} is better than \emph{B} depends on one's evidence. So
if \emph{A} is that I get a (typical) lottery ticket, and \emph{B} is
that I get a penny, then \emph{A} is better than \emph{B}, from my
perspective, iff I don't know that \emph{A} is a losing ticket. It is
far from trivial to represent claims about what one's evidence is in a
semantic model. That's in part because facts about what one's evidence
is are `first-personal' facts that are tricky to represent in standard
models, and in part because what one's evidence is changes over time,
and it's hard to represent changes over time in standard models.

Here's how I'll try to deal with, or at least sidestep, these problems.
Instead of thinking of beliefs as attitudes to sets of worlds, we'll
think of them as attitudes to world-evidence-morality triples:
⟨\emph{w},~\emph{e},~\emph{m}⟩. And we'll assume that \emph{e}
determines (perhaps among many other things) a function from times to
one's evidence at that time. Just how it does that, and just how it
attitudes distributed over \emph{e} are updated, will be left as a
black-box. (See Titelbaum (\citeproc{ref-Titelbaum2016}{2016}) for an
excellent survey of the options for how the self-locating parts of one's
credal state might be updated.)

I'll assume \emph{m} is just a number, perhaps subject to enough
constraints that we don't end up in the paradoxes of unbounded
utility\footnote{I'm assuming here that the moral value of a world can
  be represented as a number. That's not particularly plausible, but
  without this assumption the internalist views I'm opposing are very
  hard to state or defend.}. And what we want is that the value of a
proposition is the expected value of \emph{m} given that the proposition
is true. So \emph{A} is better than \emph{B}, given some evidence, just
in case the expected value of \emph{m} given \emph{A} and that evidence
is greater than the expected value of \emph{m} given \emph{B} and that
evidence. But expected values change with evidence, and evidence changes
with time, so this doesn't settle what \emph{m} should be. It turns out
that while there are a few ways one could go here, any choice ends up
violating one of the four constraints I proposed.

Assume, first, that the evidence is highly malleable. I mean two things
by that. One is that when we conditionalise on some proposition
\emph{c}, then \emph{c} gets added to the evidence. The other is that
the time in question (and remember that \emph{e} is a function from
times to evidence sets) is the time any relevant decision has to be
made. This pair of assumptions has a very nice feature - it guarantees
that the fourth constraint is met. (This turns out to be harder to do
than you might think.) Conditional on \emph{A is better than B}, thus
interpreted, I should choose \emph{A} over \emph{B}, no matter what the
other evidence is.

The problem with this assumption is that it violates the third
constraint rather dramatically. The following example is a version of
the objection that J. S. Russell and Hawthorne
(\citeproc{ref-RussellHawthorne2016}{2016, 315--16}) make to the
principle they call \textbf{Comparative Value}. Consider the following
substitutions for \emph{A} and \emph{B}.

\begin{description}
\tightlist
\item[A1]
I get a can of frosty ice-cold Foster's Lager in five minutes time.
\item[B1]
I get a poke in the eye with a burnt stick in five minutes time.
\end{description}

I think that \textbf{A1} is better than \textbf{B1}. And I even think
that conditional on them both being true, which I hope they aren't. But
on this model, we can't have that. Because conditional on them both
being true, the expected value of \emph{m} conditional on either of them
is the same as the expected value \emph{m} simpliciter. So conditional
on their both being true, it isn't true that \textbf{A1} is better than
\textbf{B1}.

This is already a violation of constraint 3. But as Russell and
Hawthorne go on to point out, a lot of strange things start to follow if
we don't want to violate constraint 2 as well. We just proved that
conditional on \textbf{A1} ∧ \textbf{B1}, it must be false that
\textbf{A1} is better than \textbf{B1}. That is, conditional on
\textbf{A1} ∧ \textbf{B1}, the probability of \textbf{A1} is better than
\textbf{B1} must be 0. If the way to update on \textbf{A1} ∧ \textbf{B1}
is by conditionalisation, it follows that the current probability of the
conjunction of \textbf{A1},~\textbf{B1} and \textbf{A1} is better than
\textbf{B1} must be 0. So conditional on \textbf{A1} is better than
\textbf{B1}, which is surely true, the conjunction of \textbf{A1} and
\textbf{B1} must have probability 0. And that's true for any \emph{A, B}
such that right now it's known that \emph{A} is better than \emph{B}.
This is all absurd. Now perhaps this isn't a violation of constraint 2,
because I'm assuming here that update is by conditionalisation, and
maybe there is a principled way to reject that in cases like this. In
any case, this option for how to understand \emph{e} fails constraint 3,
so it must be wrong.

The way this option failed suggested a distinct move. What's true about
\textbf{A1} and \textbf{B1} is not that given they are both
true,~\textbf{A1} will make the world better than \textbf{B1} will.
After all, given they are both true, they won't make any (further)
difference to the world. So perhaps when assessing \textbf{A1} and
\textbf{B1} for value, we should look at their initial value, or their
value given the (absolutely) prior probability.

The problem with this approach is that it doesn't allow learning. Assume
we learn \emph{C}, than if I get poked in the eye with a burnt stick in
five minutes, then malaria will be cured. Then it would be false that
\textbf{A1} is better than \textbf{B1}, and indeed true that \textbf{B1}
is better than \textbf{A1}. (Although, owww!) So this approach also
violates constraint 3. And, for the same reason, it violates constraint
4.

Maybe the approach is to rigidify. What it means to say that \emph{A is
better than B} is that given the actual evidence I currently
have,~\emph{A} has a higher expected \emph{m} value than \emph{B}. This
will handle the the Foster's/poke case fairly well. But it leads to
other problems. The following is a simple variant of the Rembrant case
J. S. Russell and Hawthorne (\citeproc{ref-RussellHawthorne2016}{2016,
331}) offer.

Imagine we're in the simpler of the dart cases. When a dart lands on
⟨\emph{x},~\emph{y}⟩, then each of the five possibilities that it is on
that very spot, or that it is one spot up, down, left or right are
equally likely. And the dart did in fact land on ⟨8,~3⟩. At the same
time, two fair coins have been tossed, although the results of them are
hidden. Now compare the following options:

\begin{description}
\tightlist
\item[A2]
I get a Vegemite sandwich if the dart landed on ⟨8,~4⟩, ⟨8,~2⟩, ⟨7,~3⟩
or ⟨9,~3⟩, and nothing otherwise.
\item[B2]
I get a Vegemite sandwich if at least one of the coins landed heads, and
nothing otherwise.
\end{description}

Right now \textbf{A2} is better than \textbf{B2}. That's because given
my evidence,~\textbf{A2} gets me a 0.8 chance of a Vegemite sandwich,
and \textbf{B2} gets me a 0.75 chance. (Assuming, as is completely
obvious, that more Vegemite sandwiches are better than fewer.) But
conditional on \textbf{A2} is better than \textbf{B2}, I should prefer
\textbf{B2}. That's because the only worlds where \textbf{A2} is better
than \textbf{B2} are worlds where the dart landed on ⟨8,~3⟩. And in
those worlds, I don't get a Vegemite sandwich from \textbf{A2}.

So this rigid interpretation of `better' violates constraint 4: it makes
betterness judgments not be action guiding. I prefer \textbf{A2} to
\textbf{B2}, but conditional on \textbf{A2} being better than
\textbf{B2}, I prefer \textbf{B2}. Personally, I think this is the best
interpretation of `better', but that's because I think our choices
shouldn't be guided by our beliefs about, or our evidence about, what's
better than what.

I haven't given a watertight proof here that there is no way to
interpret `better' in this kind of model, or any other kind of model,
that satisfies the four constraints. But philosophers who think moral
uncertainty matters for decision making haven't typically appreciated
how hard it is to get a model that does satisfy these constraints. The
`desire as belief' results are fairly surprising, and when combined with
anti-luminosity principles, they make it very hard to see how moral
uncertainty could be relevant to decision making.

\chapter{Screening and Regresses}\label{screeningandregresses}

Normative externalism in epistemology is false if agents should respond
not just to their evidence, but to what they believe, or should believe,
about what their evidence supports. Call that latter claim the higher
order hypothesis. Over the last three chapters I've responded to
arguments for the higher order hypothesis. I argued that the cases that
apparently support the higher order hypothesis do not do so, when viewed
in the context of a wider sweep of cases. I've argued against attempts
to derive the higher order hypothesis from anti-circularity principles.
And I've argued against attempts to derive it from enkratic principles.
In this chapter I move on to giving reasons to disbelieve the higher
order hypothesis. I argue that the higher order hypothesis is tied to a
principle about screening, a principle I call \textbf{Judgments Screen
Evidence}. And I argue that this principle, whose name I'll shorten to
JSE, leads to intolerable regresses. The return to regress based
arguments provides a stronger link than we've seen so far between this
part of the book and the earlier part; the arguments of this chapter
might seem very familiar to someone who has read chapter 4.

\section{Screening}\label{screening}

The idea of screening that's going to be central to this chapter comes
into philosophy via Hans Reichenbach
(\citeproc{ref-Reichenbach1956}{1956}). He was working on a quite
different problem, namely when we should infer that two events have a
common cause. He says that \emph{C} screens off the positive correlation
between \emph{B} and \emph{A} iff the following two conditions are met.

\begin{enumerate}
\def\labelenumi{\arabic{enumi}.}
\tightlist
\item
  \emph{A} and \emph{B} are positive correlated, i.e.,
  Pr(\emph{A}~\textbar~\emph{B})~\textgreater~Pr(\emph{A}).
\item
  Given \emph{C}, \emph{A} and \emph{B} are probabilistically
  independent, i.e.,
  Pr(\emph{A}~\textbar~\emph{B}~∧~C)~=~Pr(\emph{A}~\textbar~\emph{C}).
\end{enumerate}

I'm interested in an evidential version of screening. If we understand
evidential support probabilistically, then we could just copy over
Reichenbach's definitions, with a little reinterpretation of the
formalism. So rather than thinking of Pr in terms of objective
processes, as Reichenbach was, think of it as an evidential probability
function. Then these two clauses will say that as things stand, \emph{B}
is evidence for \emph{A}, but given \emph{C}, \emph{B} is no evidence
for \emph{A}. We can say all that without assuming any particular
connection between probability and evidence, as follows.

\begin{quote}
\emph{C} screens off the evidential support that \emph{B} provides to
\emph{A} iff:

\begin{enumerate}
\def\labelenumi{\arabic{enumi}.}
\tightlist
\item
  \emph{B} is evidence for \emph{A}; and
\item
  \emph{B} ∧ \emph{C} is exactly as good evidence for \emph{A} as
  \emph{C} is.
\end{enumerate}
\end{quote}

Both these clauses, as well as the statement that \emph{C} screens off
\emph{B} from \emph{A}, are made relative to an evidential background.
I'll leave that as tacit in what follows. Here are a couple of examples,
the second loosely based on facts, that illustrate the usefulness of
this idea.

A woman is standing at a suburban train station waiting for a train into
the city, and wondering whether she will be on time for her meeting. She
knows that there is only one train line, with no usable sidings, between
where she is and the city, so there isn't any chance of trains passing.
She knows how long trains take to get to the city if everything is
working, though she doesn't know if everything is indeed working. But
she doesn't know how frequent the trains are. She gets a call from a
friend saying that a train to the city is headed her way, and is about
five miles away. That train would, she thinks, get her to the city in
time if everything goes right. Just then she sees a train coming into
the station. Let \emph{A} be that she gets to the city on time, \emph{B}
that there is a train five miles away, and \emph{C} that there is a
train pulling into the station. Relative to her initial background,
\emph{B} is evidence for \emph{A}. But given \emph{C}, it is no evidence
at all. That's because given \emph{C}, what matters is whether this
particular train makes it in on time, without breaking down or being
held up for some reason. The later train can't pass her, so its presence
isn't relevant to whether she makes it to the city on time.

Later, she is trying to work out whether a particular person X voted for
the Democratic candidate or the Republican candidate at the last
Presidential election. She knows that X is either from Alabama or
Massachusetts, and voted, and she knows the distribution of voters in
those two states are as follows. (The numbers in the boxes are
percentages of voters, and GOP is shorthand for the Republican Party.)

~~\textbar{} Pro-Choice \textbar{} Pro-Life \textbar{} Pro-Choice
\textbar{} Pro-Life \textbar{}\\
\strut ~~\textbar{} Dem \textbar{} Dem \textbar{} GOP \textbar{} GOP
\textbar{}

\textbar:---\textbar:---:\textbar:---:\textbar:---:\textbar:---:\textbar{}
\textbar{} Alabama \textbar{} 28 \textbar{} 7 \textbar{} 7 \textbar{} 58
\textbar{} \textbar{} Massachusetts \textbar{} 52 \textbar{} 13
\textbar{} 13 \textbar{} 22 \textbar{}

Learning which state X is from is strong evidence about how they voted,
since 65\% of Massachusetts voters voted Democratic, while only 35\% of
Alabama voters did. But if she had previously learned that X was
pro-choice, then learning which state X is from would be of no
evidential significance. That's because 80\% of pro-choice voters in
each state voted Democratic. So learning that X is a pro-choice resident
of Massachusetts is of no more evidential significance than simply
learning X is pro-choice.

There is something very interesting about this theoretical possibility.
We can concede that something is usually evidentially significant even
while denying it is significant on a particular occasion. This
possibility is useful for solving a puzzle about judgment.

\section{The Counting Problem}\label{thecountingproblem}

Suppose a rational agent has some evidence \emph{E} that bears on a
proposition \emph{p}, and on that basis judges that \emph{p}. Call the
fact that the agent has made this judgment \emph{J}, and assume the
agent is self-aware enough to know that \emph{J} is true, and that she
is rational. Assume also that \emph{p} is a rational thing to judge on
the basis of \emph{E}, though the agent does not necessarily know this.
The fact that a rational person judges that \emph{p} seems to support
\emph{p}. After all, if we found out that she is rational and judged
that \emph{p}, that would ceteris paribus be evidence for \emph{p}. Now
consider this slightly informal question: \emph{How many pieces of
evidence does the agent have that bear on p}? Three options present
themselves.

\begin{enumerate}
\def\labelenumi{\arabic{enumi}.}
\tightlist
\item
  Two - both \emph{J} and \emph{E}.
\item
  One - \emph{E} subsumes whatever evidential force \emph{J} has.
\item
  One - \emph{J} subsumes whatever evidential force \emph{E} has.
\end{enumerate}

This suggests a trilemma. First, it seems \emph{J} could be evidence for
\emph{p}. We could get reason to be more confident in \emph{p} just by
learning \emph{J}. Second, it seems like double counting for the agent
to take both \emph{E} and \emph{J} to be evidence. After all, she only
formed the judgment because of \emph{E}. Yet third, it seems wrong for
her to simply ignore \emph{E}, since by stipulation it is evidence, and
it certainly seems to bear on whether \emph{p} is true.

One way out of this is to adopt the thesis I'll call \textbf{JSE}, for
Judgment Screens Evidence. This is the thesis that propositions about
rational judgments by rational agents screen off the evidential
significance of the underlying evidence behind those judgments. The
simplest argument for JSE is that it lets us answer the question above
while accommodating the idea behind all three sources of `pressure'. The
agent can treat \emph{J} just like everyone else does, i.e., as some
evidence for \emph{p}, without double counting or ignoring \emph{E}. She
can do that because she treats \emph{E} as screened off. And screened
off evidence isn't double counted or ignored. That's a rather nice
feature of JSE.

To be sure, it is a feature that JSE shares with a view we might call
ESJ, or evidence screens judgments. That view says that the agent
shouldn't take \emph{J} to be extra evidence for \emph{p},since its
evidential force is screened off by \emph{E}. This view also allows for
the agent to acknowledge that \emph{J} has the same evidential force for
her as it has for others, while also avoiding double counting. So we
need some reason to prefer JSE to ESJ.

One reason is by thinking generally about reasoning that proceeds in
steps. Assume \emph{E} is evidence for \emph{p} solely because it makes
\emph{q} more likely, and \emph{q} in turn makes \emph{p} more likely.
So if we are investigating a crime that took place in an inland village
in Cornwall, learning that a suspect had some sand in his clothes that
is only found on Cornish beaches may be some evidence that he's guilty.
That's because it establishes that the suspect was at least in the area,
unlike some other suspects. But if we knew independently that the
suspect had been in Cornwall, say because he owns a beach house there
and is often seen by his neighbours, the presence of the sand is of no
evidential significance. Perhaps the general lesson here is that later
steps screen off earlier steps. If that's right, we would expect
\emph{J} to screen \emph{E}, and not vice versa.

Another reason for preferring JSE to ESJ is that it alone supports a
number of positions that epistemologists have found independently
plausible. Indeed, it is arguable that JSE is something of a tacit
premise in a number of arguments. In the next section we will look at
three such arguments.

\section{JSE in Epistemology}\label{jseinepistemology}

\subsection{Egan and Elga on
Self-Confidence}\label{eganandelgaonself-confidence}

We'll start with some conclusions that Andy Egan and Adam Elga draw
about self-confidence in their paper ``I Can't Believe I'm Stupid''. I
suspect many of the conclusions they draw in that paper rely on JSE, but
I'll focus just on the most prominent use of JSE in the paper.

\begin{quote}
One of the authors of this paper has horrible navigational instincts.
When this author---call him ``{AE}''---has to make a close judgment call
as to which of two roads to take, he tends to take the wrong road. If it
were just {AE}'s first instincts that were mistaken, this would be no
handicap. Approaching an intersection, {AE} would simply check which way
he is initially inclined to go, and then go the opposite way.
Unfortunately, it is not merely {AE}'s first instincts that go wrong: it
is his all things considered judgments. As a result, his
worse-than-chance navigational performance persists, despite his full
awareness of it. For example, he tends to take the wrong road, even when
he second-guesses himself by choosing against his initial inclinations.

Now: {AE} faces an unfamiliar intersection. What should he believe about
which turn is correct, given the anti-reliability of his
all-things-considered judgments? Answer: {AE} should suspend judgment.
For that is the only stable state of belief available to him, since any
other state undermines itself. For example, if {AE} were at all
confident that he should turn left, that confidence would itself be
evidence that he should not turn left. In other words, {AE} should
realize that, were he to form strong navigational opinions, those
opinions would tend to be mistaken. Realizing this, he should refrain
from forming strong navigational opinions (and should outsource his
navigational decision-making to someone else whenever possible).
~(\citeproc{ref-EganElga2005}{Egan and Elga 2005, 82--83})
\end{quote}

I will argue that this reasoning goes through iff JSE is assumed. I'll
argue for this by first showing how the reasoning could fail without
JSE, and then showing how JSE could fix the argument.

Start with a slightly different case. {Katell} is trying to find out
whether \emph{p}, where this is something she knows little about. She
asks ten people whether \emph{p} is true, each of them being someone she
has good reason to believe is an expert. The experts have a chance to
consult before talking to her, so each of them knows what the others
will advise. Nine of them confidently assure her that \emph{p} is true.
The tenth is somewhat equivocal, but says that he suspects \emph{p} is
not true, although he cannot offer any reasons for this suspicion that
the other nine have not considered. It seems plausible in such a case
that she should, or at least may, accept the supermajority's verdict,
and believe \emph{p}.

Now vary the case. The first nine are experts, but the tenth is an
anti-expert. He is wrong considerably more often than not. Again, the
first nine confidently assert that \emph{p}, but now the tenth says the
same thing, i.e., \emph{p}. This doesn't change {Katell}'s epistemic
situation. She has a lot of evidence for \emph{p}, and a little evidence
against it. The evidence against has changed; it is now the confident
verdict of an anti-expert, rather than the equivocal anti-verdict of an
expert, but this doesn't matter. So she still should, or at least may,
believe \emph{p}.

Now make one final variation. {Katell} is the tenth person consulted.
She asks the first nine people, who of course all know each other's
work, and they all say \emph{p}. She knows that she has a tendency to
make a wrong judgment in this type of situation -- even when she has had
a chance to consult with experts. Perhaps \emph{p} is the proposition
that the correct road is to the left, and she is {AE}, for example. It
does require some amount of hubris to continue to be an anti-expert even
once you know you are one, and the contra-indicating judgments are made
in the presence of expert advice. But I don't think positing
delusionally narcissistic agents makes the case unrealistic. After
listening to the experts, she judges that \emph{p}. This is some
evidence that ¬\emph{p} , since she is an anti-expert. But, as in the
last two paragraphs, it doesn't seem that it must override all the other
evidence she has. So, even if she knows that in general she is fairly
anti-reliable on questions like \emph{p}, she need not suspend judgment.
Even if her judgment is some evidence that ¬\emph{p} , it might not be
strong enough to defeat her earlier evidence for \emph{p}. On those
(presumably rare) occasions where her judgment tracks the evidence, the
evidence may be strong enough for me to keep it, even once she
acknowledges she have made the judgment.

The previous paragraph assumed that JSE did not hold. It assumed that
{Katell} could still rely on the nine experts, even once she had
incorporated their testimony into a judgment. That's what JSE denies.
According to JSE, the arguments of the previous paragraph rely on
illicitly basing belief on screened-off evidence. That's bad. If JSE
holds, then once {Katell} makes a judgment, it's all the evidence she
has. Now assume JSE is true, and that {Katell} knows herself to be
something of an anti-expert. Then any judgment she makes is fatally
self-undermining, just like Egan and Elga say. When she makes a
judgment, she not only has evidence it is false, she has undefeated
evidence it is false. So if {Katell} knows she is an anti-expert, she
must suspend judgment. That's the conclusion Egan and Elga draw, and it
seems to be the right conclusion iff JSE is true. So the argument here
relies on JSE.

\subsection{White on Permissiveness}\label{whiteonpermissiveness}

Roger White (\citeproc{ref-White2005}{2005}) argues that there cannot be
a case where it could be epistemically rational, on evidence \emph{E},
to believe \emph{p}, and also rational, on the same evidence, to believe
¬\emph{p} . One of the central arguments in that paper is an analogy
between two cases.

\begin{quote}
\textbf{Random Belief}\\
S is given a pill which will lead to her forming a belief about
\emph{p}. There is a ½ chance it will lead to the true belief, and a ½
chance it will lead to the false belief. S takes the pill, forms the
belief, a belief that \emph{p} as it turns out, and then, on reflecting
on how she formed the belief, maintains that belief.
\end{quote}

\begin{quote}
\textbf{Competing Rationalities}\\
S is told, before she looks at \emph{E}, that some rational people form
the belief that \emph{p} on the basis of \emph{E}, and others form the
belief that ¬\emph{p} on the basis of \emph{E}. S then looks at \emph{E}
and, on that basis, forms the belief that \emph{p}.
\end{quote}

White claims that S is no better off in the second case than in the
former. As he says,

\begin{quote}
Supposing this is so, is there any advantage, from the point of view of
pursuing the truth, in carefully weighing the evidence to draw a
conclusion, rather than just taking a belief-inducing pill? Surely I
have no better chance of forming a true belief either way.
~(\citeproc{ref-White2005}{White 2005, 448})
\end{quote}

There are two ways to read the phrase ``from the point of view of
pursuing the truth''. One of them leads to an implausible view about the
role of rational reflection in inquiry. The other makes the argument
rely on JSE. Take these in order.

First, assume White's narrator is only concerned about having a truthful
opinion right now, and only having a truthful opinion on this very
question. Given that, it will be true that the belief-inducing pill will
do just as well as careful weighing of the evidence. But that's a very
unusual set of interests to have, and it's not clear why we should take
such a person to show us much of interest about the point of reflection.
One generally good reason for weighing the evidence carefully is that it
puts us in a better position to be able to process new evidence as it
comes in. It isn't clear how White's narrator, who takes the
belief-inducing pill, will be able to adjust to new evidence, since by
hypothesis he doesn't have any sense of how well entrenched this belief
should be, and how sensitive it should be to counterveiling evidence.
This point is closely related to the explanation Socrates gives for the
superiority of knowledge to mere true belief in \emph{Meno} 97d-98a.

Another good reason for weighing evidence carefully is that we learn
about other propositions through this process. Assume we're trying to
figure out whether \emph{p}, and there is some other proposition
\emph{q}, such that (a) we care about whether \emph{q} is true, and (b)
\emph{p} is sometimes, but not always, good evidence for \emph{q}. It is
very common that at least some such proposition exists. Then figuring
out why \emph{p} is true, or at least why we should think it is true,
will be relevant for \emph{q}. So an agent who only cares about having
at this very moment a true belief about this very proposition might be
no better off engaging in rational reflection than taking White's
belief-inducting pill, but such agents are far removed from the usual
situation we find ourselves in, and not good guides to epistemological
generalisation.

But note that with JSE we don't need to restrict attention to such
narrowly-defined agents. Assume that JSE is true. Then after S evaluates
\emph{E}, she forms a judgment, and \emph{J} is the proposition that she
formed that judgment. Now it might be true that \emph{E} itself is good
evidence for \emph{p}. (The target of White's critique says that
\emph{E} is also good evidence for ¬\emph{p} , but that's not yet
relevant.) But given JSE, that fact isn't relevant to S's current state.
For her evidence is, in its entirety, \emph{J}. And she knows that, as a
rational agent, she could just as easily have formed some other
judgment, in which case \emph{J} would have been false. Indeed, she
could have formed the opposite judgment. So \emph{J} is no evidence at
all, and she is just like the person who forms a random belief,
contradicting the assumption that believing \emph{p} could, in this
case, be rational, and that believing ¬\emph{p} could be rational.

Without JSE, White's analogy breaks down. Forming a belief via a pill,
and forming a belief on the basis of the evidence, are very different.
That's true even if you know that other rational agents take the
evidence to support a different conclusion. The random belief is
incapable of being properly updated, or of supporting the correct
strands elsewhere in the web of belief.

If we care about getting at the truth in general, and not just about
\emph{p}, then White's analogy needs JSE to go through. And we should,
and do, care about truth in general. So this argument against
permissiveness needs JSE. There may be other arguments against
permissiveness, so this isn't to say that White's conclusion requires
JSE. But his argument does.

\subsection{Disagreement and Priority}\label{disagreementandpriority}

Here is Adam Elga's version of the Equal Weight View of peer
disagreement, a theory we will discuss much more in chapter 12.

\begin{quote}
Upon finding out that an advisor disagrees, your probability that you
are right should equal your prior conditional probability that you would
be right. Prior to what? Prior to your thinking through the disputed
issue, and finding out what the advisor thinks of it. Conditional on
what? On whatever you have learned about the circumstances of the
disagreement. ~(\citeproc{ref-Elga2007}{Elga 2007, 490})
\end{quote}

It is easy to see how JSE could help defend this view. First, focus on
the role JSE can play in the clause about priority. Here is one kind of
situation that Elga wants to rule out. S has some evidence \emph{E} that
she takes to be good evidence for \emph{p}. She thinks T is an epistemic
peer. She then learns that T, whose evidence is also \emph{E}, has
concluded ¬\emph{p} . She decides, simply on that basis, that T must not
be an epistemic peer, because T has got this case wrong. This decision
violates the Equal Weight View, because it uses S's probability that T
is a peer after thinking through the disputed issue, not prior to it, in
deciding who is more likely to be right.

Now at first it might seem that S isn't doing anything wrong here. If
she knows how to apply \emph{E} properly, and can see that T is
misapplying it, then she has good reason to think that T isn't really an
epistemic peer after all. She may have thought previously that T was a
peer, indeed she may have had good reason to think that. But she now has
excellent evidence, gained from thinking through this very case, to
think that T is not a peer and so not worthy of deference.

Since Elga thinks that there is something wrong with this line of
reasoning, there must be some way to block it. The best option for
blocking it comes from ruling that \emph{E} is not available evidence
for S once she is using \emph{J} as a judgment. That is, the best block
available comes from JSE. Once we have JSE in place, we can say what S
does wrong. She is like the detective who says that we have lots of
evidence that the suspect could have committed the crime--not only does
he live in Cornwall, but he has Cornish sand in his clothes. To make the
cases more analogous, we might imagine that there are detectives with
competing theories about who could be guilty in this case. If we don't
know who was even in Cornwall, then the evidence about the sand may
favour one detective's theory over the other. If we do know that both
suspects live in Cornwall, then the evidence about the sand isn't much
help to either.

So JSE supports Elga's strong version of the Equal Weight View, which
bars agents from using the dispute at issue as evidence concerning the
peerhood of another. And if JSE is not true, then there is a kind of
simple and natural reasoning which undermines Elga's Equal Weight View.
So Elga's version of the Equal Weight View requires JSE.

\section{JSE and Higher Order Evidence}\label{jseandhigherorderevidence}

As noted above, JSE can also support the higher order hypothesis. The
idea is reasonably simple. Assume that an agent gets evidence that is in
fact good evidence for \emph{p}, concludes \emph{p} on that basis, but
also has reason to think they are in a sub-optimal epistemic
environment. The believer in higher-order evidence thinks the agent
should then lower their confidence in \emph{p}. But why is that, when
they already have excellent evidence for \emph{p}, and the evidence
about the environment doesn't seem to defeat that?

Let's make that last rhetorical question a little clearer. {Danail}
tells {Milica} that \emph{p}. {Milica} has a long relationship with
{Danail}, and he has been a very reliable testifier over that time. And
{Milica} has no reason to doubt that \emph{p}. But then {Milica} learns
she has taken a drug that makes most people very unreliable when it
comes to processing evidence by testimony. Should this last evidence
reduce her confidence in \emph{p}, by somehow defeating the support that
{Danail}'s testimony provides? The evidence about the drug isn't a
rebutting defeater; it provides no reason to think \emph{p} is false.
But nor is it the most natural kind of undercutting defeater. It
provides no reason to think that {Danail} is an unreliable testifier.
What it does is undercut any support that {Milica}'s own judgment gives
to \emph{p}. But that only matters to what {Milica} should believe if
that judgment is playing an important role in sustaining her belief. And
that's where JSE comes in. Unless JSE is true, {Milica} has a completely
sound reason to believe \emph{p}, namely {Danail}'s testimony. And that
reason isn't defeated by the drug. If a third party believed \emph{p}
because they knew that {Milica} believed it on testimonial grounds, then
the drug would be an undercutting defeater to the third party's belief.
But to make it a defeater to {Milica}'s belief, we need to assume that
{Milica}, like the third party, in some way bases her sustained belief
on her judgment. If JSE is right, then in a good sense she does do that;
her own judgment that \emph{p} is her only unscreened evidence, and if
the force of it is defeated, then she has no good reason to believe
\emph{p}. If JSE is wrong, it is harder to see the parallel between
{Milica} and the third party.

I've sketched an argument that the higher order hypothesis not just
could be supported by JSE, but would be undermined if JSE were false.
And JSE is indeed false, as we'll now show. We'll return at the end of
the chapter to whether this fact can be turned into an argument against
the higher order hypothesis.

\section{The Regress Objection}\label{theregressobjection}

{Ariella} is trying to make a forecast for how well her hometown team,
the Detroit Tigers, will do in the upcoming baseball season. Baseball
teams play 162 games\footnote{In reality they sometimes play 1 or 2 more
  or less. It will simplify the exposition to assume it is known in
  advance in Ariella's world that they play 162 exactly, and that's what
  I will assume.}, and the Tigers look like being a relatively mediocre
team. She knows that it is irrational to form any belief about precisely
how many games the Tigers will win. But she thinks, correctly as it
turns out, that it is reasonable to form a credal distribution over
propositions of the form \emph{The Tigers will win n games}, and have
that distribution be roughly a normal distribution with a standard
deviation of 5 games. The question is to work out what the most likely
win total is, which will be both the mode and the mean of the
distribution. For simplicity, we'll say that for her to \emph{predict}
that The Tigers will win \emph{n} games is to set \emph{n} to be this
centre. (I don't mean to suggest this is the ordinary use of the English
word `predict'. The definition I'm using here is stipulative.)

{Ariella} works through the known facts about the Tigers and their
opponents, and predicts that they will win 76 games. This is, as it
turns out, exactly the right prediction to make. That isn't to say the
Tigers will actually win 76 games - remember the point here is not to
form outright beliefs. Rather, the appropriate credal distribution over
propositions about the Tigers' win total, given {Ariella}'s evidence, is
centred on 76.

But {Ariella} knows something about herself. She knows that in general,
when she settles on a prediction, it is 1 game too low when it comes to
the Tigers. If someone else knew nothing other than that {Ariella} had
predicted the Tigers would win 76 games, and {Ariella}'s track record,
the rational thing for them to do would be to predict the Tigers will
win 77 games. So {Ariella} has higher-order evidence that one might
think will move her to change her prediction from 76 to 77.

Note carefully though what {Ariella} knows about herself. She knows that
it is when she settles on a prediction that it is on average 1 game too
low. If she decides that 76 wasn't a settled prediction, but 77 is, then
she has exactly the same reason to raise her prediction to 78. And if
she settles on that, she has a reason to raise her prediction to 79, and
so on. Higher-order evidence is an issue because someone can have
evidence that they make systematic mistakes in forming beliefs on the
basis of evidence. But those systematic mistakes could also concern how
they form beliefs on the basis of higher-order evidence. Indeed, they
could be the same systematic mistakes in both cases. What should be
done?

Let's start with three very bad ideas for {Ariella}. She should not
simply follow the higher-order evidence where it leads, first raising
her prediction to 76, then 77, then 78 and so on. After 87 steps, she
will predict that the Tigers will win 163 games. Given that it is a 162
game season, this is not a good idea. Nor should she follow through as
many steps of higher-order reasoning as she has the cognitive capacity
to do. Assuming she has the ability to add 1 repeatedly, that will lead
to the same flaw as above. And nor should she simply get out of the
business of making predictions about baseball. (Compare Egan and Elga's
comment that AE should simply stop making judgments about where to turn;
a comment that was about one particular case of course, and not a
general piece of advice.) Given what I've said so far about {Ariella},
she's really good at these kind of predictions. Having a small
systematic error like this is not that much of a flaw, given how good
she otherwise is.

There are three other strategies for dealing with higher-order evidence
that are at least plausible. The first is the one I will defend. It is
that {Ariella} should simply stick with her original prediction because
it is the best prediction to make given her evidence. The second is that
she should find some equilibrium point, where the higher-order evidence
does not recommend a change of view. As stated, this view won't say
anything about what {Ariella} should do, because there is no equilibrium
position. But perhaps the view could be extended to say that she should
follow the first-order evidence if there is no equilibrium, so it will
also say that she should stick with her original prediction. The third
option is that {Ariella} should follow one step of higher-order
evidence, then stop with the prediction that the Tigers will win 77
games. I'll argue for the first option by arguing against the other two.

Start with the idea that {Ariella} should, if possible, settle on an
equilibrium. The idea is that we avoid the regress by saying that when
possible, rational agents should be such that when they add the fact
that they made that judgment to their evidence, the rational judgment to
make given the new evidence has the same content as the original
judgment. So if one is rational, and predicts that \emph{p}, the
rational prediction given that one has made the prediction that \emph{p}
is still \emph{p}.

Note that this isn't as strong a requirement as it may first seem. The
requirement is not that any time an agent makes a judgment (or
prediction), rationality requires that they say on reflection that it is
the correct judgment. Rather, the requirement is that when possible,
rational agents make those judgments that, on reflection, they would
reflectively endorse. We can think of this as a kind of ratifiability
constraint on judgment, like the ratifiability constraint on decision
making that Richard Jeffrey (\citeproc{ref-Jeffrey1983}{1983}) uses to
handle Newcomb cases.

A judgment is ratifiable for agent S just in case the rational judgment
for S to make conditional on her having made that judgment has the same
strength and content as the original judgment. The regress is blocked by
saying rational agents make ratifiable judgments when possible. If the
agent does do that, there isn't much of a problem with the regress; once
she gets to the first level, she has a stable view, even once she
reflects on it.

This assumption, that only ratifiable judgments are rational, drives
much of the argumentation in Egan and Elga's paper on self-confidence;
it is a serious option. As the comparison to Jeffrey suggests, it has
some historical pedigree. And though this would take much longer to
show, it is probably the best way to make sense of the emphasis on
equilibrium concepts in game theory. Nevertheless it is false. I'll
first note one puzzling feature of the view, then one clearly false
implication of the view.

The puzzling feature is that, as we have already seen, there need not be
any ratifiable judgment to make. So the view will be somewhat
incomplete. But maybe that isn't such a bad thing. We imagine the
ratifiability theorist saying the following two things. (This isn't the
only way to extend the ratifiability view, but I won't be objecting to
this extension.)

\begin{enumerate}
\def\labelenumi{\arabic{enumi}.}
\tightlist
\item
  It is important to make ratifiable judgments. Any judgment that is not
  ratifiable is not rational.
\item
  It is better, other things being equal, to have judgments that track
  the evidence.
\end{enumerate}

This view will say that {Ariella} faces an epistemic dilemma. Anything
she does will be to some extent irrational, since it will not be
ratifiable. But the least bad option will be to predict that the Tigers
will win 76 games, as she does. If you think epistemic dilemmas are
impossible, you won't like this way of thinking about {Ariella}. But I
don't think the arguments against epistemic dilemmas are particularly
strong. If this was the worst thing to say about the ratifiability view
then it would look like a reasonable view.

But it isn't the worst thing to say about the ratifiability view. The
problem arises in cases where there is a ratifiable judgment. Change the
case a little so {Ariella} doesn't tend to overpredict Tigers losses by
1 game; she tends to overpredict them by 1\%. So if she predicts the
Tigers will lose 86 games, an outsider going off that prediction and her
track record wouldn't predict the Tigers will lose 85 games, they will
predict the Tigers will lose 85.14 games. (Remember given the stipulated
meaning of `predict' we're using here, it can be perfectly sensible to
predict that teams will win a fractional number of games. Indeed, there
is no particular reason to think that the centre of the credal
distribution over Tiger wins will fall on an integer. Remember also that
there are 162 games in a season, so predicting 76 wins just is
predicting 86 losses.)

Changing the expected error from a game to a percent doesn't seem like a
big change at first blush. But now there is a ratifiable prediction for
{Ariella}. It is that the Tigers will win 162 games, and lose 0. So if
we think {Ariella} should make ratifiable predictions where possible, we
should conclude that whatever her evidence about the Tigers hitting,
pitching and fielding, she should predict they will win all 162 games in
the season. This can't be right.

This kind of case proves that it isn't always rational to have
ratifiable credences. It would take us too far afield to discuss this in
detail, but it is interesting to think about the comparison between the
kind of case I just discussed, and the objections to backwards induction
reasoning in decision problems that have been made by Pettit and Sugden
(\citeproc{ref-PettitSugden1989}{1989}), and by Stalnaker
(\citeproc{ref-Stalnaker1998}{1998}). The backwards induction reasoning
they criticise is a development of the idea that judgments should be
ratifiable. And the clearest examples of when that idea fails are cases
where there is a unique ratifiable judgment, and it is a judgment that
first order considerations tell strongly against. The example of
{Ariella} has, quite intentionally, a similar structure.

The other option for blocking the regress is to say that there is
something special about the first revision. So if {Ariella} predicts
that the Tigers will win 76 games, that screens her evidence about the
Tigers' hitting, pitching and fielding. But if she changes her mind and
predicts that they will win 77 games, on the basis of the higher order
evidence, that doesn't screen her original prediction that they will win
76. So the regress doesn't even get started. This is structurally
similar to a move that Adam Elga (\citeproc{ref-Elga2010}{2010}) makes
about disagreement. He argues that we should adjust our views about
first-order matters in (partial) deference to our peers, but we
shouldn't adjust our views about the right response to disagreement in
the same way.

It's hard to see what could motivate such a position, either about
disagreement or about screening. It's true that we need some kind of
stopping point to avoid these regresses. But the most natural stopping
point is before the first revision. Consider a toy example. It's common
knowledge that there are two apples and two oranges in the basket, and
no other fruit. (And that no apple is an orange.) Two people disagree
about how many pieces of fruit there are in the basket. A thinks that
there are four, B thinks that there are five, and both of them are
equally confident. Two other people, C and D, disagree about what A and
B should do in the face of this disagreement. All four people regard
each other as peers. Let's say C's position is the correct one (whatever
that is) and D's position is incorrect. Elga's position is that A should
partially defer to B, but C should not defer to D. This is, intuitively,
just back to front. A has evidence that immediately and obviously
entails the correctness of her position. C is making a complicated
judgment about a philosophical question where there are plausible and
intricate arguments on each side. The position C is in is much more like
the kind of case where experience suggests a measure of modesty and
deference can lead us away from foolish errors. If anyone should be
sticking to their guns here, it is A, not C.

The same thing happens when it comes to screening. Remove B from the
example and instead assume that A has some evidence that (a) she has
made some mistakes on simple sums in the past, but (b) tends to
massively over-estimate the likelihood that she's made a mistake on any
given puzzle. What should she do? One option, in my view the correct
one, is that she should believe that there are four pieces of fruit in
the basket, because that's what the evidence obviously entails. Another
option is that she should be not very confident there are four pieces of
fruit in the basket, because she makes mistakes on these kinds of sums.
Yet another option is that she should be pretty confident (if not
completely certain) that there are four pieces of fruit in the basket,
because if she were not very confident about this, this would just be a
manifestation of her over-estimation of her tendency to err. The
`solution' to the regress we're considering here says that the second of
these three reactions is the uniquely rational reaction. The idea behind
the solution is that we should respond to the evidence provided by
first-order judgments, and correct that judgment for our known biases,
but that we shouldn't in turn correct for the flaws in our
self-correcting routine. I don't see what could motivate such a
position. Either we just rationally respond to the evidence, and in this
case just believe there are four pieces of fruit in the basket, or we
keep correcting for errors we make in any judgment and start a regress.

\section{Laundering}\label{laundering}

In the definition of JSE, I said it was restricted to rational
judgments. This was to avoid a simple counterexample to the view. (I'm
indebted here to Vann McGee for pointing out the need for this.) {Vieno}
is usually a pretty reliable judge, and he's not currently drunk or
otherwise incapacitated. But he makes a mistake, as we all do sometimes,
and forms the belief that \emph{p} on the basis of massively
insufficient evidence. This is rather irrational. Again, that's not to
say that {Vieno} himself is irrational, but he does have a particular
irrational view.

Now assume that JSE were true in an unrestricted form. {Vieno} is a
generally reliable judge. That he believes \emph{p} is, on its own,
pretty good evidence for \emph{p}. If the underlying evidence \emph{E}
is screened off, then arguably the overall evidence does suggest that
\emph{p}, so {Vieno}'s belief does track his evidence after all. More
generally, if unrestricted JSE is right, then it is impossible for
someone who knows themselves to be generally reliable to have an
irrational belief. So unrestricted JSE must be wrong.

But even if we restrict JSE to rational judgments, some problems remain.
For one thing, we need some explanation of why such a restricted thesis
should be true. That is, we need an explanation of why JSE should be
extensionally adequate in just the cases where it agrees with ESJ. The
normative externalist, who believes in ESJ, has a simple explantion for
that. JSE is extensionally adequate when and only when it agrees with
ESJ because ESJ is generally true. It isn't clear what could be a
similarly good explanation of why a restricted version of JSE holds.

Thinking through cases like {Vieno}'s can help motivate ESJ, and
normative externalism more generally. There is something very strange
about his case. On the one hand, the fact that a reliable person like
{Vieno} believes that \emph{p} should be some evidence for \emph{p}. On
the other hand, if {Vieno} still knows why he believes that \emph{p},
still knows that is that \emph{E} is the relevant evidence on which the
belief was based, then believing that \emph{p} still seems irrational.
And that's despite his knowing one important piece of evidence in favour
of \emph{p}, namely that he himself believes it.

It's important to distinguish the claims I've made in the last paragraph
from what Gilbert G. Harman (\citeproc{ref-Harman1986}{1986}) says about
a slightly different case. Imagine that a month later, {Vieno} has
forgotten the evidence that led to the belief that \emph{p}, but
nevertheless believes \emph{p}. There are two interesting variants of
this example. In one, \emph{p} has been stored in preservative memory
over that time. In the second, {Vieno} bases a new belief that \emph{p}
on the memory of believing \emph{p} a month ago, plus his general
reliability. If {Vieno} was under no obligation to retain the evidence
for \emph{p}, then it is plausible in the second case that the new
belief that \emph{p} is rational. And if the belief is rational in that
case, maybe it is rational in the case where \emph{p} was stored in
preservative memory too.

We've already discussed memory in some detail. Here i want to
distinguish the following two kinds of cases. In one, {Vieno} has an
apparent memory that \emph{p}. In the other, he has a clear memory that
\emph{E}, and irrationally infers \emph{p} from that. In the second,
{Vieno}'s belief is irrational. But it is a mystery why this should be
so, since he has this excellent evidence for \emph{p}, from his own
track record of success. ESJ explains this nicely, since that evidence
is screened off. So the case of {Vieno} is both a problem for JSE, and a
boon for ESJ. The case shows that JSE needs to be restricted, but it is
hard to motivate any particular restriction. And ESJ offers a nice
explanation of a puzzling fact, namely why {Vieno}'s track record is not
in this case evidence for \emph{p}.

Now ESJ is a strongly externalist thesis. It says that facts about one's
own judgment are not evidentially relevant to what judgment one makes,
provided one has access to the evidence behind that judgment. And that
suggests that the judgment should really just be judged on how it tracks
the evidence, which is what the externalist says.

This point about laundering also offers a nice reply to a worry that I
shouldn't have drawn a commitment to JSE from the passages I quoted
above from Egan, Elga, White and Christensen. Perhaps they are only
committed to a weaker thesis, something like that JSE is true when
mistakes have been made, or when the agent has good reason to believe
mistakes have been made. I didn't attribute such qualified theses to
these epistemologists because the qualifications seem to make the
theories worse. The qualified theories are still vulnerable to the
regress arguments that we drew out of the examples involving {Ariella}.
And the point about laundering shows that JSE is most plausible when it
is restricted to cases where mistakes have not been made.

{Ariella}'s example doesn't just show that JSE is wrong. It gives us an
extra reason to doubt the higher order hypothesis. If that hypothesis is
true, then whatever prediction {Ariella} makes, she should raise her
prediction as soon as she realises that she has made it. But that isn't
plausible, since it leads from a reasonable starting point, a prediction
of 76 wins, to an incoherent conclusion. So the higher order hypothesis
is false, and the challenge it poses to normative externalism does not
succeed.

\section{Agents, States and Actions}\label{agentsstatesandactions}

With this discussion of regresses completed, we are in a position to
evaluate an interesting alternative to my account of cases like
{Riika}'s. The alternative I'll discuss here says that if {Riika} does
nothing in response to learning the higher order evidence, her resultant
belief is perfectly acceptable, but this shows something bad about her.
I'm going to first motivate such an alternative view, then suggest that
the regresses we've discussed in this chapter pose a problem for it.

My account of {Riika}'s example is somewhat conciliatory. I say it could
be right for {Riika} to change her credences, depending on just how the
case is filled in. But there is much to be said for the less
conciliatory view that the only rational belief for {Riika} to have is
the one she started with. After all, that's what her evidence supported,
and she didn't get any counter-evidence. So how do we explain the
intuition that it would be bad to not change her mind? By postulating a
break between the evaluations of {Riika}'s beliefs, on the one hand, and
the evaluation of her actions, or of her, on the other.

It will help to have some slightly stipulative language available to
discuss the cases. When agent S forms the belief that \emph{p}, we can
evaluate that belief, and the formation of it, in a number of distinct
ways. First, we can ask whether the belief is well supported by her
evidence. Let's say that the belief is \emph{evident} if so, and not
evident if not. Second, we can ask whether the belief is supported by
the totality of her reasons to believe. Let's say that the belief is
\emph{rational} if so, and irrational if not. Third, we can ask whether
an epistemically virtuous agent would have formed that belief. Let's say
that the agent is \emph{wise} if she is so virtuous at the time the
belief is formed, and unwise if she is not.\footnote{In chapter 6 I
  noted that I'm using `wise' for this kind of evaluation of agents,
  mostly following Maria
  (\citeproc{ref-Lasonen-Aarnio2010b}{Lasonen-Aarnio 2010b},
  \citeproc{ref-Lasonen-Aarnio2014}{2014a}), though changing the
  terminology slightly.} Fourth, and last for now, we can ask whether
the practice the agent follows when forming the belief is one that she
ought, all things considered, be following. Let's say her practice is
\emph{advisable} if so, and inadvisable if not.

What's crucial to evidentialism, as I conceive of it, is that the
evident and the rational coincide. It does not commit itself on whether
following the evidence is what wise agents do, or whether following the
evidence is always advisable.

Just as we can make this four-way distinction among beliefs, we can make
a similar four-way distinction among actions. An agent looks at the
evidence in favour of different decisions, and then takes a decision.
We'll assume, to simplify matters, that the agent has decent values in
this process, so what's at issue is how the agent's doxastic system
interacts with decisions to act. So we can describe actions as evident,
rational, wise and advisable, with these terms having the same meanings
as above.

With all these distinctions in mind, we can take another look at the
cases that motivate higher-order theories. Consider, for instance, Adam
Elga's example of the pilot who has evidence that it is possible they
are suffering from hypoxia ~(\citeproc{ref-Elga2008}{Elga 2008}). Is it
obvious that it is irrational for them to believe that they have enough
fuel for the trip, as their evidence supports?

Well, it does seem inadvisable for them to act as if they had enough
fuel. But to get from premises about the the inadvisability of action to
conclusions about the irrationality of belief requires a lot of steps.
We could imagine reasoning as follows.

\begin{enumerate}
\def\labelenumi{\arabic{enumi}.}
\tightlist
\item
  It is inadvisable to act as if one had enough fuel.
\item
  So, it is inadvisable to believe one has enough fuel.
\item
  So, it is unwise to believe one has enough fuel.
\item
  So, it is irrational to believe one has enough fuel.
\end{enumerate}

Put this bluntly, every step seems questionable. There could be distinct
norms of action and belief. There could be distinct norms of advice and
evaluation. And there could be distinct norms that apply at the level of
agents to those that apply at the level of individual beliefs. Let's
look at these in order.

Once we see that there are a lot of distinct ways we can think the pilot
goes wrong, it is wrong to insist that it is simply intuitive that the
pilot has as irrational belief. The intuition is that something has gone
wrong with the pilot; what in particular has gone wrong is a matter for
theory. And perhaps what is being intuited is not anything at all about
belief, but something about action. Perhaps it would be bad in some way
to act on one's evidence, even if it would be rational to believe based
on that evidence.

Allan Coates (\citeproc{ref-Coates2012}{2012}) has developed a form of
this response to the examples that motivate internalist accounts of
higher-order evidence. It isn't just critics like Coates who have
reacted in this way. Here is David Christensen making an argument that
higher-order evidence matters to the rationality of belief.

\begin{quote}
If you doubt that my confidence should be reduced, ask yourself whether
I'd be reasonable in betting heavily on the correctness of my answer. Or
consider the variant where my conclusion concerns the drug dosage for a
critical patient, and ask yourself if it would be morally acceptable for
me to write the prescription without getting someone else to corroborate
my judgment. Insofar as I'm morally obliged to corroborate, it's because
the information about my being drugged should lower my confidence in my
conclusion. ~(\citeproc{ref-Christensen2010a}{Christensen 2010a, 195})
\end{quote}

The thought for now is that the last line of this quote is simply false.
There are all sorts of reasons it might be morally obligatory to
corroborate even if the information about being drugged should not lower
one's confidence. It's true that some forms of consequentialism about
decision making will say that if confidence is not lowered, decisions
should not change. But it is not at all compulsory to take a
consequentialist attitude towards medical ethics. (Compare what Maria
Lasonen-Aarnio (\citeproc{ref-Lasonen-Aarnio2014a}{2014b, 430}) says
about rules governing the police.) And even if one is broadly
consequentialist, Christensen's conclusion still does not
straightforwardly follow.

We should take seriously the possibility that this is a case where
agents should not change their credences, but should change how they
act. Now that will be incoherent if you think that one should always
maximise expected utility. But let's consider the possibility that this
is a case where maximising expected utility is not the thing to do. It's
a striking fact that the standard arguments for the propriety of
maximising expected utility are almost always question-begging against
the most interesting opponents ~(\citeproc{ref-Maher1997}{Maher 1997};
\citeproc{ref-Weatherson1999}{Weatherson 1999}). Imagine a theorist who
says that the right thing to do is to maximise expected expected
utility, and run your favorite argument for the properiety of maximising
expected utility against them. In most cases you'll find at some stage
you're just begging the question against them. Consider, for instance,
arguments based on representation theorems. These typically include as a
premise that if the agent is choosing between two bets, and they have
the same cost and same payoff, she should choose the bet that is the
more probable winner. But this is just to assume that, in a special
range of cases, she should maximise expected utility rather than
expected expected utility, or anything else, and that assumption is, in
this context, question-begging.

I don't mean this to be a serious argument against the view that we
should maximise expected utility. Sometimes the best arguments for true
positions are question-begging ~(\citeproc{ref-Lewis1982c}{Lewis 1982}).
And a whole chapter of this book defends the claim that we can learn
from circular arguments. Indeed I believe for independent reasons that
we should maximise expected utility. But I do think it is worth thinking
about the fact that the relevant intuitions about higher-order evidence
seem in the first place to be intuitions about actions, and require some
substantive assumptions to generate conclusions about beliefs.

After all, if {Riika} should maximise expected expected utility, then
she should order more tests, or get someone to confirm her diagnosis,
before she acts. And that is true even if she actually has good evidence
that the patient has dengue fever, as long as she lacks good evidence
that she has good evidence. And perhaps that is what we are intuiting
when we intuit that she should not act. The intuitions about the case,
then are intuitions about action, but they don't imply anything about
belief without a substantive theory of the action-belief connection
(i.e., that one should maximise expected utility), and that theory lacks
independent support.

This is a way of debunking the intuitions Christensen endorses about
{Riika}`s case. (And as noted many times, many other theorists have
similar intuitions to Christensen's about similar cases.) As it stands,
I don't accept this debunking story, because I accept the 'substantive
theory of the action-belief connection', but this is a commitment that
goes beyond normative externalism, and the rejection of level-crossing
principles.

Let's assume that that bridge has been crossed though, and we have
reason (either intuition or argument) to believe it would be inadvisable
for {Riika} to believe her patient has dengue fever. What follows?
Nothing much, unless we assume a very tight connection between
assessments of agents and assessments of states, or between assessments
of strategies and assessments of states. And there are very good reasons
to separating these assessments. Maria Lasonen-Aarnio
(\citeproc{ref-Lasonen-Aarnio2010b}{2010b},
\citeproc{ref-Lasonen-Aarnio2014}{2014a}) has argued for separating
agent assessment from state assessment, and argued that the standard
intuition here involves conflating the two. And John Hawthorne and Amia
Srinivasan have argued for separating assessment of states from
assessment of strategies for coming to those states
~(\citeproc{ref-HawthorneSrinivasan2013}{Hawthorne and Srinivasan
2013}).

Hawthorne and Srinivasan's argument is that these assessments come apart
in general, so we should not be surprised if they come apart here. In
general, it makes sense to distinguish between what someone should do in
a particular circumstance, and what the person would do if they had
instilled the habits that would be most effective in the long run. They
give an example from sports. Their example involves tennis, but the idea
generalises. Given the range of possible installable habits, it might be
that the best habit to instil is one that will lead to expending
valuable energy on occasionally chasing after lost causes. They are
particularly interested in an epistemic limit on possible habits; the
fact that we don't always know what we know means that we can't always
react perfectly to our knowledge. But there are many possible
limitations on possible habits due to our physical and cognitive
limitations. And any one of these limitations will produce a gap between
the optimal thing to do in a situation given one's knowledge or
evidence, and what would be done if one had installed the optimal
habits.

Now it may well be that the best habits we could have, given our
cognitive nature, would involve second-guessing ourselves in cases like
{Riika}`s. Certainly if we think that our instincts involve some kind of
'optimism bias' ~(\citeproc{ref-Sharot2012}{Sharot 2012}), then it will
be advisable to instil habits to counteract that bias. And it is very
plausible that the fact that someone did something because they were
acting on the best habit they could have is largely excusing. (I would
say it is completely excusing, but I'm a little sceptical that there are
complete excuses.) It seems plausible that our norms of advice are tied
more closely to the idea of what the best habits are to instil, rather
than to what is best to do in each situation, so the thing to advise
someone to do just is what they would do if they had the best possible
habits.

But all these facts should not obscure the fact that these are all
second-best situations. Our cognitive and physical limitations mean that
we sometimes cannot do what we should. That's why they are called
limitations. So there are cases where the best thing to believe is what
the evidence supports, but it is understandable and excusable to regard
the matter as unsettled. And the grounds for the excuse are that agent
has the optimal habit for situations like this. But as theorists we
should not ignore the fact that optimising habits is a second-best
solution. Best would be to believe and confidently act. And it would be
best to believe and act not because this would be a lucky guess, but
because one has sufficient reason to act.

So why didn't I just say all this in Chapters 7 through 9 rather than
going through long detours about evidence and circularity? One reason is
that we still need to explain the distinction between {Riika}'s case and
Raina's, and I'm not sure going via thoughts about advisability, wisdom
or action will help with that. (This isn't a coy way of saying I think
it won't; it's just that I haven't yet worked out a way to make it
help.)

But a bigger reason is that we need to avoid the regresses. And the
regresses suggest that policies like \emph{Adjust one's credences to the
higher-order evidence} are actually not optimal habits. That would be a
bad habit for {Ariella} to adopt. And it would be bad to advise
{Ariella} to adjust her credences to her higher-order evidence. There is
no sensible way for her to comply with that advice, and it is bad to
give advice that cannot be sensibly complied with. And it would be bad
to let higher-order evidence guide {Ariella}'s actions, since that would
lead to betting on extreme results.

So I think that a broadly evidentialist approach is the best way to
explain the cases. But it is worthwhile to note that there are good
reasons to reject level-crossing principles about act or state
evaluation, while accepting them about agent evaluation. And such
approaches might end up saying more radical things about particular
cases than I say in thoroughly rejecting level-crossing principles.

\chapter{Disagreement}\label{disagreement-1}

\section{Introducing the Issues}\label{introducingtheissues}

So far in this book I have discussed issues about disagreement only
insofar as they related to higher-order evidence. In this chapter I
change tack, and consider questions about disagreement, and especially
peer disagreement, in their own right.

Here is a schematic case of peer disagreement. {Ankita} and {Bojan} are
peers in both of the following senses:

\begin{enumerate}
\def\labelenumi{\arabic{enumi}.}
\tightlist
\item
  They know each other to be equally good at resolving a broad class of
  questions, of which the question of whether \emph{p} is true is a
  representative member.
\item
  They know that they each have the same evidence that bears on
  \emph{p}.
\end{enumerate}

They then independently consider the question of whether \emph{p} is
true, and when they report back, it turns out they have different views.
In one simple, if extreme, case, {Ankita} thinks it is true, while
{Bojan} thinks it is false. What changes, if any, should they make to
their judgments, once they know what the other thinks?

There are, as elsewhere in philosophy, slightly more actively defended
answers to this question than there are philosophers working on the
question. So we need to start not with a list of possible answers, but a
taxonomy of them. The \textbf{conciliationists} say that {Ankita} and
{Bojan} should, to a considerable extent, move their credences towards
the other's. In the case where one believes \emph{p} to be true and the
other believes it to be false, they should move to both withholding
judgment. The \textbf{anti-conciliationists} deny this; they say that at
least in many cases, at least one of the two need not change their
credences at all merely in light of the disagreement.

In theory, I'm an anti-conciliationist. In particular, I defend a view
that I'll call the \textbf{evidence aggregation} theory of disagreement.
When someone hears that a peer disagrees with them, that is defeasible
evidence the peer has evidence that they lack. Ideally, the hearer would
work out exactly what that evidence is, add it to their stock of
evidence, and react accordingly. That ideal is rarely, if ever,
realised. In more realistic cases, the hearer assigns different
probabilities to different hypotheses about what may have produced the
disagreement. The typical case is that the peer has reacted differently
to having different evidence to the hearer. And it is also typical that
that's because the peer has evidence that the hearer lacks. So in (most)
typical cases, the hearer will think there is evidence that they lack
which supports the peer's view, and typically the rational reaction to
learning that is to move one's views in the direction of the peer's
view. But this credal movement is defeasible thrice over; sometimes the
hearer knows the peer has reacted irrationally, sometimes the hearer
knows the peer has strictly less evidence than they do, and (in very
rare cases) the rational reaction to evidence of evidence for a rival
view is not to move one's view towards the rival view.

In all cases, the guiding principle is that each party should be asking
themselves, and each other, why does the other party have the views that
they have?\footnote{Note that I'm assuming here that there is no doubt
  about what the other party's views are. In realistic cases there is
  usually doubt about this. But what we are interested in here is how
  one should rationally respond to learning that another person has
  views that differ from one's own. It is useful to think about this
  question separately from the question of whether one does really know,
  in a given situation, that the other person has different views.
  Obviously if someone reports having a very improbable view, we should
  not take that report at face value; they may be lying about what they
  believe. But as theorists we can still think about the question of how
  to react to others having different views, even radically different
  views.} If the most plausible answer is that the other party has
information that is relevant to \emph{p}, then one should adjust one's
confidence in \emph{p} to suit.

The evidence aggregation view of disagreement is really the conjunction
of two separable views. The first claim is that the right theory of
disagreement is a reason aggregation theory. That is, hearers should
aggregate the reasons for belief they have with the reasons that their
interlocutor\footnote{I'm going to talk about hearers and interlocutors
  for ease of exposition, but don't read much into this. It doesn't
  matter that the evidence the `hearer' gets for the disagreement is
  testimonial. And as noted in the previous footnote, real life cases of
  testimony involve both questions about how probable it is that the
  interlocutor is sincere, and how one should react on the assumption
  that they are sincere. Our interest is solely in the latter question.}
has for the conflicting belief. The second claim is that evidence, and
evidence alone, provides reasons for belief. The focus of this chapter
will be largely on the first claim, though the way I defend it will both
presuppose the second claim, and indirectly provide some support for
it.\footnote{There is a large epistemological literature on
  disagreement, but very little of it concerns what we should say in
  cases where non-evidential reasons for belief are allowed. We'll set
  those cases aside here, though I think the evidence aggregation theory
  handles them fairly smoothly.}

One consequence of the evidence aggregation view is that a person who
has got things right, i.e., responded correctly to the evidence, should
not adjust their views if they know the other party has no evidence they
lack. So it is anti-conciliationist about the extreme case we started
with, where the parties know they have the same evidence. In practice,
the evidence aggregation view disagrees with conciliationism less than
you might expect. Given the expansive conception of evidence I have been
defending, it is vanishingly rare that parties know they have the same
evidence. Usually, the rational response to a disagreement is not to
give high credence to the proposition that the other party has exactly
the same evidence as one does. Instead, it is to give high credence to
the proposition that there is evidence that one lacks, and that supports
a view closer to that of one's interlocutor. This is what's right about
conciliationism, but it is not what is usually defended by philosophical
conciliationists.

The evidence aggregation view of disagreement that I'm promoting bears
an obvious affinity to the justificationist view of disagreement that
Jennifer Lackey (\citeproc{ref-Lackey2010}{2010}) defends. The main
differences are really points of emphasis, not deep principle. Lackey
describes her view as a way of taking the best features of each of
conciliationism and anti-conciliationism; I'm interested in a version of
the view that is clearly opposed to conciliationism. Relatedly, Lackey's
explanations of some of the cases that motivate conciliationism are
different to mine. But the similarities outweigh the differences, and I
wanted to note her theory as the closest precursor to the theory I'll
defend here.

Another big motivation for the this view of disagreement I'm defending
comes from some remarks on testimony by {Frank} Jackson
(\citeproc{ref-Jackson1987}{1987}). Jackson suggests that the primary
role of testimony is evidence aggregation.

\begin{quote}
Why should you ever accept what I say, unless you already did so before
I spoke -- in which case speech is a luxury? \ldots{} The answer cannot
be that you are taking me to be sincere. \ldots{} Sincerity relates to
whether you should infer prior agreement or disagreement in beliefs, not
to whether posterior adjustment of belief is in order. The reason
posterior adjustment in belief may be in order is that hearers (readers)
sometimes have justified opinions about the evidence that lies behind
speakers' (writers') assertions. You assert that P. I know enough about
you, and your recent situation, to know (i) that you have evidence for
P, for you would not otherwise have said it, and (ii) that your evidence
is such that had I had it, I would have believed P. I borrow your
evidence, so to speak. Typically, I won't know exactly what your
evidence is. Perhaps you visited a factory and came back and said `The
factory is well run'. I don't know just what experiences you had there
-- just what you saw, heard, smelt and so on -- but I know enough to
know that had I had these experiences -- whatever exactly they were -- I
too would have come to believe the factory well run. So I do. \ldots in
this way an epistemological division of labour is achieved. Imagine the
work (and invasion of privacy) involved if we all had to duplicate each
other's evidence. Of course, I may not come to believe exactly what the
speaker or writer believes. A friend returning from overseas may say to
me of a certain country `It is very well run'. I may know enough of my
friend to know that experiences that would make him say that, are the
kind that would make me say `Dissent is suppressed'. In this case, I
will borrow his evidence to arrive, not at what he believes, but at what
I would have, had I had his experiences.
~(\citeproc{ref-Jackson1987}{Jackson 1987, 92--93})
\end{quote}

I agree with almost all of this, though I'm not going to issue a full
defence of such an evidence aggregation account of testimony here. (Why
`almost'? Because it will be rather important later that we not be able
to move as freely between sharing experiences and sharing evidence as
Jackson does in the last line.) Rather, I'm just going to acknowledge my
debt to Jackson's ideas, and move to disagreement.

I'm hardly the first person to start with broadly evidentialist
intuitions and end up with anti-conciliationist conclusions about
disagreement; you can see a similar trajectory in recent work by Maria
Lasonen-Aarnio (\citeproc{ref-Lasonen-Aarnio2013}{2013},
\citeproc{ref-Lasonen-Aarnio2014}{2014a}), and what I say here also owes
a lot to her. But the details are different enough to justify a new
variant on similar themes.

\section{Two Concepts of Peerhood}\label{twoconceptsofpeerhood}

My setup of the {Ankita}/{Bojan} case is ambiguous at a key point. I
said that {Ankita} and {Bojan} are equally good at resolving questions
like this. There are two natural ways to interpret this. We could read
it as meaning that they are equally likely to come up with a rational
verdict, or that their verdicts are equally reliable. David Christensen
(\citeproc{ref-Christensen2014}{2016}) is very good on the importance of
this distinction.

\begin{quote}
The literature typically concentrates on people one has (independent of
one's views on the disputed issue) good reason to take as
\emph{epistemic peers}--as rough equals along certain dimensions of
epistemic evaluation. One such dimension concerns the evidence the other
person has relevant to the disputed issue, and the other concerns how
well she forms beliefs on the basis of her evidence. \ldots{[}W{]}e
should notice that there are a couple of different ways of approaching
the second dimension of evaluation--ways which are not always clearly
separated. One focuses on the other person's equal likelihood of
responding \emph{rationally} to her evidence. On this reading, \ldots{}
the disagreeing friend is what might be called a ``rationality-peer'' on
the given issue: one whose opinion is equally likely to be rational. The
second way of evaluating the other person's responses to evidence is in
terms of her likelihood of responding to that evidence by forming
\emph{accurate} beliefs. On this reading, \ldots{} the disagreeing
friend \ldots{} might be called an ``accuracy-peer'' on the given issue:
one whose opinion on the disputed issue one expects to be as likely to
be accurate as one's own. ~(\citeproc{ref-Christensen2014}{Christensen
2016, 3})
\end{quote}

Christensen cites Feldman (\citeproc{ref-Feldman2007}{2007}), Kelly
(\citeproc{ref-Kelly2005}{2005}), Christensen
(\citeproc{ref-Christensen2007c}{2007b}) and Cohen
(\citeproc{ref-Cohen2013}{2013}) as writers who understand peerhood in
terms of rationality, and Elga (\citeproc{ref-Elga2007}{2007}), White
(\citeproc{ref-White2009}{2009}), Enoch
(\citeproc{ref-Enoch2010}{2010}), Kelly
(\citeproc{ref-Kelly2010}{2010}), Lam (\citeproc{ref-Lam2011}{2011}) and
Levinstein (\citeproc{ref-Levinstein2013}{2013}) as writers who
understand it in terms of accuracy. He's not the first to notice these
two possible understandings; the distinction plays a big role in work by
Ben Levinstein (\citeproc{ref-Levinstein2013}{2013}) and by Miriam
Schoenfield (\citeproc{ref-Schoenfield2014b}{2014}).

The rationality-based understanding is most relevant to the broader
themes of this book. If peerhood is understood in terms of rationality,
then the motivation to conciliate in light of peer disagreement is
indirect. The conciliationist says that {Ankita} should do two things in
light of {Bojan}'s disagreement. First, she should use that disagreement
as evidence that her initial view is irrational, then second, she should
that fact as grounds for revising that first-order credence. Normative
externalism disagrees with the second step. The fact that she has some
higher-order evidence that she is irrational need not, on its own, be
any reason to revise her first-order credence.

But many writers have noted that the first step of this sequence is
dubious too. The most that {Ankita} gets from {Bojan}'s disagreement is
evidence that some view other than hers is rational. It does not follow
that her view is irrational, unless we have make a background assumption
that there is only one rational response to any given evidence. So it
seems that the argument for conciliationism requires the thesis Roger
White (\citeproc{ref-White2005}{2005}) calls Uniqueness: that there is a
single rational response to evidence. Whether this seeming is really
correct is actively debated: see Douven
(\citeproc{ref-Douven2009}{2009}), Kelly
(\citeproc{ref-Kelly2010}{2010}), and Ballantyne and Coffman
(\citeproc{ref-BallantyneCoffman2011}{2011},
\citeproc{ref-BallantyneCoffman2012}{2012}) for interesting moves in the
debate. I'm going to mostly not take a stance on this, since the
arguments for conciliationism have other weaknesses.

It might seem that once {Ankita} views {Bojan} as an accuracy-peer,
issues about higher-order evidence aren't relevant to determining
whether she should conciliate in light of her disagreement. After all,
in that case {Ankita} has two pieces of evidence; her own judgment and
{Bojan}'s. By hypothesis, each of them are equally accurate. So she
should act as if she had two measuring devices, one which said that
\emph{p} was true, and the other that said it was false. And in that
case one should have no settled view about \emph{p}.

But that misstates the situation. {Ankita} doesn't just have two pieces
of evidence; she also has the evidence that led to her initial judgment
that \emph{p}. We only get to describe the case in ways that make it
seem symmetric if we somehow have a reason to set that initial public
evidence aside. This point is well made by Kelly
(\citeproc{ref-Kelly2010}{2010}). And the only way I can see to justify
that set aside is by adopting some principle like JSE. And JSE, as we
saw, is false. Moreover, JSE is equivalent, given plausible assumptions,
to an internalist principle about higher-order evidence.

So however we understand peerhood, either in terms of rationality or in
terms of accuracy, the arguments for conciliationism will be tied up
with arguments about higher-order evidence and hence with normative
externalism.

\section{Evidence, Public and Private}\label{evidencepublicandprivate}

In many discussions of peer disagreement, cases are presented where it
is clear that the disputants have the same public evidence. It does not
follow that in those cases the disputants have the same evidence tout
court. Consider this simple case.

\begin{quote}
\textbf{Stars I}

{Ankita} and {Bojan} are wondering how many stars there are. They both
have the concept of a prime number, but they aren't familiar with
Euclid's proof of the infinity of primes. In fact, they both suspect,
given the decreasing frequency of primes, that they run out eventually.
In the course of their research into the stars, they run into the
Delphic Oracle, who is known to always speak the truth. The Oracle says
``There are as many stars as primes''. {Bojan} takes this to be evidence
that \emph{There are infinitely many stars} is probably false. But while
reflecting on it, {Ankita} comes up with a version of Euclid's proof
that there are infinitely many primes, and concludes that there are an
infinity of stars.
\end{quote}

This is a case where {Ankita} should not conciliate in light of her
disagreement with {Bojan}. She has a proof that there are infinitely
many primes and {Bojan} does not. So she should not change her views.
But that's not really a case that the most plausible form of
conciliationism gets wrong. For reasons that should be familiar from
previous chapters, we should treat {Ankita} as having more evidence than
{Bojan}. Her reconstruction of Euclid's proof that there are infinitely
many primes is a bit of evidence she has that {Bojan} does not.

This all suggests a very weak, and hence easier to defend, version of
conciliationism. It only applies to cases where two parties have
differing views about a proposition, and the following four conditions
are met.

\begin{enumerate}
\def\labelenumi{\arabic{enumi}.}
\tightlist
\item
  The two parties have no reason external to this disagreement to think
  that one is more likely to be rational than the other.
\item
  The two parties have no reason external to this disagreement to think
  that one is more likely to be accurate than the other.
\item
  The two parties have the same public evidence.
\item
  The two parties have the same private evidence.
\end{enumerate}

The evidence aggregation theory of disagreement is anti-conciliationist
in that in this extreme case, it denies that both parties should
conciliate. If one party is acting rationally and the other is not, the
first party should stick to their view.

But even though I'm not a conciliationist in theory, this kind of case
brings out why I'm sympathetic to conciliationism in practice. These
four conditions are met in vanishingly rare circumstances. And when they
are not met, there are quite mundane reasons for thinking that each
party should typically conciliate. A running theme through this chapter
will be that the cases thought to motivate conciliationism do not
satisfy these four criteria, and hence it is possible for an
anti-conciliationist to consistently say that each party should move
towards the others view in ordinary cases

Two more points of clarification before we move on.

First, I'm going to start by looking at a very specific form of
conciliationism, namely Adam Elga's Equal Weight View (EWV). The EWV
says that when two people are peers, and they have the same evidence,
and they learn that they have credences \emph{c}\textsubscript{1} and
\emph{c}\textsubscript{2} in a disputed proposition \emph{p}, they
should each adopt a credence half-way between their initial credences.
That is, their new credence in \emph{p} should be
(\emph{c}\textsubscript{1}~+~\emph{c}\textsubscript{2})/2. The EWV is
not by any means the only version of conciliationism. Indeed, it faces
some difficult technical problems, described by Jehle and Fitelson
(\citeproc{ref-JehleFitelson2009}{2009}) and by Levinstein
(\citeproc{ref-Levinstein2013}{2013}). But as long as we are careful, we
can see which objections are only problems for the EWV, and which form
more general problems for conciliationism.

Second, it is very important here, as almost everywhere in epistemology,
to respect the distinction Gilbert G. Harman
(\citeproc{ref-Harman1986}{1986}) draws between inference and
implication. We can see this by looking at another example about stars.

\begin{quote}
\textbf{Stars II}

In this world {Ankita} and {Bojan} are very knowledgable about primes.
Indeed, they are among the co-authors of that world's counterpart paper
to Polymath (\citeproc{ref-Polymath2014}{2014}). This time the oracle
tells them that there are as many stars as twin primes. {Ankita} infers
that there are probably infinitely many stars, but it is too soon to be
completely confident. {Bojan}, on the other hand, becomes completely
certain that there are infinitely many stars.
\end{quote}

In my opinion, and for that matter {Ankita}'s, the evidence {Ankita} and
{Bojan} have conclusively settles the question of whether there are
infinitely many stars. What they know about primes, plus what they know
about the oracle, plus what they are told by the oracle, probably
entails that there are infinitely many stars. So there is, probably, a
conclusive implication from their evidence to that conclusion. But there
is no reasonable inference from their evidence to the conclusion that
there are infinitely many stars. That inference requires knowing
something that is not in evidence, namely that there are infinitely many
twin primes. The fact that this fact is a logical truth (or at least is
logically entailed by things they know about primes) is irrelevant. A
probably conclusive implication can be a definitely unreasonable
inference, and is in this case. Unless {Bojan} has a proof of the twin
prime conjecture up his sleeve, one that he hasn't shared with his
co-authors, he should move his credences in the direction of {Ankita}'s.
That is, he should conciliate. It's possible that {Ankita} should
conciliate too; I haven't said nearly enough about the case to settle
that one way or the other. I think the mistaken idea that entailments
generate maximally strong inferences has led to some confusion about
what to say about certain cases, and that will become relevant as we
progress.

\section{Independence and
Conciliationism}\label{independenceandconciliationism}

In early writings on concliationism, such as those by Elga
(\citeproc{ref-Elga2007}{2007}) and Christensen
(\citeproc{ref-Christensen2009}{2009}), there was a line of argument
from principles like Independence (as we've discussed in previous
chapters) to conciliationism. This line is flawed, for reasons well set
out by Errol Lord (\citeproc{ref-Lord2014}{2014}). The point of this
section is simply to rehearse Lord's arguments before moving onto other
possible motivations for conciliationism.

There are weaker and stronger versions of the kind of Independence
principle that Elga, Christense and others use. The strongest such
principle says that in any dispute, a party to the dispute can only
reasonably conclude that the other party is wrong based on reasons
independent of their reasons for having a disputed view. But that leads
to very odd predictions in cases like this.

\begin{quote}
\textbf{Bus Stop}

While waiting at the bus stop, {Ankita} is approached by {Bojan} , who
tells her that he is certain she lives in a shoe. {Ankita} is fairly
confident, based on long familiarity with her apartment, that she lives
in an apartment, not a shoe.
\end{quote}

{Ankita} doesn't have to find independent evidence that {Bojan} is
mistaken to hold onto her belief that she lives in an apartment. Perhaps
in some realistic versions of Bus Stop, {Bojan} would appear drunk or be
slurring his words, and that would be the relevant independent evidence.
But those external clues are not necessary. {Bojan} could appear
perfectly sane and sensible in every respect except his firm belief that
{Ankita} lives in a shoe, and she could still dismiss his view. So this
strongest independence principle is false.

More plausible independence principles restrict the circumstances in
which one must rely on independent reasons to dismiss a conflicting
view. There are two interesting restrictions we could look at:

\begin{itemize}
\tightlist
\item
  Independence might be restricted to cases where the disagreeing
  parties are known to be just as good at reading the evidence. (We
  could break this down into two sub-cases depending on whether `good'
  is understood in terms of accuracy or rationality, but this won't
  matter.)
\item
  Independence might be restricted to cases where the disagreeing
  parties are known to have the same evidence.
\end{itemize}

But, and this is the crucial point that Lord makes, neither of these
restrictions on their own gives us a plausible principle. If we only
impose the first restriction, we end up with the implausible conclusion
that {Ankita} is expected to conciliate in this case.

\begin{quote}
\textbf{Party}

{Ankita} and {Bojan} are just as good, in both senses, at working out
where a party is given some evidence. But {Bojan} hasn't looked at the
invitation to tonight's party in weeks, so is uncertain whether the
party is on State St or Main St.~{Ankita} looked at the invitation two
minutes ago, and is certain the party is on State St.
\end{quote}

It would be absurd to think that because {Ankita}'s credence that the
party is on State St is 1, and {Bojan}'s is 0.5, and they are just as
good at working out where parties are given some evidence, that
{Ankita}'s credence that the party is on State St should move to 0.75.
Rather, she should conclude that {Bojan} hasn't looked at the invitation
recently. And she should conclude that simply because {Bojan} has a
different credence to her about where the party is. That's what an
independence principle that only imposes the first constraint would rule
out, so such an independence principle is false.

Nor will the second restriction on its own do. If we restrict the
restriction to public evidence, then Stars I is already a counterexample
to it. But we can come up with cases where arguably {Ankita} and {Bojan}
even have the same private evidence, and the restriction is still not
sufficient.

\begin{quote}
\textbf{Diagnosis}

{Ankita} is a professor at a medical school, and {Bojan} a student. The
students at her school are very good; often they are as good at
diagnosis as the professors. And {Bojan} has done, {Ankita} knows, very
well on his theory exams. But some students who know a lot of theory are
very poor at making a diagnosis based on material in a patient's file.
So {Ankita} pulls out a file at random for her and {Bojan} to look at.
Given the symptoms displayed, {Ankita} is very confident in a particular
diagnosis. But {Bojan} has no idea what to say about the case; his best
guess is that we should have low but positive credence in several
distinct diagnoses.
\end{quote}

In this case, {Ankita} shouldn't infer that she had been over-confident.
She should conclude that, despite his solid background, {Bojan} isn't
very good at making a diagnosis. I've obviously simplified a lot, but
this seems like a very natural way for professors to test whether their
students have or lack a practical skill. Now perhaps the best
explanation of this case is that {Bojan} really lacks some evidence,
despite his doing well on tests. That's actually what I suspect is going
on. But I suspect most people, and certainly most conciliationists,
don't think that. It is much easier to motivate conciliationism if we
think that there is a skill of processing evidence that goes well beyond
the possession of evidence, and that in cases like this one what's
happened is that {Bojan} lacks that skill. (Why say conciliationism is
easier to motivate if one posits large skill differences that go beyond
evidence possession? Because now we can say why one person should defer
to another without thinking the other person has evidence they lack; the
other person may have more skills.)

Now if independence just requires that the parties had the same
evidence, and this is a case where the parties have the same evidence,
it would be an independence violation for {Ankita} to infer from
{Bojan}'s lack of certainty in any diagnosis to his lack of skill in
making diagnoses. Rather, she should conciliate with him, and lose
confidence in her diagnosis. That's wrong, so this independence
principle is too strong.

So just putting each of these restrictions on independence singularly
does not yield a viable principle. What happens if we put both
restrictions on at once, and say independence holds only if the parties
are known to have the same evidence and known to be just as good (in
some sense) at processing it? Lord points out that then we don't have a
premise in an interesting argument for conciliationism. Rather, the
independence principle that is supposed to motivate conciliationism has
just become a statement of conciliationism. So it can't provide any
independent support for it.

\section{Circularity and
Conciliationism}\label{circularityandconciliationism}

Conciliationism has been supported, or at least anti-conciliationist
positions opposed, with arguments that anti-conciliationism lapses into
an implausible kind of circularity. Here are a couple of quotes setting
out this kind of worry. First, from Adam Elga.

\begin{quote}
To see the correctness of the equal weight view, start with a case of
perceptual disagreement. You and a friend are to judge the same contest,
a race between Horse A and Horse B. Initially, you think that your
friend is as good as you at judging such races. In other words, you
think that in case of disagreement about the race, the two of you are
equally likely to be mistaken. The race is run, and the two of you form
independent judgments. As it happens, you become confident that Horse A
won, and your friend becomes equally confident that Horse B won.

When you learn of your friend's opposing judgment, you should think that
the two of you are equally likely to be correct. For suppose
not--suppose it were reasonable for you to be, say, 70\% confident that
you are correct. Then you would have gotten some evidence that you are a
better judge than your friend, since you would have gotten some evidence
that you judged this race correctly, while she misjudged it. But that is
absurd. It is absurd that in this situation you get any evidence that
you are a better judge \ldots{}

Furthermore, the above judgment of absurdity is independent of who
\emph{in fact} has done a better job. Even if in fact you have judged
the series of races much more accurately than your friend, simply
comparing judgments with your friend gives you no evidence that you have
done so. ~(\citeproc{ref-Elga2007}{Elga 2007, 486--87}, emphasis in
original)
\end{quote}

And second, from Diego E. Machuca. (This quote comes just after a
presentation of Thomas Kelly defending something close enough, for
current purposes, to the evidence aggregation view I'm defending.)

\begin{quote}
Kelly maintains that one can be justified in thinking that one has
appropriately responded to the first-order evidence even in the absence
of independent evidence that one has done so. For the reason why one
takes up a given belief is precisely that one \emph{recognizes} that it
is supported by the evidence one possesses, and one would not be able to
recognize this if one were unjustified in thinking that the evidence
does support the belief in question. I confess that I cannot see how
this move is not question-begging all the way through. Just as one can
affirm that one's opinion is justified because one recognizes that the
available evidence supports it, so too one's opponent can affirm that
his opinion is justified because he recognizes that the available
evidence supports it. And if one were to argue that one's opponent is
clearly mistaken because one would not recognize that one's belief is
supported by the evidence if one were not justified in thinking that it
is, one's opponent would retort that it is he who cannot be mistaken
simply because he would not recognize that the evidence supports his
belief were he not justified in so thinking.
~(\citeproc{ref-Machuca2013}{Machuca 2013, 77--78})
\end{quote}

There are two things to say about this kind of argument. The first is
that the strong form of the objection Elga makes is not really a
response to the evidence aggregation view, but to the view that any
agent is entitled to privilege their own view over others', simply
because it is their own. But the kind of reasoning Elga worries about is
only available to the one who has got things right, not to both parties.
So the worry is not that everyone could have their self-confidence rise,
but that those who get things right could become more confident in their
ability to get things right in virtue of their recent track record of
having got things right. And that doesn't look like much of a worry.
(I'm here not far away from the replies that Andrew Rotondo
(\citeproc{ref-Rotondo2013}{2013}) makes to circularity arguments for
conciliationism.)

But at this point we run into Machuca's complaint. If the successful can
be more confident in their own ability in virtue of their successes,
won't those who merely think they are successful become more confident
in their own ability in virtue of their own perceived success? And, in
this game, doesn't everyone perceive of themselves as being successful?

There are a number of ways we could try to turn these rhetorical
questions into arguments. For the reasons I went over in chapter 9, none
of the resulting arguments will work. The underlying argument could be
that appropriate epistemic methods must be bidirectionally luminous;
everyone must be able to know if they are applying them correctly. But
that kind of argument falls to the Williamsonian anti-luminosity
arguments. Or it could be that appropriate epistemic methods must be
sensitive, in Nozick's sense. But that kind of argument falls to the
anti-sensitivity arguments. And so on for all the other ways of
precifisying the argument that evidentialism licences noxiously
question-begging practices.

Now I will note one sense in which those replies in chapter 9 might miss
the mark. Machuca is defending a form of Pyrrhonian scepticism. And many
of my defences of externalism involved showing that the principles
deployed against externalism had implausible consequences. In
particular, they implied Pyrrhonian scepticism. Now that won't look
implausible to a Pyrrhonian like Machuca. Here I must simply note that
I'm taking it as a fixed point that we do know a lot, and that
Pyrrhonian scepticism is false. This obviously loses some potential
converts, but I doubt it is possible to find philosophical arguments
that work for one's position against all possible rivals
~(\citeproc{ref-Lewis1982c}{Lewis 1982}).

\section{Six Examples}\label{sixexamples}

The last two sections reply to arguments based on general theoretical
principles in favor of conciliationism. The prospects for this way of
defending the EWV, or indeed any conciliatory position, look dim. But
these general theoretical principles have not been what have most moved
philosophers towards conciliationism. Rather, they are moved by the idea
that the EWV, or at least some form of conciliationism, is the best
explanation of the clear facts about some simple cases. The literature
here, as with the literature on higher-order evidence, suffers from that
``main cause of philosophical disease--an unbalanced diet: one nourishes
one's thinking with only one kind of example.''
~(\citeproc{ref-Wittgenstein1953}{Wittgenstein 1953, sec. 593}). I can't
claim to offer a balanced diet, but I can offer the start of a more
varied one. Here are six new morsels that will form the basis of the
discussion to follow. I'm going to argue that the evidence aggregation
view can explain the last two, while conciliationist can not. And I'll
argue there is no case that the conciliationist can explain while the
evidence aggregation theorist can not.

\subsection{Arithmetic}\label{arithmetic}

{Ankita} and {Bojan} are working on some arithmetic problems. They both
know that they have a similar track record at these problems; both are
reliable, with very similar rates of mistakes. They are trying to work
out \emph{What is 22 times 18?}. {Ankita} correctly works out that it is
396; {Bojan} says that it is 386. What should their credences in each
answer be?

\subsection{Jellybeans}\label{jellybeans}

{Ankita} and {Bojan} are trying to guess how many jellybeans are in a
sealed, transparent container. They both have equal access to the
container, and they both know that they have similarly good track
records at this kind of game. {Ankita} correctly guesses that there are
396; {Bojan} guesses that there are 386. What should their credences in
each answer be? (A similar case is considered by Jack L. Treynor
(\citeproc{ref-Treynor1987}{1987}).)

\subsection{Detectives}\label{detectives}

{Ankita} and {Bojan} are the two best murder detectives in the world.
They both know that they are the only peers they each have, and that
they have very similar track records of success, with equal (and rare)
failures. They are brought in to solve a mystery that no one has made
any progress on. Each quickly sees that it could only be the butler or
the gardener. {Bojan} has equal credence in each suspect, but {Ankita}
figures out a subtle reason that it could not have been the gardener, so
is sure the butler did it. And in fact the butler did do it, and
{Ankita} is right about why the gardener could not have done it. After
they compare credences, {Bojan} giving equal credence to each suspect,
and {Ankita} being sure it is the butler, what should their credences in
each answer be? (I owe this case to Ben Levinstein
(\citeproc{ref-Levinstein2013}{2013}).)

\subsection{Football}\label{football}

{Ankita} and {Bojan} are both very good at predicting football games of
different codes. They both typically make highly rational predictions,
and they both have excellent (and similar) records for accuracy. They
both know all this, and they have the same public evidence about this
weekend's matches. They are comparing their credences in the home team
winning ahead of two big matches: an Australian Rules match in
Melbourne, and an English Premier League match in London. For each
match, {Ankita} has a credence of 0.9 that the home team will win, and
{Bojan} has a credence of 0.1 that the home team will win. They both
regard the matches as completely independent, so {Ankita}'s credence
that both home teams will win is 0.81, while {Bojan}'s is 0.01, and each
of them have credence 0.09 in each of the hypotheses that one particular
home team will win and the other will not. Once they share their
credences with each other, what should their credence be that (a) the
home team will win the Australian Rules match, (b) the home team will
win the English Premier League match, and (c) both home teams will win?

\subsection{Simple Arithmetic}\label{simplearithmetic}

{Ankita} and {Bojan} are working on some arithmetic problems. They both
know that they have a similar track record at these problems; both are
reliable, with very similar rates of mistakes. They are answering the
question \emph{What is 2 plus 2?}. {Ankita} says it is 4; {Bojan} says
that it is 5. What should their credences in each answer be?

\subsection{Doctors}\label{doctors}

{Ankita} and {Bojan} are the two best cardiologists in the world. They
know each other to be peers, the only peers each has. They are brought
in to diagnose a case that has stumped all the other experts in the
field. {Ankita} judges that it is likely disease A, but she is just
short of fully believing it is disease A, since she thinks disease B is
an unlikely, but real, possibility. This is the rational response to the
evidence. Although the patient has disease A, the evidence available to
an expert cardiologist is just short of being sufficient to ground
knowledge that the patient has disease A, since B is also a realistic
possibility. She reports all this when she and {Bojan} compare notes,
but {Bojan} reports that he is confident that the patient has disease A.
What should their credence in each diagnosis be?

\subsection{My Verdicts}\label{myverdicts}

These cases are, in general, not so clear that we can simply know what
is true about them after a moment's thought, and use that knowledge to
evaluate theories. But for the record, here are my verdicts on the
cases.

In \textbf{Arithmetic}, I think a lot depends on the finer details of
the case, particularly on how {Ankita} got to her answer. But I think no
matter how those details are filled in, there isn't a lot of pressure on
her to conciliate. Now this isn't a popular view. Much of the motivation
for conciliationism comes from thinking that in versions of
\textbf{Arithmetic} where the sum in question is not specified, there is
rather strong pressure to conciliate. We'll come back to that idea
several times below.

In \textbf{Jellybeans}, I think they clearly should conciliate. And
unlike in \textbf{Arithmetic}, this conciliation should take the form of
not just lowering their credences in their preferred answers, but in
increasing their credence in answers between the two they offered. In
\textbf{Jellybeans}, the announced answers should increase their
confidence that the answer is 391, which is not what should happen in
\textbf{Arithmetic}.

I have no idea what the answer to the third question in
\textbf{Football}, about the appropriate credence in the compound
proposition, is. We'll say a bit below about why this is such a hard
question.

In each of the last three cases, \textbf{Detectives}, \textbf{Simple
Arithmetic} and \textbf{Doctors} {Ankita} should not conciliate, and
{Bojan} should move his credence dramatically in the direction of
{Ankita}'s. Or at least so I say.

\section{Equal Weight and the Cases}\label{equalweightandthecases}

On the face of it, the EWV gets at most one of the six cases right.
After all, the only case where it seems even prima facie right to move
to a credence half-way between the two expressed views is
\textbf{Arithmetic}. But a more nuanced understanding of the cases lets
EWV handle \textbf{Jellybeans}, and a more subtle version of
conciliationism does well (or at least well enough) with
\textbf{Detectives} and \textbf{Football}. If there is a case-based
objection to conciliationism, it comes from the last two cases. But
first I want to go over why the second, third and fourth cases are
really not problems for conciliationism. Why, as an
anti-conciliationist, should I do that? It's for two reasons. First, I
want to demonstrate how hard it is to use any case around here to show
that a particular view on disagreement is wrong. Second, we get an
interesting insight into the range of possible and indeed plausible
versions of conciliationism by working through the cases carefully.

The apparent problem with \textbf{Jellybeans} is that it seems the
rational reaction, for both {Ankita} and {Bojan}, is to increase their
credence in a particular hypothesis that neither of them endorses,
namely that there are 391 beans in the jar. But it isn't hard to see
that this is a merely apparent problem. What credences should we
attribute to {Ankita} when she announces her guess of 396? Presumably
not that she has credence 1 that there are 396, and credence 0 in
everything else. Given what we know about jars of jellybeans, and human
visual capacities, it is best to interpret her as saying that the mode
of her credal distribution over the competing hypotheses about the
content of the jar is 396. But that distribution will presumably be
fairly spread out, and indeed fairly flat around the peak. Similarly,
{Bojan} will have a credal distribution that is spread out, and fairly
flat around its peak of 386. If we average out those distributions, it
could easily be that the peak of the new distribution is at 391. That
happens, for instance, if each of {Ankita} and {Bojan}'s distributions
are normal distribution, with a mean at the number they announce, and a
standard deviation of 10.

Now there are hard questions about how we do, or even could, know that
the number they utter means that they have just this credal
distribution. But that's not particularly our problem here. The question
is what the parties to the dispute should do given that we add to their
evidence each other's credence distribution. Questions about how we
could know what another person's credence distribution are, while
fascinating, are not at issue here. This is a point worth keeping in
mind as we work through the examples.

The EWV does rather badly on \textbf{Detectives}, but other versions of
conciliationism do better. Assuming that the detectives are actually
very good at their jobs, then neither would have formed the conclusion
that the butler did it without a very good reason. If one of them
believes this, and the other does not, the one who does not should
believe that they've missed a reason. So they should largely defer to
the other.

Note that the reasoning in the last paragraph is entirely symmetric, and
doesn't directly make use of the fact that {Ankita} was right to infer
that it was the butler. So it is reasoning that should be available to
the conciliationist, even if it isn't available to the equal weight
theorist. And there is a natural method for how to get the right result
in \textbf{Detectives} in a conciliationist-friendly way. The method in
question is one I'm taking from some work by Sarah Moss
(\citeproc{ref-Moss2011}{2011}).

Imagine that {Chika} is not a detective, and has no particular expertise
in solving murders. Moreover, she has very little information that bears
directly on the case. What she does know is what {Ankita} and {Bojan}
think; she knows that {Ankita} is confident the butler did it, while
{Bojan} is uncertain. The reasoning from two paragraphs ago is available
to {Chika} too. She can think that {Ankita} wouldn't be so confident
unless she had a very good reason, so she can infer that it is very
likely that the butler did it.

One natural form of conciliationism says that the parties to a dispute
face the same normative pressures as an outsider, like {Chika}. Whatever
is rational for {Chika} to do given the knowledge just of the parties'
credences, and their track records and backgrounds, is rational for the
parties to the dispute to do. In general, that will mean conciliating,
since in general {Chika} should form a credence somewhere between the
parties' credences. But that isn't always true. If {Ankita} and {Bojan}
were both 90\% confident that it was the butler, and that's all {Chika}
knows, then {Chika} should give some credence to the possibility that
{Ankita} and {Bojan} have noticed independent reasons for thinking it is
the butler, and should have a credence in the butler's guilt slightly
higher than 0.9. Nevertheless, the view that insiders to the dispute,
like {Ankita} and {Bojan}, should end up in the same place as an
outsider, like {Chika}, who knows just the credences, seems to capture
the idea at the heart of conciliationism.

I haven't said very much in general about how {Chika} should reason
about these cases. Ben Levinstein (\citeproc{ref-Levinstein2013}{2013}),
to whom I owe this example, thinks that {Chika} should have a credence
function that minimises the sum of {Ankita} and Bogan's expected
inaccuracy. He persuasively argues that this method delivers the right
result in a number of tricky cases.

We can also think about \textbf{Football} as an `insider-outsider'
problem. This case is really rather hard. I used to think it was a
counterexample to any form of conciliationism, since conciliationists
would have to say that each party would improperly regard the games as
probabilistically dependent after learning about the disagreement. But I
now think that both premises of this little argument (that
conciliationism implies probabilistic dependence, and this is bad) are
dubious. The case is just a hard case for everyone, and we can see that
by thinking about it from {Chika}'s perspective. (The next few
paragraphs draw on work by Julia Staffel
(\citeproc{ref-Staffel2015}{2015}).)

Assume that {Chika} knows nothing about football (of any code), but does
know about {Ankita} and {Bojan}`s predictive records and their credences
concerning these games. And assume that she's an ideal aggregator.
Finally assume, more or less for reductio, that {Chika} aggregates
probabilistic judgments by taking the linear average of them. (If that's
right, the EWV and the 'insider-outsider' version of conciliationism
coincide; if it isn't right, they don't.) The following table gives
{Ankita}, {Bojan} and {Chika}'s credences and conditional credences,
assuming that {Chika} does this. I'll use \emph{p} for the home team
wins the Australian match and \emph{q} for the home team wins the
English match.

\begin{longtable}[]{@{}lcccc@{}}
\toprule\noalign{}
& \emph{p} & \emph{q} & \emph{p} ∧ \emph{q} & \emph{p} \textbar{}
\emph{q} \\
\midrule\noalign{}
\endhead
\bottomrule\noalign{}
\endlastfoot
Ankita & 0.9 & 0.9 & 0.81 & 0.9 \\
Bojan & 0.1 & 0.1 & 0.01 & 0.1 \\
Chika & 0.5 & 0.5 & 0.41 & 0.82 \\
\end{longtable}

The key number is in the bottom right. Assuming that {Chika} plans to
update by conditionalisation, that means that although her credence in
\emph{q} is now 0.5, if she learns \emph{p} , it will rise to 0.82.

It has been argued, e.g.~by Loewer and Laddaga
(\citeproc{ref-Loewer1985}{1985}) and Jehle and Fitelson
(\citeproc{ref-JehleFitelson2009}{2009}) that this is a mistake for the
following reason. {Ankita} and {Bojan} both take the games to be
probabilistically independent. So {Chika}, who only has their credences
to go on, should take them to be independent too. This argument doesn't
work, for a reason Sarah Moss (\citeproc{ref-Moss2011}{2011}) gives. The
probabilities in this table are evidential probabilities. Even if the
games are physically independent, it could be that the result of one
gives {Chika} evidence about the other. And that is what happens; if she
learns \emph{p} she gets one more data point in favor of {Ankita}'s
general accuracy in football-predicting, and against {Bojan}'s. So it is
plausible that, for her, learning \emph{p} will raise her credence in
\emph{q} .

What isn't plausible, as Staffel notes, is that it could raise her
credence that much. We can imagine, consistent with everything I've said
so far, that {Chika} has a lot of evidence about {Ankita} and {Bojan}'s
track records. If \emph{p} is true, then {Ankita} did better at
forecasting \emph{p} than {Bojan} did. So that's a reason to no longer
given exactly equal weights to their forecasts. But for all I've said so
far, this might mean that we have a data set consisting of 1001 times
that {Ankita}'s forecast was better, and 1000 times that {Bojan}'s
forecast was better. That does not look like a good reason to have a
probability for \emph{q} that is several times closer to {Ankita}'s
forecast than it is to {Bojan}'s. More generally, just what number goes
into the bottom right of the table should be sensitive to how much
information we have about {Ankita} and {Bojan}, and not just to the
balance between them. Arguably, having a conditional credence for
\emph{q} given \emph{p} of 0.82 could be reasonable if {Chika} knew
almost nothing about {Ankita} and {Bojan} before the games were played.
But it is not reasonable if she has a very substantial set of results
where they have both done very well, and within that a substantial and
balanced set of results in games where they have disagreed. But the
Equal Weight View is insensitive to the quantity of information that
{Chika} has.

So the Equal Weight View is wrong about this case. It gives an
implausible prediction, and it is insensitive to a factor that we know
to be relevant. But the failure of Equal Weight does not mean that
conciliationism fails. Saying just what values should go in the two
right-most boxes in the bottom row is a very very hard question. But
presumably it has at least one good answer. It doesn't seem like this is
an epistemic dilemma for {Chika}. So the conciliationist can still say
something substantive about how {Ankita} and {Bojan} should react to
learning about each other's forecast. The conciliationist, I suggest,
should say that {Ankita} and {Bojan} should adopt whatever credences
{Chika} should adopt. This is a substantive and interesting claim. I
think it is false, but I don't think it is obviously false. Ideally the
conciliationist who says this would have something a little more
substantive to say about what {Chika}'s credence should be. But ideally
any epistemologist who discusses the problem would have something a
little more substantive to say about what {Chika}'s credence should be,
so this isn't a particular problem for conciliationism. Nor is there any
reason to think that adopting conciliationism makes it harder to say
what {Chika} should do. So while this looks like a hard case, I don't
think it can be used in an argument against conciliationism. Or, at
least, it can't be so used just yet. Perhaps we could solve the problem
of how {Chika} should react, and then show it is implausible for
{Ankita} and/or {Bojan} to react that way. But I'm not in any position
to run that argument, because I don't know what {Chika} should do.

What I'm saying here is very similar to what I said in the chapter 6
about the problem of inter-theoretic value comparisons. In both cases,
normative internalism makes vivid a particularly hard epistemic problem.
But the problem in question, in each case a problem about aggregation,
was hard to start with, and isn't any harder in virtue of internalism.
The fact that internalism makes the problem vivid is not in itself a
reason to reject internalism.

On the other hand, \textbf{Doctors} is a problem for conciliationism,
and looking at the problem through {Chika}`s eyes doesn't help the
conciliationist. If {Chika} knows that {Bojan} is certain of a
diagnosis, and that {Ankita} gives that credence a very high credence
just short of belief, it seems prima facie plausible that {Chika} should
conclude from that that the diagnosis is correct. Unless we have some
way to motivate a theory of judgment aggregation where the aggregate
opinion is never more confident in a proposition than the weakest
member, there must be some such cases where {Chika} should believe the
diagnosis is correct. But {Ankita} should not share this confidence. She
should not find her doubts assuaged by {Bojan}'s not sharing them. So
\textbf{Doctors} is a counterexample to the 'insider-outsider' version
of conciliationism. And that's the only version that seems to get
\textbf{Football} right. So no version of conciliationism can get both
these cases right.

It is easy for the evidentialist to say what's going on in
\textbf{Simple Arithmetic}. {Ankita} has maximally strong evidence that
2 plus 2 is in fact 4. That's not just because the conclusion is a
logical truth. There are plenty of logical truths that we have
insufficient evidence to believe, either because we don't know which
logics validate them, or because we don't know what the correct logic
is. Rather, it is because the inference from \emph{x}~=~2~+~2 to
\emph{x}~= 4 is one that is immediately justified, without the need for
further steps. {Bojan}'s disagreement can't dislodge that.

But how can the conciliationist handle the case? It doesn't seem very
plausible to say that when an otherwise reasonable person says that two
plus two is five, we're obliged to doubt that it is four. The usual
response on behalf of conciliationists is to appeal to the notion of
`personal information'. The idea was first developed by Jennifer Lackey
(\citeproc{ref-Lackey2010}{2010}), but I want to first mention the
version of this defence put forward by David Christensen
(\citeproc{ref-Christensen2011}{2011}). (Christensen is describing a
scenario where the narrator plays the role of Ankita, and Bojan is their
friend.)

\begin{quote}
If such a bizarre situation were actually to occur, I think one would
reasonably take it as extremely unlikely that one's friend (a) was
feeling as clear-headed as oneself; (b) had no memories of recent
drug-ingestions or psychotic episodes; and most importantly, (c) was
being completely sincere. Thus, to use Lackey's term, one's personal
information (that one was feeling clear, lacked memories suggesting
mental malfunction, and was being sincere in one's assertion) would
introduce a relevant asymmetry, and one could reasonably maintain one's
belief.
\end{quote}

The first thing to be said here is that (c), which is what Christensen
adds to Lackey's original characterisation, is beside the point. The
question is what {Ankita} should do given that {Bojan} believes that 2
plus 2 is 5. It's not the separate question of whether she should
believe he believes that, given his utterance. So questions of sincerity
are beside the point. Then the question is whether (a) and (b), which
are the aspects of personal information that Lackey originally
highlighted, are enough to help.

And it is hard to see how they could be. If the reason for discounting
{Bojan}'s opinion rested on one's personal information, then the more
information we get about {Bojan}, the more worried we should be. But I
rather doubt that running a drug test on {Bojan}, to see whether (b) is
a relevant difference between him and {Ankita}, should make any
difference at all to {Ankita}'s confidence.

More generally, this explanation rests on an odd view about epistemic
capacities. {Ankita}'s ability to do simple arithmetic is not, according
to Christensen, a sufficient ground to believe that two plus two is
four. But her ability to detect differences in capacities and aptitudes
between two people, one of whom is herself, is enough of a ground.
Speaking personally, I'm sure I'm much better at simple arithmetic than
I am at doing such comparisons. Indeed, my abilities to make such
comparisons intuitively are so weak that I could only possibly do them
by careful statistical analysis, and that would require, among other
things, being able to add two plus two. In other words, if I can't know
what two and two is, I can't process the evidence that might tell for or
against the abilities of one party or another. So the conciliationist
doesn't have a good explanation of how we can hold on to knowledge in
these simple cases.

Simple arithmetic cases are important not just because they raise
problems for conciliationism, but because they tell us something about
what's at issue in debates about disagreement. Consider this argument by
David Enoch for thinking that in debates about disagreement as such, we
should treat the parties to the disagreement symmetrically.

\begin{quote}
Second, our question, as you will recall, was the focused one about the
epistemic significance of the disagreement itself. The question was not
that of the overall epistemic evaluation of the beliefs of the
disagreeing peers. Kelly is right, of course, that in terms of overall
epistemic evaluation (and barring epistemic permissiveness) no symmetry
holds. But from this it does not follow that the significance of the
disagreement itself is likewise asymmetrical. Indeed, it is here that
the symmetry is so compelling. The disagreement itself, after all, plays
a role similar to that of an omniscient referee who tells two thinkers
`one of you is mistaken with regard to p'. It is very hard to believe
that the epistemically responsible way to respond to such a referee
differs between the two parties. And so it is very hard to believe that
the epistemic significance of the disagreement itself is asymmetrical in
anything like the way Kelly suggests. ~(\citeproc{ref-Enoch2010}{Enoch
2010, 657})
\end{quote}

Well, consider the case when \emph{p} is the proposition that two plus
two is four, and {Ankita} is the party who believes \emph{p}, while
{Bojan} rejects it. Having an omniscient referee tell the parties that
one of them is mistaken should produce asymmetric responses in the two
parties. Now maybe there are only a small class of cases where this is
the case, and what Enoch says is right in the majority of cases. But we
can't argue for that on perfectly general grounds about the nature of
disagreement, because it fails in extreme cases like \textbf{Simple
Arithmetic}. The argument that it holds in normal cases needs a distinct
defence.

\section{The Evidence Aggregation
Approach}\label{theevidenceaggregationapproach}

Having gone over how conciliationism handles, or doesn't handle, the
cases, let's compare it to how an evidence aggregation view handles
them. We'll look at them in reverse order, because the earlier cases are
harder for the view.

Evidence Aggregation gets \textbf{Simple Arithmetic} right. {Ankita} has
clear and compelling evidence that two plus two is four. The fact that
two plus two is four is part of her evidence, and when the conclusion is
part of one's evidence, that is maximally strong support. Learning that
something has gone badly wrong with {Bojan}'s arithmetic competence or
performance does not make her lose this evidence.

It also gets \textbf{Doctors} right, by treating the case as parallel to
the case of {Roshni} from chapter 8. When {Ankita} learns that {Bojan}
is very confident that the patient has disease A, that isn't yet
evidence that {Bojan} has stronger evidence that the patient has disease
A. It might mean simply that {Bojan} hasn't considered the possibility
of B, or that he has overly hastily dismissed it. And indeed, that's
just what has happened. Until {Ankita} learns why {Bojan} has the
credences he does, she can reasonably, if provisionally, keep her
current credences. After all, there may not be any new evidence in
favour of diagnosing A. And when she does learn why {Bojan} has these
credences, she should stick to her initial view. That's not because it
was her view, but rather because it was the view best supported by the
current evidence.

From this perspective, \textbf{Detectives} is just like
\textbf{Doctors}. When {Ankita} hears {Bojan}'s credence, it is
reasonable for her to infer that she has some evidence that {Bojan}
lacks. This evidence need not be public evidence; it might be more like
the kind of evidence a mathematician gets when working through a proof.
But it is reasonable for her to infer, given just the facts about their
conflicting credences, that {Bojan} has simply missed the reason that it
must have been the butler. So she doesn't have new evidence that it
wasn't the butler, so her credence shouldn't move.

The last two cases are not like most everyday cases of disagreement. The
usual situation, when another person disagrees with us, is that they
have evidence we lack. Or, at least, it is usually the case that one
should give the possibility that the other person has extra evidence
substantial credence. That's why it is usually the case that one should
conciliate. The default view is that the other probably has good
evidence we lack, and that is reason to move one's attitude towards the
other's. It is very hard to say in general when one should abandon this
default stance. Indeed, it is very hard to even say whether the `should'
in question is moral or epistemic. It feels like an epistemic question
at first, but perhaps moral considerations to do with humility, respect
and friendship are also relevant factors. But we shouldn't let the fact
that it is hard to give a general theory here prevent us from saying
something about some cases. And we should say that \textbf{Doctors} and
\textbf{Detectives} are among the (presumably rare) cases where one
party, in this case {Ankita}, is warranted in holding firm to their
beliefs.

It's a little harder to know what the evidence aggregation view should
say about \textbf{Football}. The case as presented didn't include much
detail about how {Ankita} or {Bojan} came to their conclusions. If I was
in one or other of their positions, I would likely infer that the other
had picked up on some reason I missed, but also that they had probably
missed some reason I'd seen. So I would be tempted to conciliate,
because this is a case where the conflicting credences really are useful
evidence that there is (private) evidence that would motivate a change
of view.

While the evidence aggregation view doesn't have a firm theoretical
recommendation, it does have a firm practical recommendation. Each party
should ask the other why they have the view that they do. Assuming it is
possible to ask the other this question, and the disagreement is about
something significant enough to make it worth the bother, this is pretty
much always the practical recommendation. As far as I can tell,
intuition and folk wisdom agree with the evidence aggregation view on
this point. And it is hard to see how rival views of disagreement could
motivate such a strong recommendation to ask the other person ``Why do
you think that?''. After all, those rival views already say what the
disagreeing parties should do, and the answer is not sensitive to why
the other person has the views they do. If it turns out that all the
reasons {Bojan} can offer are ones that {Ankita} had already properly
weighed, she should revert to her initial credence. But probably he has
thought of something she missed, and probably she has thought of
something he missed, and adding those reasons together will bring their
views closer together.

The conciliationist thinks that {Ankita} and {Bojan} should aggregate
the outputs of their deliberation. The evidence aggregation view says
that they should aggregate the inputs to their deliberation. If the only
evidence they have as to those inputs is the outputs, then they should
use the outputs to make reasonable guesses as to the nature of the
inputs, and aggregate them. But this is very much a second-best
solution; the best thing to do is to find out exactly what the inputs
were. That is exactly what good interlocutors do. The primary reaction
to hearing that someone has a very different view to one's own shouldn't
be to jump to a new credence, it should be to find out why they have the
conflicting view.

In \textbf{Football} it was plausible that the two parties would have
different evidence; in \textbf{Jellybeans} it is just about certain.
{Ankita} and {Bojan} will have had different appearances when they
looked at the jar, they will have seen it from different angles, they
will be bringing different histories with these kinds of estimation
tasks to bear on the subject, and so on. In \textbf{Football} it was
likely that the parties will have different views about the question
because they have different evidence; in \textbf{Jellybeans} it is
practically certain. So the evidence aggregation view says, along with
intuition, that this is a case where they should conciliate. It has a
simpler explanation as to why their credence in hypotheses like 391
should increase than the conciliationist offers, but both parties get to
the right result for plausible reasons.

The case that's left is \textbf{Arithmetic}. This case seems to be the
one that moves people to reject evidentialist views. Cases like
\textbf{Arithmetic} are used as a primary motivation for conciliationist
views of disagreement in, for example
(\citeproc{ref-Bogardus2009}{Bogardus 2009};
\citeproc{ref-Matheson2009}{Matheson 2009};
\citeproc{ref-Carey2011}{Carey 2011}; \citeproc{ref-Kraft2012}{Kraft
2012}; \citeproc{ref-Lee2013}{Lee 2013};
\citeproc{ref-Vavova2014}{Vavova 2014};
\citeproc{ref-Worsnip2014}{Worsnip 2014};
\citeproc{ref-Mogensen2015}{Mogensen 2016}) and Ebeling
(\citeproc{ref-Ebeling2017}{2017}). In many of these papers, intuitions
about cases like \textbf{Arithmetic} are the sole motivation offered for
conciliationism, or are offered as a sufficient reason to believe
conciliationism. Worship says that cases like \textbf{Arithmetic} show
that views like evidence aggregation are ``not even slightly plausible''
~(\citeproc{ref-Worsnip2014}{Worsnip 2014, 6}). Although cases like
\textbf{Arithmetic} are commonly used by conciliationist philosophers,
none of them ever say just what arithmetic problem is under dispute in
their version of the case. The usual methodology s to describe the kind
of arithmetic problem at issue, then present the conflicting answers
that the peers give. I'm using a more concrete example because my
analysis turns on being able to talk about the particular arithmetic
problem under discussion.

The first thing to note about \textbf{Arithmetic} as I've presented it
is that it leaves out some details about how {Ankita} came to her
conclusion. (Remember that the versions offered in the literature are
even lighter on details.) So I'll go over two variants of the case. The
variations will be important enough that I'll introduce new characters
to participate in them. Each character has the same prior relationship
to {Bojan} as {Ankita} does.

{Deanna} thinks to herself that 22 times 18 is 20 times 18 plus 2 times
18, so it is 360 plus 36, so it is 396. That strikes her as conclusive,
so she announces that it is 396. {Bojan} then says he thinks 22 times 18
is 386. So {Deanna} decides to double check. She thinks that 22 times 18
is 20 plus 2 times 20 minus 2, so it is 20 squared minus 2 squared, so
it is 400 minus 4, so it is 396. She now feels confident sticking to her
original verdict.

{Efrosyni} thinks to herself that 22 times 18 is 20 times 18 plus 2
times 18, so it is 360 plus 36, so it is 396. But she feels she should
double check. So she thinks that 22 times 18 is 20 plus 2 times 20 minus
2, so it is 20 squared minus 2 squared, so it is 400 minus 4, so it is
396. She now feels confident sticking to her original verdict. She then
hears {Bojan} say that he thinks 22 times 18 is 386.

Whatever one's view about how confident {Deanna} and {Efrosyni} should
end up being in their verdict that 22 times 18 is 396, they should be
equally confident. After all, they have exactly the same evidence for
and against it: two calculations that point to 396, and {Bojan}`s
announcement of 386. But no form of conciliationism can deliver that
result. After all, conciliationism requires that a form of independence
hold.\footnote{The discussion of Lord's work above wasn't meant to
  undermine that claim. The result of that discussion was that
  conciliationism is equivalent to the strongest plausible independence
  principle, so that principle can't be used to independently defend
  conciliationism. That's all consistent with saying that
  conciliationism requires an independence principle.} The reasoning
that led to one's disagreeing views cannot be used to 're-check' that
those views are correct. So once {Efrosyni} hears {Bojan}'s
disagreement, she can't rely on either of the two routes to the
conclusion that she used. But {Deanna} is free to use the second
calculation she did as independent evidence that {Bojan} is wrong. So
the standard conciliationist has to say, falsely, that {Deanna} and
{Efrosyni} should have different credences in the proposition that 22
times 18 is 396, or, equally falsely, that {Deanna} doesn't get any
extra reason to believe that 22 times 18 is 396 when she does the
double-check.

The evidence aggregation theory suggests a better analysis of the case.
Consider the state of mind that {Efrosyni} was in when she thought,
``I'd better double check this.'' She actually had conclusive,
entailing, evidence that 22 times 18 was 396. Of course, everyone has
just the same evidence at all times, so perhaps that isn't so important.
What is more important is that after doing the first calculation she had
evidence that a reasonable person could, other things being equal, base
a belief on. {Deanna} was not unreasonable when she made her
announcement, but yet {Efrosyni} in a similar position thought she
should get more information. How should we explain that? We could treat
this as a case where a kind of permissivism is right; {Deanna} was being
reasonable in ending inquiry, and {Efrosyni} was reasonable in not
ending it, despite their being in identical positions. But it is better
to treat these cases as not quite identical. {Efrosyni} had a nagging
doubt, which {Deanna} did not have. Perhaps that is the difference; the
calculations they had both done are sufficient to end inquiry in the
absence of positive reasons to extend inquiry. A nagging doubt like
{Efrosyni} had is reasonable, and if one has such a doubt, one has a
reason to address it. But it is also reasonable to not have such a
doubt.

If that story is right, then the evidence aggregation theorist can
easily say what's going on in \textbf{Arithmetic}. If {Ankita} is like
{Deanna}, then the exchange with {Bojan} provides a good reason to
recheck her calculations. The idea here is that the evidence {Ankita}
acquired by doing the calculation is good enough to close inquiry, but
only in the absence of positive reason to keep the inquiry going. That
reason could be internal, a nagging doubt, or it could be external, such
as peer disagreement. So it is fine for an evidential aggregation
theorist to say that {Deanna} (or Ankita if she is like her) should not
necessarily conciliate, but should re-open inquiry. If {Ankita} is like
{Efrosyni}, then the evidence aggregation theorist can't make that move.
But she shouldn't want to. After all, {Efrosyni} has just as good reason
to believe 22 times 18 is 396 as {Deanna} did after rechecking. So she
has excellent reason to believe that 22 times 18 is 396. So she should
keep believing it. It is much more plausible that {Bojan} made a rare
mistake than that she made distinct mistakes on distinct calculations
that ended up at the same point.

As stated, \textbf{Arithmetic} is not detailed enough for us to know
what {Ankita} should do or believe. The advantage of the evidence
aggregation theory is that it can explain why the missing details
matter. Probably the most intuitive way to fill in the details in the
original case is to make {Ankita} like {Deanna}; she does the
calculation once, and easily could have a reason to double-check, but
does not do this. In this case we can say {Bojan}'s disagreement should
prompt {Ankita} to double-check. So we can explain the case that was
meant to be the best case for conciliationism. And, if one thinks the
differences between {Deanna}'s case and Efrosnyi's case needs to be
explained, the evidence aggregation theory can explain them even more
smoothly than the conciliationist can. So there is no argument from
intuitions about cases for conciliationism, and if any side is favoured
by considerations about cases, it is the evidence aggregation theory.

\bookmarksetup{startatroot}

\chapter{Epilogue}\label{epilogue}

I've argued at length against the idea that conformity to one's own
principles is a core part of ethics or epistemology. One should conform
to good principles. If one's own principles are good, then one should
conform to them. But that's because they are good, not because they are
one's own.

One running theme of the book has been that the idea that we should
conform to our principles leads to regresses. Philosophers like the idea
that people should conform to their own principles because this often
provides more useful, more actionable, advice than the idea that people
should do what is right. But it isn't always more useful. Just as one
might not know what the true principles are, one might not know how to
apply principles one has chosen or adopted. So even if what matters is
conformity to one's own principles, we can have disputes over who lives
up to that standard. And if we want people to only be bound by
constraints they can appreciate, indeed if that's why we thought
conformity to one's own principles was so important, then we'll have to
say that what matters is conformity to one's own judgment about what
one's own principles requires. And now we're past the point at which
subjectivism becomes implausible.

Another theme has been that the internalist wants beliefs to play
philosophical roles that only desires are fit to play. The prudent
person will perform acts that they believe will have consequences that
they actually desire. They won't, in general, perform acts that they
believe will have consequences that they believe they desire, or that
they believe to be desirable, or that they believe to be valuable. All
of these theories of prudence have only beliefs, and not desires,
determine what is a prudent act, and hence are vulnerable to the
technical objections to desire as belief theories. The moral person will
desire things that are actually good. I think this means they will have
a vast plurality of desires: to treat others with respect, to promote
the general good, to keep their promises and contracts, and so on. What
makes them moral is not that they have one desire, to do the good, plus
some beliefs about what the good consists in; it is that they have the
right desires. Similarly, the rational person will not have just one
inferential disposition: to move from \emph{it is rational to} ɸ
\emph{in my circumstances} to realizing ɸ. There is an internalist
picture that this, plus very rich beliefs about what rationality
consists in, is all the rational person needs. But this is not how the
rational person should operate. Rather, they will have any number of
distinct dispositions, corresponding to the various ways in which
rationality requires one to react to different situations.

A final theme is more ironic. Internalism is often promoted as the
theory that gives us moderation and caution. Some internalists in ethics
describe their view as `moral hedging'. Internalism in epistemology is
motivated by cases like Christensen's medical resident, and disagreeing
peers moving from extreme views to suspension of judgment. But nothing
in the internalist's theory entails that they will always be on the side
of moderation and caution. Indeed, a running theme of this part of the
book has been the epistemological internalist will end up taking the
extreme position in any number of cases. And internalism in ethics only
makes sense if you think the good agent has as a primary aim, or perhaps
as a sole aim, to do what is right by their own lights. And that is not
a recipe for moderation and caution. Rather, it is the characteristic of
that most immoderate and reckless figure: the self-righteous ideologue.

I don't object to aiming for caution and moderation in one's theory. But
a lesson of the examples we've thought about in this book is that this
must be inserted in first-order theory, not as the internalist wants to
do in the meta-theory. My first-order suggestions are that we are
thoroughly pluralist in our theory of value, and allow that mathematical
investigation is a way of acquiring evidence, not processing it. But I'm
less committed to those particular suggestions than I am to the view
that imitating an ideologue is a bad way to promote moderation.

Amia Srinivasan ends her excellent paper ``Normativity without Cartesian
Privilege'' by noting that her view, one that I'd call externalitist,
``invites us to return to a more tragic outlook of the normative.''
~(\citeproc{ref-Srinivasan2015}{Srinivasan 2015a, 287}). But that tragic
outlook, she argues, can be beneficial; it helps focus on injustices in
practice rather than injustices in theory.

The worldview motivating this book is very similar. Reflection on what
makes tragic figures tragic is a good way to appreciate this worldview.
(There is a reason I started this book by quoting Shakespeare.) And the
misguided ideologue, the person who governs their thoughts and deeds by
the theory they think is right, but in fact is off in one key respect,
is one of the great tragic figures of modernity. What might have been a
minor flaw in an average person becomes, in the ideologue, a character
defining vice.

We should avoid that tragic end. We should try to live well and, if our
minds turn to theory, we should try to have true beliefs about what it
is to live well. If all goes perfectly, there will be a pleasing harmony
between how we live and how we think one should live. But aiming for
that harmony is dangerous, and changing our lives to guarantee it can
bring more harm than good. And we should reject philosophical theories
that draw conclusions about morality or rationality from giving that
harmony too exalted a place.

\bookmarksetup{startatroot}

\chapter*{References}\label{references}
\addcontentsline{toc}{chapter}{References}

\markboth{References}{References}

\phantomsection\label{refs}
\begin{CSLReferences}{1}{0}
\bibitem[\citeproctext]{ref-Adams1985}
Adams, Robert Merrihew. 1985. {``Involuntary Sins.''}
\emph{Philosophical Review} 94 (1): 3--31.
\url{https://doi.org/10.2307/2184713}.

\bibitem[\citeproctext]{ref-Adler2002}
Adler, Jonathan E. 2002. {``Akratic Believing?''} \emph{Philosophical
Studies} 110: 1--27. \url{https://doi.org/10.1023/A:1019823330245}.

\bibitem[\citeproctext]{ref-Alexander2011}
Alexander, David. 2011. {``In Defence of Epistemic Circularity.''}
\emph{Acta Analytica} 26: 223--41.
\url{https://doi.org/10.1007/s12136-010-0100-2}.

\bibitem[\citeproctext]{ref-Allais1953}
Allais, M. 1953. {``Le Comportement de l'homme Rationnel Devant Le
Risque: Critique Des Postulats Et Axiomes de l'ecole Americaine.''}
\emph{Econometrica} 21 (4): 503--46.
\url{https://doi.org/10.2307/1907921}.

\bibitem[\citeproctext]{ref-Arntzenius2003}
Arntzenius, Frank. 2003. {``Some Problems for Conditionalization and
Reflection.''} \emph{Journal of Philosophy} 100 (7): 356--70.
\url{https://doi.org/10.5840/jphil2003100729}.

\bibitem[\citeproctext]{ref-Arpaly2003}
Arpaly, Nomy. 2003. \emph{Unprincipled Virtue}. Oxford: Oxford
University Press.

\bibitem[\citeproctext]{ref-ArpalySchroeder2014}
Arpaly, Nomy, and Timothy Schroeder. 2014. \emph{In Praise of Desire}.
Oxford: Oxford University Press.

\bibitem[\citeproctext]{ref-BallantyneCoffman2011}
Ballantyne, Nathan, and E. J. Coffman. 2011. {``Uniqueness, Evidence and
Rationality.''} \emph{Philosophers' Imprint} 11: 1--13.
\url{http://hdl.handle.net/2027/spo.3521354.0011.018}.

\bibitem[\citeproctext]{ref-BallantyneCoffman2012}
---------. 2012. {``Conciliationism and Uniqueness.''}
\emph{Australasian Journal of Philosophy} 90: 657--70.
\url{https://doi.org/10.1080/00048402.2011.627926}.

\bibitem[\citeproctext]{ref-Barnett2014}
Barnett, David James. 2014. {``What's the Matter with Epistemic
Circularity?''} \emph{Philosophical Studies} 171 (2): 177--205.
\url{https://doi.org/10.1007/s11098-013-0261-0}.

\bibitem[\citeproctext]{ref-Barnett2015}
---------. 2015. {``Is Memory Merely Testimony from One's Former
Self?''} \emph{Philosophical Review} 124 (3): 353--92.
\url{https://doi.org/10.1215/00318108-2895337}.

\bibitem[\citeproctext]{ref-BasuSchroeder2018}
Basu, Rima, and Mark Schroeder. forthcoming. {``Doxastic Wrongings.''}
In \emph{Pragmatic Encroachment in Epistemology}, edited by Brian Kim
and Matthew McGrath. Routledge.

\bibitem[\citeproctext]{ref-Bogardus2009}
Bogardus, Tomas. 2009. {``A Vindication of the Equal-Weight View.''}
\emph{Episteme} 6 (3): 324--35.
\url{https://doi.org/10.3366/E1742360009000744}.

\bibitem[\citeproctext]{ref-Boghossian2003}
Boghossian, Paul. 2003. {``Blind Reasoning.''} \emph{Proceedings of the
Aristotelian Society, Supplementary Volume} 77 (1): 225--48.
\url{https://doi.org/10.1111/1467-8349.00110}.

\bibitem[\citeproctext]{ref-BonjourSosa}
BonJour, Laurence, and Ernest Sosa. 2003. \emph{Epistemic Justification:
Internalism Vs. Externalism, Foundations Vs. Virtues}. Great Debates in
Philosophy. Malden, MA: Blackwell.

\bibitem[\citeproctext]{ref-Bostrom2003}
Bostrom, Nick. 2003. {``Are You Living in a Computer Simulation?''}
\emph{Philosophical Quarterly} 53 (211): 243--55.
\url{https://doi.org/10.1111/1467-9213.00309}.

\bibitem[\citeproctext]{ref-Brown2011-BROCT}
Brown, Campbell. 2011. {``Consequentialize This.''} \emph{Ethics} 121
(4): 749--71. \url{https://doi.org/10.1086/660696}.

\bibitem[\citeproctext]{ref-BuchakRisk}
Buchak, Lara. 2013. \emph{Risk and Rationality}. Oxford: Oxford
University Press.

\bibitem[\citeproctext]{ref-Buchak2013}
---------. 2014. {``Belief, Credence and Norms.''} \emph{Philosophical
Studies} 169 (2): 285--311.
\url{https://doi.org/10.1007/s11098-013-0182-y}.

\bibitem[\citeproctext]{ref-BurnsSwedlow2003}
Burns, Jeffrey M, and Russell H Swerdlow. 2003. {``Right Orbitofrontal
Tumor with Pedophilia Symptom and Constructional Apraxia Sign.''}
\emph{Archives of Neurology} 60 (3): 437--40.
\url{https://doi.org/10.1001/archneur.60.3.437}.

\bibitem[\citeproctext]{ref-Calhoun1989}
Calhoun, Cheshire. 1989. {``Responsibility and Reproach.''}
\emph{Ethics} 99 (2): 389--406. \url{https://doi.org/10.1086/293071}.

\bibitem[\citeproctext]{ref-CappelenDever2014}
Cappelen, Herman, and Josh Dever. 2014. \emph{The Inessential
Indexical}. Oxford: Oxford University Press.

\bibitem[\citeproctext]{ref-Carey2011}
Carey, Brandon. 2011. {``Possible Disagreements and Defeat.''}
\emph{Philosophical Studies} 155 (3): 371--81.
\url{https://doi.org/10.1007/s11098-010-9581-5}.

\bibitem[\citeproctext]{ref-Carroll1895}
Carroll, Lewis. 1895. {``What the Tortoise Said to Achilles.''}
\emph{Mind} 4 (14): 278--80.
\url{https://doi.org/10.1093/mind/iv.14.278}.

\bibitem[\citeproctext]{ref-ChoKreps1987}
Cho, In-Koo, and David M. Kreps. 1987. {``Signalling Games and Stable
Equilibria.''} \emph{The Quarterly Journal of Economics} 102 (2):
179--221. \url{https://doi.org/10.2307/1885060}.

\bibitem[\citeproctext]{ref-Christensen2005}
Christensen, David. 2005. \emph{Putting Logic in Its Place}. Oxford:
Oxford University Press.

\bibitem[\citeproctext]{ref-Christensen2007a}
---------. 2007a. {``Does Murphy's Law Apply in Epistemology? Self-Doubt
and Rational Ideals.''} \emph{Oxford Studies in Epistemology} 2: 3--31.

\bibitem[\citeproctext]{ref-Christensen2007c}
---------. 2007b. {``Epistemology of Disagreement: The Good News.''}
\emph{Philosophical Review} 116 (2): 187--217.
\url{https://doi.org/10.1215/00318108-2006-035}.

\bibitem[\citeproctext]{ref-Christensen2009}
---------. 2009. {``Disagreement as Evidence: The Epistemology of
Controversy.''} \emph{Philosophy Compass} 4 (5): 756--67.
\url{https://doi.org/10.1111/j.1747-9991.2009.00237.x}.

\bibitem[\citeproctext]{ref-Christensen2010a}
---------. 2010a. {``Higher-Order Evidence.''} \emph{Philosophy and
Phenomenological Research} 81 (1): 185--215.
\url{https://doi.org/10.1111/j.1933-1592.2010.00366.x}.

\bibitem[\citeproctext]{ref-Christensen2010b}
---------. 2010b. {``Rational Reflection.''} \emph{Philosophical
Perspectives} 24: 121--40.
\url{https://doi.org/10.1111/j.1520-8583.2010.00187.x}.

\bibitem[\citeproctext]{ref-Christensen2011}
---------. 2011. {``Disagreement, Question-Begging and Epistemic
Self-Criticism.''} \emph{Philosophers' Imprint} 11 (6): 1--22.
\url{http://hdl.handle.net/2027/spo.3521354.0011.006}.

\bibitem[\citeproctext]{ref-Christensen2014}
---------. 2016. {``Conciliation, Uniqueness and Rational Toxicity.''}
\emph{No{û}s} 50 (3): 584--603.
\url{https://doi.org/10.1111/nous.12077}.

\bibitem[\citeproctext]{ref-Coady1995}
Coady, C. A. J. 1995. \emph{Testimony: A Philosophical Study}. Oxford:
Clarendon Press.

\bibitem[\citeproctext]{ref-Coates2012}
Coates, Allen. 2012. {``Rational Epistemic Akrasia.''} \emph{American
Philosophical Quarterly} 49 (2): 113--24.

\bibitem[\citeproctext]{ref-Cohen1986}
Cohen, Stewart. 1986. {``Knowledge and Context.''} \emph{The Journal of
Philosophy} 83: 574--83. \url{https://doi.org/10.2307/2026434}.

\bibitem[\citeproctext]{ref-Cohen2002}
---------. 2002. {``Basic Knowledge and the Problem of Easy
Knowledge.''} \emph{Philosophy and Phenomenological Research} 65 (2):
309--29. \url{https://doi.org/10.1111/j.1933-1592.2002.tb00204.x}.

\bibitem[\citeproctext]{ref-Cohen2005}
---------. 2005. {``Why Basic Knowledge Is Easy Knowledge.''}
\emph{Philosophy and Phenomenological Research} 70 (2): 417--30.
\url{https://doi.org/10.1111/j.1933-1592.2005.tb00536.x}.

\bibitem[\citeproctext]{ref-Cohen2013}
---------. 2013. {``A Defence of the (Almost) Equal Weight View.''} In
\emph{The Epistemology of Disagreement: New Essays}, edited by David
Christensen and Jennifer Lackey, 98--117. Oxford: Oxford University
Press. \url{https://doi.org/10.1093/acprof:oso/9780199698370.003.0006}.

\bibitem[\citeproctext]{ref-Conee1992}
Conee, Earl. 1992. {``The Truth Connection.''} \emph{Philosophy and
Phenomenological Research} 52 (3): 657--69.
\url{https://doi.org/10.2307/2108213}.

\bibitem[\citeproctext]{ref-DeRose1995}
DeRose, Keith. 1995. {``Solving the Skeptical Problem.''}
\emph{Philosophical Review} 104: 1--52.
\url{https://doi.org/10.2307/2186011}.

\bibitem[\citeproctext]{ref-Douven2009}
Douven, Igor. 2009. {``Uniqueness Revisited.''} \emph{American
Philosophical Quarterly} 46: 347--61.

\bibitem[\citeproctext]{ref-Dretske2005}
Dretske, Fred. 2005. {``Is Knowledge Closed Under Known Entailment? The
Case Against Closure.''} In \emph{Contemporary Debates in Epistemology},
edited by Matthias Steup and Ernest Sosa, 13--26. Malden, MA: Blackwell.

\bibitem[\citeproctext]{ref-Ebeling2017}
Ebeling, Martin. 2017. \emph{Conciliatory Democracy: From Deliberation
Toward a New Politics of Disagreement}. New York: Palgrave Macmillan.

\bibitem[\citeproctext]{ref-EganElga2005}
Egan, Andy, and Adam Elga. 2005. {``{I Can't Believe I'm Stupid}.''}
\emph{Philosophical Perspectives} 19 (1): 77--93.
\url{https://doi.org/10.1111/j.1520-8583.2005.00054.x}.

\bibitem[\citeproctext]{ref-Elga2007}
Elga, Adam. 2007. {``Reflection and Disagreement.''} \emph{No{û}s} 41
(3): 478--502. \url{https://doi.org/10.1111/j.1468-0068.2007.00656.x}.

\bibitem[\citeproctext]{ref-Elga2008}
---------. 2008. {``Lucky to Be Rational.''}

\bibitem[\citeproctext]{ref-Elga2010}
---------. 2010. {``How to Disagree about How to Disagree.''} In
\emph{Disagreement}, edited by Ted Warfield and Richard Feldman,
175--87. Oxford: Oxford University Press.
\url{https://doi.org/10.1093/acprof:oso/9780199226078.003.0008}.

\bibitem[\citeproctext]{ref-Shapiro2007}
Elizabeth, Princess of Bohemia, and René Descartes. 2007. \emph{The
Correspondence Between Princess Elizabeth of Bohemia and Ren{é}
Descartes}. Translated by Lisa Shapiro. Chicago: University of Chicago
Press.

\bibitem[\citeproctext]{ref-Enoch2010}
Enoch, David. 2010. {``Not Just a Truthometer: Taking Oneself Seriously
(but Not Too Seriosuly) in Cases of Peer Disagreement.''} \emph{Mind}
119 (476): 953--97. \url{https://doi.org/10.1093/mind/fzq070}.

\bibitem[\citeproctext]{ref-FantlMcGrath2009}
Fantl, Jeremy, and Matthew McGrath. 2009. \emph{Knowledge in an
Uncertain World}. Oxford: Oxford University Press.

\bibitem[\citeproctext]{ref-Feldman2007}
Feldman, Richard. 2007. {``Reasonable Religious Disagreements.''} In
\emph{Philosophers Without Gods: Meditations on Atheism and the
Secular}, 194--214. Oxford: Oxford University Press.

\bibitem[\citeproctext]{ref-Field2017}
Field, Claire. forthcoming. {``It's OK to Make Mistakes: Against the
Fixed Point Thesis.''} \emph{Episteme}, forthcoming.
\url{https://doi.org/10.1017/epi.2017.33}.

\bibitem[\citeproctext]{ref-Finnis2011}
Finnis, John. 2011. \emph{Natural Law and Natural Rights}. Second.
Oxford: Oxford University Press.

\bibitem[\citeproctext]{ref-Fitzpatrick2008}
FitzPatrick, William J. 2008. {``Moral Responsibility and Normative
Ignorance: Answering a New Skeptical Challenge.''} \emph{Ethics} 118
(4): 589--613. \url{https://doi.org/10.1086/589532}.

\bibitem[\citeproctext]{ref-Fricker2010}
Fricker, Miranda. 2010. {``The Relativism of Blame and Williams's
Relativism of Distance.''} \emph{Aristotelian Society Supplementary
Volume} 84 (1): 151--77.
\url{https://doi.org/10.1111/j.1467-8349.2010.00190.x}.

\bibitem[\citeproctext]{ref-Fumerton2010}
Fumerton, Richard. 2010. {``You Can't Trust a Philosopher.''} In
\emph{Disagreement}, edited by Ted Warfield and Richard Feldman,
91--110. Oxford: Oxford University Press.
\url{https://doi.org/10.1093/acprof:oso/9780199226078.003.0006}.

\bibitem[\citeproctext]{ref-Galbraith1964}
Galbraith, John Kenneth. 1964. {``Let Us Begin: An Invitation to Action
on Poverty.''} \emph{Harper's}, 16--26.

\bibitem[\citeproctext]{ref-Ganson2008}
Ganson, Dorit. 2008. {``Evidentialism and Pragmatic Constraints on
Outright Belief.''} \emph{Philosophical Studies} 139 (3): 441--58.
\url{https://doi.org/10.1007/s11098-007-9133-9}.

\bibitem[\citeproctext]{ref-Gendler2000}
Gendler, Tamar Szabó. 2000. {``The Puzzle of Imaginative Resistance.''}
\emph{Journal of Philosophy} 97 (2): 55--81.
\url{https://doi.org/10.2307/2678446}.

\bibitem[\citeproctext]{ref-Gettier1963}
Gettier, Edmund L. 1963. {``Is Justified True Belief Knowledge?''}
\emph{Analysis} 23 (6): 121--23. \url{https://doi.org/10.2307/3326922}.

\bibitem[\citeproctext]{ref-Goldman1986}
Goldman, Alvin. 1986. \emph{Epistemology and Cognition}. Cambridge, MA:
Harvard University Press.

\bibitem[\citeproctext]{ref-Goodman1955}
Goodman, Nelson. 1955. \emph{Fact, Fiction and Forecast}. Cambridge:
Harvard University Press.

\bibitem[\citeproctext]{ref-Graham2012}
Graham, Peter A. 2014. {``A Sketch of a Theory of Moral
Blameworthiness.''} \emph{Philosophy and Phenomenological Research} 88
(2): 388--409. \url{https://doi.org/10.1111/j.1933-1592.2012.00608.x}.

\bibitem[\citeproctext]{ref-GreavesOrd2017}
Greaves, Hilary, and Toby Ord. 2017. {``Moral Uncertainty about
Population Axiology.''} \emph{Journal of Ethics and Social Philosophy}
12 (2): 135--67. \url{https://doi.org/10.26556/jesp.v12i2.223}.

\bibitem[\citeproctext]{ref-Greco2014}
Greco, Daniel. 2014. {``A Puzzle about Epistemic Akrasia.''}
\emph{Philosophical Studies} 167: 201--19.
\url{https://doi.org/10.1007/s11098-012-0085-3}.

\bibitem[\citeproctext]{ref-Guerrero2007}
Guerrero, Alexander. 2007. {``Don't Know, Don't Kill: Moral Ignorance,
Culpability and Caution.''} \emph{Philosophical Studies} 136 (1):
59--97. \url{https://doi.org/10.1007/s11098-007-9143-7}.

\bibitem[\citeproctext]{ref-GustafssonTorpman2014}
Gustafsson, Johan E., and Olle Torpman. 2014. {``In Defence of My
Favorite Theory.''} \emph{Pacific Philosophical Quarterly} 95 (2):
159--74. \url{https://doi.org/10.1111/papq.12022}.

\bibitem[\citeproctext]{ref-Harman2011a}
Harman, Elizabeth. 2011. {``Does Moral Ignorance Exculpate?''}
\emph{Ratio} 24 (4): 443--68.
\url{https://doi.org/10.1111/j.1467-9329.2011.00511.x}.

\bibitem[\citeproctext]{ref-Harman2014}
---------. 2015. {``The Irrelevance of Moral Uncertainty.''}
\emph{Oxford Studies in Metaethics} 10: 53--79.
\url{https://doi.org/10.1093/acprof:oso/9780198738695.003.0003}.

\bibitem[\citeproctext]{ref-Harman1986}
Harman, Gilbert. 1986. \emph{Change in View}. Cambridge, MA: Bradford.

\bibitem[\citeproctext]{ref-HawthorneMagidor2009}
Hawthorne, John, and Ofra Magidor. 2009. {``Assertion, Context, and
Epistemic Accessibility.''} \emph{Mind} 118 (470): 377--97.
\url{https://doi.org/10.1093/mind/fzp060}.

\bibitem[\citeproctext]{ref-HawthorneMagidor2011}
---------. 2011. {``Assertion and Epistemic Opacity.''} \emph{Mind} 119
(476): 1087--1105. \url{https://doi.org/10.1093/mind/fzq093}.

\bibitem[\citeproctext]{ref-HawthorneSrinivasan2013}
Hawthorne, John, and Amia Srinivasan. 2013. {``Disagreement Without
Transparency: Some Bleak Thoughts.''} In \emph{The Epistemology of
Disagreement: New Essays}, edited by David Christensen and Jennifer
Lackey, 9--30. Oxford: Oxford University Press.
\url{https://doi.org/10.1093/acprof:oso/9780199698370.003.0002}.

\bibitem[\citeproctext]{ref-HeBolzBaillargeon2011}
He, Zijing, Matthias Bolz, and Renée Baillargeon. 2011. {``False-Belief
Understanding in 2.5-Year-Olds: Evidence from Violation-of-Expectation
Change-of-Location and Unexpected-Contents Tasks.''} \emph{Developmental
Science} 14 (2): 292--305.
\url{https://doi.org/10.1111/j.1467-7687.2010.00980.x}.

\bibitem[\citeproctext]{ref-Hedden2016}
Hedden, Brian. 2016b. {``Does MITE Make Right? On Decision-Making Under
Normative Uncertainty.''} \emph{Oxford Studies in Metaethics} 11:
102--28.
\url{https://doi.org/10.1093/acprof:oso/9780198784647.001.0001}.

\bibitem[\citeproctext]{ref-Hedden2015}
---------. 2016a. {``Does MITE Make Right? On Decision-Making Under
Normative Uncertainty.''} \emph{Oxford Studies in Metaethics} 11:
102--28.
\url{https://doi.org/10.1093/acprof:oso/9780198784647.001.0001}.

\bibitem[\citeproctext]{ref-Holton1999}
Holton, Richard. 1999. {``Intention and Weakness of Will.''} \emph{The
Journal of Philosophy} 96 (5): 241--62.
\url{https://doi.org/10.2307/2564667}.

\bibitem[\citeproctext]{ref-Holton2014}
---------. 2014. {``Intention as a Model for Belief.''} In
\emph{Rational and Social Agency: Essays on the Philosophy of Michael
Bratman}, edited by Manuel Vargas and Gideon Yaffe, 12--37. Oxford:
Oxford University Press.

\bibitem[\citeproctext]{ref-Hookway2001}
Hookway, Christopher. 2001. {``Epistemic Akrasia and Epistemic
Virtue.''} In \emph{Virtue Epistemology: Essays on Epistemic Virtue and
Responsibility}, edited by Abrol Fairweather and Linda Trinkaus
Zagzebski, 178--99. Oxford: Oxford University Press.

\bibitem[\citeproctext]{ref-Horowitz2014}
Horowitz, Sophie. 2014. {``Epistemic Akrasia.''} \emph{No{û}s} 48 (4):
718--44. \url{https://doi.org/10.1111/nous.12026}.

\bibitem[\citeproctext]{ref-Hurley1989}
Hurley, Susan. 1989. \emph{Natural Reasons}. Oxford: Oxford University
Press.

\bibitem[\citeproctext]{ref-Ichikawa2009}
Ichikawa, Jonathan. 2009. {``Explaining Away Intuitions.''} \emph{Studia
Philosophica Estonica} 2 (2): 94--116.
\url{https://doi.org/10.12697/spe.2009.2.2.06}.

\bibitem[\citeproctext]{ref-IchikawaJarvis2009}
Ichikawa, Jonathan, and Benjamin Jarvis. 2009. {``Thought-Experiment
Intuitions and Truth in Fiction.''} \emph{Philosophical Studies} 142
(2): 221--46. \url{https://doi.org/10.1007/s11098-007-9184-y}.

\bibitem[\citeproctext]{ref-Jackson1977}
Jackson, Frank. 1977. \emph{Perception: A Reprsentative Theory}.
Cambridge: Cambridge University Press.

\bibitem[\citeproctext]{ref-Jackson1987}
---------. 1987. \emph{Conditionals}. Blackwell: Oxford.

\bibitem[\citeproctext]{ref-Jackson1991}
---------. 1991. {``Decision Theoretic Consequentialism and the Nearest
and Dearest Objection.''} \emph{Ethics} 101: 461--82.
\url{https://doi.org/10.1086/293312}.

\bibitem[\citeproctext]{ref-Jeffrey1983}
Jeffrey, Richard C. 1983. \emph{The Logic of Decision}. 2nd ed. Chicago:
University of Chicago Press.

\bibitem[\citeproctext]{ref-JehleFitelson2009}
Jehle, David, and Branden Fitelson. 2009. {``What Is the {`Equal Weight
View'}?''} \emph{Episteme} 6 (3): 280--93.
\url{https://doi.org/10.3366/E1742360009000719}.

\bibitem[\citeproctext]{ref-Keller2009}
Keller, Simon. 2009. {``Welfare as Success.''} \emph{No{û}s} 43 (4):
656--83. \url{https://doi.org/10.1111/j.1468-0068.2009.00723.x}.

\bibitem[\citeproctext]{ref-Kelly2005}
Kelly, Thomas. 2005. {``The Epistemic Significance of Disagreement.''}
\emph{Oxford Studies in Epistemology} 1: 167--96.

\bibitem[\citeproctext]{ref-Kelly2010}
---------. 2010. {``Peer Disagreement and Higher Order Evidence.''} In
\emph{Disagreement}, edited by Ted Warfield and Richard Feldman,
111--74. Oxford: Oxford University Press.
\url{https://doi.org/10.1093/acprof:oso/9780199226078.003.0007}.

\bibitem[\citeproctext]{ref-KleinSEP}
Klein, Peter. 2015. {``Skepticism.''} In \emph{The Stanford Encyclopedia
of Philosophy}, edited by Edward N. Zalta, Summer 2015. Metaphysics
Research Lab, Stanford University.
\url{http://plato.stanford.edu/archives/sum2015/entries/skepticism/}.

\bibitem[\citeproctext]{ref-Kolodny2005}
Kolodny, Niko. 2005. {``Why Be Rational?''} \emph{Mind} 114 (455):
509--63. \url{https://doi.org/10.1093/mind/fzi509}.

\bibitem[\citeproctext]{ref-Kraft2012}
Kraft, James. 2012. \emph{The Epistemology of Religious Disagreement: A
Better Understanding}. New York: Palgrave Macmillan.

\bibitem[\citeproctext]{ref-Lackey2010}
Lackey, Jennifer. 2010. {``What Should We Do When We Disagree.''}
\emph{Oxford Studies in Epistemology} 3: 274--93.

\bibitem[\citeproctext]{ref-Lam2011}
Lam, Barry. 2011. {``On the Rationality of Belief-Invariance in Light of
Peer Disagreement.''} \emph{Philosophical Review} 120: 207--45.
\url{https://doi.org/10.1215/00318108-2010-028}.

\bibitem[\citeproctext]{ref-Lasonen-Aarnio2010}
Lasonen-Aarnio, Maria. 2010a. {``Is There a Viable Account of
Well-Founded Belief.''} \emph{Erkenntnis} 72 (2): 205--31.
\url{https://doi.org/10.1007/s10670-009-9200-z}.

\bibitem[\citeproctext]{ref-Lasonen-Aarnio2010b}
---------. 2010b. {``Unreasonable Knowledge.''} \emph{Philosophical
Perspectives} 24: 1--21.
\url{https://doi.org/10.1111/j.1520-8583.2010.00183.x}.

\bibitem[\citeproctext]{ref-Lasonen-Aarnio2013}
---------. 2013. {``Disagreement and Evidential Attenuation.''}
\emph{No{û}s} 47 (4): 767--94. \url{https://doi.org/10.1111/nous.12050}.

\bibitem[\citeproctext]{ref-Lasonen-Aarnio2014}
---------. 2014a. {``Higher-Order Evidence and the Limits of Defeat.''}
\emph{Philosophy and Phenomenological Research} 88 (2): 314--45.
\url{https://doi.org/10.1111/phpr.12090}.

\bibitem[\citeproctext]{ref-Lasonen-Aarnio2014a}
---------. 2014b. {``The Dogmastism Puzzle.''} \emph{Australasian
Journal of Philosophy} 92 (3): 417--32.
\url{https://doi.org/10.1080/00048402.2013.834949}.

\bibitem[\citeproctext]{ref-Lee2013}
Lee, Matthew. 2013. {``Conciliationism Without Uniqueness.''}
\emph{Grazer Philosophische Studien} 88 (1): 161--88.

\bibitem[\citeproctext]{ref-Levinstein2013}
Levinstein, Ben. 2013. {``Accuracy as Epistemic Utility.''} PhD thesis,
Rutgers University.

\bibitem[\citeproctext]{ref-Levy2005}
Levy, Neil. 2005. {``The Good, the Bad and the Blameworthy.''}
\emph{Journal of Ethics and Social Philosophy} 1 (2): 1--16.
\url{https://doi.org/10.26556/jesp.v1i2.6}.

\bibitem[\citeproctext]{ref-Levy2009}
---------. 2009. {``Culpable Ignorance and Moral Responsibility: A Reply
to FitzPatrick.''} \emph{Ethics} 119 (4): 729--41.
\url{https://doi.org/10.1086/605018}.

\bibitem[\citeproctext]{ref-Lewis1973a}
Lewis, David. 1973. \emph{Counterfactuals}. Oxford: Blackwell
Publishers.

\bibitem[\citeproctext]{ref-Lewis1978b}
---------. 1978. {``Truth in Fiction.''} \emph{American Philosophical
Quarterly} 15 (1): 37--46.

\bibitem[\citeproctext]{ref-Lewis1979}
---------. 1979. {``Attitudes \emph{de Dicto} and \emph{de Se}.''}
\emph{Philosophical Review} 88 (4): 513--43.
\url{https://doi.org/10.2307/2184843}.

\bibitem[\citeproctext]{ref-Lewis1982c}
---------. 1982. {``Logic for Equivocators.''} \emph{No{û}s} 16 (3):
431--41. \url{https://doi.org/10.2307/2216219}.

\bibitem[\citeproctext]{ref-Lewis1988b}
---------. 1988. {``Desire as Belief.''} \emph{Mind} 97 (387): 323--32.
\url{https://doi.org/10.1093/mind/XCVII.387.323}.

\bibitem[\citeproctext]{ref-Lewis1994b}
---------. 1994. {``Reduction of Mind.''} In \emph{A Companion to the
Philosophy of Mind}, edited by Samuel Guttenplan, 412--31. Oxford:
Blackwell. \url{https://doi.org/10.1017/CBO9780511625343.019}.

\bibitem[\citeproctext]{ref-Lewis1996a}
---------. 1996a. {``Desire as Belief {II}.''} \emph{Mind} 105 (418):
303--13. \url{https://doi.org/10.1093/mind/105.418.303}.

\bibitem[\citeproctext]{ref-Lewis1996b}
---------. 1996b. {``Elusive Knowledge.''} \emph{Australasian Journal of
Philosophy} 74 (4): 549--67.
\url{https://doi.org/10.1080/00048409612347521}.

\bibitem[\citeproctext]{ref-Lillehammer1997}
Lillehammer, Hallvard. 1997. {``Smith on Moral Fetishism.''}
\emph{Analysis} 57 (3): 187--95.
\url{https://doi.org/10.1111/1467-8284.00073}.

\bibitem[\citeproctext]{ref-Linton2013}
Linton, Marisa. 2013. \emph{Choosing Terror: Virtue, Friendship, and
Authenticity in the French Revolution}. Oxford: {O}xford {U}niversity
{P}ress.

\bibitem[\citeproctext]{ref-LipseyLancaster}
Lipsey, R. G., and Kelvin Lancaster. 1956-1957. {``The General Theory of
Second Best.''} \emph{Review of Economic Studies} 24 (1): 11--32.
\url{https://doi.org/10.2307/2296233}.

\bibitem[\citeproctext]{ref-Littlejohn2012}
Littlejohn, Clayton. 2012. \emph{Justification and the
Truth-Connection}. Cambridge: Cambridge University Press.

\bibitem[\citeproctext]{ref-Littlejohn2015}
---------. 2018. {``Stop Making Sense? On a Puzzle about Rationality.''}
\emph{Philosophy and Phenomenological Research} 96 (2): 257--72.
\url{https://doi.org/10.1111/phpr.12271}.

\bibitem[\citeproctext]{ref-Lockhart2000}
Lockhart, Ted. 2000. \emph{Moral Uncertainty and Its Consequences}.
Oxford University Press.

\bibitem[\citeproctext]{ref-Loewer1985}
Loewer, Barry, and Robert Laddaga. 1985. {``Destroying the Consensus.''}
\emph{Synthese} 62 (1): 79--95.
\url{https://doi.org/10.1007/BF00485388}.

\bibitem[\citeproctext]{ref-Lord2014}
Lord, Errol. 2014. {``From Independence to Conciliationism: An
Obituary.''} \emph{Australasian Journal of Philosophy} 92 (2): 365--77.
\url{https://doi.org/10.1080/00048402.2013.829506}.

\bibitem[\citeproctext]{ref-MacAskillThesis}
MacAskill, William. 2014. {``Normative Uncertainty.''} PhD thesis,
Oxford University.

\bibitem[\citeproctext]{ref-MacAskill2016}
---------. 2016. {``Normative Uncertainty as a Voting Problem.''}
\emph{Mind} 125 (500): 967--1004.
\url{https://doi.org/10.1093/mind/fzv169}.

\bibitem[\citeproctext]{ref-Machuca2013}
Machuca, Diego E. 2013. {``A Neo-Pyrrhonian Approach to the Epistemology
of Disagreement.''} In \emph{Disagreement and Skepticism}, edited by
Diego E. Machuca, 66--89. New York: Routledge.

\bibitem[\citeproctext]{ref-Mahajan2010}
Mahajan, Sanjoy. 2010. \emph{Street-Fighting Mathematics: The Art of
Educated Guessing and Opportunistic Problem Solving}. Second. Cambridge,
MA: {MIT} Press.

\bibitem[\citeproctext]{ref-Maher1997}
Maher, Patrick. 1997. {``Depragmatized Dutch Book Arguments.''}
\emph{Philosophy of Science} 64: 291--305.
\url{https://doi.org/10.1086/392552}.

\bibitem[\citeproctext]{ref-WeathersonMaitra2010}
Maitra, Ishani, and Brian Weatherson. 2010. {``Assertion, Knowledge and
Action.''} \emph{Philosophical Studies} 149 (1): 99--118.
\url{https://doi.org/10.1007/s11098-010-9542-z}.

\bibitem[\citeproctext]{ref-Markovits2010}
Markovits, Julia. 2010. {``Acting for the Right Reasons.''}
\emph{Philosophical Review} 119 (2): 201--42.
\url{https://doi.org/10.1215/00318108-2009-037}.

\bibitem[\citeproctext]{ref-Markovits2014}
---------. 2014. \emph{Moral Reason}. Oxford: Oxford University Press.

\bibitem[\citeproctext]{ref-Mason2015}
Mason, Elinor. 2015. {``Moral Ignorance and Blameworthiness.''}
\emph{Philosophical Studies} 172 (11): 3037--57.
\url{https://doi.org/10.1007/s11098-015-0456-7}.

\bibitem[\citeproctext]{ref-Matheson2009}
Matheson, Jonathan. 2009. {``Conciliatory Views of Disagreement and
Higher-Order Evidence.''} \emph{Episteme} 6 (3): 269--79.
\url{https://doi.org/10.3366/E1742360009000707}.

\bibitem[\citeproctext]{ref-McKie1998}
McKie, John, Peter Singer, Helga Kuhse, and Jeff Richardson. 1998.
\emph{The Allocation of Health Care Resources: An Ethical Evaluation of
the 'QALY' Approach}. Aldergate: Ashgate.

\bibitem[\citeproctext]{ref-McPhee2012}
McPhee, Peter. 2012. \emph{Robespierre: A Revolutionary Life}. New
Haven: Yale University Press.

\bibitem[\citeproctext]{ref-Mogensen2015}
Mogensen, Andreas L. 2016. {``Contingency Anxiety and the Epistemology
of Disagreement.''} \emph{Pacific Philosophical Quarterly} 97 (4):
590--611. \url{https://doi.org/10.1111/papq.12099}.

\bibitem[\citeproctext]{ref-Moller2011}
Moller, D. 2011. {``Abortion and Moral Risk.''} \emph{Philosophy} 86
(3): 425--43. \url{https://doi.org/10.1017/S0031819111000222}.

\bibitem[\citeproctext]{ref-MoodyAdams1994}
Moody-Adams, Michelle M. 1994. {``Culture, Responsibility and Affected
Ignorance.''} \emph{Ethics} 104 (2): 291--309.
\url{https://doi.org/10.1086/293601}.

\bibitem[\citeproctext]{ref-Moore1903}
Moore, G. E. 1903. \emph{Principia Ethica}. Cambridge: Cambridge
University Press.

\bibitem[\citeproctext]{ref-Morison2014}
Morison, Benjamin. 2014. {``Sextus Empiricus.''} In \emph{The Stanford
Encyclopedia of Philosophy}, edited by Edward N. Zalta, Spring 2014.
Metaphysics Research Lab, Stanford University.
\url{http://plato.stanford.edu/archives/spr2014/entries/sextus-empiricus/}.

\bibitem[\citeproctext]{ref-Moss2011}
Moss, Sarah. 2011. {``Scoring Rules and Epistemic Compromise.''}
\emph{Mind} 120 (480): 1053--69.
\url{https://doi.org/10.1093/mind/fzs007}.

\bibitem[\citeproctext]{ref-Moss2012}
---------. 2012. {``Updating as Communication.''} \emph{Philosophy and
Phenomenological Research} 85 (2): 225--48.
\url{https://doi.org/10.1111/j.1933-1592.2011.00572.x}.

\bibitem[\citeproctext]{ref-Moss2015}
---------. 2015. {``Time-Slice Epistemology and Action Under
Indeterminacy.''} \emph{Oxford Studies in Epistemology} 5: 172--94.
\url{https://doi.org/10.1093/acprof:oso/9780198722762.003.0006}.

\bibitem[\citeproctext]{ref-Nagel2013}
Nagel, Jennifer. 2013. {``Defending the Evidential Value of Epistemic
Intuitions: A Reply to Stich.''} \emph{{P}hilosophy and
{P}henomenological {R}esearch} 86 (1): 179--99.
\url{https://doi.org/10.1111/phpr.12008}.

\bibitem[\citeproctext]{ref-Nagel2014}
---------. 2014. \emph{Knowledge: A Very Short Introduction}. Oxford:
Oxford University Press.

\bibitem[\citeproctext]{ref-NissanRozen2015}
Nissan-Rozen, Ittay. 2015. {``Against Moral Hedging.''} \emph{Economics
and Philosophy} 31 (3): 349--69.
\url{https://doi.org/10.1017/S0266267115000206}.

\bibitem[\citeproctext]{ref-Nozick1969}
Nozick, Robert. 1969. {``Newcomb's Problem and Two Principles of
Choice.''} In \emph{Essays in Honor of Carl g. Hempel: A Tribute on the
Occasion of His Sixty-Fifth Birthday}, edited by Nicholas Rescher,
114--46. Riedel: Springer.

\bibitem[\citeproctext]{ref-Nozick1981}
---------. 1981. \emph{Philosophical Explorations}. Cambridge, MA:
Harvard University Press.

\bibitem[\citeproctext]{ref-Nozick1994}
---------. 1994. \emph{The Nature of Rationality}. Princeton: Princeton
University Press.

\bibitem[\citeproctext]{ref-Owens2002}
Owens, David. 2002. {``Epistemic Akrasia.''} \emph{The Monist} 85 (3):
381--97.

\bibitem[\citeproctext]{ref-Palmer1941}
Palmer, R. R. 1941. \emph{Twelve Who Ruled}. Princeton, NJ: Princeton
University Press.

\bibitem[\citeproctext]{ref-Parfit1984}
Parfit, Derek. 1984. \emph{Reasons and Persons}. Oxford: Clarendon
Press.

\bibitem[\citeproctext]{ref-Peels2010}
Peels, Rik. 2010. {``What Is Ignorance?''} \emph{Philosophia} 38 (1):
57--67. \url{https://doi.org/10.1007/s11406-009-9202-8}.

\bibitem[\citeproctext]{ref-PettitSugden1989}
Pettit, Philip, and Robert Sugden. 1989. {``The Backward Induction
Paradox.''} \emph{Journal of Philosophy} 86 (4): 169--82.
\url{https://doi.org/10.2307/2026960}.

\bibitem[\citeproctext]{ref-Polymath2014}
Polymath, D. H. J. 2014. {``New Equidistribution Estimates of Zhang
Type, and Bounded Gaps Between Primes.''}
\url{http://arxiv.org/abs/1402.0811}.

\bibitem[\citeproctext]{ref-Price1989}
Price, Huw. 1989. {``Defending Desire-as-Belief.''} \emph{Mind} 98
(389): 119--27. \url{https://doi.org/10.1093/mind/XCVIII.389.119}.

\bibitem[\citeproctext]{ref-Pryor2000}
Pryor, James. 2000. {``The Sceptic and the Dogmatist.''} \emph{No{û}s}
34 (4): 517--49. \url{https://doi.org/10.1111/0029-4624.00277}.

\bibitem[\citeproctext]{ref-Pryor2004}
---------. 2004. {``What's Wrong with Moore's Argument?''}
\emph{Philosophical Issues} 14 (1): 349--78.
\url{https://doi.org/10.1111/j.1533-6077.2004.00034.x}.

\bibitem[\citeproctext]{ref-Quiggin1982}
Quiggin, John. 1982. {``A Theory of Anticipated Utility.''}
\emph{Journal of Economic Behavior \& Organization} 3 (4): 323--43.
\url{https://doi.org/10.1016/0167-2681(82)90008-7}.

\bibitem[\citeproctext]{ref-Railton1984}
Railton, Peter. 1984. {``Alienation, Consequentialism, and the Demands
of Morality.''} \emph{Philosophy and Public Affairs} 13 (2): 134--71.

\bibitem[\citeproctext]{ref-Regan1980}
Regan, Donald. 1980. \emph{Utilitarianism and Cooperation}. Oxford:
{O}xford {U}niversity {P}ress.

\bibitem[\citeproctext]{ref-Reichenbach1956}
Reichenbach, Hans. 1956. \emph{The Direction of Time}. Berkeley:
University of California Press.

\bibitem[\citeproctext]{ref-Ribeiro2011}
Ribeiro, Brian. 2011. {``Epistemic Akrasia.''} \emph{International
Journal for the Study of Scepticism} 1: 18--25.
\url{https://doi.org/10.1163/221057011X554151}.

\bibitem[\citeproctext]{ref-Rosati2014}
Rosati, Connie S. 2016. {``Moral Motivation.''} In \emph{The Stanford
Encyclopedia of Philosophy}, edited by Edward N. Zalta, Winter 2016.
Metaphysics Research Lab, Stanford University.
\url{http://plato.stanford.edu/archives/win2016/entries/moral-motivation/}.

\bibitem[\citeproctext]{ref-Rosen2003}
Rosen, Gideon. 2003. {``Culpability and Ignorance.''} \emph{Proceedings
of the Aristotelian Society} 103 (1): 61--84.
\url{https://doi.org/10.1111/j.0066-7372.2003.00064.x}.

\bibitem[\citeproctext]{ref-Rosen2004}
---------. 2004. {``Skepticism about Moral Responsibility.''}
\emph{Philosophical Perspectives} 18 (1): 295--313.
\url{https://doi.org/10.1111/j.1520-8583.2004.00030.x}.

\bibitem[\citeproctext]{ref-Rosen2008}
---------. 2008. {``Kleinbart the Oblivious and Other Tales of Ignorance
and Responsibility.''} \emph{Journal of Philosophy} 105 (10): 591--610.
\url{https://doi.org/10.5840/jphil20081051023}.

\bibitem[\citeproctext]{ref-Ross2006}
Ross, Jacob. 2006. {``Rejecting Ethical Deflationism.''} \emph{Ethics}
116 (4): 742--68. \url{https://doi.org/10.1086/505234}.

\bibitem[\citeproctext]{ref-SchroederRoss2014}
Ross, Jacob, and Mark Schroeder. 2014. {``Belief, Credence, and
Pragmatic Encroachment.''} \emph{Philosophy and Phenomenological
Research} 88 (2): 259--88.
\url{https://doi.org/10.1111/j.1933-1592.2011.00552.x}.

\bibitem[\citeproctext]{ref-Rotondo2013}
Rotondo, Andrew. 2013. {``Undermining, Circularity, and Disagreement.''}
\emph{Synthese} 190 (3): 563--84.
\url{https://doi.org/10.1007/s11229-011-0050-2}.

\bibitem[\citeproctext]{ref-Russell1912}
Russell, Bertrand. 1912/1997. \emph{The Problems of Philosophy}. Oxford:
Oxford University Press.

\bibitem[\citeproctext]{ref-RussellHawthorne2016}
Russell, Jeffrey Sanford, and John Hawthorne. 2016. {``General Dynamic
Triviality Theorems.''} \emph{Philosophical Review} 125 (3): 307--39.
\url{https://doi.org/10.1215/00318108-3516936}.

\bibitem[\citeproctext]{ref-Sartre1946}
Sartre, Jean-Paul. 1946/2007. {``Existentialism Is a Humanism.''} In
\emph{Existentialism Is a Humanism}, translated by Annie Cohen-Solal,
17--72. New Haven: Yale University Press.

\bibitem[\citeproctext]{ref-Schechter2013}
Schechter, Josh. 2013. {``Rational Self-Doubt and the Failure of
Closure.''} \emph{Philosophical Studies} 163 (2): 429--52.
\url{https://doi.org/10.1007/s11098-011-9823-1}.

\bibitem[\citeproctext]{ref-Schoenfield2014b}
Schoenfield, Miriam. 2014. {``Permission to Believe: Why Permissivism Is
True and What It Tells Us about Irrelevant Influences on Belief.''}
\emph{No{û}s} 48: 193--218. \url{https://doi.org/10.1111/nous.12006}.

\bibitem[\citeproctext]{ref-Schoenfield2014}
---------. 2015. {``A Dilemma for Calibrationism.''} \emph{Philosophy
and Phenomenological Research} 91 (2): 425--55.
\url{https://doi.org/10.1111/phpr.12125}.

\bibitem[\citeproctext]{ref-Schwitzgebel2008}
Schwitzgebel, Eric. 2008. {``The Unreliability of Naive
Introspection.''} \emph{Philosophical Review} 117 (2): 245--73.
\url{https://doi.org/10.1215/00318108-2007-037}.

\bibitem[\citeproctext]{ref-Schwitzgebel2011}
---------. 2011. {``Self-Ignorance.''} In \emph{Consciousness and the
Self: New Essays}, edited by JeeLoo Liu and John Perry, 184--97.
Cambridge: Cambridge University Press.
\url{https://doi.org/10.1017/cbo9780511732355.009}.

\bibitem[\citeproctext]{ref-ScottBaillargeon2013}
Scott, Rose M., and Renée Baillargeon. 2013. {``Do Infants Really Expect
Others to Act Efficiently? A Critical Test of the Rationality
Principle.''} \emph{Psychological Science} 24 (4): 466--74.
\url{https://doi.org/10.1177/0956797612457395}.

\bibitem[\citeproctext]{ref-Scurr2006}
Scurr, Ruth. 2006. \emph{Fatal Purity: Robespierre and the French
Revolution}. London: Chatto \& Windus.

\bibitem[\citeproctext]{ref-Sepielli2009}
Sepielli, Andrew. 2009. {``What to Do When You Don't Know What to Do.''}
\emph{Oxford Studies in Metaethics} 4: 5--28.

\bibitem[\citeproctext]{ref-Sharot2012}
Sharot, Tali. 2012. \emph{The Optimism Bias: Why We're Wired to Look at
the Bright Side}. London: Constable; Robinson.

\bibitem[\citeproctext]{ref-Sidgwick1874}
Sidgwick, Henry. 1874. \emph{The Methods of Ethics}. London: Macmillan.

\bibitem[\citeproctext]{ref-SingerFAM}
Singer, Peter. 1972. {``Famine, Affluence and Morality.''}
\emph{Philosophy and Public Affairs} 1 (3): 229--43.

\bibitem[\citeproctext]{ref-Slote1992}
Slote, Michael. 1992. \emph{From Morality to Virtue}. Oxford: Oxford
University Press.

\bibitem[\citeproctext]{ref-Smart1961}
Smart, J. J. C. 1961. \emph{An Outline of a System of Utilitarian
Ethics}. Melbourne: University of Melbourne Press.

\bibitem[\citeproctext]{ref-AngelaSmith2005}
Smith, Angela M. 2005. {``Responsibility for Attitudes: Activity and
Passivity in Mental Life.''} \emph{Ethics} 115 (2): 236--71.
\url{https://doi.org/10.1086/426957}.

\bibitem[\citeproctext]{ref-Smith1994}
Smith, Michael. 1994. \emph{The Moral Problem}. Oxford: Blackwell.

\bibitem[\citeproctext]{ref-Smith1996}
---------. 1996. {``The Argument for Internalism: Reply to Miller.''}
\emph{Analysis} 56 (3): 175--84.
\url{https://doi.org/10.1111/j.0003-2638.1996.00175.x}.

\bibitem[\citeproctext]{ref-Smith2006}
---------. 2006. {``Moore on the Right, the Good, and Uncertainty.''} In
\emph{Metaethics After Moore}, edited by Terrence Horgan and Mark
Timmons, 133--48. Oxford: Oxford University Press.
\url{https://doi.org/10.1093/acprof:oso/9780199269914.003.0007}.

\bibitem[\citeproctext]{ref-Smith2009}
---------. 2009. {``Consequentialism and the Nearest and Dearest
Objection.''} In \emph{Minds, Ethics, and Conditionals: Themes from the
Philosophy of Frank Jackson}, edited by Ian Ravenscroft, 237--66.
Oxford: Oxford.
\url{https://doi.org/10.1093/acprof:oso/9780199267989.003.0011}.

\bibitem[\citeproctext]{ref-Smithies2012}
Smithies, Declan. 2012. {``Moore's Paradox and the Accessibility of
Justification.''} \emph{Philosophy and Phenomenological Research} 85
(2): 273--300. \url{https://doi.org/10.1111/j.1933-1592.2011.00506.x}.

\bibitem[\citeproctext]{ref-Srinivasan2015}
Srinivasan, Amia. 2015a. {``{A}re {W}e {L}uminous?''} \emph{{P}hilosophy
and {P}henomenological {R}esearch} 90 (2): 294--319.
\url{https://doi.org/10.1111/phpr.12067}.

\bibitem[\citeproctext]{ref-Srinivasan2015b}
---------. 2015b. {``Normativity Without Cartesian Privilege.''}
\emph{Philosophical Issues} 25: 273--99.
\url{https://doi.org/10.1111/phis.12059}.

\bibitem[\citeproctext]{ref-Staffel2015}
Staffel, Julia. 2015. {``Disagreement and Epistemic Utility-Based
Compromise.''} \emph{Journal of Philosophical Logic} 44 (3): 273--86.
\url{https://doi.org/10.1007/s10992-014-9318-6}.

\bibitem[\citeproctext]{ref-Stalnaker1998}
Stalnaker, Robert. 1998. {``Belief Revision in Games: Forward and
Backward Induction.''} \emph{Mathematical Social Sciences} 36 (1):
31--56. \url{https://doi.org/10.1016/S0165-4896(98)00007-9}.

\bibitem[\citeproctext]{ref-Stanley2005}
Stanley, Jason. 2005. \emph{{Knowledge and Practical Interests}}. Oxford
University Press.

\bibitem[\citeproctext]{ref-Steup2013}
Steup, Matthias. 2013. {``Is Epistemic Circularity Bad?''} \emph{Res
Philosophica} 90 (2): 215--35.
https://doi.org/\href{http://\%20dx.doi.org/\%2010.11612/\%20resphil.2013.90.2.8}{http://
dx.doi.org/ 10.11612/ resphil.2013.90.2.8}.

\bibitem[\citeproctext]{ref-Stoppard1967}
Stoppard, Tom. 1967/1994. \emph{Rosencrantz and Guildernstern Are Dead}.
New York: Grove Press.

\bibitem[\citeproctext]{ref-Strawson1962}
Strawson, P. F. 1962. {``Freedom and Resentment.''} \emph{Proceedings of
the British Academy} 48: 1--25.

\bibitem[\citeproctext]{ref-Svavarsdottir1999}
Svavarsdóttir, Sigrún. 1999. {``Moral Cognition and Motivation.''}
\emph{Philosophical Review} 108 (2): 161--219.
\url{https://doi.org/10.2307/2998300}.

\bibitem[\citeproctext]{ref-Tarsney2017}
Tarsney, Christian. 2017. {``Rationality and Moral Risk: A Moderate
Defense of Hedging.''} PhD thesis, University of Maryland, College Park.

\bibitem[\citeproctext]{ref-Titelbaum2014}
Titelbaum, Michael. 2014. \emph{Quitting Certainties: A Bayesian
Framework for Modeling Degrees of Belief}. Oxford: Oxford.

\bibitem[\citeproctext]{ref-Titelbaum2015}
---------. 2015. {``Rationality's Fixed Point (or: In Defence of Right
Reason).''} \emph{Oxford Studies in Epistemology} 5: 253--94.
\url{https://doi.org/10.1093/acprof:oso/9780198722762.003.0009}.

\bibitem[\citeproctext]{ref-Titelbaum2016}
---------. 2016. {``Self-Locating Credences.''} In \emph{Oxford Handbook
of Probability and Philosophy}, edited by Alan Hájek and Christopher
Hitchcock, 666--80. {O}xford {U}niversity {P}ress.
\url{https://doi.org/10.1093/oxfordhb/9780199607617.013.34}.

\bibitem[\citeproctext]{ref-Treynor1987}
Treynor, Jack L. 1987. {``Market Efficiency and the Bean Jar
Experiment.''} \emph{Financial Analysts Journal} 43 (3): 50--53.
\url{https://doi.org/10.2469/faj.v43.n3.50}.

\bibitem[\citeproctext]{ref-Vargas2005}
Vargas, Manuel. 2005. {``The Trouble with Tracing.''} \emph{Midwest
Studies in Philosophy} 29 (1): 269--91.
\url{https://doi.org/10.1111/j.1475-4975.2005.00117.x}.

\bibitem[\citeproctext]{ref-Vavova2014}
Vavova, Katia. 2014. {``Moral Disagreement and Moral Skepticism.''}
\emph{Philosophical Perspectives} 28 (1): 302--33.
\url{https://doi.org/10.1111/phpe.12049}.

\bibitem[\citeproctext]{ref-Vogel1990}
Vogel, Jonathan. 1990. {``Cartesian Skepticism and Inference to the Best
Explanation.''} \emph{Journal of Philosophy} 87 (11): 658--66.
\url{https://doi.org/10.5840/jphil1990871123}.

\bibitem[\citeproctext]{ref-Vogel2000}
---------. 2000. {``Reliabilism Leveled.''} \emph{Journal of Philosophy}
97 (11): 602--23. \url{https://doi.org/10.2307/2678454}.

\bibitem[\citeproctext]{ref-Watson1996}
Watson, Gary. 1996. {``Two Faces of Responsibility.''}
\emph{Philosophical Topics} 24 (2): 227--48.

\bibitem[\citeproctext]{ref-Weatherson1999}
Weatherson, Brian. 1999. {``Begging the Question and Bayesians.''}
\emph{Studies in the History and Philosophy of Science Part A} 30:
687--97. \url{https://doi.org/10.1016/S0039-3681(99)00020-5}.

\bibitem[\citeproctext]{ref-Weatherson2003-sim}
---------. 2003. {``Are You a Sim?''} \emph{Philosophical Quarterly} 53
(212): 425--31. \url{https://doi.org/10.1111/1467-9213.00323}.

\bibitem[\citeproctext]{ref-Weatherson2004}
---------. 2004. {``Luminous Margins.''} \emph{Australasian Journal of
Philosophy} 82 (3): 373--83. \url{https://doi.org/10.1080/713659874}.

\bibitem[\citeproctext]{ref-Weatherson2005}
---------. 2005. {``Scepticism, Rationalism and Externalism.''}
\emph{Oxford Studies in Epistemology} 1: 311--31.

\bibitem[\citeproctext]{ref-Weatherson2012}
---------. 2012. {``Knowledge, Bets and Interests.''} In \emph{Knowledge
Ascriptions}, edited by Jessica Brown and Mikkel Gerken, 75--103.
Oxford: Oxford University Press.
\url{https://doi.org/10.1093/acprof:oso/9780199693702.003.0004}.

\bibitem[\citeproctext]{ref-Weatherson2013Lewis}
---------. 2013. {``The Role of Naturalness in Lewis's Theory of
Meaning.''} \emph{Journal for the History of Analytical Philosophy} 1
(10): 1--19. \url{https://doi.org/10.4148/jhap.v1i10.1620}.

\bibitem[\citeproctext]{ref-Weatherson2014}
---------. 2014a. {``Games, Beliefs and Credences.''} \emph{Philosophy
and Phenomenological Research} 92 (2): 209--36.
\url{https://doi.org/10.1111/phpr.12088}.

\bibitem[\citeproctext]{ref-Weatherson2014-ProbScept}
---------. 2014b. {``Probability and Scepticism.''} In \emph{Scepticism
and Perceptual Justification}, edited by Dylan Dodd and Elia Zardini,
71--86. Oxford: Oxford University Press.
\url{https://doi.org/10.1093/acprof:oso/9780199658343.003.0004}.

\bibitem[\citeproctext]{ref-Weatherson2015}
---------. 2015. {``Memory, Belief and Time.''} \emph{Canadian Journal
of Philosophy} 45 (5-6): 692--715.
\url{https://doi.org/10.1080/00455091.2015.1125250}.

\bibitem[\citeproctext]{ref-Wedgwood2012}
Wedgwood, Ralph. 2012. {``Justified Inference.''} \emph{Synthese} 189
(2): 273--95. \url{https://doi.org/10.1007/s11229-011-0012-8}.

\bibitem[\citeproctext]{ref-Weyrich2012}
Weirich, Paul. 2016. {``Causal Decision Theory.''} In \emph{The Stanford
Encyclopedia of Philosophy}, edited by Edward N. Zalta, Winter 2016.
Metaphysics Research Lab, Stanford University.
\url{http://plato.stanford.edu/archives/win2016/entries/decision-causal/}.

\bibitem[\citeproctext]{ref-Weisberg2010}
Weisberg, Jonathan. 2010. {``Bootstrapping in General.''}
\emph{Philosophy and Phenomenological Research} 81 (3): 525--48.
\url{https://doi.org/10.1111/j.1933-1592.2010.00448.x}.

\bibitem[\citeproctext]{ref-White2005}
White, Roger. 2005. {``Epistemic Permissiveness.''} \emph{Philosophical
Perspectives} 19 (1): 445--59.
\url{https://doi.org/10.1111/j.1520-8583.2005.00069.x}.

\bibitem[\citeproctext]{ref-White2006}
---------. 2006. {``Problems for Dogmatism.''} \emph{Philosophical
Studies} 131: 525--57. \url{https://doi.org/10.1007/s11098-004-7487-9}.

\bibitem[\citeproctext]{ref-White2009}
---------. 2009. {``On Treating Oneself and Others as Thermometers.''}
\emph{Episteme} 6 (3): 233--50.
\url{https://doi.org/10.3366/E1742360009000689}.

\bibitem[\citeproctext]{ref-Williams1981}
Williams, Bernard. 1981. {``Persons, Character, and Morality.''} In
\emph{Moral Luck}, 1--19. Cambridge: Cambridge University Press.
\url{https://doi.org/10.1017/cbo9781139165860.002}.

\bibitem[\citeproctext]{ref-Williams1995}
---------. 1995. {``Saint-Just's Illusion.''} In \emph{Making Sense of
Humanity and Other Philosophical Essays}, 135--50. Cambridge:
{C}ambridge {U}niversity {P}ress.
\url{https://doi.org/10.1017/cbo9780511621246.013}.

\bibitem[\citeproctext]{ref-Williamson2000}
Williamson, Timothy. 2000. \emph{{Knowledge and its Limits}}. Oxford
University Press.

\bibitem[\citeproctext]{ref-Williamson2007}
---------. 2007. \emph{The Philosophy of Philosophy}. Oxford: Blackwell.

\bibitem[\citeproctext]{ref-Williamson2011}
---------. 2011. {``{I}mprobable {K}nowing.''} In \emph{{E}videntialism
and Its {D}iscontents}, edited by T. Dougherty, 147--64. {O}xford
{U}niversity {P}ress.
\url{https://doi.org/10.1093/acprof:oso/9780199563500.003.0010}.

\bibitem[\citeproctext]{ref-Williamson2014}
---------. 2014. {``Very Improbable Knowing.''} \emph{Erkenntnis} 79
(5): 971--99. \url{https://doi.org/10.1007/s10670-013-9590-9}.

\bibitem[\citeproctext]{ref-Wittgenstein1953}
Wittgenstein, Ludwig. 1953. \emph{Philosophical Investigations}. London:
Macmillan.

\bibitem[\citeproctext]{ref-Wolf1987}
Wolf, Susan. 1987. {``Sanity and the Metaphysics of Responsibility.''}
In \emph{{R}esponsibility, {C}haracter, and the {E}motions: {N}ew
{E}ssays in {M}oral {P}sychology}, edited by Ferdinand David Schoeman,
46--62. {C}ambridge {U}niversity {P}ress.
\url{https://doi.org/10.1017/cbo9780511625411.003}.

\bibitem[\citeproctext]{ref-Worsnip2014}
Worsnip, Alex. 2014. {``Disagreement about Disagreement? What
Disagreement about Disagreement?''} \emph{Philosophers' Imprint} 14
(18): 1--20. \url{http://hdl.handle.net/2027/spo.3521354.0014.018}.

\bibitem[\citeproctext]{ref-Wright2000}
Wright, Crispin. 2000. {``Cogency and Question-Begging: Some Reflections
on McKinsey's Paradox and Putnam's Proof.''} \emph{Philosophical Issues}
10: 140--63. \url{https://doi.org/10.1111/j.1758-2237.2000.tb00018.x}.

\bibitem[\citeproctext]{ref-Wright2002}
---------. 2002. {``(Anti-)sceptics Simple and Subtle: G.e. Moore and
John McDowell.''} \emph{Philosophy and Phenomenological Research} 65
(2): 330--48. \url{https://doi.org/10.1111/j.1933-1592.2002.tb00205.x}.

\bibitem[\citeproctext]{ref-Zimmerman2008}
Zimmerman, Michael J. 2008. \emph{Living with Uncertainty: The Moral
Significance of Ignorance}. Cambridge: Cambridge University Press.

\end{CSLReferences}


\backmatter

\end{document}
